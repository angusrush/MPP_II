\documentclass[main.tex]{subfiles}
\begin{document}

\section{A new Barwick's theorem}
\label{sec:a_new_barwick_s_theorem}

It turns out Barwick's construction can be obviously modified. Here is a form which is more useful to us. Little to no effort has been made to make this as general as possible, as long as it works. It is relatively clear from the proof that there are many sets of conditions that suffice, and probably no unique weakest one.

\begin{theorem}
  \label{thm:new_barwick}
  Let $p\colon \triple{C} \to \triple{D}$ be a functor between adequate triples such that $p\colon \category{C} \to \category{D}$ is an inner fibration which satisfies the following conditions.
  \begin{enumerate}
    \item A morphism in $\category{C}$ is egressive if and only if it is $p$-cartesian (hence also $p\updag$-cartesian).

    \item Every egressive morphism in $\category{D}$ admits a lift in $\category{C}$ (given a lift of the target) which is both egressive and $p$-cartesian.

    \item Consider a square
      \begin{equation*}
          \sigma = \quad
          \begin{tikzcd}
            y'
            \arrow[r, "f'"]
            \arrow[d, two heads, swap, "g'"]
            & x'
            \arrow[d, two heads, "g"]
            \\
            y
            \arrow[r, rightarrowtail, "f"]
            & x
          \end{tikzcd}
      \end{equation*}
      in $\category{C}$ with egressive and ingressive morphisms as marked, in which $f$ is $p$-cocartesian and $g$ and $g'$ are $p$-cartesian, lying over an ambigressive pullback square
      \begin{equation*}
        p(\sigma) = \quad
        \begin{tikzcd}
          py'
          \arrow[r, rightarrowtail]
          \arrow[d, two heads]
          & px'
          \arrow[d, two heads]
          \\
          py
          \arrow[r, rightarrowtail]
          & px
        \end{tikzcd}
      \end{equation*}
      in $\category{D}$. Then  $f'$ is ingressive, $p$-cocartesian, and and $p\downdag$-cocartesian.
  \end{enumerate}
  Then spans of the form
  \begin{equation*}
    \begin{tikzcd}
      & z
      \arrow[dl, two heads, swap, "g"]
      \arrow[dr, rightarrowtail, "f"]
      \\
      x
      && y
    \end{tikzcd}
  \end{equation*}
  are cocartesian, where $g$ is $p\updag$-cartesian and $f$ is $p$-cocartesian.
\end{theorem}
\begin{proof}
  We first note that $\sigma$ is automatically pullback; since the sides are $p$-cartesian, it is automatically a relative pullback, and it lies over a pullback square.

  The rest is even simpler than the proof in my thesis of the ordinary Barwick's theorem. The square that we have to show is homotopy pullback factors as before, but this time we don't have to take the final homotopy pullback to find path components corresponding to cartesian 2-simplices, since our 2-simplices are automatically cartesian.
\end{proof}



\end{document}
