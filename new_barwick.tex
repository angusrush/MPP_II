\documentclass[main.tex]{subfiles}
\begin{document}

\section{A new Barwick's theorem}
\label{sec:a_new_barwick_s_theorem}

It turns out Barwick's construction can be obviously modified. Here is a form which is more useful to us. Little to no effort has been made to make this as general as possible, as long as it works. It is relatively clear from the proof that there are many sets of conditions that suffice, and probably no unique weakest one.

\begin{theorem}[Old Barwick]
  \label{thm:old_barwick}
  Let $p\colon \triple{C} \to \triple{D}$ be a functor between adequate triples such that $p\colon \category{C} \to \category{D}$ is an inner fibration which satisfies the following conditions.
  \begin{enumerate}
    \item Each morphism $g \in \category{D}\updag$ admits a lift to a morphism in $\category{C}\updag$ (given a lift of the source) which is both $p$-cocartesian and $p\updag$-cocartesian.

    \item Consider a commutative square
      \begin{equation*}
        \sigma = \quad
        \begin{tikzcd}
          y'
          \arrow[r, rightarrowtail, "f'"]
          \arrow[d, two heads, swap]
          & x'
          \arrow[d, "g"]
          \\
          y
          \arrow[r, rightarrowtail, "f"]
          & x
        \end{tikzcd}
      \end{equation*}
      in $\category{C}$ where $g$ belongs to $\category{C}\updag$, and $f$ and $f'$ belong to $\category{C}\downdag$. Suppose that $f$ is $p$-cocartesian. Then $f'$ is $p'$-cocartesian if and only if $\sigma$ is an ambigressive pullback square (and in particular $g \in \category{C}\updag$).
  \end{enumerate}
\end{theorem}

\begin{theorem}[New Barwick]
  \label{thm:new_barwick}
  Let $p\colon \triple{C} \to \triple{D}$ be a functor between adequate triples such that $p\colon \category{C} \to \category{D}$ is an inner fibration which satisfies the following conditions.
  \begin{enumerate}
    \item The subcategory $\category{C}\updag \subseteq \category{C}$ consists of all $p$-cartesian morhisms in $\category{C}$; that is, an $n$-simplex in $\category{C}$ belongs to $\category{C}\updag$ if and only if each $1$-simplex it contains is $p$-cartesian.

    \item The map $p\updag\colon \category{C}\updag \to \category{D}\updag$ is a cartesian fibration.

    \item Consider a square
      \begin{equation*}
          \sigma = \quad
          \begin{tikzcd}
            y'
            \arrow[r, "f'"]
            \arrow[d, two heads, swap, "g'"]
            & x'
            \arrow[d, two heads, "g"]
            \\
            y
            \arrow[r, rightarrowtail, "f"]
            & x
          \end{tikzcd}
      \end{equation*}
      in $\category{C}$ where $g$ and $g'$ belong to $\category{C}\updag$, and $f$ belongs to $\category{C}\downdag$. Further suppose that $f$ is $p$-cocartesian. Then $f'$ belongs to $\category{C}\downdag$, and is both $p$-cocartesian and $p\downdag$-cocartesian.
  \end{enumerate}
  Then spans of the form
  \begin{equation*}
    \begin{tikzcd}
      & z
      \arrow[dl, two heads, swap, "g"]
      \arrow[dr, rightarrowtail, "f"]
      \\
      x
      && y
    \end{tikzcd}
  \end{equation*}
  are cocartesian, where $g$ is $p\updag$-cartesian and $f$ is $p$-cocartesian.
\end{theorem}
\begin{proof}
  We first note that $\sigma$ is automatically pullback; since the sides are $p$-cartesian, it is automatically a relative pullback, and it lies over a pullback square.

  The rest is even simpler than the proof in my thesis of the ordinary Barwick's theorem. The square that we have to show is homotopy pullback factors as before, but this time we don't have to take the final homotopy pullback to find path components corresponding to cartesian 2-simplices, since our 2-simplices are automatically cartesian.
\end{proof}



\end{document}
