\documentclass[main.tex]{subfiles}
\begin{document}

\section{Background}
\label{sec:background}

\subsection{Scaled simplicial sets}
\label{ssc:scal}

Note: This is basically a summary of the parts of The Goodwillie Calculus that we're gonna need.

\begin{definition}
  \label{def:scaled_simplicial_set}
  A \defn{scaled simplicial set} is a pair $(X, A_{X}) = \overline{X}$, where $X$ is a simplicial set, and $A \subseteq X_{2}$ is a set of $2$-simplices of $X$ which contains every degenerate simplex. A simplex belonging to $A$ is known as \emph{thin.} A morphism of scaled simplicial sets is a morphism of the underlying simplicial sets which takes thin simplices to thin simplices.
\end{definition}

We will need the following facts about scaled simplicial sets.

\begin{enumerate}
  \item For any category $\category{C}$ enriched in marked simplicial sets, we can build a scaled simplicial set $\Nsc(\category{C}) = (\N(\category{C}), T)$, the \defn{scaled nerve} of $\category{C}$, whose underlying simplicial set $\N(\category{C})$ is the simplicial nerve of $\category{C}$. The scaling $T$ is defined in the following way. Let $\sigma\colon \Delta^{2} \to \N(\category{C})$ be any 2-simplex.
    This corresponds to a diagram
    \begin{equation*}
      \begin{tikzcd}
        & Y
        \arrow[dr, "g"]
        \\
        X
        \arrow[ur, "f"]
        \arrow[rr, "h"]
        && Z
      \end{tikzcd}
    \end{equation*}
    in $\category{C}$, together with a 1-simplex $\alpha_{\sigma}\colon \Delta^{1} \to \Map_{\category{C}}(X, Z)$ corresponding to a map $gf \to h$. We will then define $\sigma$ to be thin if and only if $\alpha_{\sigma}$ is marked in $\Map_{\category{C}}(X, Z)$.

  \item The functor $\Nsc$ admits a left adjoint $\Csc\colon \SSetsc \to \SCatmk$, the \defn{scaled rigidification}, such that for any scaled simplicial set $\overline{X}$ with underlying simplicial set $X$, the underlying simplicially enriched category of $\Csc(\overline{X})$ is simply $\C(X)$, the rigidification of $X$. The scaling on $\Csc(X)$ also admits a relatively simple description.

  \item A map of scaled simplicial sets $\overline{X} \to \overline{Y}$ is said to be a \defn{bicategorical equivalence} if the map $F\colon \Csc(X) \to \Csc(Y)$ is a weak equivalence of $\SSet$-enriched categories; that is, if for all objects $x$, $x'$ of $\Csc(\overline{X})$, the map
    \begin{equation*}
      \Hom_{\Csc(\overline{X})}(x, x') \to \Hom_{\Csc(\overline{Y})}(F(x), F(x'))
    \end{equation*}
    is a weak equivalence of marked simplicial sets, and if for all $y \in \Csc(\overline{Y})$, there exists some $x \in \Csc(\overline{X})$ such that $F(x)$ and $y$ are isomorphic in the homotopy category $\h \Csc(\overline{Y})$.

  \item There is a left proper, combinatorial model structure on $\SSetsc$, whose cofibrations are monomorphisms, and whose weak equivalences are bicategorical fibrations. This gives a Quillen equvialence
    \begin{equation*}
      \Csc : \SSetsc \leftrightarrow \SCatmk : \Nsc.
    \end{equation*}

  \item An \defn{$\infty$-bicategory} is a fibrant object with respect to this model structure. In particular, for any fibrant object $\category{C} \in \SCatmk$ (i.e.\ any quasicategory-enriched category with equvialences marked), $\Nsc(\category{C})$ is an $\infty$-bicategory.

  \item The category of scaled simplicial sets has a natural simplicial enrichment as follows: For any two scaled simplicial sets $\overline{X}$ and $\overline{Y}$, we define
    \begin{equation*}
      \Mapsc(\overline{X}, \overline{Y})_{n} = \Hom(\overline{X} \times \Delta^{n}_{\sharp}, \overline{Y}),
    \end{equation*}
    where $\Delta^{n}_{\sharp}$ is the scaled simplicial set $(\Delta^{n}, \Delta^{n}_{2})$. For $\overline{Y}$ an $\infty$-bicategory, $\Mapsc(\overline{X}, \overline{Y})$ is a quasicategory. Taking the core gives a Kan complex.

  \item With this,

  \item We will denote the simplicially enriched category whose objects are $\infty$-bicategories, with mapping spaces as above by $\catname{Bicat}_{\infty}$. We then define $\Cat_{(\infty, 2)} = \N(\catname{Bicat}_{\infty})$. This is a quasicategory.

  \item We have also a simplicially enriched category whose objects are quasicategories, and whose morphisms are $\SSetmk$-enriched mapping spaces. The coherent nerve of this is
\end{enumerate}

We will need multiple notions of the quasicategory of $(\infty,2)$-categories.
\begin{itemize}
  \item The model category $\SSetsc$ of scaled simplicial sets, whose cofibrations are precisely the monomorphisms.

  \item The weak equivalences are bicategorical equivalences,
\end{itemize}

\subsection{Enhanced twisted arrow categories}
\label{ssc:enhanced_twisted_arrow_categories}

This is a summary of the parts of the theory that we'll need.
\begin{enumerate}
  \item For each $n \in \N$, we consider the simplicial set $\Delta^{n} \star (\Delta^{n})\op$. We will denote the vertices of $\Delta^{n}$ by $i$ and the vertices of $(\Delta^{n})\op$ by $\bar{i}$. This is really just a fancy notation for the $(2n+1)$-simplex with vertices
    \begin{equation*}
      1, \ldots, n, \bar{n}, \ldots, \bar{1}.
    \end{equation*}

  \item We define a cosimplicial object
    \begin{equation*}
      Q\colon \Delta \to \SSetsc;\qquad [n] \mapsto (\Delta^{n} \star (\Delta^{n})\op, T_{n}),
    \end{equation*}
    where $T_{n}$ is the set of $2$-simplices $\sigma\colon \Delta^{2} \to \Delta^{n} \star (\Delta^{n})\op$ which
    \begin{itemize}
      \item factor through $\Delta^{n}$; or
      \item factor through $(\Delta^{n})\op$; or
      \item are of the form $\Delta^{\{i,j,\bar{k}\}}$, where $i < j \leq k$; or
      \item are of the form $\Delta^{\{k, \bar{j}, \bar{i}\}}$, where $i < j \leq k$.
    \end{itemize}

  \item For any scaled simplicial set $\overline{X}$, we define a simplicial set
    \begin{equation*}
      \Tw(\overline{X})_{n} = \Hom_{\SSetsc}(Q([n]), \overline{X}).
    \end{equation*}
    We further define a marking on $\Tw(\overline{X})$ by saying that a 1-simplex in $\Tw(\overline{X})$ is marked if and only if it corresponds to a map $\Delta^{3}_{\sharp} \to \overline{X}$.

  \item This construction clearly extends to an ordinary functor $\Tw\colon \SSetsc \to \SSetmk$. One checks that this functor preserves weak equivalences, and thus extends to a functor of quasicategories or $\Kan$-enriched categories or whatever your favorite model is
    \begin{equation*}
      \Tw\colon \ITCat \to \ICat.
    \end{equation*}

  \item For any scaled simplicial set $\overline{X}$, a zero-simplex in $\Tw(\overline{X})$ is a 1-simplex in $\overline{X}$, and a 1-simplex in $\Tw(\overline{X})$ is a 3-simplex in $\overline{X}$ of the form
    \begin{equation*}
      \begin{tikzcd}[column sep=large, row sep=large]
        x
        \arrow[r]
        \arrow[d, ""{name=L, right}]
        \arrow[dr]
        & x'
        \arrow[d, ""{name=R, left}]
        \arrow[from=l, to=R, phantom, near end, "\circlearrowleft"{description}]
        \\
        y
        & y'
        \arrow[l]
        \arrow[from=L, Rightarrow, shorten=3ex]
      \end{tikzcd}
      \qquad
      \begin{tikzcd}[column sep=large, row sep=large]
        x
        \arrow[r]
        \arrow[d, ""{name=L, right}]
        & x'
        \arrow[d, ""{name=R, left}]
        \arrow[dl]
        \\
        y
        \arrow[from=R, phantom, near start, "\circlearrowleft"{description}]
        & y'
        \arrow[l]
        \arrow[from=L, to=u, shorten=3ex, Rightarrow]
      \end{tikzcd},
    \end{equation*}
    where the simplices marked with $\circlearrowleft$ are thin. The $\Rightarrow$-marked simplices are simply to remind us that given a $\SSetmk$-enriched category $\category{C}$, two-simplices in $\Csc(\category{C})$ correspond to diagrams in $\category{C}$ of the form
    \begin{equation*}
      \begin{tikzcd}
        & c'
        \arrow[dr, "g"]
        \\
        c
        \arrow[ur, "f"]
        \arrow[rr, "h"{below}, ""{name=B, above}]
        && c''
        \arrow[from=B, to=ul, Rightarrow, "\eta"]
      \end{tikzcd},
    \end{equation*}
    where $\eta$ is a 2-morphism $h \Rightarrow g \circ f$.
\end{enumerate}

\begin{definition}
  We will call a 1-simplex $\Delta^{1} \to \Tw(\overline{X})$ \defn{thin} if it corresponds to a map $(\Delta^{1} \star {\Delta^{1}}\op)_{\sharp} \to \overline{X}$; that is, a 1-simplex in $\Delta^{1} \to \Tw(\overline{X})$ is thin if and only if each of the faces of the adjunct map $\Delta^{1} \star (\Delta^{1})\op$ is a thin 2-simplex.
\end{definition}

\subsection{The universal property of Kan extensions}
\label{ssc:the_universal_property_of_kan_extensions}

We start by providing a proof of the standard universal property for Kan extensions.

\begin{definition}
  Let $\category{C}$, $\category{D}$, and $\category{E}$ be 1-categories, and let $f\colon \category{C} \to \category{D}$ and $\mathcal{F}\colon \category{C} \to \category{E}$ be functors. A left Kan extension $\mathcal{F}$ along $f$ is a pair $( f_{!}\mathcal{F}, \eta )$, where $f_{!}\colon \category{D} \to \category{E}$ is a functor, and $\eta$ is a natural transformation $\eta\colon \mathcal{F} \Rightarrow f_{!}\mathcal{F} \circ f$ satisfying the following universal property: for any functor $\mathcal{G}\colon \category{D} \to \category{E}$ and natural transformation $\gamma\colon \mathcal{F} \Rightarrow \mathcal{G} \circ f$, there exists a unique natural transformation $f_{!}\mathcal{F} \Rightarrow \mathcal{G}$ such that the pasting
  \begin{equation*}
    \begin{tikzcd}[row sep=huge, column sep=huge]
      \category{C}
      \arrow[rr, ""{name=M, below}, "\mathcal{F}"]
      \arrow[dr, swap, "f"]
      && \category{E}
      \\
      & \category{D}
      \arrow[from=M, Rightarrow, "\eta", swap]
      \arrow[ur, bend left, ""{name=LU, swap}, "f_{!}\mathcal{F}"]
      \arrow[ur, bend right, ""{name=LD, swap}, swap, "\mathcal{G}"]
      \arrow[from=LU, to=LD, Rightarrow, "\exists!\chi"]
    \end{tikzcd}
  \end{equation*}
  is equal to $\gamma$.
\end{definition}

\begin{proposition}
  Suppose we are given categories $\category{C}$, $\category{D}$, and $\category{E}$, and functors $\mathcal{F}$ and $f$ as above, and assume that for all objects $c \in \category{C}$, the colimit
  \begin{equation*}
    \colim\left[ \category{C}_{/d} \to \category{C} \to \category{E} \right]
  \end{equation*}
  exists. Then the left Kan extension $(f_{!}\mathcal{F}, \eta)$ exists, and is specified up to natural isomorphism by the prescription that on objects the functor $\mathcal{F}_{!}f$ should take the values
  \begin{equation*}
    f_{!}\mathcal{F}(d) = \colim \left[ \category{C}_{/d} \to \category{C} \to \category{E} \right].
  \end{equation*}
\end{proposition}
\begin{proof}
  We have specified our functor on objects. We need to specify how it acts on morphisms. Let $d \to d'$ be a morphism in $\category{D}$. We need to provide a morphism
  \begin{equation}
    \label{eq:map_between_colimits_defining_kan_extension_on_morphism}
    \colim\left[ \category{C}_{/d} \to \category{C} \to \category{E} \right] \to \colim\left[ \category{C}_{/d'} \to \category{C} \to \category{E} \right].
  \end{equation}

  The map $\category{C}_{/d} \to \category{C}$ factors through the map $\category{C}_{/d'} \to \category{C}$ via the map $\category{C}_{/d} \to \category{C}_{/d'}$ given by composing everything in sight with our map $d \to d'$, sending
  \begin{equation*}
    (c, f(c) \to d) \mapsto (c, f(c) \to d \to d').
  \end{equation*}
  This allows us to turn the cocone $(\kappa', f_{!}\mathcal{F}(d'))$ under $\category{C}_{/d'} \to \category{C} \to \category{E}$ to a cocone $(\kappa, f_{!}\mathcal{F}(d'))$ under $\category{C}_{/d} \to \category{C} \to \category{E}$. The cocone $\kappa$ has components 
  \begin{equation*}
    \kappa_{(c, f(c) \to d)} = \kappa'_{(c, f(c) \to d \to d')}\colon \mathcal{F}(c) \to f_{!}\mathcal{F}(d'),
  \end{equation*}
  and it is easy to see that these really do form a cocone under $\category{C}_{/d} \to \category{C} \to \category{E}$; for any morphism $c \to c'$ in $\category{C}_{/d}$, the diagram
  \begin{equation}
    \label{eq:first_triangle_kan_extension}
    \begin{tikzcd}[row sep=small]
      \mathcal{F}(c)
      \arrow[dd]
      \arrow[dr]
      \\
      & f_{!}\mathcal{F}(d')
      \\
      \mathcal{F}(c')
      \arrow[ur]
    \end{tikzcd}
  \end{equation}
  commutes because $c \to c'$ is also a morphism in $\mathcal{C}_{/d'}$. We therefore get a map out of the \emph{universal} cocone under this diagram, i.e.\ $f_{!}\mathcal{F}(d)$, into $f_{!}\mathcal{F}(d')$; according to the universal property for colimits, this is the uniquely defined such map so that for all objects $(c, f(c) \to d)$ in $\category{C}_{/d}$, the diagram
  \begin{equation}
    \label{eq:second_triangle_kan_extension}
    \begin{tikzcd}[row sep=small]
      & f_{!}\mathcal{F}(d)
      \arrow[dd]
      \\
      \mathcal{F}(c)
      \arrow[ur]
      \arrow[dr]
      \\
      & f_{!}\mathcal{F}(d')
    \end{tikzcd}
  \end{equation}
  commutes. This is how we define our functor on morphisms. Functoriality follows from uniqueness of these maps.

  Next, we need to provide the components of the natural transformation $\eta$; that is, for each $c \in \category{C}$, we need a map
  \begin{equation*}
    \eta_{c}\colon \mathcal{F}(c) \to f_{!}\mathcal{F}(f(c)) = \colim \left[ \category{C}_{/f(c)} \to \category{C} \to \category{E} \right].
  \end{equation*}
  The overcategory $\category{C}_{/f(c)}$ contains a distinguished object $(c, f(\id_{c})\colon f(c) \to f(c))$, and the image of this object in $\category{E}$ is $\mathcal{F}(c)$. The corresponding leg of the cocone thus provides the morphism $\eta_{c}$ we are interested in.

  Next, we show naturality. Let $c \to c'$ be a morphism in $\category{C}$. Then $f(c) \to f(c')$ is a morphism in $\category{D}$, and this is what gives us the map $f_{!}\mathcal{F}(f(c)) \to f_{!}\mathcal{F}(f(c'))$. Consider the diagram
  \begin{equation*}
    \begin{tikzcd}
      \mathcal{F}(c)
      \arrow[r]
      \arrow[d]
      \arrow[dr]
      & \mathcal{F}(c')
      \arrow[d]
      \\
      \colim\left[ \category{C}_{/f(c)} \to \category{C} \to \category{E} \right]
      \arrow[r]
      & \colim\left[ \category{C}_{/f(c')} \to \category{C} \to \category{E} \right]
    \end{tikzcd},
  \end{equation*}
  where the left- and right-hand arrows are the components of the yet-to-be-proven-natural transformation $\eta$, and the diagonal map is part of the colimit cone for $f_{!}\mathcal{F}(f(c'J))$ corresponding to the object $(c, f(c) \to f(c'))$. The upper triangle is of the type mentioned in \hyperref[eq:first_triangle_kan_extension]{Diagram~\ref*{eq:first_triangle_kan_extension}}. The lower triangle is of the type mentioned in \hyperref[eq:second_triangle_kan_extension]{Diagram~\ref*{eq:second_triangle_kan_extension}}.

  It remains only to check that our data satisfies the correct universal property. Let $\mathcal{G}\colon \category{D} \to \category{E}$ be a functor, and let $\gamma\colon \mathcal{F} \Rightarrow \mathcal{G} \circ f$ be a natural transformation. We first need to find the components of our natural transformation $\chi$, i.e.\ maps
  \begin{equation*}
    \chi_{d}\colon f_{!}\mathcal{F}(d) = \colim \left[ \mathcal{C}_{/d} \to \category{C} \to \category{E} \right] \to \category{G}(d).
  \end{equation*}
  This is induced by the universal property for colimits via the cocone with components indexed by $(c, \alpha\colon f(c) \to d) \in \category{C}_{/d}$ given by
  \begin{equation*}
    \mathcal{F}(c) \overset{\gamma_{c}}{\to} \mathcal{G}(f(c)) \overset{\mathcal{G}(\alpha)}{\to} \mathcal{G}(d).
  \end{equation*}

  In order to show that this really is a cocone, we need to show that for any morphism $c \to c'$ in $\category{C}_{/d}$, the outer pentagon in the diagram
  \begin{equation*}
    \begin{tikzcd}[row sep=small]
      \mathcal{F}(c)
      \arrow[r]
      \arrow[dd]
      & \mathcal{G}(f(c))
      \arrow[dr]
      \arrow[dd]
      \\
      && \mathcal{G}(d)
      \\
      \mathcal{F}(c')
      \arrow[r]
      & \mathcal{G}(f(c'))
      \arrow[ur]
    \end{tikzcd}
  \end{equation*}
  commutes. But the left-hand square commutes because $\gamma$ is natural, and the right-hand triangle commutes by definition of the overcategory.

  Now we check that the natural transformation defined in this way really is natural. In order to do that, we need to check that the following solid diagram commutes.
  \begin{equation}
    \label{eq:pentagon_naturality_of_univ_prop}
    \begin{tikzcd}[row sep=small]
      & \colim\left[ \category{C}_{/d} \to \category{C} \to \category{E} \right]
      \arrow[r]
      \arrow[dd]
      \arrow[ddr, dashed]
      & \category{G}(d)
      \arrow[dd]
      \\
      \mathcal{F}(c)
      \arrow[ur, dotted]
      \arrow[dr, dotted]
      \\
      & \colim\left[ \category{C}_{/d'} \to \category{C} \to \category{E} \right]
      \arrow[r]
      & \category{G}(d')
    \end{tikzcd}
  \end{equation}
  There is a cocone under $\category{C}_{/d} \to \category{C} \to \category{E}$ with nadir $\category{G}(d')$, where the leg corresponding to $(c, f(c) \to d)$ is given by either solid composition in the diagram
  \begin{equation}
    \label{eq:inner_square_naturality_of_univ_prop}
    \begin{tikzcd}[row sep=small]
      & \colim\left[ \category{C}_{/d} \to \category{C} \to \category{E} \right]
      \arrow[dd]
      \arrow[ddr, dashed]
      \\
      \mathcal{F}(c)
      \arrow[ur]
      \arrow[dr]
      \\
      & \colim\left[ \category{C}_{/d'} \to \category{C} \to \category{E} \right]
      \arrow[r]
      & \category{G}(d')
    \end{tikzcd}.
  \end{equation}
  (Both composites agree because the triangle commutes). The dashed diagonal map is the defined by the universal property of the colimit to be the unique map such that for all objects $(c, f(c) \to d)$, the outside of \hyperref[eq:inner_square_naturality_of_univ_prop]{Diagram~\ref*{eq:inner_square_naturality_of_univ_prop}} commutes. But both ways of composing around the solid square in \hyperref[eq:pentagon_naturality_of_univ_prop]{Diagram~\ref*{eq:pentagon_naturality_of_univ_prop}} make this triangle commute, so they must agree.
\end{proof}

Note that this is equivalent to the following definition. Consider a map
\begin{equation*}
  \tau_{\flat}\colon \Lambda^{3}_{0} \to \Cat
\end{equation*}
such that $d_{2}\tau$ is the triangle
\begin{equation*}
  \begin{tikzcd}
    \category{C}
    \arrow[rr, "\mathcal{F}", ""{swap, name=M}]
    \arrow[dr, swap, "\mathcal{F}"]
    && \category{E}
    \\
    & \category{D}
    \arrow[ur, swap, "f_{!}\mathcal{F}"]
    \arrow[from=M, Rightarrow, "\eta"]
  \end{tikzcd}
\end{equation*}
and $d_{3}\tau$ is strictly commutative. Then there is a unique filling to a full 3-simplex.

We would like to show that, assuming our Kan extension is pointwise, this remains true if one drops the condition that $d_{3}\tau$ be strictly commutative; in fact, it needn't even be thin!

We consider such a horn:
\begin{equation*}
  \begin{tikzcd}
    && \category{D}
    \arrow[drrr, "f_{!}\mathcal{F}"]
    \arrow[dddrr, near start, "g"]
    \\
    &&&&& \category{E}
    \\
    \category{C}
    \arrow[uurr, "f"]
    \arrow[urrrrr, crossing over, near start, "\mathcal{F}", ""{name=CE}]
    \arrow[drrrr, swap, "h"]
    \\
    &&&& \category{D}'
    \arrow[uur, swap, "\mathcal{G}"]
  \end{tikzcd}
\end{equation*}
This is a bad drawing. Anyway\dots the back left face is supposed to be thin, and the top-front face is supposed to be a (pointwise) left Kan extension. We need to produce a natural transformation $f_{!}\mathcal{F} \Rightarrow \mathcal{G} \circ g$ making the pasting diagram given by the two different compositions along the outside of the 3-simplex agree.

In order to show this, we first need to produce a natural transformation $\chi\colon f_{!}\mathcal{F} \Rightarrow \mathcal{G} \circ g$. First, we produce the components
\begin{equation*}
  \chi_{d}\colon \colim\left[ \category{C}_{/d} \to \category{C} \overset{\mathcal{F}}{\to} \category{E} \right] \to \mathcal{G}(g(d)).
\end{equation*}
This should be induced by a cocone under our diagram indexed by objects $(c, \alpha\colon f(c) \to d)$. There is an obvious choice: we take the cocone with legs
\begin{equation*}
  \mathcal{F}(c) \to \mathcal{G}(h(c)) \to \mathcal{G}(g(f(c))) \to \mathcal{G}(g(d)).
\end{equation*}
It follows easily that these legs really do form a cocone, giving us our morphisms $\chi_{d}$.

Next we have to show that these components $\chi_{d}$ really do form a natural transformation. This is the same as the ordinary proof.


Okay, so it's definitely true for 1-categories. What about


\end{document}
