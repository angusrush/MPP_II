\documentclass[main.tex]{subfiles}

\begin{document}

\section{Background}
\label{sec:background}

\subsection{Scaled simplicial sets}
\label{ssc:scal}

Note: This is basically a summary of the parts of The Goodwillie Calculus that we're gonna need.

\begin{definition}
  \label{def:scaled_simplicial_set}
  A \defn{scaled simplicial set} is a pair $(X, A_{X}) = \overline{X}$, where $X$ is a simplicial set, and $A \subseteq X_{2}$ is a set of $2$-simplices of $X$ which contains every degenerate simplex. A simplex belonging to $A$ is known as \emph{thin.} A morphism of scaled simplicial sets is a morphism of the underlying simplicial sets which takes thin simplices to thin simplices.
\end{definition}

We will need the following facts about scaled simplicial sets.

\begin{enumerate}
  \item For any category $\category{C}$ enriched in marked simplicial sets, we can build a scaled simplicial set $\Nsc(\category{C}) = (\N(\category{C}), T)$, the \defn{scaled nerve} of $\category{C}$, whose underlying simplicial set $\N(\category{C})$ is the simplicial nerve of $\category{C}$. The scaling $T$ is defined in the following way. Let $\sigma\colon \Delta^{2} \to \N(\category{C})$ be any 2-simplex.
    This corresponds to a diagram
    \begin{equation*}
      \begin{tikzcd}
        & Y
        \arrow[dr, "g"]
        \\
        X
        \arrow[ur, "f"]
        \arrow[rr, "h"]
        && Z
      \end{tikzcd}
    \end{equation*}
    in $\category{C}$, together with a 1-simplex $\alpha_{\sigma}\colon \Delta^{1} \to \Map_{\category{C}}(X, Z)$ corresponding to a map $gf \to h$. We will then define $\sigma$ to be thin if and only if $\alpha_{\sigma}$ is marked in $\Map_{\category{C}}(X, Z)$.

  \item The functor $\Nsc$ admits a left adjoint $\Csc\colon \SSetsc \to \SCatmk$, the \defn{scaled rigidification}, such that for any scaled simplicial set $\overline{X}$ with underlying simplicial set $X$, the underlying simplicially enriched category of $\Csc(\overline{X})$ is simply $\C(X)$, the rigidification of $X$. The scaling on $\Csc(X)$ also admits a relatively simple description.

  \item A map of scaled simplicial sets $\overline{X} \to \overline{Y}$ is said to be a \defn{bicategorical equivalence} if the map $F\colon \Csc(X) \to \Csc(Y)$ is a weak equivalence of $\SSet$-enriched categories; that is, if for all objects $x$, $x'$ of $\Csc(\overline{X})$, the map
    \begin{equation*}
      \Hom_{\Csc(\overline{X})}(x, x') \to \Hom_{\Csc(\overline{Y})}(F(x), F(x'))
    \end{equation*}
    is a weak equivalence of marked simplicial sets, and if for all $y \in \Csc(\overline{Y})$, there exists some $x \in \Csc(\overline{X})$ such that $F(x)$ and $y$ are isomorphic in the homotopy category $\h \Csc(\overline{Y})$.

  \item There is a left proper, combinatorial model structure on $\SSetsc$, whose cofibrations are monomorphisms, and whose weak equivalences are bicategorical fibrations. This gives a Quillen equvialence
    \begin{equation*}
      \Csc : \SSetsc \leftrightarrow \SCatmk : \Nsc.
    \end{equation*}

  \item An \defn{$\infty$-bicategory} is a fibrant object with respect to this model structure. In particular, for any fibrant object $\category{C} \in \SCatmk$ (i.e.\ any quasicategory-enriched category with equvialences marked), $\Nsc(\category{C})$ is an $\infty$-bicategory.

  \item The category of scaled simplicial sets has a natural simplicial enrichment as follows: For any two scaled simplicial sets $\overline{X}$ and $\overline{Y}$, we define
    \begin{equation*}
      \Mapsc(\overline{X}, \overline{Y})_{n} = \Hom(\overline{X} \times \Delta^{n}_{\sharp}, \overline{Y}),
    \end{equation*}
    where $\Delta^{n}_{\sharp}$ is the scaled simplicial set $(\Delta^{n}, \Delta^{n}_{2})$. For $\overline{Y}$ an $\infty$-bicategory, $\Mapsc(\overline{X}, \overline{Y})$ is a quasicategory. Taking the core gives a Kan complex.

  \item With this,

  \item We will denote the simplicially enriched category whose objects are $\infty$-bicategories, with mapping spaces as above by $\catname{Bicat}_{\infty}$. We then define $\Cat_{(\infty, 2)} = \N(\catname{Bicat}_{\infty})$. This is a quasicategory.
\end{enumerate}

We will need multiple notions of the quasicategory of $(\infty,2)$-categories.
\begin{itemize}
  \item The model category $\SSetsc$ of scaled simplicial sets, whose cofibrations are precisely the monomorphisms.

  \item The weak equivalences are bicategorical equivalences,
\end{itemize}

\end{document}
