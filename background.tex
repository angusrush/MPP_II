\documentclass[main.tex]{subfiles}
\begin{document}

\section{Introduction and conventions}
\label{sec:introduction_and_conventions}

\subsection{General conventions}
\label{ssc:general_conventions}

By $\S$, we mean the $\infty$-category of \emph{small} spaces.

\subsection{The marked-scaled model structure}
\label{ssc:marked-scaled_model_structure}

One common model for $(\infty,2)$-categories is Lurie's theory of $\infty$-bicategories

In \cite{garcia20212}, a marked-biscaled model structure modelling cartesian fibrations of $\infty$-bicategories is defined. When taken over a point, the two scalings collapse into one, giving the following.

\begin{definition}
  \label{def:marked-scaled_simplicial_set}
  A marked-scaled simplicial set is a triple $(X, E_{X}, T_{X})$, where $E_{X} \subseteq X_{1}$ is a collection of edges of $X$ containing all degenerate edges, and $T_{X} \subseteq X_{2}$ is a collection of triangles of $X$ containing all degenerate triangles.
\end{definition}

\begin{notation}
  \leavevmode
  \begin{itemize}
    \item We will denote the category of marked-scaled simplicial sets by $\SSetms$.

    \item To save on notation, we will sometimes denote the marked-scaled simplicial set $(X, E_{X}, T_{X})$ by $X^{E_{X}}_{T_{X}}$, particularly in the case that $E_{X}$ or $T_{X}$ are $\sharp$ or $\flat$, the maximum (resp. minimum) markings and scalings. For example, the bimarked simplicial set $X^{\sharp}_{\flat}$ has all 1-simplices marked and only degenerate 2-simplices scaled, etc.
  \end{itemize}
\end{notation}

We will use the following abuse of notation freely.

\begin{notation}
  Let $(\Delta^{n}, E, T)$ be a marked-scaled $n$-simplex. For any simplicial subset $S \subseteq \Delta^{n}$, we will denote by $(S, E, T)$ the simplicial subset $S$ together with the inherited marking and scaling.
\end{notation}

\begin{notation}
  Given a cosimplicial object $Q\colon \Delta \to \category{C}$, we will write $Q(n)$ instead of $Q([n])$.
\end{notation}

\begin{definition}
  \label{def:ms-anodyne_morphisms}
  The set of ms-anodyne morphisms is a saturated set of morphisms between marked-scaled simplicial sets containing the following classes of morphisms:
  \begin{enumerate}[label=(A\arabic*)]
    \item\label{item:innerms} Inner horn inclusions
      \begin{equation*}
        (\Lambda^{n}_{i}, \flat, \{\Delta^{\{i-1,i,i+1\}}\}) \to (\Delta^{n}, \flat, \{\Delta^{\{i-1,i,i+1\}}\}),
      \end{equation*}
      for $n \geq 2$ and $0 < i < n$.

    \item\label{item:outerms} Outer horn inclusions
      \begin{equation*}
        (\Lambda^{n}_{n}, \{\Delta^{\{n-1,n\}}\}, \{\Delta^{\{0, n-1, n\}}\}),
      \end{equation*}
      for $n \geq 1$.
  \end{enumerate}
\end{definition}

\begin{proposition}
  \label{prop:sharp_marked_right_anodyne}
  For any right anodyne morphism $A \hookrightarrow B$, the morphism $A^{\sharp}_{\sharp} \hookrightarrow B^{\sharp}_{\sharp}$ is marked-scaled anodyne.
\end{proposition}

\begin{theorem}
  There is a model structure on the category $\SSetms$, whose trivial cofibrations are given by the ms-anodyne maps, and whose fibrant objects are $\infty$-bicategories with precisely the equivalences marked.
\end{theorem}

\end{document}
