\documentclass[main.tex]{subfiles}
\begin{document}

\section{Background}
\label{sec:background}

\subsection{Scaled simplicial sets}
\label{ssc:scal}

Note: This is basically a summary of the parts of The Goodwillie Calculus that we're gonna need.

\begin{definition}
  \label{def:scaled_simplicial_set}
  A \defn{scaled simplicial set} is a pair $(X, A_{X}) = \overline{X}$, where $X$ is a simplicial set, and $A \subseteq X_{2}$ is a set of $2$-simplices of $X$ which contains every degenerate simplex. A simplex belonging to $A$ is known as \emph{thin.} A morphism of scaled simplicial sets is a morphism of the underlying simplicial sets which takes thin simplices to thin simplices.
\end{definition}

We will need the following facts about scaled simplicial sets.

\begin{enumerate}
  \item For any category $\category{C}$ enriched in marked simplicial sets, we can build a scaled simplicial set $\Nsc(\category{C}) = (\N(\category{C}), T)$, the \defn{scaled nerve} of $\category{C}$, whose underlying simplicial set $\N(\category{C})$ is the simplicial nerve of $\category{C}$. The scaling $T$ is defined in the following way. Let $\sigma\colon \Delta^{2} \to \N(\category{C})$ be any 2-simplex.
    This corresponds to a diagram
    \begin{equation*}
      \begin{tikzcd}
        & Y
        \arrow[dr, "g"]
        \\
        X
        \arrow[ur, "f"]
        \arrow[rr, "h"]
        && Z
      \end{tikzcd}
    \end{equation*}
    in $\category{C}$, together with a 1-simplex $\alpha_{\sigma}\colon \Delta^{1} \to \Map_{\category{C}}(X, Z)$ corresponding to a map $gf \to h$. We will then define $\sigma$ to be thin if and only if $\alpha_{\sigma}$ is marked in $\Map_{\category{C}}(X, Z)$.

  \item The functor $\Nsc$ admits a left adjoint $\Csc\colon \SSetsc \to \SCatmk$, the \defn{scaled rigidification}, such that for any scaled simplicial set $\overline{X}$ with underlying simplicial set $X$, the underlying simplicially enriched category of $\Csc(\overline{X})$ is simply $\C(X)$, the rigidification of $X$. The scaling on $\Csc(X)$ also admits a relatively simple description.

  \item A map of scaled simplicial sets $\overline{X} \to \overline{Y}$ is said to be a \defn{bicategorical equivalence} if the map $F\colon \Csc(X) \to \Csc(Y)$ is a weak equivalence of $\SSet$-enriched categories; that is, if for all objects $x$, $x'$ of $\Csc(\overline{X})$, the map
    \begin{equation*}
      \Hom_{\Csc(\overline{X})}(x, x') \to \Hom_{\Csc(\overline{Y})}(F(x), F(x'))
    \end{equation*}
    is a weak equivalence of marked simplicial sets, and if for all $y \in \Csc(\overline{Y})$, there exists some $x \in \Csc(\overline{X})$ such that $F(x)$ and $y$ are isomorphic in the homotopy category $\h \Csc(\overline{Y})$.

  \item There is a left proper, combinatorial model structure on $\SSetsc$, whose cofibrations are monomorphisms, and whose weak equivalences are bicategorical fibrations. This gives a Quillen equvialence
    \begin{equation*}
      \Csc : \SSetsc \leftrightarrow \SCatmk : \Nsc.
    \end{equation*}

  \item An \defn{$\infty$-bicategory} is a fibrant object with respect to this model structure. In particular, for any fibrant object $\category{C} \in \SCatmk$ (i.e.\ any quasicategory-enriched category with equvialences marked), $\Nsc(\category{C})$ is an $\infty$-bicategory.

  \item The category of scaled simplicial sets has a natural simplicial enrichment as follows: For any two scaled simplicial sets $\overline{X}$ and $\overline{Y}$, we define
    \begin{equation*}
      \Mapsc(\overline{X}, \overline{Y})_{n} = \Hom(\overline{X} \times \Delta^{n}_{\sharp}, \overline{Y}),
    \end{equation*}
    where $\Delta^{n}_{\sharp}$ is the scaled simplicial set $(\Delta^{n}, \Delta^{n}_{2})$. For $\overline{Y}$ an $\infty$-bicategory, $\Mapsc(\overline{X}, \overline{Y})$ is a quasicategory. Taking the core gives a Kan complex.

  \item With this,

  \item We will denote the simplicially enriched category whose objects are $\infty$-bicategories, with mapping spaces as above by $\catname{Bicat}_{\infty}$. We then define $\Cat_{(\infty, 2)} = \N(\catname{Bicat}_{\infty})$. This is a quasicategory.
\end{enumerate}

We will need multiple notions of the quasicategory of $(\infty,2)$-categories.
\begin{itemize}
  \item The model category $\SSetsc$ of scaled simplicial sets, whose cofibrations are precisely the monomorphisms.

  \item The weak equivalences are bicategorical equivalences,
\end{itemize}

\subsection{Enhanced twisted arrow categories}
\label{ssc:enhanced_twisted_arrow_categories}

This is a summary of the parts of the theory that we'll need.
\begin{enumerate}
  \item For each $n \in \N$, we consider the simplicial set $\Delta^{n} \star (\Delta^{n})\op$. We will denote the vertices of $\Delta^{n}$ by $i$ and the vertices of $(\Delta^{n})\op$ by $\bar{i}$. This is really just a fancy notation for the $(2n+1)$-simplex with vertices
    \begin{equation*}
      1, \ldots, n, \bar{n}, \ldots, \bar{1}.
    \end{equation*}

  \item We define a cosimplicial object
    \begin{equation*}
      Q\colon \Delta \to \SSetsc;\qquad [n] \mapsto (\Delta^{n} \star (\Delta^{n})\op, T_{n}),
    \end{equation*}
    where $T_{n}$ is the set of $2$-simplices $\sigma\colon \Delta^{2} \to \Delta^{n} \star (\Delta^{n})\op$ which
    \begin{itemize}
      \item factor through $\Delta^{n}$; or
      \item factor through $(\Delta^{n})\op$; or
      \item are of the form $\Delta^{\{i,j,\bar{k}\}}$, where $i < j \leq k$; or
      \item are of the form $\Delta^{\{k, \bar{j}, \bar{i}\}}$, where $i < j \leq k$.
    \end{itemize}

  \item For any scaled simplicial set $\overline{X}$, we define a simplicial set
    \begin{equation*}
      \Tw(\overline{X})_{n} = \Hom_{\SSetsc}(Q([n]), \overline{X}).
    \end{equation*}
    We further define a marking on $\Tw(\overline{X})$ by saying that a 1-simplex in $\Tw(\overline{X})$ is marked if and only if it corresponds to a map $\Delta^{3}_{\sharp} \to \overline{X}$.

  \item This construction clearly extends to an ordinary functor $\Tw\colon \SSetsc \to \SSetmk$. One checks that this functor preserves weak equivalences, and thus extends to a functor of quasicategories or $\Kan$-enriched categories or whatever your favorite model is
    \begin{equation*}
      \Tw\colon \ITCat \to \ICat.
    \end{equation*}

  \item For any scaled simplicial set $\overline{X}$, a zero-simplex in $\Tw(\overline{X})$ is a 1-simplex in $\overline{X}$, and a 1-simplex in $\Tw(\overline{X})$ is a 3-simplex in $\overline{X}$ of the form
    \begin{equation*}
      \begin{tikzcd}[column sep=large, row sep=large]
        x
        \arrow[r]
        \arrow[d, ""{name=L, right}]
        \arrow[dr]
        & x'
        \arrow[d, ""{name=R, left}]
        \arrow[from=l, to=R, phantom, near end, "\circlearrowleft"{description}]
        \\
        y
        & y'
        \arrow[l]
        \arrow[from=L, Rightarrow, shorten=3ex]
      \end{tikzcd}
      \qquad
      \begin{tikzcd}[column sep=large, row sep=large]
        x
        \arrow[r]
        \arrow[d, ""{name=L, right}]
        & x'
        \arrow[d, ""{name=R, left}]
        \arrow[dl]
        \\
        y
        \arrow[from=R, phantom, near start, "\circlearrowleft"{description}]
        & y'
        \arrow[l]
        \arrow[from=L, to=u, shorten=3ex, Rightarrow]
      \end{tikzcd},
    \end{equation*}
    where the simplices marked with $\circlearrowleft$ are thin. The $\Rightarrow$-marked simplices are simply to remind us that given a $\SSetmk$-enriched category $\category{C}$, two-simplices in $\Csc(\category{C})$ correspond to diagrams in $\category{C}$ of the form
    \begin{equation*}
      \begin{tikzcd}
        & c'
        \arrow[dr, "g"]
        \\
        c
        \arrow[ur, "f"]
        \arrow[rr, "h"{below}, ""{name=B, above}]
        && c''
        \arrow[from=B, to=ul, Rightarrow, "\eta"]
      \end{tikzcd},
    \end{equation*}
    where $\eta$ is a 2-morphism $h \Rightarrow g \circ f$.
\end{enumerate}

\begin{definition}
  We will call a 1-simplex in $\Delta^{1} \to \Tw(\overline{X})$ \defn{svelte} if it corresponds to a map $(\Delta^{1} \star {\Delta^{1}}\op)_{\sharp} \to \overline{X}$.
\end{definition}

We would like to understand (relative) pullbacks in $p\colon \Tw(\overline{X}) \to \category{C} \times \category{C}\op$. More specifically, we would like to have the following.

\begin{proposition}
  Let $\sigma\colon \Lambda^{2}_{2} \to \Tw(\CC)$ corresponding to a diagram
  \begin{equation*}
    \begin{tikzcd}
      & C'
      \arrow[d, "F"]
      \\
      C''
      \arrow[r]
      & C
    \end{tikzcd}
  \end{equation*}
  in $\Tw(\CC)$, where $F$ is svelte. Then a filling
  \begin{equation*}
    \overline{\sigma} = \quad
    \begin{tikzcd}
      \tilde{C}
      \arrow[r]
      \arrow[d, swap, "F'"]
      & C'
      \arrow[d, "F"]
      \\
      C''
      \arrow[r]
      & C
    \end{tikzcd}
  \end{equation*}
  to a square is a pullback if and only if the following conditions are satisfied:
  \begin{itemize}
    \item The image of $\overline{\sigma}$ in $\category{C} \times \category{C}\op$ is pullback (that is, the top square of $\overline{\sigma}$ is pullback, and the bottom square is pushout)

    \item The morphism $F'$ is svelte.
  \end{itemize}
\end{proposition}
\begin{proof}
  We need to show that we can lift
  \begin{equation*}
    \begin{tikzcd}
      \partial \Delta^{n}
      \arrow[r]
      \arrow[d]
      & \Tw(\CC)_{/ \overline{\sigma}}
      \arrow[d]
      \\
      \Delta^{n}
      \arrow[r]
      \arrow[ur, dashed]
      & \Tw(\CC)_{/ \sigma} \times_{(\category{C} \times \category{C}\op)_{/\sigma}}(\category{C} \times \category{C}\op)_{/\overline{\sigma}}
    \end{tikzcd},
  \end{equation*}
  or equivalently
  \begin{equation*}
    \begin{tikzcd}
      \partial \Delta^{n} \star (\Lambda^{2}_{0})^{\triangleleft} \displaystyle\coprod_{\partial \Delta^{n} \star \Lambda^{2}_{0}} \Delta^{n} \star (\Lambda^{2}_{0})^{\triangleleft}
      \arrow[r]
      \arrow[d]
      & \Tw(\CC)
      \arrow[d]
      \\
      \Delta^{n} \star (\Lambda^{2}_{0})^{\triangleleft}
      \arrow[r]
      \arrow[ur, dashed]
      & \category{C} \times \category{C}\op
    \end{tikzcd}.
  \end{equation*}

  We can write $(\Lambda^{2}_{0}) \cong \Delta^{\{012\}} \amalg_{\Delta^{\{02\}}} \Delta^{\{01'2\}}$.
\end{proof}

%Unfortunately, this construction is a bit too unwieldy. Thankfully we are working with an $\infty$-category, $\ICat$, which has a very natural model as a $\SSetmk$ enriched category. We can therefore achieve what we want by modifying the cosimplicial object $Q$.
%
%\subsection{Enhanced twisted arrow categories for simplicially-enriched categories}
%\label{ssc:enhanced_twisted_arrow_categories_for_simplicially_enriched_categories}
%
%Let's unwrap the above construction in the particular case that $\overline{X} = \ICat$. The $n$-simplices of $\Tw(\ICat)$ are then
%\begin{align*}
%  \Tw(\ICat)_{n} &= \Hom_{\SSet}(\Delta^{n}, \Tw(\ICat)) \\
%  &= \Hom_{\SSetmk}(Q(\Delta^{n}), \ICat) \\
%  &= \Hom_{\SCatmk}(\Csc Q(\Delta^{n}), \QCat),
%\end{align*}
%where $\QCat$ is the $\SSetmk$-enriched category of quasicategories. We should understand the cosimplicial object $\Csc Q$.
%\begin{itemize}
%  \item For any $[n] \in \Delta$, the object $\Csc Q ([n])$ in $\SCatmk$ has the following description.
%
%  \item The objects are the elements of $[n] \star [n]\op \cong [2n+1]$. We will conventionally write the elements of $[n]$ by $i$, $j$, etc., and the elements of $[n]\op$ by $\bar{i}$, $\bar{j}$, etc. We will denote a generic element of $[n] \star [n]\op$ by $a$, $b$, etc.
%
%  \item For any $a$, $b \in [n] \star [n]\op$, the simplicial set underlying the marked simplicial set $\Hom_{\Csc Q [n]}(a, b)$ is the poset of subsets of $\{a, \ldots, b\}$ containing $a$ and $b$.
%
%  \item For subsets
%    \begin{equation*}
%      S' \subseteq S \subseteq [n] \star [n]\op,
%    \end{equation*}
%    both starting at $a \in [n] \star [n]\op$ and ending at $b$, the corresponding map in $\Hom_{\Csc Q[n]}(a, b)$ is marked if and only if
%
%\end{itemize}

\section{A new Barwick's theorem}
\label{sec:a_new_barwick_s_theorem}

It turns out Barwick's construction can be obviously modified. Here is a form which is more useful to us. Little to no effort has been made to make this as general as possible, as long as it works. It is relatively clear from the proof that there are many sets of conditions that suffice, and probably no unique weakest one.

\begin{theorem}
  Let $p\colon \triple{C} \to \triple{D}$ be a functor between adequate triples which is an inner fibration, satisfying the following conditions.
  \begin{enumerate}
    \item Every egressive morphism in $\category{D}$ admits a lift which is both egressive and $p$-cartesian.

    \item Consider a square
      \begin{equation*}
          \sigma = \quad
          \begin{tikzcd}
            y'
            \arrow[r, "f'"]
            \arrow[d, two heads, swap, "g'"]
            & x'
            \arrow[d, two heads, "g"]
            \\
            y
            \arrow[r, rightarrowtail, "f"]
            & x
          \end{tikzcd}
      \end{equation*}
      in $\category{C}$ in which $f$ is $p$-cocartesian and $g$ and $g'$ are $p$-cartesian, lying over an ambigressive pullback square
      \begin{equation*}
        p(\sigma) = \quad
        \begin{tikzcd}
          py'
          \arrow[r, rightarrowtail]
          \arrow[d, two heads]
          & px'
          \arrow[d, two heads]
          \\
          py
          \arrow[r, rightarrowtail]
          & px
        \end{tikzcd}
      \end{equation*}
      in $\category{D}$. Then $\sigma$ is pullback, and $f'$ is ingressive, $p$-cocartesian, and and $p\downdag$-cocartesian.
  \end{enumerate}
  Then spans of the form
  \begin{equation*}
    \begin{tikzcd}
      & z
      \arrow[dl, two heads, swap, "g"]
      \arrow[dr, rightarrowtail, "f"]
      \\
      x
      && y
    \end{tikzcd}
  \end{equation*}
  are cocartesian, where $g$ is $p\updag$ is cocartesian and $f$ is $p$-cartesian.
\end{theorem}
\begin{proof}
  Even simpler than before. The square that we have to show is pullback factors as before, but this time we don't have to take the final pullback; the square is automatically pullback.
\end{proof}



\end{document}
