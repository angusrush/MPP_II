\documentclass[main.tex]{subfiles}

\begin{document}

\section{Horn filling via left Kan extensions}
\label{sec:horn_filling_via_left_kan_extensions}

Kan extensions of $\infty$-categories are usually defined pointwise via the colimit formula \cite{highertopostheory} \cite{cisinski2019higher}. In this section, we will show that such pointwise $\infty$-Kan extensions enjoy a horn filling property in $\ICCat$ which generalizes the universal property for Kan extensions of functors between $1$-categories.

We will explain the precise meaning of the below definition in \hyperref[ssc:basic_facts_about_kan_extensions]{Subsection~\ref*{ssc:basic_facts_about_kan_extensions}}.

\begin{definition}
  \label{def:left_kan}
  We will say that a $2$-simplex $\tau\colon \Delta^{2}_{\flat} \to \ICCat$,
  \begin{equation}
    \label{eq:prototypical_left_kan_simplex}
    \tau =
    \begin{tikzcd}
      X
      \arrow[dr, swap, "f"]
      \arrow[rr, "F", ""{below, name=M}]
      && \category{C}
      \\
      & Y
      \arrow[ur, swap, "G"]
      \arrow[from=M, Rightarrow, swap, "\eta"]
    \end{tikzcd},
  \end{equation}
  is \defn{left Kan} if the natural transformation $\eta$ exhibits $G$ as the left Kan extension of $F$ along $f$.
\end{definition}

The goal of this section is to prove the following.

\begin{theorem}
  \label{thm:left_kan_implies_globally_left_kan}
  Let $\tau\colon \Delta^{2}_{\flat} \to \ICCat$ be a left Kan 2-simplex. Then for each $n \geq 2$, every solid lifting problem of the form
  \begin{equation}
    \label{eq:lifting_problems_for_global_kan_extensions}
    \begin{tikzcd}
      \Delta^{\{0,1,n+1\}}_{\flat}
      \arrow[d, hook]
      \arrow[dr, "\tau"]
      \\
      (\Lambda^{n+1}_{0})_{\flat}
      \arrow[r]
      \arrow[d, hook]
      & \ICCat
      \\
      \Delta^{n+1}_{\flat}
      \arrow[ur, dashed]
    \end{tikzcd}
  \end{equation}
  admits a dashed solution.
\end{theorem}

\begin{note}
  One can even show that if the terminal vertex in $\tau$ is mapped to an $\infty$-category $\category{C}$ with sufficient colimits (i.e.\ all colimits in $\category{C}$ indexed by the homotopy fibers of $f$ should exist), then $\tau$ is left kan \emph{if and only if} it enjoys these lifting properties. That is, these lifting properties generalize the universal property for Kan extensions of ordinary categories. However, we will not make use of this result.
\end{note}

%In this section, we show that a simplex $\Delta^{2}_{\flat} \to \ICCat$ which is left Kan is also globally left Kan. This will amount to showing that a simplex $\tau\colon \Delta^{2}_{\flat} \to \ICCat$ of the form
%\begin{equation*}
%  \begin{tikzcd}
%    X
%    \arrow[dr, swap, "f"]
%    \arrow[rr, "F", ""{below, name=M}]
%    && \category{C}
%    \\
%    & Y
%    \arrow[ur, swap, "f_{!}F"]
%    \arrow[from=M, Rightarrow, swap, "\eta"]
%  \end{tikzcd},
%\end{equation*}
%where $\category{C}$ is cocomplete, $X$ is small, $f_{!}F$ is a left Kan extension of $F$ along $f$, and $\eta\colon F \Rightarrow f^{*}f_{!}F$ is the unit map, has the property that for any $n \geq 2$, any solid filling problem
%\begin{equation}
%  \label{eq:lifting_problems_for_global_kan_extensions}
%  \begin{tikzcd}
%    \Delta^{\{0,1,n+1\}}_{\flat}
%    \arrow[d, hook]
%    \arrow[dr, "\tau"]
%    \\
%    (\Lambda^{n+1}_{0})_{\flat}
%    \arrow[r]
%    \arrow[d, hook]
%    & \ICCat
%    \\
%    \Delta^{n+1}_{\flat}
%    \arrow[ur, dashed]
%  \end{tikzcd}
%\end{equation}
%has a dashed solution.

Let us examine the lifting problems of \hyperref[eq:lifting_problems_for_global_kan_extensions]{Equation~\ref*{eq:lifting_problems_for_global_kan_extensions}} in the case $n = 2$. In this case we have the data of categories and functors
\begin{equation*}
  \begin{tikzcd}
    & Y
    \arrow[drr, "f_{!}F"]
    \\
    X
    \arrow[dr, swap, "h"]
    \arrow[ur, "f"]
    \arrow[rrr, near end, "F"]
    &&& \category{C}
    \\
    & Z
    \arrow[urr, swap, "G"]
    \arrow[from=uu, crossing over, near start, "g"]
  \end{tikzcd}
\end{equation*}
together with natural transformations $\eta\colon F \Rightarrow f_{!}F \circ f$, $\beta\colon h \Rightarrow g \circ f$, and $\delta\colon F \Rightarrow G \circ h$, making up the sides of $\Lambda^{3}_{0}$. We need to produce a natural transformation $\alpha\colon f_{!}F \Rightarrow G \circ g$ and a filling of the full 3-simplex, which is the data of a homomotopy-commutative diagram
\begin{equation*}
  \begin{tikzcd}
    F
    \arrow[r, "\eta"]
    \arrow[d, swap, "\delta"]
    & f_{!}F \circ f
    \arrow[d, "\alpha f"]
    \\
    G \circ h
    \arrow[r, "G \beta"]
    & G \circ g \circ f
  \end{tikzcd}
\end{equation*}
in $\Fun(X, \category{C})$.\footnote{In particular, if we take $h = f$, $g = \id_{Y}$, and $\beta$ to be the identity $f \Rightarrow f$, we recover the classical universal property satisfied by left Kan extension.}

We can rephrase this as follows. We are given the data
\begin{equation*}
  \overbrace{
  \begin{tikzcd}[ampersand replacement=\&]
    F
    \arrow[r, "\eta"]
    \arrow[d, swap, "\delta"]
    \& f_{!}F \circ f
    \\
    G \circ h
    \arrow[r, "G \beta"]
    \& G \circ g \circ f
  \end{tikzcd}}^{\text{in }\Fun(X, \category{C})}
  \qquad\qquad
  \overbrace{
  \begin{tikzcd}
    f_{!}F
    \\
    G \circ g
  \end{tikzcd}}^{\text{ in }\Fun(Y, \category{C})}
\end{equation*}
where the left-hand diagram is a map $LC^{2} \to \Fun(X, C)$, where $LC^{2}$ is the simplicial subset of the boundary of $\Delta^{1} \times \Delta^{1}$ except the right-hand face, and the right-hand diagram is a map $\partial \Delta^{1} \to \Fun(Y, C)$. We need to construct a filler $\partial \Delta^{1} \hookrightarrow \Delta^{1}$ on the right, and from it a filler $LC^{2} \hookrightarrow \Delta^{1} \times \Delta^{1}$ on the left. We do this in the following sequence of steps.

\begin{enumerate}
  \item We first can fill the lower-left half of the diagram on the left simply by taking the composition $G\beta \circ \delta$. Doing this, we are left with the filling problem
    \begin{equation*}
      \overbrace{
      \begin{tikzcd}[ampersand replacement=\&]
        F
        \arrow[r, "\eta"]
        \arrow[dr]
        \& f_{!}F \circ f
        \\
        \& G \circ g \circ f
      \end{tikzcd}}^{\text{in }\Fun(X, \category{C})}
      \qquad\qquad
      \overbrace{
      \begin{tikzcd}
        f_{!}F
        \\
        G \circ g
      \end{tikzcd}}^{\text{ in }\Fun(Y, \category{C})}.
    \end{equation*}

  \item Using the adjunction $f_{!} \dashv f^{*}$, we can extend the diagram on the right to a diagram which is adjunct to the diagram on the left (in the sense to be described in \hyperref[ssc:adjunct_data]{Subsection~\ref*{ssc:adjunct_data}}):
    \begin{equation*}
      \overbrace{
      \begin{tikzcd}[ampersand replacement=\&]
        F
        \arrow[r, "\eta"]
        \arrow[dr]
        \& f_{!}F \circ f
        \\
        \& G \circ g \circ f
      \end{tikzcd}}^{\text{in }\Fun(X, \category{C})}
      \qquad\qquad
      \overbrace{
        \begin{tikzcd}[ampersand replacement=\&]
          f_{!}F
          \arrow[r, "\id"]
          \arrow[dr]
        \& f_{!}F
        \\
        \& G \circ g
      \end{tikzcd}}^{\text{ in }\Fun(Y, \category{C})}
    \end{equation*}

  \item The diagram on the right has an obvious filler, which is adjunct to a filler on the left.
    \begin{equation*}
      \overbrace{
      \begin{tikzcd}[ampersand replacement=\&]
        F
        \arrow[r, "\eta"]
        \arrow[dr]
        \& f_{!}F \circ f
        \arrow[d, "\alpha f"]
        \\
        \& G \circ g \circ f
      \end{tikzcd}}^{\text{in }\Fun(X, \category{C})}
      \qquad\qquad
      \overbrace{
        \begin{tikzcd}[ampersand replacement=\&]
          f_{!}F
          \arrow[r, "\id"]
          \arrow[dr]
        \& f_{!}F
        \arrow[d, "\alpha"]
        \\
        \& G \circ g
      \end{tikzcd}}^{\text{ in }\Fun(Y, \category{C})}
    \end{equation*}
\end{enumerate}

The lifting problems which we have to solve in \hyperref[eq:lifting_problems_for_global_kan_extensions]{Equation~\ref*{eq:lifting_problems_for_global_kan_extensions}} for $n > 2$ amount to replacing $\Delta^{1} \times \Delta^{1}$ by $(\Delta^{1})^{n}$, etc; the basic process remains unchanged, but the combinatorics involved in filling the necessary cubes becomes more involved. In \hyperref[ssc:filling_cubes_relative_to_their_boundary]{Subsection~\ref*{ssc:filling_cubes_relative_to_their_boundary}} we explain the combinatorics of filling cubes relative to their boundaries. In \hyperref[ssc:adjunct_data]{Subsection~\ref*{ssc:adjunct_data}}, we give a formalization the concept of adjunct data, and provide a means of for filling partial data to total adjunct data. In \hyperref[ssc:left_kan_implies_globally_left_kan]{Subsection~\ref*{ssc:left_kan_implies_globally_left_kan}}, we show that we can always solve the lifting problems of \hyperref[eq:lifting_problems_for_global_kan_extensions]{Equation~\ref*{eq:lifting_problems_for_global_kan_extensions}}.

\subsection{Basic facts about Kan extensions}
\label{ssc:basic_facts_about_kan_extensions}

In this section, we recall a few basic facts about Kan extensions. These are mostly results found in \cite{kerodon} which we will need. Since Kan extensions are defined by the colimit formula, we will start by defining colimits. Clasically, colimits in $\infty$-categories are defined using colimit cones.

\begin{definition}
  \label{def:colimit_via_cocones}
  Let $F\colon K \to \category{C}$ be a diagram, where $\category{C}$ is an $\infty$-category. A cocone $\tilde{F}\colon K^{\triangleright} \to \category{C}$ is a \defn{colimit cone} if it is an initial object in the $\infty$-category $\category{C}_{F/}$.
\end{definition}

It is sometimes more convenient to define colimits via natural transformations to a constant functor.

\begin{definition}
  Let $f\colon K \to \category{C}$ be a map between simplicial sets, where $\category{C}$ is an $\infty$-category. Let $c \in \category{C}$ be an object, and denote by $\underline{c}$ the constant functor $K \to \category{C}$ with value $c$. A natural transformation $f \Rightarrow \underline{c}$ \defn{exhibits $c$ as the colimit of $f$} if it is an initial object in the $\infty$-category $\category{C}^{F/}$.
\end{definition}

Fortunately, these definitions are compatible: an object $K^{\triangleright} \to \category{C}$ in $\category{C}_{/F}$ is a colimit cone with cone tip $c$ if and only if the composite map
\begin{equation}
  \label{eq:natural_transformation_from_colimit_cone}
  K \times \Delta^{1} \to K \times \Delta^{1} \amalg_{K \times \{1\}} \Delta^{0} \to K \star \Delta^{0} \to \category{C}
\end{equation}
exhibits $c$ as a colimit of $F$. This follows from the fact that for any $\infty$-category $\category{C}$ and any functor $F\colon K \to \category{C}$, the comparison map $\category{C}_{/F} \to \category{C}^{/F}$ of \cite[Prop.~4.2.1.5]{highertopostheory} is a categorical equivalence.

\begin{note}
  It follows immediately from this description that colimit cones are homotopically invariant: if a natural transformation $\eta$ exhibits some object as a colimit of some diagram, than any natural transformation $\eta'$ which is homotopic to $\eta$ will do just as well.
\end{note}

We take a moment to record a definition, which we will need later.

\begin{definition}
  \label{def:preserve_colimits_in_each_slot}
  Let $\category{C}$ be an $\infty$-category, and let $F\colon \category{C} \times \category{C} \to \category{C}$ be a functor. We say that \defn{$F$ preserves colimits in each slot} if and only if the following condition is satisfed:
  \begin{itemize}
    \item Let $K$ and $K'$ be simplicial sets; let $G\colon K \to \category{C}$ and $G'\colon K' \to \category{C}$ be functors; and suppose $\eta\colon G \Rightarrow \underline{c}$ exhibits $c$ as the colimit of $G$, and $\eta'\colon G' \Rightarrow \underline{c'}$ exhibits $c'$ as the colimit of $G'$. Then $F(\eta \times \eta')\colon F \circ (G \times G') \Rightarrow \underline{F(c, c')}$ exhibits $F(c, c')$ as the colimit of $F \circ (G \times G')$.
  \end{itemize}
  %\begin{equation*}
  %  \begin{tikzcd}[row sep=huge, column sep=large]
  %    K \times K'
  %    \arrow[d]
  %    \arrow[r, "G \times G'"]
  %    & \category{C} \times \category{C}
  %    \arrow[r, "F"]
  %    \arrow[dl, Rightarrow, shorten=2ex, swap, "F(\eta \times \eta')"{description}]
  %    & \category{C}
  %    \\
  %    *
  %    \arrow[urr, swap, "\underline{F(c, c')}"]
  %  \end{tikzcd}
  %\end{equation*}
\end{definition}

\begin{notation}
  \label{notation:rund_um_undercategories}
  Let $f\colon X \to Y$ be a map of simplicial sets, and $y \in Y$ an object. We will use the following notation.
  \begin{itemize}
    \item Denote by $X_{/y}$ the fiber product $X \times_{Y} Y_{/y}$.

    \item Denote by $\pi\colon X_{/y} \to X$ the projection map.

    \item Denote by $\alpha\colon f \circ \pi \Rightarrow \underline{y}$ the natural transformation $X_{/y} \times \Delta^{1} \to X$ coming from the comparison map $X_{/y} \to X^{/y}$.
  \end{itemize}
\end{notation}

\begin{definition}[\protect{\cite[Variant~7.3.1.5]{kerodon}}]
  \label{def:nat_xfo_exhibiting_left_kan_ext}
  Let $X$, $Y$, and $\category{C}$ be $\infty$-categories, $f\colon X \to Y$, $F\colon X \to \category{C}$ and $G\colon Y \to \category{C}$ functors, and $\eta\colon F \Rightarrow G \circ f$ a natural transformation.
  \begin{equation*}
    \begin{tikzcd}
      X
      \arrow[dr, swap, "f"]
      \arrow[rr, "F", ""{below, name=M}]
      && \category{C}
      \\
      & Y
      \arrow[ur, swap, "G"]
      \arrow[from=M, Rightarrow, swap, "\eta"]
    \end{tikzcd}
  \end{equation*}
  We say that $\eta$ \defn{exhibits $G$ as the left Kan extension of $F$ along $f$} if for each object $y \in Y$, the natural transformation $F \circ \pi \Rightarrow \underline{G(y)}$ furnished by the pasting diagram
  \begin{equation*}
    \begin{tikzcd}[row sep=large, column sep=large]
      X_{/y}
      \arrow[r, "\pi", ""{name=LA, swap}]
      \arrow[d, ""{name=LD}]
      & X
      \arrow[d, swap, "f"]
      \arrow[r, "F", ""{below, name=M}]
      & \category{C}
      \\
      \{y\}
      \arrow[r, hook, ""{name=LB}]
      & Y
      \arrow[ur, swap, "G"]
      \arrow[from=M, Rightarrow, shorten=2ex, swap, "\eta"]
      \arrow[from=LA, to=LB, Rightarrow, shorten=1ex, "\alpha"]
    \end{tikzcd}
  \end{equation*}
  exhibits $G(y)$ as the colimit of the functor $F \circ \pi$. Here, $\underline{G(y)}$ is the `counterclockwise' path from $X_{/y}$ to $\category{C}$.
\end{definition}

The following guarantees the existence of local left Kan extensions.
\begin{theorem}[\protect{\cite[Proposition 7.3.5.1]{kerodon}}]
  \label{thm:existence_local_left_kan_exts}
  Suppose that $f\colon X \to Y$ is a map of simplicial sets, and suppose $F\colon X \to \category{C}$ is a map of simplicial sets such that $\category{C}$ is a quasicategory. The functor $F$ admits a left Kan extension along $f$ if and only if for all objects $x \in X$, the colimit of the functor
  \begin{equation*}
    X_{/y} \overset{\pi}{\to} X \overset{F}{\to} \category{C}
  \end{equation*}
  exists in $\category{C}$.
\end{theorem}

The following is a combination of special cases of \cite[Proposition~7.3.1.15]{kerodon} and \cite[Corollary~7.3.1.16]{kerodon}
\begin{example}
  \label{eg:strictly_commuting_left_kan}
  For any homotopy-commuting diagram of simplicial sets
  \begin{equation*}
    \begin{tikzcd}
      X
      \arrow[dr, swap, "f"]
      \arrow[rr, "F", ""{below, name=M}]
      && \category{C}
      \\
      & Y
      \arrow[ur, swap, "G"]
      \arrow[from=M, Rightarrow, "\simeq", "\eta"{swap}]
    \end{tikzcd}
  \end{equation*}
  where $X$, $Y$, and $\category{C}$ are $\infty$-categories and $f\colon X \to Y$ is a categorical equivalence, $\eta$ exhibits $G$ as a left Kan extension of $F$ along $f$.
\end{example}

The following guarantees existence of global left Kan extensions.
\begin{theorem}[\protect{\cite[Corollary~7.3.6.3]{kerodon}}]
  \label{thm:existence_global_left_kan_exts}
  Let $f\colon X \to Y$ be a map of simplicial sets, and suppose that $\category{C}$ is a quasicategory which admits $X_{/y}$-shaped colimits for all $y \in Y$. Then the restriction functor $f^{*}\colon \Fun(Y, \category{C}) \to \Fun(X, \category{C})$ admits a left adjoint $f_{!}$, sending a functor $F\colon X \to \category{C}$ to $f_{!}F\colon Y \to \category{C}$, its left Kan extension along $f$.
\end{theorem}

The following is an easy consequence of \cite[Remark~7.3.1.11]{kerodon}, to be explained later.
\begin{proposition}
  Let $f\colon X \to Y$ be a map of simplicial sets, and let $\category{C}$ be a category such that all functors $X \to \category{C}$ admit left Kan extensions along $f$, so that we have an adjunction $f_{!} \vdash f^{*}$. Let $G\colon Y \to \category{C}$ and $\eta\colon F \Rightarrow G \circ f$. The natural transformation $\eta$ exhibits $G$ as a left Kan extension of $F$ along $f$ if and only if it is adjunct (in the sense of \hyperref[def:adjunct_data]{Definition~\ref*{def:adjunct_data}}) to an equivalence in $\Fun(Y, \category{C})$ relative to $s = \id_{\Delta^{1}}$.
\end{proposition}

\begin{proposition}[\protect{\cite[Proposition 7.3.7.18]{kerodon}}]
  \label{prop:kan_extend_along_composition}
  Consider $\infty$-categories and functors
  \begin{equation*}
    \begin{tikzcd}
      X
      \arrow[dr, swap, "f"]
      \arrow[rrrr, "F"]
      &&&& \category{C}
      \\
      & Y
      \arrow[urrr, "G"]
      \arrow[dr, swap, "g"]
      \\
      && Z
      \arrow[uurr, swap, "H"]
    \end{tikzcd},
  \end{equation*}
  and natural transformations $\alpha\colon F \Rightarrow G \circ f$ and $\beta\colon H \circ g \Rightarrow G$, and suppose that $\alpha$ exhibits $G$ as the left Kan extension of $F$ along $f$. Then $\beta$ exhibits $H$ as the left Kan extension of $G$ along $g$ if and only if $\beta f \circ \alpha$ exhibits $H$ as the left Kan extension of $F$ along $g \circ f$.
\end{proposition}

\begin{proposition}[\protect{\cite[Remark 7.3.1.12]{kerodon}}]
  \label{prop:kan_ext_invariance_of_target}
  Suppose $H\colon \category{C} \to \category{D}$ is an equivalence of categories, $f\colon X \to Y$ is a map of simplicial sets, $G\colon Y \to \category{C}$ is a functor, and $\eta\colon F \Rightarrow G \circ f$ is a natural transformation. Then $\eta$ exhibits $G$ as a left Kan extension of $F$ along $f$ if and only if $H\eta\colon H \circ F \Rightarrow H \circ G \circ f$ exhibits $H \circ G$ as a left Kan extension of $H \circ F$ along $f$.
\end{proposition}

\begin{proposition}[\protect{\cite[Remark 7.3.1.9]{kerodon}}]
  \label{prop:homotopy_invariance_of_witness}
  The property that $\eta$ exhibits $G$ as a left Kan extension of $F$ along $f$ is homotopy-invariant; any homotopic $\eta$ will do just as well.
\end{proposition}

\subsection{Filling cubes relative to their boundaries}
\label{ssc:filling_cubes_relative_to_their_boundary}

Our main goal in this \hyperref[sec:horn_filling_via_left_kan_extensions]{Section~\ref*{sec:horn_filling_via_left_kan_extensions}} is to prove \hyperref[thm:left_kan_implies_globally_left_kan]{Theorem~\ref*{thm:left_kan_implies_globally_left_kan}}. To do this, we must understand lifting problems of the form
\begin{equation*}
  \begin{tikzcd}
    (\Lambda^{n+1}_{0})_{\flat}
    \arrow[r]
    \arrow[d, hook]
    & \ICCat
    \\
    \Delta^{n+1}_{\flat}
    \arrow[ur, dashed]
  \end{tikzcd}.
\end{equation*}

The category $\ICCat$ is defined to be the homotopy coherent nerve of the $\SSetmk$-enriched category $\QCat$ of quasicategories. Therefore, solving such lifting problems is equivalent to solving the adjunct lifting problems
\begin{equation*}
  \begin{tikzcd}
    \C[\Lambda^{n+1}_{0}]
    \arrow[r]
    \arrow[d, hook]
    & \QCat
    \\
    \C[\Delta^{n+1}]
    \arrow[ur, dashed]
  \end{tikzcd}
\end{equation*}
where the diagram in question is now a diagram of $\SSetmk$-enriched categories (where the mapping spaces of $\C[\Lambda^{n+1}_{0}]$ and $\C[\Delta^{n+1}]$ are taken to carry the flat marking). In order to solve such lifting problems, we need to fill the missing data in each mapping space. As we will see in \hyperref[ssc:left_kan_implies_globally_left_kan]{Subsection~\ref*{ssc:left_kan_implies_globally_left_kan}}, these fillings take the form of filling a cube $(\Delta^{1})^{n}$ relative to its boundary missing one face.

In this subsection, we give the combinatorics of filling the $n$-cube. Our main result is \hyperref[prop:cube_filling]{Proposition~\ref*{prop:cube_filling}}, which writes the inclusion the boundary of the $n$-cube missing a certain face into the full cube as a composition of an inner anodyne map and a marked anodyne map.

\begin{definition}
  The \defn{$n$-cube} is the simplicial set $C^{n} := (\Delta^{1})^{n}$.
\end{definition}

\begin{note}
  We will consider the the factors $\Delta^{1}$ of $C^{n}$ to be ordered, so that we can speak about the first factor, the second factor, etc.
\end{note}

Our first step is to understand the nondegenerate simplices of the $n$-cube.

\begin{definition}
  Denote by $a^{i}\colon \Delta^{n} \to \Delta^{1}$ the map which sends $\Delta^{\{0, \ldots, i-1\}}$ to $\{0\}$, and $\Delta^{\{i, \ldots, n\}}$ to $\{1\}$.
\end{definition}

\begin{note}
  The superscript of $a^{i}$ counts how many vertices of $\Delta^{n}$ are sent to $\Delta^{\{0\}} \subset \Delta^{1}$.
\end{note}

Every nondegenerate simplex $\Delta^{n} \to C^{n}$ can be specified by giving a walk along the edges of $C^{n}$ starting at $(0, \ldots, 0)$ and ending at $(1, \ldots, 1)$, and any simplex specified in this way is nondegenerate. We now use this to define a bijection between $S_{n}$, the symmetric group on the set $\{1, \ldots, n\}$, and the nondegenerate simplices $\Delta^{n} \to C^{n}$ as follows.

\begin{definition}
  For any permutation $\tau\colon \{1, \ldots, n\} \to \{1, \ldots, n\}$, we define an $n$-simplex
  \begin{equation*}
    \phi(\tau)\colon \Delta^{n} \to C^{n};\qquad \phi(\tau)_{i} = a^{\tau(i)}.
  \end{equation*}
  That is,
  \begin{equation*}
    \phi(\tau) = (a^{\tau(1)}, \ldots, a^{\tau(n)}).
  \end{equation*}
\end{definition}

\begin{note}
  It is easy to check that the above definition really results in a bijection between $S_{n}$ and the nondegenerate simplices of $C^{n}$.
\end{note}

\begin{notation}
  We will denote any permutatation $\tau\colon \{1, \ldots, n\} \to \{1, \ldots, n\}$ by the corresponding $n$-tuple $(\tau(1), \ldots, \tau(n))$. Thus, the identity permutation is $(1, \ldots, n)$, and the permutation $\gamma_{1,2}$ which swaps $1$ and $2$ is $(2, 1, 3, \ldots, n)$.
\end{notation}

Given a permutation $\tau \in S_{n}$ corresponding to an $n$-simplex
\begin{equation*}
  \phi(\tau) = (a^{\tau(1)}, \ldots, a^{\tau(n)})\colon \Delta^{n} \to C^{n},
\end{equation*}
the $i$th face of $\phi(\tau)$ is a nondegenerate $(n-1)$-simplex in $C^{n}$, i.e.\ a map $\Delta^{n-1} \to C^{n}$. We calculate this as follows: for any $a^{i}\colon \Delta^{n} \to \Delta^{1}$ and any face map $\partial_{j}\colon \Delta^{n-1} \to \Delta^{n}$, we note that
\begin{equation*}
  \partial_{j}^{*}a^{i} =
  \begin{cases}
    a^{i-1}, & j < i \\
    a^{i}, & j \geq i
  \end{cases},
\end{equation*}
where by minor abuse of notation we denote the map $\Delta^{n-1} \to \Delta^{1}$ sending $\Delta^{\{0, \ldots i-1\}}$ to $\{0\}$ and the rest to $\{1\}$ also by $a^{i}$. We then have that
\begin{gather*}
  \label{eq:simplicial_identity_for_cubes}
  d_{i}\tau = d_{i}(a^{\tau(1)}, \ldots, a^{\tau(n)}) := (\partial_{i}^{*} a^{\tau(1)}, \ldots, \partial_{i}^{*} a^{\tau(n)}).
\end{gather*}

\begin{example}
  Consider the 3-simplex $(a^{1}, a^{2}, a^{3})$ in $C^{3}$. This has spine
  \begin{equation*}
    (0, 0, 0) \to (1, 0, 0) \to (1, 1, 0) \to (1, 1, 1),
  \end{equation*}
  as is easy to see quickly:
  \begin{itemize}
    \item The function $\Delta^{3} \to \Delta^{1}$ corresponding to the first coordinate is $a^{1}$, so the first coordinate of the first vertex is 0, and the first coordinate of the rest of the vertices are 1.

    \item The function $\Delta^{3} \to \Delta^{1}$ corresponding to the second coordinate is $a^{2}$, so the second coordinates of the first two points are equal to zero, and the second coordinate of the remaining vertices is 1.

    \item The function $\Delta^{3} \to \Delta^{1}$ corresponding to the third coordinate is $a^{3}$, so the third coordinates of the first three points is equal to zero, and the second coordinate of the final vertex is equal to 1.
  \end{itemize}
  Using \hyperref[eq:simplicial_identity_for_cubes]{Equation~\ref*{eq:simplicial_identity_for_cubes}}, we then calculate that
  \begin{equation*}
    d_{2} (a^{1}, a^{2}, a^{3}) = (a^{1}, a^{2}, a^{2}).
  \end{equation*}
  This corresponds to the 2-simplex in $C^{n}$ with spine
  \begin{equation*}
    (0, 0, 0) \to (1, 0, 0) \to (1, 1, 1).
  \end{equation*}
\end{example}

Our next task is to understand the boundary of the $n$-cube. We can view the boundary of the cube $C^{n}$ as the union of its faces.

\begin{definition}
  \label{def:boundary_of_n-cube}
  The \defn{boundary of the $n$-cube} is the simplicial subset
  \begin{equation}
    \label{eq:boundary_of_cube}
    \partial C^{n} := \bigcup_{i = 1}^{n} \bigcup_{j = 0}^{1} \overset{(1)}{\Delta^{1}} \times \cdots \times \overset{(i)}{\{j\}} \times \cdots \times \overset{(n)}{\Delta^{1}} \subset C^{n}.
  \end{equation}
\end{definition}

The following is easy to see.
\begin{proposition}
  \label{prop:simplex_intersect_faces}
  An $(n-1)$-simplex
  \begin{equation*}
    \gamma = (\gamma_{1}, \ldots, \gamma_{n})\colon \Delta^{n-1} \to C^{n}
  \end{equation*}
  lies entirely within the face
  \begin{equation*}
    \overset{(1)}{\Delta^{1}} \times \cdots \times \overset{(i)}{\{0\}} \times \cdots \times \overset{(n)}{\Delta^{1}}
  \end{equation*}
  if and only if $\gamma_{i} = a^{n} = \const_{0}$, and to the face
  \begin{equation*}
    \overset{(1)}{\Delta^{1}} \times \cdots \times \overset{(i)}{\{1\}} \times \cdots \times \overset{(n)}{\Delta^{1}}
  \end{equation*}
  if and only if $\gamma_{i} = a^0 = \const_{1}$.
\end{proposition}

\begin{definition}
  \label{def:left_box}
  The \defn{left box of the $n$-cube}, denoted $LC^{n}$, is the simplicial subset of $C^{n}$ given by the union of all of the faces in \hyperref[eq:boundary_of_cube]{Equation~\ref*{eq:boundary_of_cube}} except for
  \begin{equation*}
    \Delta^{1} \times \cdots \times \Delta^{1} \times \Delta^{\{1\}}.
  \end{equation*}

  Note that drawing the box with the final coordinate going from left to right, the right face is open here; the terminology is chosen to match with `left horn'.
\end{definition}

\begin{example}
  The simplicial subset $LC^{2} \subset C^{2}$ can be drawn as follows.
  \begin{equation*}
    \begin{tikzcd}
      {(0, 0)}
      \arrow[r]
      \arrow[d]
      & {(0, 1)}
      \\
      {(1, 0)}
      \arrow[r]
      & {(1, 1)}
    \end{tikzcd}
  \end{equation*}
\end{example}

Note that in $LC^{2}$, the morphisms $(0,0) \to (1, 0) \to (1, 1)$ form an inner horn, which we can fill by pushing out along an inner horn inclusion. Our main goal in this section is to show that this is generically true: given a left cube $LC^{n}$, we can fill much of $C^{n}$ using pushouts along inner horn inclusions. To this end, it will be helpful to know how each nondegenerate simplex in $C^{n}$ intersects $LC^{n}$.

\begin{lemma}
  \label{lemma:first_and_last_face}
  Let $\phi\colon \Delta^{n} \to C^{n}$ be a nondegenerate $n$-simplex corresponding to the permutaton $\tau \in S_{n}$. We have the following.
  \begin{enumerate}
    \item The zeroth face $d_{0}\phi$ belongs to $LC^{n}$ if and only if $\tau(n) \neq 1$.

    \item For $0 < i < n$, the face $d_{i}\phi$ never belongs to $LC^{n}$.

    \item The $n$th face $d_{n}\phi$ always belongs to $LC^{n}$.
  \end{enumerate}
\end{lemma}
\begin{proof}
  \begin{enumerate}
    \item We have
      \begin{equation*}
        d_{0}\phi = (a^{\tau(0) - 1}, \ldots a^{\tau(n)-1})
      \end{equation*}
      since $\tau(i) > 0$ for all $i$. For $j = \tau^{-1}(1)$, we have that $\tau(j) = 1$, so the $j$th entry of $d_{0}\phi$ is $a^{0}$. Thus, \hyperref[prop:simplex_intersect_faces]{Proposition~\ref*{prop:simplex_intersect_faces}} guarantees that $d_{0}\phi$ is contained in the face
      \begin{equation*}
        \overset{(1)}{\Delta^{1}} \times \cdots \times \overset{(j)}{\{1\}} \times \cdots \times \overset{(n)}{\Delta^{1}}.
      \end{equation*}
      This face belongs to $LC^{n}$ except when $j = n$.

    \item In this case, the superscript of each entry of $d_{i}\phi$ is between $1$ and $n-1$, hence not equal to $n$ or $0$.

    \item In this case,
      \begin{equation*}
        d_{n}(a^{\tau(1)}, \ldots, a^{\tau(n)}) = (a^{\tau(1)}, \ldots, a^{\tau(n)})
      \end{equation*}
      since $n \geq \tau(i)$ for all $1 \leq i \leq n$. Thus, the $\tau^{-1}(n) = j$th entry is $a^{n}$, so $d_{n}\phi$ belongs to the face
      \begin{equation*}
        \overset{(1)}{\Delta^{1}} \times \cdots \times \overset{(j)}{\{0\}} \times \cdots \times \overset{(n)}{\Delta^{1}}.
      \end{equation*}
  \end{enumerate}
\end{proof}

This is promising: it means that for many simplices in $C^{n}$ which we want to fill relative to $LC^{n}$, we already have the data of the first and last faces. The following lemma will allow us to take advantage of this.

\begin{notation}
  For any subset $T \subseteq [n]$, write
  \begin{equation*}
    \Lambda^{n}_{T} := \bigcup_{t \in T} d_{t} \Delta^{n} \subset \Delta^{n}.
  \end{equation*}
\end{notation}

\begin{lemma}
  \label{lemma:subset_of_faces_inner_anodyne}
  For any proper subset $T \subset [n]$ containing 0 and $n$, the inclusion $\Lambda^{n}_{T} \hookrightarrow \Delta^{n}$ is inner anodyne.
\end{lemma}
\begin{proof}
  Induction. For $n = 2$, $T$ must be equal to $\{0, 2\}$, so $\Delta^{2}_{T} = \Lambda^{2}_{1}$.

  Assume the result holds for $n-1$, and let $T \subset [n]$ be a proper subset containing $0$ and $n$. Using the inductive step, we can fill all but one of the faces $d_{1}\Delta^{n}, \ldots, d_{n-1}\Delta^{n}$. The result follows
\end{proof}

We can use \hyperref[lemma:subset_of_faces_inner_anodyne]{Lemma~\ref*{lemma:subset_of_faces_inner_anodyne}} to show that we can fill much of $C^{n}$ as a sequence of inner anodyne pushouts, but to do this we need to pick an order in which to fill our simplices. We do this as follows.

\begin{definition}
  \label{def:order_on_nondegenerate_simplices}
  We define a total order on $S_{n}$ as follows. For any two permutations $\tau$, $\tau' \in S_{n}$, we say that $\tau < \tau'$ if there exists $k \in [n]$ such that the following conditions are satisfied.
  \begin{itemize}
    \item For all $i < k$, we have that $\tau^{-1}(i) = \tau'^{-1}(i)$.

    \item We have that $\tau^{-1}(k) < \tau'^{-1}(k)$.
  \end{itemize}

  We then define a total order on the nondegenerate simplices of $C^{n}$ by saying that $\phi(\tau) < \phi(\tau')$ if and only if $\tau < \tau'$.
\end{definition}

Note that this is \emph{not} the lexicographic order on $S_{n}$; instead, we have that $\tau < \tau'$ if and only if $\tau^{-1}$ is less than $\tau'^{-1}$ under the lexicographic order.

\begin{example}
  The elements of the permutation group $S_{3}$ have the order
  \begin{equation*}
    (1,2,3) < (1,3,2) < (2,1,3) < (3,1,2) < (2,3,1) < (3,2,1).
  \end{equation*}
\end{example}

We use this ordering to define our filtration.

\begin{notation}
  For $\tau \in S_{n}$, we mean by
  \begin{equation*}
    \bigcup_{\tau' \in S_{n}}^{\tau}
  \end{equation*}
  the union over all $\tau' \in S_{n}$ for which $\tau' < \tau$ with respect to the ordering defined above. Note the strict inequality.
\end{notation}

We would like to show that each step of the filtration
\begin{align}
  \label{eq:inclusions_inner_anodyne}
  \begin{split}
    LC^{n} &\hookrightarrow LC^{n} \cup \phi(1, \ldots, n) \\
    & \hookrightarrow \cdots \\
    & \hookrightarrow LC^{n} \cup \bigcup_{\tau \in S_{n}}^{(2, \ldots, n, 1)} \phi(\tau)
  \end{split}
\end{align}
is inner anodyne. To do this, it suffices to show that for each $\tau < (2, \ldots, n, 1)$, the intersection
\begin{equation}
  \label{eq:intersections_inner_anodyne}
  B_{\tau} = \left(LC^{n} \cup \bigcup_{\tau' \in S_{n}}^{\tau} \phi(\tau')\right) \cap \phi(\tau)
\end{equation}
is of the form $\Lambda^{n}_{T}$ for some proper $T \subseteq [n]$ containing 0 and $n$.

\begin{proposition}
  \label{prop:inner_anodyne_cube_filling}
  Each inclusion in \hyperref[eq:inclusions_inner_anodyne]{Equation~\ref*{eq:inclusions_inner_anodyne}} is inner anodyne.
\end{proposition}
\begin{proof}
  The first inclusion
  \begin{equation*}
    LC^{n} \hookrightarrow LC^{n} \cup \phi(1, \ldots, n)
  \end{equation*}
  is inner anodyne because by \hyperref[lemma:first_and_last_face]{Lemma~\ref*{lemma:first_and_last_face}} the intersection $LC^{n} \cap \phi(1, \ldots, n)$ is of the form $d_{0}\Delta^{n} \cup d_{n} \Delta^{n}$.

  Each subsequent intersection contains the faces $d_{0}\Delta^{n}$ and $d_{n}\Delta^{n}$ (again by \hyperref[lemma:first_and_last_face]{Lemma~\ref*{lemma:first_and_last_face}}), so by \hyperref[lemma:subset_of_faces_inner_anodyne]{Lemma~\ref*{lemma:subset_of_faces_inner_anodyne}}, it suffices to show that for each $\tau$ under consideration, there is at least one face not shared with any previous $\tau'$.

  To this end, fix $0 < i < n$, and consider $\tau \in S_{n}$ such that $\tau \neq (1, \ldots, n)$. We consider the face $d_{i}\phi(\tau)$. Let $j = \tau^{-1}(i)$ and $j' = \tau^{-1}(i+1)$. Then $d_{i} a^{\tau(j)} = d_{i}a^{\tau(j')}$, so $d_{i} \phi(\tau) = d_{i}\phi(\tau \circ \gamma_{jj'})$, where $\gamma_{jj'} \in S_{n}$ is the permutation swapping $j$ and $j'$. Thus, the face $d_{i}\sigma(\tau)$ is equal to the face $d_{i}\sigma(\tau \circ \gamma_{jj'})$, which is already contained in the union under consideration if and only $\tau \circ \gamma_{jj'} < \tau$. This in turn is true if and only if $j < j'$.

  Assume that every face of $\phi(\tau)$ is contained in the union. Then
  \begin{equation*}
    \tau^{-1}(1) < \tau^{-1}(2) < \cdots \tau^{-1}(n),
  \end{equation*}
  which implies that $\tau = (1, \ldots, n)$. This is a contradiction. Thus, each boundary inclusion under consideration is of the form $\Lambda^{n}_{T} \hookrightarrow \Delta^{n}$, for $T \subset [n]$ a proper subset containing 0 and $n$, and is thus inner anodyne.
\end{proof}

Each nondegenerate simplex of $C^{n}$ which we have yet to fill is of the form $\phi(\tau)$, where $\tau(n) = 1$. Thus, each of their spines begins
\begin{equation*}
  (0, \ldots, 0, 0) \to (0, \ldots, 0, 1) \to \cdots,
\end{equation*}
and then remains confined to the face $\Delta^{1} \times \cdots \times \Delta^{1} \times \{1\}$. This implies that the part of $LC^{n}$ yet to be filled is of the form $\Delta^{0} \ast C^{n-1}$, and the boundary of this with respect to which we must do the filling is of the form $\Delta^{0} \ast \partial C^{n-1}$.

Our next task is to show that we can also perform this filling, under the assumption that $(0, \ldots, 0, 0) \to (0, \ldots, 0, 1)$ is marked. We do not give the details of this proof, at is quite similar to the proof of \hyperref[prop:inner_anodyne_cube_filling]{Proposition~\ref*{prop:inner_anodyne_cube_filling}}. One continues to fill simplices in the order prescribed in \hyperref[def:order_on_nondegenerate_simplices]{Definition~\ref*{def:order_on_nondegenerate_simplices}}, and shows that each of these inclusions is marked anodyne. The main tool is the following lemma, proved using the same inductive argument as \hyperref[lemma:subset_of_faces_inner_anodyne]{Lemma~\ref*{lemma:subset_of_faces_inner_anodyne}}.

\begin{lemma}
  \label{lemma:subset_of_faces_marked_anodyne}
  For any subset $T \subset [n]$ containing 1 and $n$, and \emph{not} containing 0, the inclusion
  \begin{equation*}
    (\Lambda^{n}_{T}, \mathcal{E}) \hookrightarrow (\Delta^{n}, \mathcal{F})
  \end{equation*}
  is (cocartesian-)marked anodyne, where by $\mathcal{F}$ we mean the set of degenerate $1$-simplces together with $\Delta^{\{0, 1\}}$, and by $\mathcal{E}$ we mean the restriction of this marking to $\Lambda^{n}_{T}$.
\end{lemma}

We collect our major results from this section in the following proposition.

\begin{proposition}
  \label{prop:cube_filling}
  Denote by $J^{n} \subseteq C^{n}$ the simplicial subset spanned by those nondegenerate $n$-simplices whose spines do not begin $(0, \ldots, 0, 0) \to (0, \ldots, 0, 1)$. We can write the filling $LC^{n} \hookrightarrow C^{n}$ as a composition
  \begin{equation*}
    \begin{tikzcd}
      LC^{n}
      \arrow[r, hook, "i"]
      & J^{n}
      \arrow[r, hook, "j"]
      & C^{n}
    \end{tikzcd},
  \end{equation*}
  where $i$ is inner anodyne, and $j$ fits into a pushout square
  \begin{equation*}
    \begin{tikzcd}
      \Delta^{0} \ast \partial C^{n-1}
      \arrow[dr, phantom, "\ulcorner"{at end}]
      \arrow[r, hook]
      \arrow[d, hook]
      & \Delta^{0} \ast C^{n-1}
      \arrow[d, hook]
      \\
      J^{n}
      \arrow[r, hook]
      & C^{n}
    \end{tikzcd},
  \end{equation*}
  where the top inclusion underlies a marked anodyne morphism
  \begin{equation*}
    (\Delta^{0} \ast \partial C^{n-1}, \mathcal{G}) \hookrightarrow (\Delta^{0} \ast C^{n-1}, \mathcal{G}),
  \end{equation*}
  where the marking $\mathcal{G}$ contains all degenerate morphisms together with the morphism $(0, \ldots, 0, 0) \to (0, \ldots,0, 1)$.

  %is the morphism underlying a
  %(cocartesian-) marked anodyne morphism of marked simplicial sets $(J^{n})^{\clubsuit} \hookrightarrow (C^{n})^{\clubsuit}$. Furthermore, we have a pushout square
  %\begin{equation*}
  %  \begin{tikzcd}
  %    \Delta^{0} \ast \partial C^{n-1}
  %    \arrow[dr, phantom, "\ulcorner"{at end}]
  %    \arrow[r, hook]
  %    \arrow[d, hook]
  %    & \Delta^{0} \ast C^{n-1}
  %    \arrow[d, hook]
  %    \\
  %    J^{n}
  %    \arrow[r, hook]
  %    & C^{n}
  %  \end{tikzcd},
  %\end{equation*}
  %where the right-hand downward-facing inclusion is the unique one taking $\Delta^{0}$ to $(0, \ldots, 0) \in C^{n}$, and $C^{n-1}$ to the right face
  %\begin{equation*}
  %  \Delta^{1} \times \cdots \times \Delta^{1} \times \{1\} \subset C^{n}.
  %\end{equation*}
\end{proposition}

\subsection{Adjunct data}
\label{ssc:adjunct_data}

When confronted with a lifting problem, one frequently finds a lift by passing to an adjoint lifting problem which has a solution, and transporting the solution back along the adjunction. This relies on the fact that providing data on one side of an adjunction is, in a certain sense, equivalent to providing adjunct data on the other side. In this section, we give a formalization of the the notion of adjunct data in the $\infty$-categorical context. Our main result is \hyperref[prop:can_transport_adjunct_data]{Proposition~\ref*{prop:can_transport_adjunct_data}}, which shows that under certain conditions, we can use data on one side of an adjunction of an $\infty$-categories to fill in data on the other side. First, we recall somewhat explicitly the definition of an adjunction given in \cite{highertopostheory}.

\begin{definition}
  \label{def:adjunction}
  An \defn{adjunction} is a bicartesian fibration $p\colon \mathcal{M} \to \Delta^{1}$. We say that $p$ is \defn{associated} to functors $f\colon \category{C} \to \category{D}$ and $g\colon \category{D} \to \category{C}$ if there exist equivalences $h_{0}\colon \category{C} \to \category{M}_{0}$ and $h_{1}\colon \category{D} \to \category{M}_{1}$, and a commutative diagram
  \begin{equation*}
    \begin{tikzcd}
      \category{C} \times \Delta^{1}
      \arrow[r, "u"]
      \arrow[dr, swap, "\pr"]
      & \mathcal{M}
      \arrow[d, "p"]
      & \category{D} \times \Delta^{1}
      \arrow[l, swap, "v"]
      \arrow[dl, "\pr"]
      \\
      & \Delta^{1}
    \end{tikzcd}
  \end{equation*}
  such that the following conditions are satisfied.
  \begin{itemize}
    \item \emph{The map $u$ is associated to the functor $f$:} the restriction $u|\category{C} \times \{0\} = h_{0}$, the restriction $u|\category{C} \times \{1\} = f \circ h_{1}$, and for each $c \in \category{C}$, the edge $u|\{c\} \times \Delta^{1}$ is $p$-cocartesian.

    \item \emph{The map $v$ is associated to the functor $g$:} the restriction $v|\category{D} \times \{0\} = g \circ h_{0}$, the restriction $u|\category{D} \times \{1\} = h_{1}$, and for each $d \in \category{D}$, the edge $u|\{d\} \times \Delta^{1}$ is $p$-cartesian.
  \end{itemize}

  If $f$ and $g$ are functors to which an adjunction is associated as above, then we say that $f$ is left adjoint to $g$, and equivalently that $g$ is right adjoint to $f$.
\end{definition}

\begin{note}
  As noted in \cite{highertopostheory}, this terminology is slightly imprecise; it would be more correct to say that $p$ is associated to $f$ and $g$ via the data $(h_{0}, h_{1}, u, v)$.
\end{note}

\begin{note}
  It is shown in \cite{highertopostheory} that if $f$ and $g$ are functors such that $f$ is left adjoint to $g$, then we can choose an adjunction $p\colon \mathcal{M} \to \Delta^{1}$ and data $(h_{0}, h_{1}, u, v)$ such that $h_{0}$ and $h_{1}$ are isomorphisms. We can use this to identify $\mathcal{M}_{0}$ with $\category{C}$, and $\mathcal{M}_{1}$ with $\category{D}$. In what follows we will always assume that we have chosen such data, and will thus leave the isomorphisms $h_{0}$ and $h_{1}$ implicit.
\end{note}

\begin{definition}
  \label{def:adjunct_data}
  Let $s\colon K \to \Delta^{1}$ be a map of simplicial sets whose fibers we denote by $K_{0}$ and $K_{1}$, and let $p\colon \category{M} \to \Delta^{1}$ be a bicartesian fibration associated to adjoint functors
  \begin{equation*}
    f : \category{C} \longleftrightarrow \category{D} : g
  \end{equation*}
  via data $(u\colon \category{C} \times \Delta^{1} \to \category{M}, v\colon \category{D} \times \Delta^{1} \to \category{M})$. We say that a map $\alpha\colon K \to \category{C}$ is \defn{adjunct} to a map $\tilde{\alpha}\colon K \to \category{D}$ relative to $s$, and equivalently that $\tilde{\alpha}$ is adjunct to $\alpha$ relative to $s$, if there exists a map $A\colon K \times \Delta^{1} \to \category{M}$ such that the diagram
  \begin{equation*}
    \begin{tikzcd}
      K \times \Delta^{1}
      \arrow[r, "A"]
      \arrow[dr, swap, "\pr_{\Delta^{1}}"]
      & \category{M}
      \arrow[d, "p"]
      \\
      & \Delta^{1}
    \end{tikzcd}
  \end{equation*}
  commutes, and such that the following conditions are satisfied.
  \begin{enumerate}
    \item The restriction $A|_{K \times \{0\}} = \alpha$.

    \item The restriction $A|_{K \times \{1\}} = \tilde{\alpha}$.

    \item The restriction $A|K_{0} \times \Delta^{1}$ is equal to the composition
      \begin{equation*}
        K_{0} \times \Delta^{1} \hookrightarrow K \times \Delta^{1} \overset{\alpha \circ \times \id}{\to} \category{C} \times \Delta^{1} \overset{u}{\to} \mathcal{M}.
      \end{equation*}

    \item The restriction $A|K_{1} \times \Delta^{1}$ is equal to the composition
      \begin{equation*}
        K_{1} \times \Delta^{1} \hookrightarrow K \times \Delta^{1} \overset{\tilde{\alpha} \times \id}{\to} \category{D} \times \Delta^{1} \overset{v}{\to} \mathcal{M}.
      \end{equation*}
  \end{enumerate}
\end{definition}

In the next examples, fix adjoint functors
\begin{equation*}
  f : \category{C} \longleftrightarrow \category{D} : g
\end{equation*}
corresponding to a bicartesian fibration $p\colon \category{M} \to \Delta^{1}$ via data $(u, v)$.

\begin{example}
  Any morphism in $\category{D}$ of the form $fC \to D$ is adjunct to some morphism in $\category{C}$ of the form $C \to gD$, witnessed by any square
  \begin{equation*}
    \begin{tikzcd}
      C
      \arrow[r, "a"]
      \arrow[d]
      & fC
      \arrow[d]
      \\
      gD
      \arrow[r, "b"]
      & D
    \end{tikzcd}
  \end{equation*}
  in $\category{M}$ where $a$ is $u|\{c\} \times \Delta^{1}$ and $b$ is $v|\{d\} \times \Delta^{1}$. This corresponds to the map $s = \id\colon \Delta^{1} \to \Delta^{1}$.
\end{example}

\begin{example}
  Pick some object $D \in \category{D}$, and consider the identity morphism $\id\colon gD \to gD$ in $\category{C}$. This morphism is adjunct to the component of the unit map $\eta_{D}\colon D \to fgD$ relative to $s = \id\colon \Delta^{1} \to \Delta^{1}$.
\end{example}


\begin{example}
  For any simplicial sets $K$ and $K'$, any diagram $K \ast K' \to \category{D}$ such that $K$ is in the image of $f$ is adjunct to some diagram $K \ast K' \to \category{C}$ such that $K'$ is in the image of $g$. This will follow from \hyperref[prop:can_transport_adjunct_data]{Proposition~\ref*{prop:can_transport_adjunct_data}}.
\end{example}

\begin{lemma}
  \label{lemma:left_cylinder_inclusion_marked_anodyne}
  The inclusion of marked simplicial sets
  \begin{equation}
    \label{eq:cocartesian_anodyne_morphism}
    \left( \Delta^{n} \times \Delta^{\{0\}} \coprod_{\partial \Delta^{n} \times \Delta^{\{0\}}} \partial \Delta^{n} \times \Delta^{1}, \mathcal{E} \right) \hookrightarrow (\Delta^{n} \times \Delta^{1}, \mathcal{F})
  \end{equation}
  where the marking $\mathcal{F}$ is the flat marking together with the edge $\Delta^{\{0\}} \times \Delta^{1}$, and $\mathcal{E}$ is the restriction of this marking, is marked (cocartesian) anodyne.
\end{lemma}

The next proposition shows the power of adjunctions in lifting problems: given data on either side of an adjunction, we can fill in adjunct data on the other side.

\begin{proposition}
  \label{prop:can_transport_adjunct_data}
  Let $\category{M} \to \Delta^{1}$, $\category{C}$, $\category{D}$, $f$, $g$, $u$ and $v$ be as in \hyperref[def:adjunction]{Definition~\ref*{def:adjunction}}, and let
  \begin{equation*}
    \begin{tikzcd}[column sep=small]
      K'
      \arrow[rr, hook, "i"]
      \arrow[dr, swap, "s \circ i"]
      && K
      \arrow[dl, "s"]
      \\
      & \Delta^{1}
    \end{tikzcd},
  \end{equation*}
  be a commuting triangle of simplicial sets, where $i$ is a monomorphim such that $i|_{\{1\}}$ is an isomorphism. Let $\tilde{\alpha}'\colon K' \to \category{D}$ and $\alpha\colon K \to \category{C}$ be maps, and denote $\alpha' = \alpha \circ i$. Suppose that $\tilde{\alpha}'$ is adjunct to $\alpha'$ relative to $s \circ i$. Then there exists a dashed extension
  \begin{equation}
    \label{eq:adjunct_filling_problems}
    \begin{tikzcd}
      K'
      \arrow[r, "\alpha'"]
      \arrow[d, swap, hook, "i"]
      & \category{C}
      \\
      K
      \arrow[ur, swap, "\alpha"]
    \end{tikzcd}
    \quad \rightsquigarrow \quad
    \begin{tikzcd}
      K'
      \arrow[r, "\tilde{\alpha}'"]
      \arrow[d, swap, hook, "i"]
      & \category{D}
      \\
      K
      \arrow[ur, dashed, swap, "\tilde{\alpha}"]
    \end{tikzcd}
  \end{equation}
  such that $\tilde{\alpha}$ is adjunct to $\alpha$ relative to $s$. Furthermore, any two such lifts are equivalent as functors $K \to \category{D}$.

  %Furthermore, denoting by
  %\begin{equation*}
  %  \catname{Adj} \subseteq \Fun(K, \category{D}) \times_{\Fun(K', \category{D})} \{\alpha'\}
  %\end{equation*}
  %the full simplicial subset on those functors $\tilde{\alpha}\colon K \to \category{D}$ which which are adjoint to $\alpha$, so that $\catname{Adj}$ is the space of solutions to the above lifting problem, we have that $\catname{Adj}$ is a contractible Kan complex.
\end{proposition}
\begin{proof}
  Pick some commutative diagram
  \begin{equation*}
    \begin{tikzcd}
      K' \times \Delta^{1}
      \arrow[r, "A"]
      \arrow[dr, swap, "\pr_{\Delta^{1}}"]
      & \category{M}
      \arrow[d]
      \\
      & \Delta^{1}
    \end{tikzcd}
  \end{equation*}
  displaying $\tilde{\alpha}'$ and $\alpha'$ as adjunct. From the map $A$ and the map $\alpha\colon K \to \category{C} \cong \category{M}_{0}$, we can construct the solid commutative square of cocartesian-marked simplicial sets
  \begin{equation*}
    \begin{tikzcd}
      \left(K \times \Delta^{\{0\}} \coprod_{K' \times \Delta^{\{0\}}} K' \times \Delta^{1}\right)^{\heartsuit}
      \arrow[r]
      \arrow[d, swap, "w"]
      & \category{M}^{\natural}
      \arrow[d]
      \\
      (K \times \Delta^{1})^{\heartsuit}
      \arrow[ur, dashed, "\ell"]
      \arrow[r]
      & (\Delta^{1})^{\sharp}
    \end{tikzcd},
  \end{equation*}
  where $(K \times \Delta^{1})^{\heartsuit}$ is the marked simplicial set where the only nondenerate morphisms marked are those of the form $\{k\} \times \Delta^{1}$ for $k \in K|_{s^{-1}\{0\}}$, and the $\heartsuit$-marked pushout-product denotes the restriction of this marking to the pushout-product. We can write the morphism of simplicial sets underlying $w$ as a smash-product $(K' \hookrightarrow K) \wedge (\Delta^{\{0\}} \hookrightarrow \Delta^{1})$. Building $K' \hookrightarrow K$ simplex by simplex, in increasing order of dimension, we can write $w$ as a transfinite composition of pushouts along inclusions of the form given in \hyperref[eq:cocartesian_anodyne_morphism]{Equation~\ref*{eq:cocartesian_anodyne_morphism}} (by assumption each simplex $\sigma$ we adjoin to $K$ is not completely contained in $K'$, so the initial vertex of $\sigma$ must lie in the fiber of $s$ over $0$). We can thus also build our lift $\ell$ by lifting against each of these in turn, and we can always choose each lift so that it is compatible with $v$. We can then take $\tilde{\alpha} = \ell|K \times \{1\}$.

  It is manifest from this technique that the space of lifts is contractible; in particular, any two such lifts are equivalent in $\Map(K \times \Delta^{1}, \category{M})$, so their restrictions to $K \times \{1\}$ are equivalent as functors $K \to \category{D}$.

  %Each of these is marked anodyne, so the left-hand map is as well. Thus, a dashed lift $\ell$ exists, giving us a map $h_{1}^{-1} \circ \ell|_{K \times \Delta^{\{1\}}}\colon K \to \category{D}$. This, together with the composition
  %\begin{equation*}
  %  K' \times \Delta^{1} \overset{\tilde{\alpha}' \times \id}{\to} \category{D} \times \Delta^{1} \overset{c}{\to} \category{D}
  %\end{equation*}
  %where $c\colon h_{1}^{-1} \circ h_{1} \to \id_{\category{D}}$ is a natural isomorphism, gives a solid diagram
  %\begin{equation*}
  %  \begin{tikzcd}
  %    K \times \Delta^{\{0\}} \coprod_{K' \times \Delta^{\{0\}}} K' \times (\Delta^{1})^{\sharp}
  %    \arrow[r]
  %    \arrow[d]
  %    & \category{D}^{\natural}
  %    \\
  %    K \times (\Delta^{1})^{\sharp}
  %    \arrow[ur, dashed, swap, "m"]
  %  \end{tikzcd},
  %\end{equation*}
  %where $\category{D}^{\natural}$ is the marked simplicial set whose underlying simplicial set is $\category{D}$ and where all equivalences are marked. This admits a dashed lift, and restricting $m|_{K \times \Delta^{\{1\}}}$ gives us a map $\tilde{\alpha}\colon K \to \category{D}$ as in \hyperref[eq:adjunct_filling_problems]{Equation~\ref*{eq:adjunct_filling_problems}}. However, we still need to display this as adjunct to $\alpha$: the lift $\ell$ has $\ell|_{K \times \Delta^{\{0\}}} = h_{0} \circ \alpha$, but only $\ell|_{K \times \Delta^{\{1\}}} \simeq h_{1} \circ \tilde{\alpha}$, not equality as we require.

  %Our last task is to remedy this. Denote the spine of the $3$-simplex by
  %\begin{equation*}
  %  I_{3} := \Delta^{\{0, 1\}} \amalg_{\Delta^{\{1\}}} \Delta^{\{1,2\}} \amalg_{\Delta^{\{2\}}} \Delta^{\{2,3\}} \hookrightarrow \Delta^{3}.
  %\end{equation*}
  %We build a map $I_{3} \times K \overset{b}{\to} \category{M}$ which:
  %\begin{itemize}
  %  \item On $\Delta^{\{0, 1\}} \times K$ is given by the lift $\ell$.

  %  \item On $\Delta^{\{1, 2\}} \times K$ is given by the composition
  %    \begin{equation*}
  %      K \times \Delta^{1} \overset{\ell|_{K \times \Delta^{\{1\}}} \times \id}{\to} \category{M}_{1} \times \Delta^{1} \overset{H}{\to} \category{M}_{1} \hookrightarrow \category{M},
  %    \end{equation*}
  %    where $H$ is a natural isomorphism $\id_{\category{M}_{1}} \to h_{1} \circ h_{1}^{-1}$.

  %  \item On $\Delta^{\{2, 3\}} \times K$ is given by the natural isomorphism
  %    \begin{equation*}
  %      K \times \Delta^{1} \overset{m}{\to} \category{D} \overset{h_{1}}{\to} \category{M}_{1} \hookrightarrow \category{M}.
  %    \end{equation*}
  %\end{itemize}
  %The map $I_{3} \hookrightarrow \Delta^{3}$ is inner anodyne, so there exists a map $\Delta^{3} \times K \to \category{M}$ whose restriction to $I_{3} \times K$ is equal to $b$. The restriction of this filling to $\Delta^{\{0,3\}} \times K$ satisfies the conditions of \hyperref[def:adjunct_data]{Definition~\ref*{def:adjunct_data}} (since the composition of a (co)cartesian morphism with an equivalence is again (co)cartesian), exhibiting $\tilde{\alpha}$ as adjunct to $\alpha$.
\end{proof}

\begin{example}
  Let $f : \category{C} \leftrightarrow \category{D} : g$ be an adjunction of $1$-categories, giving an adjunction between quasicategories upon taking nerves. Consider the diagram
  \begin{equation*}
    \begin{tikzcd}[column sep=small]
      K' = \Delta^{\{0,1,3\}} \cup \Delta^{\{0,2,3\}}
      \arrow[dr]
      \arrow[rr, hook, "i"]
      && \Delta^{3} = K
      \arrow[dl]
      \\
      & \Delta^{1}
    \end{tikzcd},
  \end{equation*}
  where both downwards-facing arrows take $0$, $1 \mapsto 0$ and $2$, $3 \mapsto 1$, and fix a map $\tilde{\alpha}\colon K \to N(\category{C})$ and a map $\alpha'\colon K' \to N(\category{D})$ such that $\tilde{\alpha} \circ i$ is adjunct to $\alpha'$. This corresponds to the diagrams
  \begin{equation*}
    \overbrace{
      \begin{tikzcd}[ampersand replacement=\&]
        C
        \arrow[r]
        \arrow[d]
        \& gD
        \arrow[d]
        \\
        C'
        \arrow[r]
        \arrow[ur, dashed]
        \& gD'
      \end{tikzcd}
    }^{\text{in } \category{C}}
    \qquad\qquad
    \overbrace{\begin{tikzcd}[ampersand replacement=\&]
      fC
      \arrow[r]
      \arrow[d]
      \& D
      \arrow[d]
      \\
      fC'
      \arrow[r]
      \& D'
    \end{tikzcd}}^{\text{in }\category{D}}
  \end{equation*}
  where the solid diagrams in each category are adjunct to one another.
  \hyperref[prop:can_transport_adjunct_data]{Proposition~\ref*{prop:can_transport_adjunct_data}} implies that the solution to the lifting problem on the left yields one on the right. Similarly, its dual implies the converse. Thus, \hyperref[prop:can_transport_adjunct_data]{Proposition~\ref*{prop:can_transport_adjunct_data}} implies in particular that such lifting problems are equivalent.
\end{example}


%\begin{proposition}
%  Let $s\colon K \to \Delta^{1}$, and let $\alpha\colon K \to \category{C}$ and $\tilde{\alpha}\colon K \to \category{D}$ be adjunct diagrams. Then the squares
%  \begin{equation*}
%    \begin{tikzcd}
%      K_{0}
%      \arrow[r, "\alpha|_{0}"]
%      \arrow[d, hook]
%      & \category{C}
%      \arrow[d, "f"]
%      \\
%      K
%      \arrow[r, "\tilde{\alpha}"]
%      & \category{D}
%    \end{tikzcd}
%    \qquad\text{and}\qquad
%    \begin{tikzcd}
%      K_{1}
%      \arrow[r, "\tilde{\alpha}|_{1}"]
%      \arrow[d, hook]
%      & \category{D}
%      \arrow[d, "g"]
%      \\
%      K
%      \arrow[r, "\alpha"]
%      & \category{C}
%    \end{tikzcd}
%  \end{equation*}
%  are commutative.
%\end{proposition}
%\begin{proof}
%  One sees equally easily that 2.\ implies 3.
%\end{proof}

We note that in the proof of \hyperref[prop:can_transport_adjunct_data]{Proposition~\ref*{prop:can_transport_adjunct_data}}, we did not need to use the full power of the statement that $f$ was left adjoint to $g$ (i.e.\ that $p\colon \category{M} \to \Delta^{1}$ was a bicartesian fibration), only that $f$ was \emph{locally} left adjoint to $g$, (i.e.\ that $p$ admitted cocartesian lifts at certain objects). This allows us to generalize \hyperref[prop:can_transport_adjunct_data]{Proposition~\ref*{prop:can_transport_adjunct_data}} somewhat.

\begin{definition}
  \label{def:partial_adjunction}
  Let $g\colon \category{D} \to \category{C}$ be a functor between quasicategories, and pick some cartesian fibration $p\colon \category{M} \to \Delta^{1}$ which classifies it. We assume without loss of generality that $h_{0}\colon \category{M}_{0} \cong \category{C}$ and  $h_{1}\colon \category{M}_{1} \cong \category{D}$ are isomorphisms, and we will notationally suppress them.

  We say that $g$ \defn{admits a left adjoint at $c \in \category{C}$} if $p$ admits a cocartesian lift of the morphism $\id_{\Delta^{1}}$ with source $c \in \category{M}_{0}$.
\end{definition}

\begin{definition}
  \label{def:adjunct_local}
  Let $s\colon K \to \Delta^{1}$ be a map of simplicial sets whose fibers we denote by $K_{0}$ and $K_{1}$, and let $p\colon \category{M} \to \Delta^{1}$ be a cartesian fibration associated to $f\colon \category{D} \to \category{C}$ via a map $v\colon \category{D} \times \Delta^{1} \to \category{M}$. We say that a map $\alpha\colon K \to \category{C}$ is \defn{adjunct} to a map $\tilde{\alpha}\colon K \to \category{D}$ relative to $s$, and equivalently that $\tilde{\alpha}$ is adjunct to $\alpha$ relative to $s$, if there exists a map $A\colon K \times \Delta^{1} \to \category{M}$ such that the diagram
  \begin{equation*}
    \begin{tikzcd}
      K \times \Delta^{1}
      \arrow[r, "A"]
      \arrow[dr, swap, "\pr_{\Delta^{1}}"]
      & \category{M}
      \arrow[d, "p"]
      \\
      & \Delta^{1}
    \end{tikzcd}
  \end{equation*}
  commutes, and such that the following conditions are satisfied.
  \begin{enumerate}
    \item The restriction $A|_{K \times \{0\}} = \alpha$.

    \item The restriction $A|_{K \times \{1\}} = \tilde{\alpha}$.

    \item For each vertex $k \in K_{0}$, the image of $\{k\} \times \Delta^{1}$ under $A$ in $\category{M}$ is $p$-cartesian.

    \item The restriction $A|K_{1} \times \Delta^{1}$ is equal to the composition
      \begin{equation*}
        K_{1} \times \Delta^{1} \hookrightarrow K \times \Delta^{1} \overset{\tilde{\alpha} \times \id}{\to} \category{D} \times \Delta^{1} \overset{v}{\to} \mathcal{M}.
      \end{equation*}
  \end{enumerate}
\end{definition}

\begin{example}
  \label{eg:partial_adjunctions_and_kan_extensions}
  Suppose $f\colon \category{C} \to \category{C}'$ and $F\colon \category{C} \to \category{D}$ are functors. The statement that the left Kan extension $f_{!}F\colon \category{C'} \to \category{D}$ exists is equivalent to the statement that $f^{*}$ admits a left adjoint at $F$.
\end{example}

The proof of \hyperref[prop:can_transport_adjunct_data]{Proposition~\ref*{prop:can_transport_adjunct_data}} can be used as is to show the following statement.

\begin{proposition}
  \label{prop:can_transport_adjunct_data_partial}
  Let $p\colon \category{M} \to \Delta^{1}$ classify $g\colon \category{D} \to \category{C}$ via $v\colon \category{D} \times\Delta^{1} \to \category{M}$, and let
  \begin{equation*}
    \begin{tikzcd}[column sep=small]
      K'
      \arrow[rr, hook, "i"]
      \arrow[dr, swap, "s \circ i"]
      && K
      \arrow[dl, "s"]
      \\
      & \Delta^{1}
    \end{tikzcd},
  \end{equation*}
  be a commuting triangle of simplicial sets, where $i$ is a monomorphim such that $i|_{\{1\}}$ is an isomorphism. Let $\tilde{\alpha}'\colon K' \to \category{D}$ and $\alpha\colon K \to \category{C}$ be maps, and denote $\alpha' = \alpha \circ i$. Suppose that $\tilde{\alpha}'$ is adjunct to $\alpha'$ relative to $s \circ i$. Further suppose that $g$ admits a left adjoint at all vertices belonging to the image $\alpha(K_{0}) \subseteq \category{C}$.

  Then there exists a dashed extension
  \begin{equation}
    \label{eq:adjunct_filling_problems_partial}
    \begin{tikzcd}
      K'
      \arrow[r, "\alpha'"]
      \arrow[d, swap, hook, "i"]
      & \category{C}
      \\
      K
      \arrow[ur, swap, "\alpha"]
    \end{tikzcd}
    \quad \rightsquigarrow \quad
    \begin{tikzcd}
      K'
      \arrow[r, "\tilde{\alpha}'"]
      \arrow[d, swap, hook, "i"]
      & \category{D}
      \\
      K
      \arrow[ur, dashed, swap, "\tilde{\alpha}"]
    \end{tikzcd}
  \end{equation}
  such that $\tilde{\alpha}$ is adjunct to $\alpha$ relative to $s$. Furthermore, any two such lifts are equivalent as functors $K \to \category{D}$.
\end{proposition}


\subsection{Left Kan implies globally left Kan}
\label{ssc:left_kan_implies_globally_left_kan}

In this subsection, we will prove \hyperref[thm:left_kan_implies_globally_left_kan]{Theorem~\ref*{thm:left_kan_implies_globally_left_kan}}. In order to do that, we must show that we can solve lifting problems of the form
\begin{equation*}
  \label{eq:specific_left_horns_in_cat_infty_bicategories}
  \begin{tikzcd}
    \Delta^{\{0, 1, n+1\}}_{\flat}
    \arrow[d, hook]
    \arrow[dr, "\tau"]
    \\
    (\Lambda^{n+1}_{0})_{\flat}
    \arrow[r]
    \arrow[d, hook]
    & \ICCat
    \\
    \Delta^{n+1}_{\flat}
    \arrow[ur, dashed]
  \end{tikzcd},
\end{equation*}
for $n \geq 2$, where $\tau$ is left Kan. Forgetting for a moment the upper triangle singling out the $2$-simplex $\tau$, and using the adjunction $\Csc \dashv \Nsc$, we will do this by solving equivalent lifting problems of the form\footnote{Note that we are implicitly taking $\C[\Lambda^{n+1}_{0}]$ and $\C[\Delta^{n+1}]$ to carry the flat markings on their mapping spaces.}
\begin{equation}
  \label{eq:left_horns_in_cat_infty_scaled}
  \begin{tikzcd}
    \C[\Lambda^{n+1}_{0}]
    \arrow[r]
    \arrow[d, hook]
    & \QCat
    \\
    \C[\Delta^{n+1}]
    \arrow[ur, dashed, swap, "\mathcal{F}"]
  \end{tikzcd}.
\end{equation}
We expect that the reader is familiar with the basics of (scaled) rigidification, so we give only a rough description of this lifting problem here. The objects of the simplicial category $\C[\Delta^{n+1}]$ are given by the set $\{0, \ldots, n+1\}$, and the mapping spaces are defined by
\begin{equation*}
  \C[\Delta^{n+1}](i, j) =
  \begin{cases}
    N(P_{ij}), &i \leq j \\
    \emptyset, & i > j
  \end{cases}.
\end{equation*}

where $P_{ij}$ is the poset of subsets of the linearly ordered set $\{i, \ldots, j\}$ containing $i$ and $j$, ordered by inclusion. The simplicial category $\C[\Lambda^{n+1}_{0}]$ has the same objects as $\C[\Delta^{n+1}]$, and each morphism space $\C[\Lambda^{n+1}_{0}](i, j)$ is a simplicial subset of $\C[\Delta^{n+1}](i, j)$, as we shall soon describe.

The map $\C[\Lambda^{n+1}_{0}](i, j) \hookrightarrow \C[\Delta^{n+1}](i, j)$ is an isomorphism except for $(i, j) = (1, n+1)$ and $(i, j) = (0, n+1)$; in these cases, it is an inclusion. The missing data corresponds in the case of $(1, n+1)$ to the missing face $d_{0}\Delta^{n+1}$ of $\Lambda^{n+1}_{0}$, and in the case of $(0, n+1)$ to the missing interior. Thus, to find a filling as in \hyperref[eq:left_horns_in_cat_infty_scaled]{Equation~\ref*{eq:left_horns_in_cat_infty_scaled}}, we need to solve the lifting problems
\begin{equation}
  \label{eq:lifting_problem_1}
  \begin{tikzcd}
    \C[\Lambda^{n+1}_{0}](1, n+1)
    \arrow[r]
    \arrow[d, hook]
    & \Fun(\mathcal{F}(1), \mathcal{F}(n+1))
    \\
    \C[\Delta^{n+1}](1, n+1)
    \arrow[ur, dashed, swap, "\ell'"]
  \end{tikzcd}
\end{equation}
and
\begin{equation}
  \label{eq:lifting_problem_2}
  \begin{tikzcd}
    \C[\Lambda^{n+1}_{0}](0, n+1)
    \arrow[r]
    \arrow[d, hook]
    & \Fun(\mathcal{F}(0), \mathcal{F}(n+1))
    \\
    \C[\Delta^{n+1}](0, n+1)
    \arrow[ur, dashed, swap, "\ell"]
  \end{tikzcd}.
\end{equation}
However, these problems are not independent; the filling $\ell$ of the full simplex needs to agree with the filling $\ell'$ we found for the missing face of the horn, corresponding to the condition that the square
\begin{equation}
  \label{eq:lifting_problem_compatibility_condition}
  \begin{tikzcd}
    \C[\Delta^{n+1}](1, n+1)
    \arrow[r, "\ell'"]
    \arrow[d, swap, "{\{0,1\}}^{*}"]
    & \Fun(\mathcal{F}(1), \mathcal{F}(n+1))
    \arrow[d, "\mathcal{F}({\{0, 1\}})^{*}"]
    \\
    \C[\Delta^{n+1}](0, n+1)
    \arrow[r, "\ell"]
    & \Fun(\mathcal{F}(0), \mathcal{F}(n+1))
  \end{tikzcd}
\end{equation}
commute.

\begin{notation}
  \label{notation:simplicial_lifting}
  Recall our desired filling $\mathcal{F}\colon \C[\Delta^{n+1}] \to \QCat$ of \hyperref[eq:left_horns_in_cat_infty_scaled]{Equation~\ref*{eq:left_horns_in_cat_infty_scaled}}. We will denote $\mathcal{F}(i)$ by $X_{i}$, and for each subset $S = \{i_{1}, \ldots, i_{k}\} \subseteq [n]$, we will denote $\category{F}(S) \in \Map(X_{i_{1}}, X_{i_{k}})$ by $f_{i_{k}\cdots i_{1}}$. For any inclusion $S' \subseteq S \subseteq [n]$ preserving minimim and maximum elements, we will denote the corresponding morphism by $\alpha^{S'}_{S}$.
\end{notation}

\begin{theorem}
  For any $n \geq 2$ and any globally left Kan $2$-simplex $\tau$, the solid extension problem
  \begin{equation*}
    \begin{tikzcd}
      \Delta^{\{0, 1, n+1\}}_{\flat}
      \arrow[d, hook]
      \arrow[dr, "\tau"]
      \\
      (\Lambda^{n+1}_{0})_{\flat}
      \arrow[r]
      \arrow[d, hook]
      & \ICCat
      \\
      \Delta^{n+1}_{\flat}
      \arrow[ur, dashed]
    \end{tikzcd}
  \end{equation*}
  admits a dashed filler.
\end{theorem}

\begin{note}
  The author would like to offer the friendly recommendation that in reading the proof below, it is helpful to follow along with the explanation given at the beginning of \hyperref[sec:horn_filling_via_left_kan_extensions]{Section~\ref*{sec:horn_filling_via_left_kan_extensions}}.
\end{note}

\begin{proof}
  We need to solve the lifting problems of \hyperref[eq:lifting_problem_1]{Equation~\ref*{eq:lifting_problem_1}} and \hyperref[eq:lifting_problem_2]{Equation~\ref*{eq:lifting_problem_2}}, and check that our solutions satisfy the condition of \hyperref[eq:lifting_problem_compatibility_condition]{Equation~\ref*{eq:lifting_problem_compatibility_condition}}. In the discussuion surrounding these equations, we ignored that the simplex $\tau\colon \Delta^{\{0,1,n+1\}}_{\flat} \to \ICCat$ was left Kan. We now reintroduce this information. The information that the simplex $\tau\colon \Delta^{\{0,1,n+1\}}_{\flat} \to \ICCat$ is left Kan tells us the following.
  \begin{itemize}
    \item  Concretely, it tells us that the map $f_{n,1}$ is the left Kan extension of $f_{n, 0}$ along $f_{1, 0}$, and that $\alpha_{\{0,n+1\}}^{\{0,1,n+1\}}\colon f_{n+1,0} \to f_{n+1, 1} \circ f_{1, 0} $ is the unit map.

    \item In particular, \hyperref[eg:partial_adjunctions_and_kan_extensions]{Example~\ref*{eg:partial_adjunctions_and_kan_extensions}} then tells us that the map $f_{1,0}^{*}$ admits a left adjoint at $f_{n+1,0}$.
  \end{itemize}

  The lifting problems as described above are somewhat unwieldy, given in terms of mapping spaces of simplicial categories $\C[\Delta^{n+1}]$ and $\C[\Lambda^{n+1}_{0}]$. It will be useful to give these mapping spaces more down-to-earth descriptions, in terms of simplicial sets with which we are familiar. We have the following succinct descriptions of the inclusions of \hyperref[eq:lifting_problem_1]{Equation~\ref*{eq:lifting_problem_1}} and \hyperref[eq:lifting_problem_2]{Equation~\ref*{eq:lifting_problem_2}} along which we have to extend.
  \begin{itemize}
    \item There is an isomorphism of simplicial sets
      \begin{equation*}
        \C[\Delta^{n+1}](0, n+1) \overset{\cong}{\to} C^{n}
      \end{equation*}
      specified completely by sending a subset $S \subseteq [n+1]$ to the point
      \begin{equation*}
        (z_{1}, \ldots, z_{n}) \in C^{n},\qquad z_{i} =
        \begin{cases}
          1, & i \in S \\
          0, &\text{otherwise.}
        \end{cases}
      \end{equation*}

      The inclusion $\C[\Lambda^{n+1}_{0}](0, n+1) \hookrightarrow \C[\Delta^{n+1}](0, n+1)$ corresponds to the simplicial subset $LC^{n} \hookrightarrow C^{n}$.

    \item There is a similar isomorphism of simplicial sets
      \begin{equation*}
        \C[\Delta^{n+1}](1, n+1) \overset{\cong}{\to} C^{n-1}
      \end{equation*}
      specified completely by sending $S \subseteq \{1, \ldots, n+1\}$ to the point
      \begin{equation*}
        (z_{1}, \ldots, z_{n-1}) \in C^{n-1},\qquad z_{i} =
        \begin{cases}
          1, & i+1 \in S \\
          0, &\text{otherwise.}
        \end{cases}
      \end{equation*}
      The inclusion $\C[\Lambda^{n+1}_{0}](1, n+1) \hookrightarrow \C[\Delta^{n+1}](1, n+1)$ corresponds to the inclusion $\partial C^{n-1} \hookrightarrow C^{n-1}$.
  \end{itemize}

  Using these descriptions, we can write our lifting problems in the more inviting form
  \begin{equation*}
    (*) =
    \begin{tikzcd}
      LC^{n}
      \arrow[r]
      \arrow[d, hook]
      & \Fun(X_{0}, X_{n+1})
      \\
      C^{n}
      \arrow[ur, dashed]
    \end{tikzcd},\qquad (**) =
    \begin{tikzcd}
      \partial C^{n-1}
      \arrow[r, "\tilde{\alpha}'"]
      \arrow[d, hook]
      & \Fun(X_{1}, X_{n+1})
      \\
      C^{n-1}
      \arrow[ur, dashed]
    \end{tikzcd},
  \end{equation*}
  and our condition becomes that the square
  \begin{equation*}
    (\star) =
    \begin{tikzcd}
      C^{n-1}
      \arrow[r]
      \arrow[d, hook]
      & \Fun(X_{1}, X_{n+1})
      \arrow[d, "f_{1,0}^{*}"]
      \\
      C^{n}
      \arrow[r]
      & \Fun(X_{0}, X_{n+1})
    \end{tikzcd}
  \end{equation*}
  commutes, where the left-hand vertical morphism is the inclusion of the right face.

  Using \hyperref[prop:cube_filling]{Proposition~\ref*{prop:cube_filling}}, we can partially solve the lifting problem $(*)$, reducing it to the lifting problem
  \begin{equation*}
    (*') =
    \begin{tikzcd}
      \Delta^{0} \ast \partial C^{n-1}
      \arrow[r, "\alpha"]
      \arrow[d, hook]
      & \Fun(X_{0}, X_{n+1})
      \\
      \Delta^{0} \ast C^{n-1}
      \arrow[ur, dashed]
    \end{tikzcd}.
  \end{equation*}
  The image of the 1-simplex $(0, \ldots, 0) \to (0, \ldots, 1)$ of $\Delta^{0} \ast \partial C^{n-1} \subseteq C^{n}$ under $\alpha$ is the unit map $\alpha^{\{0, 1, n+1\}}_{\{0, n+1\}}\colon f_{n+1, 0} \to f_{n+1, 1} \circ f_{1, 0}$. This is not in general an equivalence, so we cannot use \hyperref[prop:cube_filling]{Proposition~\ref*{prop:cube_filling}} to solve the lifting problem $(*')$ directly. However, the unit map is adjunct to an equivalence in $\Fun(X_{1}, X_{n+1})$ relative to $s = \id_{\Delta^{1}}$. Furthermore, the restriction of $\alpha$ to $\partial C^{n-1}$ is in the image of $f^{*}_{1,0}$, so by \hyperref[prop:can_transport_adjunct_data_partial]{Proposition~\ref*{prop:can_transport_adjunct_data_partial}} and our assumption that $f_{10}^{*}$ admits a left adjoint at $f_{n+1,0}$, we can augment the map $\tilde{\alpha}'$ to a map $\tilde{\alpha}\colon \Delta^{0} \ast \partial C^{n-1} \to \Fun(X_{1}, X_{n-1})$ which is adjunct to $\alpha$ relative to the map
  \begin{equation*}
    s\colon \Delta^{0} \ast \partial C^{n-1} \to \Delta^{1}
  \end{equation*}
  sending $\Delta^{0}$ to $\Delta^{\{0\}}$ and $\partial C^{n-1}$ to $\Delta^{\{1\}}$; in particular, the image of the morphism $(0, \ldots, 0) \to (0, \ldots 1)$ under $\tilde{\alpha}'$ is an equivalence. This allows us to replace the lifting problem $(**)$ by the superficially more complicated lifting problem
  \begin{equation*}
    (**') =
    \begin{tikzcd}
      \Delta^{0} \ast \partial C^{n-1}
      \arrow[r, "\tilde{\alpha}"]
      \arrow[d, hook]
      & \Fun(X_{1}, X_{n+1})
      \\
      \Delta^{0} \ast C^{n-1}
      \arrow[ur, dashed, swap, "\tilde{\beta}"]
    \end{tikzcd}
  \end{equation*}
  However, since image of $(0, \ldots 0) \to (0, \ldots, 1)$ is an equivalence, \hyperref[prop:cube_filling]{Proposition~\ref*{prop:cube_filling}} implies that we can solve the lifting problem $(**')$. Again using (the dual to) \hyperref[prop:can_transport_adjunct_data_partial]{Proposition~\ref*{prop:can_transport_adjunct_data_partial}}, we can transport this filling to a solution to the lifting problem $(*')$. The condition $(\star)$ amounts to demanding that the restriction $\beta|_{C^{n-1}}$ be the image of $\tilde{\beta}|_{C^{n-1}}$ under $f_{1,0}^{*}$, which is true by construction.
\end{proof}

\subsection{A few other tricks with Kan extensions}
\label{ssc:a_few_other_tricks_with_kan_extensions}

We end this section with a few miscellaneous tricks which we can play with maps $\Delta^{n} \to \ICCat$ involving left Kan simplices.

\begin{lemma}
  \label{lemma:transport_left_kan_simplices}
  Denote by $\mathcal{E} \subseteq \Hom(\Delta^{2}, \Delta^{3})$ the subset containing all degenerate $2$-simplices together with the simplices $\Delta^{\{0,2,3\}}$ and $\Delta^{\{1,2,3\}}$. Let $\sigma\colon \Delta^{3}_{\mathcal{E}} \to \ICCat$ such that $\Delta^{\{2,3\}}$ is mapped to an equivalence. Then $\sigma|\Delta^{\{0,1,2\}}$ is left Kan if and only if $\sigma|\Delta^{\{0,1,3\}}$ is left Kan.
\end{lemma}
\begin{proof}
  Replacing thin simplices by strict compositions using \hyperref[prop:homotopy_invariance_of_witness]{Proposition~\ref*{prop:homotopy_invariance_of_witness}}, this statement reduces to that of \hyperref[prop:kan_ext_invariance_of_target]{Proposition~\ref*{prop:kan_ext_invariance_of_target}}.
\end{proof}

The next lemma is similar but easier.
\begin{lemma}
  \label{lemma:transport_thin_simplices}
  Denote by $\mathcal{E} \subseteq \Hom(\Delta^{2}, \Delta^{3})$ the subset containing all degenerate $2$-simplices together with the simplices $\Delta^{\{0,2,3\}}$ and $\Delta^{\{1,2,3\}}$. Let $\sigma\colon \Delta^{3}_{\mathcal{E}} \to \ICCat$ such that $\Delta^{\{2,3\}}$ is mapped to an equivalence. Then $\sigma|\Delta^{\{0,1,2\}}$ is thin if and only if $\sigma|\Delta^{\{0,1,3\}}$ is thin.
\end{lemma}

\begin{lemma}
  \label{lemma:compose_left_kan_simplices}
  Denote by $\mathcal{E}' \subseteq \Hom(\Delta^{2}, \Delta^{3})$ the subset containing all degenerate $2$-simplices, together with the simplex $\Delta^{\{0,1,2\}}$. Let $\sigma\colon \Delta^{3}_{\mathcal{F}} \to \ICCat$ such that $\sigma|\Delta^{\{0,1,3\}}$ is left Kan. Then $\sigma|\Delta^{\{0,2,3\}}$ is left Kan if and only if $\sigma|\Delta^{\{1,2,3\}}$ is left Kan.
\end{lemma}
\begin{proof}
  Replacing thin simplices by strict compositions using \hyperref[prop:homotopy_invariance_of_witness]{Proposition~\ref*{prop:homotopy_invariance_of_witness}}, this reduces to \hyperref[prop:kan_extend_along_composition]{Propsition~\ref*{prop:kan_extend_along_composition}}.
\end{proof}


\end{document}
