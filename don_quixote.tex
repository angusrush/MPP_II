\documentclass[main.tex]{subfiles}

\begin{document}

\section{Tilting at cosimplicial objects}
\label{sec:tilting_at_cosimplicial_objects}

\textbf{Future Angus:} This definitely isn't going to work. We're throwing away any semblance of coherence. This is just the twisted arrow category of $\QCat$ as a 2-category.

Let's consider $\Tw(\ICat)$. The $n$-simplices are given by maps
\begin{align*}
  \Hom_{\SSet}(\Delta^{n}, \Tw(\ICat)) &\simeq \Hom_{\SSetmk}(Q(n), \ICat) \\
  &\simeq \Hom_{\SCatmk}(\Csc Q(n), \QCat),
\end{align*}
so we can think of $\Tw(\ICat)$ as coming from the cosimplicial object $\Csc Q\colon \Delta \to \SCatmk$, with the following description:
\begin{itemize}
  \item The objects of $\Csc Q(n)$ are the objects of the linearly ordered set
    \begin{equation*}
      S_{n} = [n] \star [\bar{n}] = \{0 < 1 < \cdots < n < \bar{n} < \cdots < \bar{0}\}.
    \end{equation*}
    We will denote a generic element of $S_{n}$ by $a$, $b$, etc. We will denote an element of $[n] \subset S_{n}$ by $i$, and a generic element of $[\bar{n}] \subset S_{n}$ by $\bar{i}$. That is, for each $a \in S_{n}$ we always have either $a = i$ or $a = \bar{i}$ for some $0 \leq i \leq n$. We refer to $i$ as the \emph{digit} of $a$.

  \item For any $a$, $b \in S_{n}$, the underlying simplicial set of $\Csc Q(n)(a, b)$ is the (nerve of the) poset of subsets of $\{a < \cdots < b\}$ containing $a$ and $b$.

  \item A morphism $A \subseteq B \subseteq \{a < \cdots < b\}$ in such a mapping space is marked if and only if the highest digit between 0 and $n$ appearing in both $A$ and $B$ is the same. For instance, for mapping spaces between objects $i$, $j \in [n]$, the situation corresponding to $A$ and $B$ both belonging to $[n] \subseteq S_{n}$, everything is marked because by definition both $A$ and $B$ must contain $i$ (and similarly if both belong to $[\bar{n}]$).
\end{itemize}

We then define a cosimplicial object $R\colon \Delta \to \SCatmk$ as follows.
\begin{itemize}
  \item The objects of $R(n)$ are the objects of the linearly ordered set
    \begin{equation*}
      S_{n} = \{0 < 1 < \cdots < n < \bar{n} < \cdots < \bar{0}\}.
    \end{equation*}

  \item The mapping space $R(n)(a, b)$ is given by the (nerve of the) linearly ordered set
    \begin{equation*}
      \{\max(a, b) < \cdots < n\},
    \end{equation*}
    where by $\max(a, b)$, we mean the maximum digit appearing. In particular: 
    \begin{itemize}
      \item For $b < a$, the mapping simplicial sets are empty. To avoid having to repeatedly mention that this case is trivial, we implicitly consider only mapping spaces $\Map(a, b)$ with $a \leq b$.

      \item For $a = i$ and $b = j$, 
    \end{itemize}

  \item Let $\phi\colon [m] \to [n]$. We define a map $R(m) \to R(n)$ as follows.
    \begin{itemize}
      \item On objects, we send $i \mapsto \phi(i)$ and $\bar{i} \mapsto \overline{\phi(i)}$. We write both of these $a \mapsto \phi(a)$.

      \item On morphisms, we define a map
        \begin{equation*}
          R(m)(a, b) \to R(n)(\phi(a), \phi(b))
        \end{equation*}
        sending $i \in \{\max(a, b) < \cdots < m\}$ to $\phi(i) \in \{\max(\phi(a), \phi(b)) < \cdots < n\}$.
    \end{itemize}
\end{itemize}

For each $n$, we get a map
\begin{equation*}
  \iota_{n}\colon R(n) \to \Csc Q(n)
\end{equation*}
which is the identity on objects, and acts on morphisms in the following way:
\begin{itemize}
  \item For $a = i$ and $b = j$, $c$ is given by some digit $k$ with $i \leq k \leq j$, and we send
\end{itemize}

\end{document}
