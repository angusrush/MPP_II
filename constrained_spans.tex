\documentclass[main.tex]{subfiles}

\begin{document}

\section{Segal spaces of spans}
\label{sec:segal_spaces_of_spans}

\subsection{Basic definitions}

We recall the basic definitions of Segal spaces of spans. Note that spans as we will define them form not only a Segal space, but a \emph{complete} Segal space; for the most part, this will not concern us. For more information, we direct the reader to \cite{spectralmackeyfunctors1}. This section is intended to be a summary of the relevant results of loc.\ cit.

The objects of our study will be $\infty$-categories whose morphisms are spans in some quasicategory $\category{C}$, i.e.\ diagrams in $\category{C}$ of the form
\begin{equation*}
  \begin{tikzcd}
    & y
    \arrow[dl, swap, "\phi"]
    \arrow[dr, "\psi"]
    \\
    x
    && x'
  \end{tikzcd}.
\end{equation*}
We will want to be able to place certain conditions on the legs $\phi$ and $\psi$ of our spans. For example, we may want to restrict our attention to spans such that $\phi$ is an equivalence. More precisely, we pick out two subcategories of $\category{C}$ to which the respective legs of our spans must belong.

\begin{definition}
  A \defn{triple} of categories is a triple $\triple{C}$, where $\category{C}$ is a quasicategory where $\category{C}_{\dagger}$ and $\category{C}^{\dagger}$ are subcategories, each of which contain all equivalences.
\end{definition}

We will use the following terminology and notation for the morphsism in our subcategories.
\begin{itemize}
  \item We denote the morphisms in $\category{C}_{\dagger}$ with tails (as in $x \ing y$), and call them \emph{ingressive}.

  \item We denote the morphisms in $\category{C}^{\dagger}$ with two heads (as in $x \eg y$), and call them \emph{egressive}.
\end{itemize}

The egressive morphisms will correspond to the backwards (i.e.\ leftwards) facing legs of our spans, and the ingressive morphisms will correspond to the forwards (i.e.\ rightwards) facing legs. We provide the following diagram as a summary of this terminology.
\begin{equation*}
  \begin{tikzcd}
    & Y
    \arrow[dl, two heads, swap, "\substack{\text{egressive} \\ \category{C}\updag}"]
    \arrow[dr, rightarrowtail, "\substack{\text{ingressive} \\ \category{C}\downdag}"]
    \\
    X
    && X'
  \end{tikzcd}
\end{equation*}

In order for a triple of categories to be able to support a category of spans, it will have to satisfy certain properties. Following Barwick, we will call such triples \emph{adequate.}

\begin{definition}
  \label{def:adequate_triple}
  A triple $(\category{C}, \category{C}_{\dagger}, \category{C}^{\dagger})$ is said to be \defn{adequate} if it has the following properties:
  \begin{enumerate}
    \item For any ingressive morphism $f \in \category{C}_{\dagger}$ and any egressive morphism $g \in \category{C}^{\dagger}$, there exists a pullback square
      \begin{equation*}
        \begin{tikzcd}
          y'
          \arrow[r]
          \arrow[d]
          & x'
          \arrow[d, two heads, "g"]
          \\
          y
          \arrow[r, rightarrowtail, "f"]
          & x
        \end{tikzcd}.
      \end{equation*}

    \item For any pullback square
      \begin{equation*}
        \begin{tikzcd}
          y'
          \arrow[r, "f'"]
          \arrow[d]
          & x'
          \arrow[d]
          \\
          y
          \arrow[r, "f"]
          & x
        \end{tikzcd},
      \end{equation*}
      if the arrow $f$ belongs to $\category{C}_{\dagger}$ (resp.\ $\category{C}^{\dagger}$), then the arrow $f'$ belongs to $\category{C}_{\dagger}$ (resp.\ $\category{C}^{\dagger}$).
  \end{enumerate}
\end{definition}

We will call a square of the form
\begin{equation*}
  \begin{tikzcd}
    y'
    \arrow[r, rightarrowtail]
    \arrow[d, two heads]
    & x'
    \arrow[d, two heads]
    \\
    y
    \arrow[r, rightarrowtail]
    & x
  \end{tikzcd}
\end{equation*}
\emph{ambigressive.} If such an ambigressive square is also a pullback square, we will call it \emph{ambigressive pullback.}

It is now time to set about building our $\infty$-categories of spans. Fix some triple $\triple{C}$. We will define two models for such a category: a Segal space $\SPAN\triple{C}$, and the quasicategory $\Span\triple{C}$.

Given two $1$-simplices in $\Span\triple{C}$ represented by spans $X \twoheadleftarrow Y \rightarrowtail X'$ and $X' \twoheadleftarrow Y' \rightarrowtail X''$ in $\category{C}$, we need to specify what it means to compose them. We define the composition to be the span $X \twoheadleftarrow Y \times_{X'} Y' \rightarrowtail X''$ given by the diagram below, where the top square is pullback.
\begin{equation*}
  \begin{tikzcd}
    && Y \times_{X'} Y'
    \arrow[dl, two heads, dashed, swap, "q'"]
    \arrow[dr, tail, dashed, "f'"]
    \\
    & Y
    \arrow[dl, two heads, swap, "g"]
    \arrow[dr, tail, "f"]
    && Y'
    \arrow[dl, two heads, swap, "q"]
    \arrow[dr, tail, "p"]
    \\
    X
    && X'
    && X''
  \end{tikzcd}
\end{equation*}
Note that the conditions in \hyperref[def:adequate_triple]{Definition~\ref*{def:adequate_triple}} tell us precisely that such a composition always exists.

Equivalently, we this means that a $2$-simplex in $\Span\triple{C}$ should be a diagram in $\category{C}$ of the above form. More generally, we would like an $n$-simplex in $\Span\triple{C}$ to correspond to $n$-fold composition of spans. For example, for $n = 4$, such an $n$-simplex in $\Span\triple{C}$ should consist of a diagram in $\category{C}$ of the form
\begin{equation*}
  \begin{tikzcd}[column sep=tiny, row sep=small]
    \\
    &&&& X_{04}
    \arrow[dl, two heads]
    \arrow[dr, tail]
    \\
    &&& X_{03}
    \arrow[dl, two heads]
    \arrow[dr, tail]
    && X_{14}
    \arrow[dl, two heads]
    \arrow[dr, tail]
    \\
    && X_{02}
    \arrow[dl, two heads]
    \arrow[dr, tail]
    && X_{13}
    \arrow[dl, two heads]
    \arrow[dr, tail]
    && X_{24}
    \arrow[dl, two heads]
    \arrow[dr, tail]
    \\
    & X_{01}
    \arrow[dl, two heads]
    \arrow[dr, tail]
    && X_{12}
    \arrow[dl, two heads]
    \arrow[dr, tail]
    && X_{23}
    \arrow[dl, two heads]
    \arrow[dr, tail]
    && X_{34}
    \arrow[dl, two heads]
    \arrow[dr, tail]
    \\
    X_{00}
    && X_{11}
    && X_{22}
    && X_{33}
    && X_{44}
  \end{tikzcd},
\end{equation*}
where the morphisms are ingressive and egressive as shown, and each square is pullback.

All that remains is to make this definition rigorous. To this end, we make the following definition.

\begin{definition}
  for each $n \geq 0$, we define a poset $\Sigma_{n}$ as follows:
  \begin{itemize}
    \item The elements of $\Sigma_{n}$ are pairs of integers $(i, j)$, with $0 \leq i \leq j \leq n$.
  
    \item For $(i, j)$, $(i', j') \in \Sigma_{n}$, we define
      \begin{equation*}
        (i, j) \leq (i', j') \iff i \leq i' \leq j' \leq j.
      \end{equation*}
  \end{itemize}
\end{definition}

\begin{example}
  \label{eg:sigma_4}
  We can draw $\Sigma_{4}$ as follows.
  \begin{equation*}
    \begin{tikzcd}[column sep=tiny, row sep=small]
      \\
      &&&& {(0, 4)}
      \arrow[dl]
      \arrow[dr]
      \\
      &&& {(0, 3)}
      \arrow[dl]
      \arrow[dr]
      && {(1, 4)}
      \arrow[dl]
      \arrow[dr]
      \\
      && {(0, 2)}
      \arrow[dl]
      \arrow[dr]
      && {(1, 3)}
      \arrow[dl]
      \arrow[dr]
      && {(2, 4)}
      \arrow[dl]
      \arrow[dr]
      \\
      & {(0, 1)}
      \arrow[dl]
      \arrow[dr]
      && {(1, 2)}
      \arrow[dl]
      \arrow[dr]
      && {(2, 3)}
      \arrow[dl]
      \arrow[dr]
      && {(3, 4)}
      \arrow[dl]
      \arrow[dr]
      \\
      {(0, 0)}
      && {(1, 1)}
      && {(2, 2)}
      && {(3, 3)}
      && {(4, 4)}
    \end{tikzcd}
  \end{equation*}
\end{example}

These posets $\Sigma_{n}$ assemble into a cosimplicial object 
\begin{equation*}
  \Delta \to \SSet;\qquad [n] \mapsto N(\Sigma_{n}),
\end{equation*}
Left Kan extending along the Yoneda embedding $\Delta \hookrightarrow \SSet$ yields a functor $\sd\colon \SSet \to \SSet$. General abstract nonsense gives us right adjoint $\Span'\colon \SSet \to \SSet$ forming an adjunction
\begin{equation*}
  \sd: \SSet \longleftrightarrow \SSet : \Span'.
\end{equation*}
For any simplicial set $A$, the simplicial set $\Span'(A)$ has $n$-simplices
\begin{equation*}
  \Span'(A)_{n} = \Hom_{\SSet}(\sd(\Delta^{n}), A).
\end{equation*}

Note that for a quasicategory $\category{C}$, the simplicial set $\Span'(\category{C})$ is very close to what we want; its $n$-simplices are maps $\Sigma_{n} \to \category{C}$, but the backward- and forward-facing legs of the spans do not necessarily belong to the categories $\category{C}\downdag$ and $\category{C}\updag$, and the squares are not necessarily pullback. We will use the term to refer to a map $\sd(\Delta^{n}) \to \category{C}$ which has the desired form.

%For any $n$-simplex $\sd(\Delta^{n}) \to \category{C}$ corresponding to a diagram as above, we will call the morphisms $X_{ij} \to X_{ij'}$ \emph{backward-facing,} and the morphisms $X_{ij} \to X_{i'j}$ \emph{forward-facing.} 

\begin{definition}
  \label{def:ambigressive_cartesian}
  For any adequate triple $\triple{C}$, we will call a functor $\sd(\Delta^{n}) \to \category{C}$ \emph{ambigressive Cartesian} if each square in $\sd(\Delta^{n}) = N(\Sigma_{n})$ of the form
  \begin{equation*}
    \begin{tikzcd}[column sep=tiny]
      & {(i, j)}
      \arrow[dr]
      \arrow[dl]
      \\
      {(i, j - \ell)}
      \arrow[dr]
      & & {(i + k , j)}
      \arrow[dl]
      \\
      & {(i + k , j - \ell)}
    \end{tikzcd}
  \end{equation*}
  (where we include the possibilities $k = 0$ and $\ell = 0$) is mapped to an ambigressive pullback square,
  \begin{equation*}
    \begin{tikzcd}[column sep=tiny]
      & X_{ij}
      \arrow[dr]
      \arrow[dl]
      \\
      X_{i(j-\ell)}
      \arrow[dr]
      & & X_{(i+k)(j)}
      \arrow[dl]
      \\
      & X_{(i+k)(j-\ell)}
    \end{tikzcd}
  \end{equation*}
  where the backwards-facing morphsims are egressive and the forwards-facing morphisms are ingressive. 
\end{definition}

We are now ready to define our category $\Span(\category{C})$.

\begin{definition}
  We define a simplicial set $\Span\triple{C}$ level-wise to be the subset
  \begin{equation*}
    \Span\triple{C}_{n} \subseteq \Span'(\category{C})_{n}
  \end{equation*}
  on functors $\sd(\Delta^{n}) \to \category{C}$ which are ambigressive Cartesian.
\end{definition}

While it is not hard to check that the face and degeneracy maps on $\Span'\triple{C}$ restrict to $\Span\triple{C}$ (and thus that the above construction really produces a simplicial set), solving the lifting problems necessary to prove directly that $\Span\triple{C}$ is a quasicategory turns out to be combinatorially strenuous. It turns out to be easier to switch to a new model of quasicategories, (complete) Segal spaces, which give easier access to homotopical data.

To that end, for an adequate triple $\triple{C}$, we make the following defintion.
\begin{definition}
  \label{def:acart}
  For any simplicial set $A$, we define a simplicial set $\Map^{\mathrm{aCart}}(\sd(A), \category{C})^{\simeq}$ to be the full simplicial subset
  \begin{equation*}
    \Map^{\mathrm{aCart}}(\sd(A), \category{C})^{\simeq} \subseteq \Map(\sd(A), \category{C})^{\simeq}
  \end{equation*}
  on functors $\sd(A) \to \category{C}$ such that for each standard simplex $\sigma$ of $A$, the image of $\sd(\sigma)$ in $\category{C}$ is ambigressive Cartesian. 
\end{definition}

\begin{definition}
  The (complete) Segal space of spans $\SPAN\triple{C}$ is defined level-wise by 
  \begin{equation*}
    \SPAN\triple{C}_{n} = \Map^{\mathrm{aCart}}(\sd(\Delta^{n}), \category{C})^{\simeq}.
  \end{equation*}
\end{definition}

For a proof that this really is a Segal space, the reader is once again referred to \cite{spectralmackeyfunctors1}. Note that the quasicategory of spans $\Span\triple{C}$ is the bottom row of the complete Segal space of span $\SPAN\triple{C}$; that is,
\begin{equation*}
  \Span\triple{C} = \SPAN\triple{C}/\Delta^{0}.
\end{equation*}
\hyperref[prop:joyal-tierney_quillen_equivalence]{Proposition~\ref*{prop:joyal-tierney_quillen_equivalence}} thus immediately implies that $\Span\triple{C}$ is a quasicategory.

\subsection{Important example: spans and equivalences}
\label{ssc:important_example_spans_and_equivalences}

Let $\category{C}$ be a quasicategory. One can think about spans in $\category{C}$ as `fractions' of morphisms in $\category{C}$. Consider a span in $\category{C}$
\begin{equation*}
  \begin{tikzcd}
    & Y
    \arrow[dl, swap, "g"]
    \arrow[dr, "f"]
    \\
    X
    && X'
  \end{tikzcd}.
\end{equation*}
If the morphism $g$ were invertible, we could fill this to a full 2-simplex
\begin{equation*}
  \begin{tikzcd}
    & Y
    \arrow[dl, swap, "g"]
    \arrow[dr, "f"]
    \\
    X
    \arrow[rr, swap, dashed, "f \circ g^{-1}"]
    && X'
  \end{tikzcd}.
\end{equation*}
Even if $g$ is not invertible, we can think of our span as formally representing a `fraction of morphisms' $f \circ g^{-1}\colon X \to X'$, which may or may not exist in $\category{C}$.

With this point of view in mind, if we build from a quasicategory $\category{C}$ a category of spans whose forwards-facing legs are arbitrary morphisms in $\category{C}$, and whose backward-facing legs are equivalences, then we have not really done anything; each span, thought of as a formal morphism in $\category{C}$, is represented by an actual morphism in $\category{C}$. Thus, the category of spans in $\category{C}$ whose backward-facing legs are equivalences should be equivalent to the category $\category{C}$. In the following example, we construct this equivalence.

\begin{example}
  \label{eg:spans_with_equivalences_on_one_leg}
  For any quasicategory $\category{C}$, we can define a triple $(\category{C}, \category{C}\updag = \category{C}, \category{C}\downdag = \category{C}^{\simeq})$. These choices of ingressive and egressive morphisms correspond to spans of the form
  \begin{equation*}
    \begin{tikzcd}
      & Y
      \arrow[dl, swap, "\simeq"]
      \arrow[dr]
      \\
      X
      && X'
    \end{tikzcd},
  \end{equation*}
  i.e.\ spans such that the backwards-facing map $Y \to X$ is an equivalence in $\category{C}$. With this triple, a functor $\sd(\Delta^{n}) \to \category{C}$ is ambigressive cartesian if and only if each backward-facing leg is an equivalence. Note that the triple $(\category{C}, \category{C}, \category{C}^{\simeq})$ defined above is adequate even if $\category{C}$ does not admit pullbacks: for any solid diagram below, where $g$ is an equivalence, the dashed square completion is a pullback, where $g^{-1}$ is any homotopy inverse for $g$.
  \begin{equation*}
    \begin{tikzcd}
      X
      \arrow[r, dashed, "g^{-1} \circ f"]
      \arrow[d, swap, dashed, "\id"]
      & X'
      \arrow[d, "g"]
      \\
      X
      \arrow[r, "f"]
      & Y'
    \end{tikzcd}
  \end{equation*}

  We will denote the resulting Segal space of spans by $\SPAN^{\simeq}(\category{C})$, and the quasicategory $\SPAN^{\simeq}(\category{C})/\Delta^{0}$ by $\Span^{\simeq}(\category{C})$. 

  The quasicategory $\Span^{\simeq}(\category{C})$ is categorically equivalent to the category $\category{C}$ from which we started. More specifically, there exists a weak categorical equivalence $\category{C} \to \Span^{\simeq}(\category{C})$. We will now construct this map. 

  For each $n \geq 0$ there is a retraction of posets
  \begin{equation*}
    [n] \hookrightarrow \Sigma_{n} \twoheadrightarrow [n],
  \end{equation*}
  where the first map takes $i \mapsto (i, n)$, and the second takes $(i, j) \mapsto i$. In the case $n = 2$, the inclusion can be drawn

  \tikzset{LA/.style = {line width=#1, -{Straight Barb[length=3pt]}}, LA/.default=1.5pt}
  \begin{equation*}
    \begin{tikzcd}[row sep=small, column sep=tiny]
      0
      \arrow[dr, LA]
      \\
      & 1
      \arrow[dr, LA]
      \\
      && 2
    \end{tikzcd}
    \qquad \hookrightarrow \qquad
    \begin{tikzcd}[row sep=small, column sep=tiny]
      && (0, 2)
      \arrow[dl]
      \arrow[dr, LA]
      \\
      & (0, 1)
      \arrow[dl]
      \arrow[dr]
      && (1, 2)
      \arrow[dl]
      \arrow[dr, LA]
      \\
      (0, 0)
      && (1, 1)
      && (2, 2)
    \end{tikzcd},
  \end{equation*}
  and the map $\sd(\Delta^{n}) \to \Delta^{n}$ `collapses' the backward-facing legs of the spans. The general case is analogous.

  Applying the nerve, we find a retraction of simplicial sets
  \begin{equation*}
    \Delta^{n} \hookrightarrow \sd(\Delta^{n}) \twoheadrightarrow \Delta^{n}.
  \end{equation*}
  Pulling back along these maps gives us, for each $n$, a retraction
  \begin{equation*}
    \begin{tikzcd}
      \Map(\Delta^{n}, \category{C})^{\simeq}
      \arrow[r, hook, "i'_{n}"]
      & \Map(\sd(\Delta^{n}), \category{C})^{\simeq} 
      \arrow[r, hook, "r'_{n}"]
      & \Map(\Delta^{n}, \category{C})^{\simeq}
    \end{tikzcd}.
  \end{equation*}
  Note that the image of the inclusion $i'_{n}$ consists of spans whose backwards-facing legs are identities (hence certainly equivalences), and hence that we can restrict the total space of our retraction:
  \begin{equation*}
    \begin{tikzcd}
      \Map(\Delta^{n}, \category{C})^{\simeq} 
      \arrow[r, hook, "i_{n}"]
      & \Map^{\aCart}(\sd(\Delta^{n}), \category{C})^{\simeq}
      \arrow[r, two heads, "r_{n}"]
      & \Map(\Delta^{n}, \category{C})^{\simeq}
    \end{tikzcd}.
  \end{equation*}

  It is not hard to convince oneself that a map $\sd(\Delta^{n}) \to \category{C}$ is ambigressive cartesian if and only if it is a right Kan extension of its restriction along $\Delta^{n} \hookrightarrow \sd(\Delta^{n})$. In the case $n = 2$, for example, it is clear from the limit formula that any right Kan extension of a diagram $\Delta^{2} \to \category{C}$ along the inclusion $\Delta^{2} \hookrightarrow \sd(\Delta^{2})$ is of the form
  \begin{equation*}
    \begin{tikzcd}[row sep=small, column sep=tiny]
      X
      \arrow[dr]
      \\
      & Y
      \arrow[dr]
      \\
      && Z
    \end{tikzcd}
    \qquad \hookrightarrow \qquad
    \begin{tikzcd}[row sep=small, column sep=tiny]
      && X
      \arrow[dl, swap, "\simeq"]
      \arrow[dr]
      \\
      & X'
      \arrow[dl, swap, "\simeq"]
      \arrow[dr]
      && Y
      \arrow[dl, swap, "\simeq"]
      \arrow[dr]
      \\
      X''
      && Y'
      && Z
    \end{tikzcd},
  \end{equation*}
  (and conversely, any diagram of the above form is a right Kan extension), and diagrams of the above form are precisely those which are ambigressive cartesian. Put differently, $\Fun^{\aCart}(\sd(\Delta^{n}), \category{C})$ is the full subcategory of $\Fun(\sd(\Delta^{n}), \category{C})$ on functors which are right Kan extensions of their restrictions to $\Delta^{n}$. This, together with \cite[Prop.\ 4.3.2.15]{highertopostheory} implies that $r_{n}$ is a trivial fibration for all $n$. The identity $\id_{\Fun(\Delta^{n}, \category{C})}$ is certainly a weak Kan equivalence, so by the $2/3$ property for weak equivalences, $i_{n}$ is also a weak equivalence for all $n$.

  The projections $\sd(\Delta^{n}) \to \Delta^{n}$ assemble into a cosimplicial object $\sd(\Delta^{\bullet}) \to \Delta^{\bullet}$. From this it easily follows that the maps $i_{n}$ assemble into a map of Segal spaces
  \begin{equation*}
    i\colon \Gamma(\category{C}) = \Fun(\Delta^{\bullet}, \category{C}) \to \Fun^{\aCart}(\sd(\Delta^{\bullet}), \category{C}) = \SPAN^{\simeq}(\category{C})
  \end{equation*}
  which is a level-wise Kan weak equivalence, and hence a weak equivalence in the complete Segal spaces model structure.\footnote{For a brief review of the model structures in question, see \hyperref[sss:a_brief_list_of_model_structures]{Appendix~\ref*{sss:a_brief_list_of_model_structures}}. For more information, see \cite{qcats_vs_segal_spaces}, especially the discussion following Thm.\ 4.1.} Here $\Gamma(\category{C})$ is the complete Segal space defined in \cite{qcats_vs_segal_spaces} in the discussion preceding Thm.\ 4.11. Thus, because $\Gamma(\category{C})$ and $\SPAN^{\simeq}(\category{C})$ are complete Segal spaces (hence fibrant-cofibrant objects in the complete Segal space model structure), the restricion of $i$ to first rows $i / \Delta^{0}$ is a weak categorical equivalence, which is what we wanted to show.
\end{example}

As \hyperref[eg:spans_with_equivalences_on_one_leg]{Example~\ref*{eg:spans_with_equivalences_on_one_leg}} shows, taking spans whose backward-facing legs are equivalences allows us to find an equivalence between $\category{C}$ and a quasicategory of spans in $\category{C}$. Similarly, if we look at spans whose forward-facing legs are equivalences, we find a category of spans equivalent to $\category{C}\op$.

\begin{example}
  \label{eg:spans_with_equivalences_on_forwards_leg}
  For any quasicategory $\category{C}$, we can define an adequate triple $(\category{C}, \category{C}\downdag = \category{C}^{\simeq}, \category{C}\updag = \category{C})$. These selections of ingressive and egressive morphisms correspond to spans of the form
  \begin{equation*}
    \begin{tikzcd}
      & Y
      \arrow[dl]
      \arrow[dr, "\simeq"]
      \\
      X
      && X'
    \end{tikzcd},
  \end{equation*}
  i.e.\ spans such that the forward-facing map $Y \to X$ is an equivalence in $\category{C}$. We will denote the corresponding category of spans by $\Span_{\simeq}(\category{C})$. Exactly analogous reasoning to that in \hyperref[eg:spans_with_equivalences_on_one_leg]{Example~\ref*{eg:spans_with_equivalences_on_one_leg}} gives us a categorical equivalence
  \begin{equation*}
    \category{C}\op \to \Span_{\simeq}(\category{C}).
  \end{equation*}
\end{example}

\subsection{Maps between triples}
\label{ssc:maps_between_triples}

We have now seen that given any quasicategory $\category{C}$ with pullbacks, one can create a quasicategory of spans $\Span(\category{C})$, and that by defining an adequate triple structure $\triple{C}$ on $\category{C}$, one can specify certain classes of morphisms to which the legs of the spans are allowed to belong. It is easy to see that any functor $\category{C} \to \category{D}$ between quasicategories $\category{C}$ and $\category{D}$ with pullbacks induces a functor $\Span(\category{C}) \to \Span(\category{D})$. Similarly, we make the following definition.

\begin{definition}
  Let $\triple{C}$ and $\triple{D}$ be adequate triples. A functor $p\colon \category{C} \to \category{D}$ is said to be a \defn{functor between adequate triples} $\triple{C} \to \triple{D}$ 
  if the following conditions hold. 
  \begin{enumerate}
    \item The functor $p$ preserves ingressive morphisms. That is, $p(\category{C}_{\dagger}) \subseteq \category{D}_{\dagger}$

    \item The functor $p$ preserves egressive morphsims. That is, $p(\category{C}^{\dagger}) \subseteq \category{D}^{\dagger}$. 

    \item The functor $p$ preserves ambigressive pullbacks.
  \end{enumerate}
\end{definition}

We will denote the restriction of $p$ to $\category{C}_{\dagger}$ by
\begin{equation*}
  p_{\dagger}\colon\category{C}_{\dagger} \to \category{D}_{\dagger},
\end{equation*}
and similarly for $p^{\dagger}$. Clearly, a functor $p$ between adequate triples gives functors $\Span(p)$ and $\SPAN(p)$ respectively between quasicategories and Segal spaces of spans.

It is natural to wonder about the relationship between $p$ and $\Span(p)$. The goal of the remainder of this section is to provide a proof of the following theorem, originally appearing as \cite[Thm.\ 12.2]{spectralmackeyfunctors1}. This theorem provides conditions on a functor $p$ between adequate triples such that the induced map between category of spans is an inner fibration, and provides the form of a $p$-cocartesian morphism in $\Span\triple{C}$.

\begin{theorem}
  \label{thm:main}
  Let $p\colon \triple{C} \to \triple{D}$ be a functor of adequate triples which is an inner fibration. Further suppose that $p$ satisfies the following conditions:
  \begin{enumerate}
    \item For any ingressive morphism $g\colon s \ing t$ of $\category{D}$ and any object $x \in \category{C}_{s}$, there exists an ingressive morphism $f\colon x \ing y$ covering $g$ which is both $p$-cocartesian and $p_{\dagger}$-cocartesian.

    \item Suppose $\sigma$ is a commutative square
      \begin{equation*}
        \begin{tikzcd}
          y'
          \arrow[r, rightarrowtail, "f'"]
          \arrow[d, two heads]
          & x'
          \arrow[d, "\psi"]
          \\
          y
          \arrow[r, rightarrowtail, swap, "f"]
          & x
        \end{tikzcd}
      \end{equation*}
      in $\category{C}$ such that $p(\sigma)$ is ambigressive pullback in $\category{D}$, with in- and egressive morphisms as marked, and $f$ is $p$-cocartesian. Then $f'$ is $p$-cocartesian if and only if $\sigma$ is an ambigressive pullback square (and in particular $\psi$ is egressive).
  \end{enumerate}

  Then $\pi$ is an inner fibration. Further, if a morphism in $\Span\triple{C}$ is of the form
  \begin{equation*}
    \begin{tikzcd}
      & y
      \arrow[dl, two heads, swap, "\phi"]
      \arrow[dr, rightarrowtail, "\psi"]
      \\
      x
      && x'
    \end{tikzcd},
  \end{equation*}
  where $\phi$ is egressive and $p^{\dagger}$-cartesian, and $\psi$ is ingressive and $p$-cocartesian, it is $\pi$-cocartesian.%\footnote{This is not precisely the same condition that Barwick gives: he demands that $\phi$ be $p$-cartesian rather than $p^{\dagger}$-cartesian. As far as I can tell, this is simply an error; Barwick seems to use the condition as stated above in his proof.}
\end{theorem}

This theorem originally appears in \cite{spectralmackeyfunctors1}. However, the proof there is combinatorial in nature. Our proof uses the material developed in \hyperref[sec:cocartesian_fibrations_between_complete_segal_spaces]{Section~\ref*{sec:cocartesian_fibrations_between_complete_segal_spaces}}, leveraging the fact that $\infty$-categories of spans have natural incarnations as (complete) Segal spaces.

Before we get to the meat of the proof, however, we have to prove some preliminary results.

\subsection{Housekeeping}

In this section we recall some basic facts we will need in the course of our proof of the main theorem. It is recommended to skip this section on first reading, and refer back to the results as necessary.

\subsubsection{Isofibrations}

We will need some results about isofibrations, sometimes called categorical fibrations. These are standard, and proofs can be found in \cite{kerodon}.

\begin{proposition}
  \label{prop:characterization_of_isofibrations}
  Let $f\colon \category{C} \to \category{D}$ be a map between quasicategories. The following are equivalent:
  \begin{itemize}
    \item The map $f$ is an isofibration.

    \item The map $f$ is an inner fibration, and each equivalence in $\category{D}$ has a lift in $\category{C}$ which is an equivalence.
  \end{itemize}
\end{proposition}

We will denote by $\Map(-, -)$ the internal hom on $\SSet$.
\begin{proposition}
  Let $f\colon \category{C} \to \category{D}$ be an isofibration between quasicategories, and let $i\colon A \hookrightarrow B$ be an inclusion of simplicial sets. Then the map
  \begin{equation*}
    \Map(B, \category{C}) \to \Map(A, \category{C}) \times_{\Map(A, \category{D})} \Map(B, \category{D})
  \end{equation*}
  is an isofibration.
\end{proposition}

We will denote by $(-)^{\simeq}$ the \emph{core} functor, i.e.\ the functor which takes a simplicial set $X$ to the largest Kan complex it contains.
\begin{proposition}
  Let $f\colon X \to Y$ be an isofibration of simplicial sets. Then
  \begin{equation*}
    f^{\simeq}\colon X^{\simeq} \to Y^{\simeq}
  \end{equation*}
  is a Kan fibration.
\end{proposition}

We will denote the core of the mapping space functor $\Map(-, -)$ by $\Fun(-, -)$. That is, for simplicial sets $A$ and $B$, we have
\begin{equation*}
  \Fun(A, B) \cong \Map(A, B)^{\simeq}.
\end{equation*}

\begin{corollary}
  \label{cor:kan_fib_and_isofib}
  Let $f\colon \category{C} \to \category{D}$ be an isofibration between quasicategories, and let $A \hookrightarrow B$ be an inclusion of simplicial sets. Then the map
  \begin{equation*}
    \Fun(B, \category{C}) \to \Fun(A, \category{C}) \times_{\Fun(A, \category{D})} \Fun(B, \category{D})
  \end{equation*}
  is a Kan fibration.
\end{corollary}

\subsubsection{Connected components}

Our proof will rely heavily on the fact that the many properties that morphisms can have (ingressive, egressive, $p$-cocartesian, etc.) are well-behaved with respect to the homotopical structure of the complete Segal spaces in which they live; for example, if a morphism is $p$-cocartesian, then every morphism in its path component is $p$-cocartesian.

Let $X'$ and $X$ be simplicial sets, and let $f\colon X' \hookrightarrow X$ be an inclusion. We will say that $f$ is an \emph{inclusion of connected components} if for all simplices $\sigma \in X$, if $\sigma$ has any vertex in common with $X'$, then $\sigma$ is wholly contained in $X'$.

\begin{lemma}
  \label{lemma:lifting_wrt_inclusions_of_connected_components}
  Let $A$ be a simplicial set such that between any two vertices $x$ and $y$ of $A$ there exists a finite zig-zag of 1-simplices of $A$ connecting $x$ to $y$. Let $A_{0} \subseteq$ A be a nonempty simplicial subset of $A$. Let $f\colon X' \hookrightarrow X$ be a morphism between simplicial sets which is an inclusion of connected components. Then the following dashed lift always exists.
  \begin{equation*}
    \begin{tikzcd}
      A_{0}
      \arrow[r, "f_{0}"]
      \arrow[d, hook]
      & X'
      \arrow[d, hook]
      \\
      A
      \arrow[r, "f"]
      \arrow[ur, dashed]
      & X
    \end{tikzcd}
  \end{equation*}
\end{lemma}
\begin{proof}
  We have a map $f\colon A \to X$; in order to construct our dashed lift, it suffices to show that under the above assumptions, $f$ takes every simplex of $A$ to a simplex which belongs to $X'$. To this end, let $\sigma \in A$ be a simplex. There exists a finite zig-zag of $1$-simplices connecting some vertex of $\sigma$ to a vertex $\alpha$ of $A_{0}$; under $f$, this is mapped to a zig-zag of $1$-simplices connecting $f(\alpha)$ to $f(\sigma)$. The vertex $f(\alpha)$ belongs to $X'$; proceeding inductively, we find that the image of every 1-simplex belonging to the zig-zag also belongs to $X'$, and that $f(\sigma)$ therefore also belongs to $X'$.
\end{proof}

\begin{corollary}
  \label{cor:connected_components_kan_fibration}
  Given any commuting square of simplicial sets
  \begin{equation*}
    \begin{tikzcd}
      X'
      \arrow[r, hook, "i"]
      \arrow[d, swap, "f'"]
      & X
      \arrow[d, "f"]
      \\
      Y'
      \arrow[r, hook]
      & Y
    \end{tikzcd}
  \end{equation*}
  with monomorphisms as marked, where $i$ is an inclusion of connected components between Kan complexes and $f$ is a Kan fibration, the map $f'$ is a Kan fibration.

  Furthermore, if $f$ is a trivial fibration and $f'$ is surjective on vertices, then $f'$ is a trivial fibration.
\end{corollary}
\begin{proof}
  First, we show that if $f$ is a Kan fibration, then $f'$ is a Kan fibration. We need to show that a dashed lift below exists for each $n \geq 1$ and each $0 \leq i \leq n$.
  \begin{equation*}
    \begin{tikzcd}
      \Lambda^{n}_{i}
      \arrow[r]
      \arrow[d]
      & X'
      \arrow[r, hook, "i"]
      \arrow[d, "f'"]
      & X
      \arrow[d, "f"]
      \\
      \Delta^{n}
      \arrow[r]
      \arrow[ur, dashed]
      & Y'
      \arrow[r, hook]
      & Y
    \end{tikzcd}
  \end{equation*}
  We can always solve our outer lifting problem, and by \hyperref[lemma:lifting_wrt_inclusions_of_connected_components]{Lemma~\ref*{lemma:lifting_wrt_inclusions_of_connected_components}}, our lift factors through $X'$.

  Now suppose that $f$ is a trivial Kan fibration. The logic above says that $f'$ has the right lifting property with respect to $\partial \Delta^{n} \hookrightarrow \Delta^{n}$ for $n \geq 1$; the right lifting property with respect to $\partial \Delta^{0} \hookrightarrow \Delta^{0}$ is equivalent to the surjectivity of $f'$ on vertices.
\end{proof}

\subsubsection{Contractibility}

We will need at certain points to show that various spaces of lifts are contractible. These are mostly common-sense results, but it will be helpful to have them written down somewhere so we can refer to them later.

\begin{lemma}
  \label{lemma:inner_fib_contractible_fibers}
  Let $f\colon \category{C} \to \category{D}$ be an inner fibration between quasicategories. Then for any commuting square
  \begin{equation*}
    \begin{tikzcd}
      \Lambda^{2}_{1}
      \arrow[r, "\alpha"]
      \arrow[d, hook]
      & \category{C}
      \arrow[d, "f"]
      \\
      \Delta^{2}
      \arrow[r, "\beta"]
      & \category{D}
    \end{tikzcd},
  \end{equation*}
  the fiber $F$ in the pullback square below is contractible.
  \begin{equation*}
    \begin{tikzcd}[column sep=large]
      F
      \arrow[r]
      \arrow[d]
      & \Fun(\Delta^{2}, \category{C})
      \arrow[d]
      \\
      \Delta^{0}
      \arrow[r, "{(\alpha, f\alpha, \beta)}"]
      & \Fun(\Lambda^{2}_{1}, \category{C}) \times_{\Fun(\Lambda^{2}_{1}, \category{D})} \Fun(\Delta^{2}, \category{D})
    \end{tikzcd}
  \end{equation*}
\end{lemma}
\begin{proof}
  The right-hand map is a trivial Kan fibration.
\end{proof}

One should interpret this as telling us that given a $\Lambda^{2}_{2}$-horn $\alpha$ in $\category{C}$ lying over a 2-simplex $\beta$ in $\category{D}$, the space of fillings of $\alpha$ lying over $\beta$ is contractible. We will need a pantheon of similar results.

\begin{lemma}
  \label{lemma:cartesian_horn_fillings_contractible}
  Let $f\colon \category{C} \to \category{D}$ be an inner fibration between quasicategories. Let $e$ be any $f$-cartesian edge. Then for any square
  \begin{equation*}
    \begin{tikzcd}
      \Lambda^{2}_{2}
      \arrow[r, "\alpha"]
      \arrow[d, hook]
      & \category{C}
      \arrow[d, "f"]
      \\
      \Delta^{2}
      \arrow[r, "\beta"]
      & \category{D}
    \end{tikzcd}
  \end{equation*}
  such that $\alpha|_{\{1,2\}} = e$, the fiber $F$ in the pullback diagram
  \begin{equation}
    \label{eq:fiber_of_inn_fib_with_cocart}
    \begin{tikzcd}[column sep=large]
      F
      \arrow[r]
      \arrow[d]
      & \Fun(\Delta^{2}, \category{C})
      \arrow[d]
      \\
      \Delta^{0}
      \arrow[r, "{(\alpha, f\alpha, \beta)}"]
      & \Fun(\Lambda^{2}_{2}, \category{C}) \times_{\Fun(\Lambda^{2}_{1}, \category{D})} \Fun(\Delta^{2}, \category{D})
    \end{tikzcd}
  \end{equation}
  is contractible.
\end{lemma}
\begin{proof}
  Denote $f(e) = \bar{e}$. Define a marking $\mathcal{E}'$ on $\category{D}$ containing all degenerate edges and $\bar{e}$. Define a marking $\mathcal{E}$ on $\category{C}$ containing the $f$-cocartesian lifts of the edges in $\mathcal{E}$. By \cite[Prop.\ 3.1.1.6]{highertopostheory}, the map $\category{C}^{\mathcal{E}} \to \category{D}^{\mathcal{E}'}$ has the right-lifting property with respect to all cocartesian-marked anodyne morphisms. It follows that
  \begin{equation*}
    \begin{tikzcd}[column sep=large]
      \Map^{\sharp}((\Delta^{2})^{\mathcal{L}}, \category{C}^{\mathcal{E}})
      \arrow[d]
      \\
      \Map^{\sharp}((\Lambda^{2}_{2})^{\mathcal{L}}, \category{C}^{\mathcal{E}}) \times_{\Map^{\sharp}((\Lambda^{2}_{1})^{\mathcal{L}}, \category{D}^{\mathcal{E}'})} \Map^{\sharp}((\Delta^{2})^{\mathcal{L}}, \category{D}^{\mathcal{E}'})
    \end{tikzcd}
  \end{equation*}
  is a trivial Kan fibration; the map on the right-hand side of \hyperref[eq:fiber_of_inn_fib_with_cocart]{Diagram~\ref*{eq:fiber_of_inn_fib_with_cocart}} is the core of this map, and is thus also a trivial Kan fibration.
\end{proof}

\begin{lemma}
  \label{lemma:twice_cartesian_lifts_contractible}
  Let $\triple{C} \to \triple{D}$ be an inner fibration between triples satisfying the conditions of \hyperref[thm:main]{Theorem~\ref*{thm:main}}. Denote by $\Fun'(\Delta^{1}, \category{C}^{\dagger})$ the full simplicial subset of $\Fun(\Delta^{1}, \category{C}^{\dagger})$ on edges $\Delta^{1} \to \category{C}^{\dagger}$ which are both $p$-cocartesian and $p^{\dagger}$-cocartesian. Then the fibers of the map
  \begin{equation*}
    \Fun'(\Delta^{1}, \category{C}^{\dagger}) \to
    \Fun(\Delta^{\{0\}}, \category{C}^{\dagger})
    \times_{\Fun(\Delta^{\{0\}}, \category{D}^{\dagger})}
    \Fun(\Delta^{1}, \category{D}^{\dagger})
  \end{equation*}
  are contractible.
\end{lemma}
\begin{proof}
  Denote by $\Fun''(\Delta^{1}, \category{C}^{\dagger})$ the full simplicial subset of $\Fun(\Delta^{1}, \category{C}^{\dagger})$ on edges which are $p^{\dagger}$-cartesian. Then we have a square
  \begin{equation*}
    \noindent\makebox[\textwidth]{%
      \begin{tikzcd}[ampersand replacement=\&]
        \Fun'(\Delta^{1}, \category{C}^{\dagger})
        \arrow[r, hook]
        \arrow[d]
        \& \Fun''(\Delta^{1}, \category{C}^{\dagger})
        \arrow[d]
        \\
        \Fun(\Delta^{\{0\}}, \category{C}^{\dagger})
        \times_{\Fun(\Delta^{\{0\}}, \category{D}^{\dagger})}
        \Fun(\Delta^{1}, \category{D}^{\dagger})
        \arrow[r, equals]
        \& \Fun(\Delta^{\{0\}}, \category{C}^{\dagger})
        \times_{\Fun(\Delta^{\{0\}}, \category{D}^{\dagger})}
        \Fun(\Delta^{1}, \category{D}^{\dagger})
      \end{tikzcd}
    }
  \end{equation*}
  in which the top map is an inclusion of connected components, the right map is a trivial Kan fibration, and the left map is surjective on vertices. Thus, the left map is a trivial Kan fibration by \hyperref[cor:connected_components_kan_fibration]{Corollary~\ref*{cor:connected_components_kan_fibration}}
\end{proof}

\subsection{Proof of main theorem}

Our goal is now to prove \hyperref[thm:main]{Theorem~\ref*{thm:main}}. As in the statement of the theorem, we fix a functor of triples $p\colon \triple{C} \to \triple{D}$ giving us a functor of quasicategories
\begin{equation*}
  \pi\colon \Span\triple{C} \to \Span\triple{D}.
\end{equation*}
We wish to show that the following hold.
\begin{enumerate}
  \item The map $\pi$ is an inner fibration.

  \item Morphsims of the form
    \begin{equation}
      \label{eq:form_of_p_cocartesian_morphisms}
      \begin{tikzcd}
        & y
        \arrow[dl, two heads, swap, "\phi"]
        \arrow[dr, rightarrowtail, "\psi"]
        \\
        x
        && x'
      \end{tikzcd},
    \end{equation}
    are $\pi$-cocartesian, where $\phi$ is egressive and $p_{\dagger}$-cartesian, and $\psi$ is ingressive and $p$-cocartesian.
\end{enumerate}

The way forward is clear: the functor $\pi$ is the zeroth row of a functor of Segal spaces
\begin{equation*}
  \pi'\colon \SPAN\triple{C} \to \SPAN\triple{D}.
\end{equation*}

The results of \hyperref[sec:cocartesian_fibrations_between_complete_segal_spaces]{Section~\ref*{sec:cocartesian_fibrations_between_complete_segal_spaces}} therefore guarantee the following.
\begin{enumerate}
  \item By \hyperref[cor:reedy_implies_inner]{Corollary~\ref*{cor:reedy_implies_inner}}, in order to show that $\pi$ is an inner fibration, it suffices to show that $\pi'$ is a Reedy fibration.

  \item In order to show that morphisms of the required form are $\pi$-cocartesian, it will suffice to show by \hyperref[cor:cocart_fib_between_css_gives_cocart_fib_of_quasicats]{Corollary~\ref*{cor:cocart_fib_between_css_gives_cocart_fib_of_quasicats}} that morphisms of this form are $\pi'$-cocartesian. 
\end{enumerate}

\subsubsection{The map \texorpdfstring{$\pi$}{pi} is an inner fibration}

We begin by checking that the map $\pi$ is an inner fibration. As stated above, we do this by checking that $\pi'$ is a Reedy fibration. This follows from the following useful fact, implied by the first assumption of \hyperref[thm:main]{Theorem~\ref*{thm:main}}.
\begin{lemma}
  Let $p\colon \triple{C} \to \triple{D}$ be a functor of triples satisfying the conditions of \hyperref[thm:main]{Theorem~\ref*{thm:main}}. Then the map $p\colon \category{C} \to \category{D}$ is an isofibration.
\end{lemma}
\begin{proof}
  By assumption, $p$ is an inner fibration and $\category{C}$ and $\category{D}$ are quasicategories, so it suffices by \hyperref[prop:characterization_of_isofibrations]{Proposition~\ref*{prop:characterization_of_isofibrations}} to show that $p$ admits lifts of equivalences. Each equivalence in $\category{D}$ is in particular ingressive, and hence admits a $p$-cocartesian lift by the assumptions of \hyperref[thm:main]{Theorem~\ref*{thm:main}}; these are automatically equivalences.
\end{proof}

\begin{proposition}
  For any functor of adequate triples $p\colon \triple{C} \to \triple{D}$, the map
  \begin{equation*}
    \pi'\colon \SPAN\triple{C} \to \SPAN\triple{D}
  \end{equation*}
  is a Reedy fibration.
\end{proposition}
\begin{proof}
  Consider the following diagram.
  \begin{equation*}
    \noindent\makebox[\textwidth]{%
      \begin{tikzcd}[ampersand replacement=\&]
        \Fun^{\aCart}(\sd(\Delta^{n}), \category{C})
        \arrow[r, hook]
        \arrow[d]
        \& \Fun(\sd(\Delta^{n}), \category{C})
        \arrow[d]
        \\
        \Fun^{\aCart}(\sd(\partial\Delta^{n}), \category{C})
        \times_{\Fun^{\aCart}(\sd(\partial\Delta^{n}), \category{D})}
        \Fun^{\aCart}(\sd(\Delta^{n}), \category{D})
        \arrow[r, hook]
        \& \Fun(\sd(\partial\Delta^{n}), \category{C})
        \times_{\Fun(\sd(\partial\Delta^{n}), \category{D})}
        \Fun(\sd(\Delta^{n}), \category{D})
      \end{tikzcd}.
    }
  \end{equation*}
  The right-hand map is a Kan fibration by \hyperref[cor:kan_fib_and_isofib]{Corollary~\ref*{cor:kan_fib_and_isofib}} (because $\sd$ preserves monomorphisms), and the top map is an inclusion of connected components, so \hyperref[cor:connected_components_kan_fibration]{Corollary~\ref*{cor:connected_components_kan_fibration}} implies that the left-hand map is a Kan fibration, which is what we needed to show.
\end{proof}

\begin{corollary}
  The map $\pi$ is an inner fibration
\end{corollary}
\begin{proof}
  \hyperref[cor:reedy_implies_inner]{Corollary~\ref*{cor:reedy_implies_inner}}.
\end{proof}

\subsubsection{Cocartesian morphisms have the promised form}
\label{sss:cocartesian_morphisms_have_the_promised_form}

We are now ready to begin in earnest our proof that morphisms of the form \hyperref[eq:form_of_p_cocartesian_morphisms]{Equation~\ref*{eq:form_of_p_cocartesian_morphisms}} are $\pi'$-cocartesian. Morally, this result should not be surprising. We should think of the homotopy pullback condition defining cocartesian morphisms (\hyperref[def:cocartesian_morphism]{Definition~\ref*{def:cocartesian_morphism}}) as telling us that we can fill relative $\Lambda^{2}_{2}$-horns in $\pi'\colon \SPAN\triple{C} \to \SPAN\triple{D}$, and that such fillings are unique up to contractible choice. Finding a filling 
\begin{equation*}
  \begin{tikzcd}
    \Lambda^{2}_{2}
    \arrow[r]
    \arrow[d, hook]
    & \SPAN\triple{C}
    \arrow[d, "\pi'"]
    \\
    \Delta^{2}
    \arrow[r]
    \arrow[ur, dashed]
    & \SPAN\triple{D}
  \end{tikzcd}
\end{equation*}
is equivalent finding a filling
\begin{equation*}
  \begin{tikzcd}
    \sd(\Lambda^{2}_{2})
    \arrow[r]
    \arrow[d, hook]
    & \category{C}
    \arrow[d, "p"]
    \\
    \sd(\Delta^{2})
    \arrow[r]
    \arrow[ur, dashed]
    & \category{D}
  \end{tikzcd}
\end{equation*}
(such that the filling is ambigressive and the necessary square is pullback); the conditions on $p$ guarantee us that we can perform this filling in a series of steps pictured in \hyperref[fig:factorization]{Figure~\ref*{fig:factorization}}, each of which is unique up to contractible choice.

In fact, our proof mainly consists of making this idea rigorous. To this end, we first introduce some notation. Let $p\colon \triple{C} \to \triple{D}$ be a functor of triples.
\begin{itemize}
  \item We will denote $p$-cartesian morphisms in $\category{C}$ with a circle:
    \begin{equation*}
      \begin{tikzcd}
        x
        \arrow[r, "\circ" marking]
        & y
      \end{tikzcd}
    \end{equation*}

  \item We will denote $p$-cocartesian morphisms in $\category{C}$ with a bullet:
    \begin{equation*}
      \begin{tikzcd}
        x
        \arrow[r, "\bullet" marking]
        & y
      \end{tikzcd}
    \end{equation*}

  \item We will denote $p^{\dagger}$-cartesian morphisms in $\category{C}$ with a triangle:
    \begin{equation*}
      \begin{tikzcd}
        x
        \arrow[r, "\vartriangleright" marking]
        & y
      \end{tikzcd}
    \end{equation*}

  \item We will denote $p_{\dagger}$-cocartesian morphisms in $\category{C}$ with a filled triangle:
    \begin{equation*}
      \begin{tikzcd}
        x
        \arrow[r, "\blacktriangleright" marking]
        & y
      \end{tikzcd}
    \end{equation*}
\end{itemize}
Thus, an ingressive morphism which is both $p$-cocartesian and $p_{\dagger}$-cocartesian will be denoted by
\begin{equation*}
  \begin{tikzcd}[column sep=large]
    x
    \arrow[r, rightarrowtail, "\blacktriangleright \bullet" marking]
    & y
  \end{tikzcd}
\end{equation*}

Our proof will rest on the factorization $\sd(\Lambda^{2}_{0}) \hookrightarrow \sd(\Delta^{2})$ pictured in \hyperref[fig:factorization]{Figure~\ref*{fig:factorization}}. Denote the underlying factorization of simplicial sets by
\begin{equation*}
  A_{1} \overset{i_{1}}{\hookrightarrow}
  A_{2} \overset{i_{2}}{\hookrightarrow}
  A_{3} \overset{i_{3}}{\hookrightarrow}
  A_{4} \overset{i_{4}}{\hookrightarrow}
  A_{5} \overset{i_{5}}{\hookrightarrow}
  A_{6}
\end{equation*}

\begin{sidewaysfigure}[p]
  \begin{equation*}
    \begin{tikzcd}
      && 11
      \\
      & 01
      \arrow[ur, rightarrowtail, "\bullet" marking]
      \arrow[dl, two heads, "\vartriangleleft" marking]
      \\
      00
      && 02
      \arrow[ll, two heads]
      \arrow[rr, rightarrowtail]
      && 22
    \end{tikzcd}
    \overset{i_{1}}{\longrightarrow}
    \begin{tikzcd}
      && 11
      \\
      & 01
      \arrow[ur, rightarrowtail, "\bullet" marking]
      \arrow[dl, two heads, "\vartriangleleft" marking]
      \\
      00
      && 02
      \arrow[ll, two heads]
      \arrow[rr, rightarrowtail]
      \arrow[ul, two heads]
      && 22
    \end{tikzcd}
    \overset{i_{2}}{\longrightarrow}
    \begin{tikzcd}
      && 11
      \\
      & 01
      \arrow[ur, rightarrowtail, "\bullet" marking]
      \arrow[dl, two heads, "\vartriangleleft" marking]
      \\
      00
      && 02
      \arrow[ll, two heads]
      \arrow[rr, rightarrowtail]
      \arrow[uu]
      \arrow[ul, two heads]
      && 22
    \end{tikzcd}
  \end{equation*}
  \vspace{2cm}
  \begin{equation*}
    \overset{i_{3}}{\longrightarrow}
    \begin{tikzcd}
      && 11
      \\
      & 01
      \arrow[ur, rightarrowtail, "\bullet" marking]
      \arrow[dl, two heads, "\vartriangleleft" marking]
      && 12
      \\
      00
      && 02
      \arrow[ur, sloped, "\bullet \blacktriangleright" marking, rightarrowtail]
      \arrow[ll, two heads]
      \arrow[rr, rightarrowtail]
      \arrow[uu]
      \arrow[ul, two heads]
      && 22
    \end{tikzcd}
    \overset{i_{4}}{\longrightarrow}
    \begin{tikzcd}
      && 11
      \\
      & 01
      \arrow[ur, rightarrowtail, "\bullet" marking]
      \arrow[dl, two heads, "\vartriangleleft" marking]
      && 12
      \arrow[ul]
      \\
      00
      && 02
      \arrow[ur, sloped, "\bullet \blacktriangleright" marking, rightarrowtail]
      \arrow[ll, two heads]
      \arrow[rr, rightarrowtail]
      \arrow[uu]
      \arrow[ul, two heads]
      && 22
    \end{tikzcd}
    \overset{i_{5}}{\longrightarrow}
    \begin{tikzcd}
      && 11
      \\
      & 01
      \arrow[ur, rightarrowtail, "\bullet" marking]
      \arrow[dl, two heads, "\vartriangleleft" marking]
      && 12
      \arrow[ul]
      \arrow[dr, rightarrowtail]
      \\
      00
      && 02
      \arrow[ur, sloped, "\bullet \blacktriangleright" marking, rightarrowtail]
      \arrow[ll, two heads]
      \arrow[rr, rightarrowtail]
      \arrow[uu]
      \arrow[ul, two heads]
      && 22
    \end{tikzcd}
  \end{equation*}
  \caption{A factorization of the inclusion $\sd(\Lambda^{2}_{0}) \hookrightarrow \sd(\Delta^{2})$, where certain morphisms have been labelled.}
  \label{fig:factorization}
\end{sidewaysfigure}

For each of the $A_{i}$ above, denote by
\begin{equation}
  \label{eq:inclusion_of_functors_respecting_labelling}
  \Fun'(A_{i}, \category{C}) \subseteq \Fun(A_{i}, \category{C})
\end{equation}
the full simplicial subset on those functors $A_{i} \to \category{C}$ which respect each labelling in \hyperref[fig:factorization]{Figure~\ref*{fig:factorization}}:
\begin{itemize}
  \item ingressive
  \item egressive
  \item $p$-cartesian
  \item $p$-cocartesian
  \item $p^{\dagger}$-cartesian
  \item $p_{\dagger}$-cocartesian
\end{itemize}
Note that because equivalences in $\category{C}$ belong to each of these classes of morphisms, the inclusion in \hyperref[eq:inclusion_of_functors_respecting_labelling]{Equation~\ref*{eq:inclusion_of_functors_respecting_labelling}} is an inclusion of connected components.

Similarly, we will denote by $\Fun'(A_{i}, \category{D})$ the full simplicial subset on functors which respect the ingressive and egressive labellings.

\begin{lemma}
  \label{lemma:homotopy_pullback_for_filling_procedure}
  The square
  \begin{equation*}
    \begin{tikzcd}
      \Fun'(A_{6}, \category{C})
      \arrow[r]
      \arrow[d]
      & \Fun'(A_{1}, \category{C})
      \arrow[d]
      \\
      \Fun'(A_{6}, \category{D})
      \arrow[r]
      & \Fun'(A_{1}, \category{D})
    \end{tikzcd}
  \end{equation*}
  is homotopy pullback.
\end{lemma}
\begin{proof}
  The factorization of \hyperref[fig:factorization]{Figure~\ref*{fig:factorization}} gives us a factorization of the above square into five squares
  \begin{equation*}
    \begin{tikzcd}
      \Fun'(A_{k+1}, \category{C})
      \arrow[r]
      \arrow[d]
      & \Fun'(A_{k}, \category{C})
      \arrow[d]
      \\
      \Fun'(A_{k+1}, \category{D})
      \arrow[r]
      & \Fun'(A_{k}, \category{D})
    \end{tikzcd},
    \qquad1 \leq k < 6,
  \end{equation*}
  each corresponding to one of the inclusions $i_{k}$. We will be done if we can show that each of these squares is homotopy pullback. It suffices to show that for each $k$, the map
  \begin{equation*}
    j_{k}\colon \Fun'(A_{k+1}, \category{C})
    \to
    \Fun'(A_{k}, \category{C})
    \times_{\Fun'(A_{k+1}, \category{D})}
    \Fun'(A_{k}, \category{D})
  \end{equation*}
  is a weak equivalence. First, we show that each of these maps is a Kan fibration. To see this, consider the square
  \begin{equation*}
    \noindent\makebox[\textwidth]{%
      \begin{tikzcd}[ampersand replacement=\&]
        \Fun'(A_{k+1}, \category{C})
        \arrow[r, hook]
        \arrow[d, swap, "j_{k}"]
        \& \Fun(A_{k+1}, \category{C})
        \arrow[d]
        \\
        \Fun'(A_{k}, \category{C})
        \times_{\Fun'(A_{k+1}, \category{D})}
        \Fun'(A_{k}, \category{D})
        \arrow[r, hook]
        \& \Fun(A_{k}, \category{C})
        \times_{\Fun(A_{k+1}, \category{D})}
        \Fun(A_{k}, \category{D})
      \end{tikzcd}.
    }
  \end{equation*}
  The right-hand map is a Kan fibration because of \hyperref[cor:kan_fib_and_isofib]{Corollary~\ref*{cor:kan_fib_and_isofib}}, and we have already seen that the top map is an inclusion of connected componenents. Hence each $j_{k}$ is a Kan fibration by \hyperref[cor:connected_components_kan_fibration]{Corollary~\ref*{cor:connected_components_kan_fibration}}. Thus, in order to show that each $j_{k}$ is a weak equivalence, it suffices to show that the fibers are contractible.

  First consider the case $k = 1$. For any map $\alpha$ below, consider the following pullback square.
  \begin{equation*}
    \begin{tikzcd}
      F_{\alpha}
      \arrow[r]
      \arrow[d]
      & \Fun'(A_{1}, \category{C})
      \arrow[d, "j_{1}"]
      \\
      \Delta^{0}
      \arrow[r, "\alpha"]
      & \Fun'(A_{0}, \category{C})
      \times_{\Fun'(A_{1}, \category{D})}
      \Fun'(A_{1}, \category{D})
    \end{tikzcd}.
  \end{equation*}
  The fiber $F_{\alpha}$ over $\alpha$ is the space of ways of completing a diagram of shape $A_{0}$ in $\category{C}$ to a diagram of shape $A_{1}$ in $\category{C}$ given a diagram of shape $A_{1}$ in $\category{D}$. This is the space of ways of filling $\Lambda^{2}_{2} \hookrightarrow \Delta^{2}$ in $\category{C}^{\dagger}$ lying over a $2$-simplex in $\category{D}^{\dagger}$. This is contractible by \hyperref[lemma:cartesian_horn_fillings_contractible]{Lemma~\ref*{lemma:cartesian_horn_fillings_contractible}}.

  The case $k = 2$ is similar, using \hyperref[lemma:inner_fib_contractible_fibers]{Lemma~\ref*{lemma:inner_fib_contractible_fibers}}.

  The case $k = 3$ is similar, using \hyperref[lemma:twice_cartesian_lifts_contractible]{Lemma~\ref*{lemma:twice_cartesian_lifts_contractible}}.

  The cases $k = 4$ and $k = 5$ use the dual to \hyperref[lemma:cartesian_horn_fillings_contractible]{Lemma~\ref*{lemma:cartesian_horn_fillings_contractible}}.
\end{proof}
We are now ready to show that morphisms of the form \hyperref[eq:form_of_p_cocartesian_morphisms]{Diagram~\ref*{eq:form_of_p_cocartesian_morphisms}} are $\pi'$-cocartesian. 

\begin{proposition}
  For any morphism $e$ of the form given in \hyperref[eq:form_of_p_cocartesian_morphisms]{Equation~\ref*{eq:form_of_p_cocartesian_morphisms}}, the square
  \begin{equation*}
    \noindent\makebox[\textwidth]{%
      \begin{tikzcd}[ampersand replacement=\&]
        \Fun^{\aCart}(\sd(\Delta^{2}), \category{C}) \times_{\Fun^{\aCart}(\sd(\Delta^{\{0, 1\}}), \category{C})} \{e\}
        \arrow[r]
        \arrow[d]
        \& \Fun^{\aCart}(\sd(\Lambda^{2}_{0}), \category{C}) \times_{\Fun^{\aCart}(\sd(\Delta^{\{0, 1\}}), \category{C})} \{e\}
        \arrow[d]
        \\
        \Fun^{\aCart}(\sd(\Delta^{2}), \category{D}) \times_{\Fun^{\aCart}(\sd(\Delta^{\{0, 1\}}), \category{D})} \{\pi e\}
        \arrow[r]
        \& \Fun^{\aCart}(\sd(\Lambda^{2}_{0}), \category{D}) \times_{\Fun^{\aCart}(\sd(\Delta^{\{0, 1\}}), \category{D})} \{\pi e\}
      \end{tikzcd}
    }
  \end{equation*}
  is homotopy pullback.
\end{proposition}
\begin{proof}
  This square factors horizontally into the two squares
  \begin{equation*}
    \noindent\makebox[\textwidth]{%
      \begin{tikzcd}[ampersand replacement=\&, column sep=tiny]
        \Fun^{\aCart}(\sd(\Delta^{2}), \category{C}) \times_{\Fun(\sd(\Delta^{\{0, 1\}}), \category{C})} \{e\}
        \arrow[r]
        \arrow[d]
        \& \Fun'(A_{6}, \category{C}) \times_{\Fun(\sd(\Delta^{\{0, 1\}}), \category{C})} \{e\}
        \arrow[d]
        \\
        \Fun^{\aCart}(\sd(\Delta^{2}), \category{D}) \times_{\Fun(\sd(\Delta^{\{0, 1\}}), \category{D})} \{\pi e\}
        \arrow[r]
        \& \Fun'(A_{6}, \category{D}) \times_{\Fun(\sd(\Delta^{\{0, 1\}}), \category{C})} \{\pi e\}
      \end{tikzcd}
    }
  \end{equation*}
  and
  \begin{equation*}
    \noindent\makebox[\textwidth]{%
      \begin{tikzcd}[ampersand replacement=\&, column sep=tiny]
        \Fun'(A_{6}, \category{C}) \times_{\Fun(\sd(\Delta^{\{0, 1\}}), \category{C})} \{e\}
        \arrow[r]
        \arrow[d]
        \& \Fun'(A_{1}, \category{C}) \times_{\Fun(\sd(\Delta^{\{0, 1\}}), \category{C})} \{e\}
        \arrow[d]
        \\
        \Fun'(A_{6}, \category{D}) \times_{\Fun(\sd(\Delta^{\{0, 1\}}), \category{C})} \{\pi e\}
        \arrow[r]
        \& \Fun'(A_{1}, \category{D}) \times_{\Fun(\sd(\Delta^{\{0, 1\}}), \category{C})} \{\pi e\}
      \end{tikzcd}.
    }
  \end{equation*}

  The first is homotopy pullback because the bottom map is an inclusion of connected components, and the second condition of \hyperref[thm:main]{Theorem~\ref*{thm:main}} guarantees that the fiber over an ambigressive $\sd(\Delta^{2}) \to \category{D}$ belonging to $\Fun'(A_{6}, \category{D})$ whose restriction to $\sd(\Delta^{\{0, 1\}})$ is $\pi e$ is precisely an ambigressive Cartesian functor $\sd(\Delta^{2}) \to \category{C}$ whose restriction to $\sd(\Delta^{\{0, 1\}})$ is $e$.

  That the second is homotopy pullback follows immediately from \hyperref[lemma:homotopy_pullback_for_filling_procedure]{Lemma~\ref*{lemma:homotopy_pullback_for_filling_procedure}}.
\end{proof}

This proves \hyperref[thm:main]{Theorem~\ref*{thm:main}}.

\end{document}
