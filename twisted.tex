\documentclass[main.tex]{subfiles}

\begin{document}

\section{Local systems}
\label{sec:local_systems}

For any $\infty$-category $\category{C}$, a $\category{C}$-local system on some space $X$ is simply a functor $X \to \category{C}$. Thus, the $\infty$-category of $\category{C}$-local systems on $X$ is simply $\LS(\category{C})_{X}:=\Fun(X, \category{C})$. We would like to consider local systems on all spaces $X$ simultaneously, defining an $\infty$-category $\LS(\category{C})$ of $\category{C}$-local systems.

We follow the general strategy laid out in \cite{luriehopkins2013ambidexterity}. We will consider a cartesian fibration 
\begin{equation*}
  p'\colon\smallint\Fun(-, -) \to \ICat \times \ICat\op
\end{equation*}
which classifies the functor
\begin{equation*}
  \Fun(-,-)\colon \ICat\op \times \ICat \to \ICat;\qquad (\category{C}, \category{D}) \mapsto \Fun(\category{C}, \category{D}).
\end{equation*}
The strict pullback $p$ of $p'$ in the diagram
\begin{equation*}
  \begin{tikzcd}
    \LS(\category{C})
    \arrow[r]
    \arrow[d, swap, "p"]
    & \int\Fun(-, -)
    \arrow[d, "p'"]
    \\
    \S \times \{\category{C}\}
    \arrow[r]
    & \ICat \times \ICat\op
  \end{tikzcd}
\end{equation*}
will then classify the functor
\begin{equation*}
  \S \to \ICat;\qquad X \mapsto \LS(\category{C})_{X}.
\end{equation*}
The total space $\LS(\category{C})$ of $p$ will thus be our candidate for our $\infty$-category of local systems. The fibration $p$ remembers the source of the local system.

In \cite{garcia2020enhanced}, a suitable model for $\int\Fun(-, -)$ is given: the so-called \emph{enhanced twisted arrow category} $\Tw(\ICCat)$. In \hyperref[ssc:the_twisted_arrow_category]{Subsection~\ref*{ssc:the_twisted_arrow_category}}, we reproduce the pertinent points of \cite{garcia2020enhanced}, defining the enhanced twisted arrow category. In \hyperref[ssc:the_infinity_category_of_local_systems]{Subsection~\ref*{ssc:the_infinity_category_of_local_systems}}, we use the enhanced twisted arrow category to define the category $\LS(\category{C})$ of local systems, together with a cartesian fibration $p\colon \LS(\category{C}) \to \S$ classifying the functor
\begin{equation*}
  \S\op \to \ICat;\qquad X \to \Fun(X, \category{C}).
\end{equation*}
We note that this functor sends a map of spaces $f\colon X \to Y$ to the pullback functor
\begin{equation*}
  f^{*}\colon \Fun(Y, \category{C}) \to \Fun(X, \category{C}).
\end{equation*}
If $\category{C}$ is cocomplete, each functor $f^{*}$ has a left adjoint $f_{!}$ given by left Kan extension. By abstract nonsense, the cartesian fibration $p$ is also a cocartesian fibration, whose cocartesian edges correspond to left Kan extension. In \hyperref[ssc:the_fibration]{Subsection~\ref*{ssc:the_fibration}} we will show this explicitly, using results built up in \hyperref[ssc:two_lemmas_about_marked_scaled_anodyne_morphisms]{Subsection~\ref*{ssc:two_lemmas_about_marked_scaled_anodyne_morphisms}} and \hyperref[ssc:the_infinity_category_of_local_systems]{Subsection~\ref*{ssc:the_infinity_category_of_local_systems}}; if the reader is willing to take this on faith, these sections can be safely skipped.

\begin{note}
  The reader may notice that we are being inefficient here. If we were only interested in the construction outlined above, then using the twisted arrow category would be much more combinatorially strenuous than necessary; the objects of the twisted arrow category $\Tw(\ICCat)$ are functors of $\infty$-categories $\category{D} \to \category{C}$, with $\category{C}$ and $\category{D}$ both allowed to vary. If, in the very next breath, we fix the target $\category{C}$, then it makes more sense to use some sort of overcategory $(\ICCat)_{/\category{C}}$ from the beginning. There is a method to our madness; in \hyperref[sec:the_monoidal_construction]{Section~\ref*{sec:the_monoidal_construction}}, we will introduce a monoidal version of this construction, and here we will need to allow the targets of our functors to vary.
\end{note}

\subsection{The enhanced twisted arrow category}
\label{ssc:the_twisted_arrow_category}

For a 2-category $\mathsf{C}$, the twisted arrow 1-category $\Tw(\mathsf{C})$ has the following description.
\begin{itemize}
  \item The objects of $\Tw(\mathsf{C})$ are the morphisms of $\mathsf{C}$:
    \begin{equation*}
      f\colon c \to c'.
    \end{equation*}

  \item For two objects $(f\colon c \to c')$ and $(g\colon d \to d')$, the morphisms $f \to g$ are given by diagrams
    \begin{equation*}
      \begin{tikzcd}
        c
        \arrow[r, "a"]
        \arrow[d, "f"{swap}, ""{name=L}]
        & d
        \arrow[d, "g", ""{name=R, swap}]
        \\
        c'
        & d'
        \arrow[l, "a'"]
        \arrow[from=L, to=R, Rightarrow, shorten=2ex, "\alpha"]
      \end{tikzcd},
    \end{equation*}
    where $\alpha$ is a 2-morphism $f \Rightarrow a' \circ g \circ a$.


  \item The composition of morphisms is given by concatenating the corresponding diagrams.
    \begin{equation*}
      \begin{tikzcd}
        c
        \arrow[r, "a"]
        \arrow[d, "f"{swap}, ""{name=L}]
        & d
        \arrow[d, "g"{description}, ""{name=LM}, ""{name=RM, swap}]
        \arrow[r, "b"]
        & e
        \arrow[d, "h", ""{name=R, swap}]
        \\
        c'
        & d'
        \arrow[l, "a'"]
        & e'
        \arrow[l, "b'"]
        \arrow[from=L, to=RM, Rightarrow, shorten=2ex, "\alpha"]
        \arrow[from=LM, to=R, Rightarrow, shorten=2ex, "\beta"]
      \end{tikzcd}
    \end{equation*}
\end{itemize}

In \cite{garcia2020enhanced}, a homotopy-coherent version of the twisted arrow category of an $\infty$-bicategory is defined. For any $\infty$-bicategory $\CC$, the twisted arrow $\infty$-category of $\CC$, denoted $\Tw(\CC)$, is a quasicategory whose $n$-simplices are diagrams $\Delta^{n} \star (\Delta^{n})\op \to \CC$, together with a scaling to ensure that the information encoded in an $n$-simplex is determined, up to contractible choice, by data
\begin{equation*}
  \overbrace{
    \begin{tikzcd}[ampersand replacement=\&]
      c_{0}
      \arrow[r, "a_{0}"]
      \arrow[d, "f_{0}"{swap}, ""{name=L1}]
      \& c_{1}
      \arrow[d, "f_{1}"{description}, ""{swap, name=R1}]
      \arrow[r, dotted]
      \& \cdots
      \arrow[r, dotted]
      \& c_{n-1}
      \arrow[r, "a_{n-1}"]
      \arrow[d, "f_{n-1}"{description}, ""{name=Ln}]
      \& c_{n}
      \arrow[d, "f_{n}", ""{swap, name=Rn}]
      \\
      c'_{0}
      \& c'_{1}
      \arrow[l, "a'_{0}"]
      \& \cdots
      \arrow[l, dotted]
      \& c'_{n-1}
      \arrow[l, dotted]
      \& c'_{n}
      \arrow[l, "a'_{n-1}"]
      \arrow[from=L1, to=R1, Rightarrow, "\alpha_{0}", shorten=2ex]
      \arrow[from=Ln, to=Rn, Rightarrow, "\alpha_{n-1}", shorten=2ex]
    \end{tikzcd}
  }^{n\text{ squares}}
\end{equation*}
in $\CC$. Here, the top row of morphisms corresponds to the spine of the $n$-simplex $\Delta^{n} \subset \Delta^{n} \star (\Delta^{n})\op$, and the bottom row of morphisms to the spine of $(\Delta^{n})\op \subset \Delta^{n} \star (\Delta^{n})\op$. To make this correspondence more clear, we will introduce the following notation.

\begin{notation}
  For $i \in [n] \subset [2n+1]$, we will write $\overline{i} := 2n+1-i$. Thus, $\overline{0} = 2n+1$, $\overline{1} = 2n$, etc.
\end{notation}

Thus, the $i$th column in the above diagram corresponds to the image of the morphism $i \to \overline{i}$ in $\Delta^{n} \star (\Delta^{n})\op$.

The scaling mentioned above is defined in the following way.

\begin{definition}
  We define a cosimplicial object $\tilde{Q}\colon \Delta \to \SSetsc$ by sending
  \begin{equation*}
    \tilde{Q}([n]) = (\Delta^{n} \star (\Delta^{n})\op, \dagger),
  \end{equation*}
  where $\dagger$ is the scaling consisting of all degenerate 2-simplices, together with all 2-simplices of the following kinds:
  \begin{enumerate}
    \item All simplices $\Delta^{2} \to \Delta^{n} \star (\Delta^{n})\op$ factoring through $\Delta^{n}$.

    \item All simplices $\Delta^{2} \to \Delta^{n} \star (\Delta^{n})\op$ factoring through $(\Delta^{n})\op$.

    \item All simplices $\Delta^{\{i, j, \overline{k}\}} \subseteq \Delta^{n} \star (\Delta^{n})\op$, $i < j \leq k$.

    \item All simplices $\Delta^{\{k, \overline{j}, \overline{i}\}} \subseteq \Delta^{n} \star (\Delta^{n})\op$, $i < j \leq k$.
  \end{enumerate}
\end{definition}

Note that $\Delta^{n} \star (\Delta^{n})\op \cong \Delta^{2n+1}$. We will use this identification freely.

This extends to a nerve-realization adjunction
\begin{equation*}
  Q : \SSet \longleftrightarrow \SSetsc : \Tw,
\end{equation*}
where the functor $Q$ is the extension by colimits of the functor $\tilde{Q}$, and for any scaled simplicial set $X$, the simplicial set $\Tw(X)$ has $n$-simplices
\begin{equation*}
  \Tw(\CC)_{n} = \Hom_{\SSetsc}(Q(\Delta^{n}), X).
\end{equation*}

\begin{notation}
  For any simplicial set $X$, $Q(X)$ carries the scaling given simplex-wise by that described above. We will denote this also by $\dagger$. Using the identification $\Delta^{n} \star (\Delta^{n})\op \cong \Delta^{2n+1}$, we can write $Q(\Delta^{n})$ explicitly and compactly as $\Delta^{2n+1}_{\dagger}$, which we will often do.
\end{notation}

The inclusion $\Delta^{n} \amalg (\Delta^{n})\op \hookrightarrow \Delta^{n} \star (\Delta^{n})\op$ provides, for any $\infty$-bicategory $\CC$ with underlying $\infty$-category $\category{C}$, a morphism of simplicial sets
\begin{equation*}
  \Tw(\CC) \to \category{C} \times \category{C}\op.
\end{equation*}
In \cite{garcia2020enhanced}, the following is shown.

\begin{theorem}
  \label{thm:mainthm_walker_fernando}
  Let $\CC$ be an $\infty$-bicategory (presented as a fibrant scaled simplicial set), and let $\category{C}$ be the underlying $\infty$-category (presented as a quasicategory). Then the map
  \begin{equation*}
    p_{\CC}\colon \Tw(\CC) \to \category{C} \times \category{C}\op.
  \end{equation*}
  is a cartesian fibration between quasicategories, and a morphism $f\colon \Delta^{1} \to \Tw(\CC)$ is $p_{\CC}$-cartesian if and only if the morphism $\sigma\colon \Delta^{3}_{\dagger} \to \CC$ to which it is adjunct is fully scaled, i.e.\ factors through the map $\Delta^{3}_{\dagger} \to \Delta^{3}_{\sharp}$. Furthermore, the cartesian fibration $p_{\CC}$ classifies the functor
  \begin{equation*}
    \Map_{\CC}(-, -)\colon \category{C}\op \times \category{C} \to \ICat.
  \end{equation*}
\end{theorem}

%In fact, this assignment is functorial.
%
%\begin{proposition}
%  This extends to a functor $\ICCat \to \ICat^{\Delta^{1}}$.
%\end{proposition}

\subsection{Two lemmas about marked-scaled anodyne morphisms}
\label{ssc:two_lemmas_about_marked_scaled_anodyne_morphisms}

In the section following this one, we provide a fairly explicit construction of a family of marked-scaled anodyne inclusions. Constructing these inclusions directly from the classes of generating marked-scaled anodyne inclusions given in \hyperref[def:ms-anodyne_morphisms]{Definition~\ref*{def:ms-anodyne_morphisms}} would be possible but tedious. To lighten this load somewhat, we reproduce in this section two lemmas from \cite{garcia2cartesianfibrationsii}. The majority of this section is a retelling of \cite[Sec.~2.3~and~2.4]{garcia2cartesianfibrationsii}. Note however that our needs will be rather different than those of \cite{garcia2cartesianfibrationsii}, and this is reflected in some minor differences of notation.

We first need a compact notation for specifying simplicial subsets of $\Delta^{n}$. Denote by $P\colon \Set \to \Set$ the power set functor.

\begin{definition}
  Let $T$ be a finite linearly ordered set, so $T \cong [n]$ for some $n \in \N$. For any subset $\mathcal{A} \subseteq  P(T)$, define a simplicial subset
  \begin{equation*}
    \S^{\mathcal{A}}_{T} := \bigcup_{S \in \mathcal{A}} \Delta^{T \smallsetminus S} \subseteq \Delta^{T}.
  \end{equation*}
\end{definition}

\begin{notation}
  If the linearly ordered set $T$ is clear from context, we will drop it, writing $\S^{\mathcal{A}}$.
\end{notation}

\begin{note}
  \label{note:correspondence_simplicial_subsets_power_set}
  The assignment $\mathcal{A} \mapsto \S^{\mathcal{A}}$ does not induce is not a one-to-one correspondence between subsets $\mathcal{A} \subseteq P([n])$ and simplicial subsets $S^{\mathcal{A}} \subseteq \Delta^{n}$, since for any two subsets $S \subsetneq S' \subset [n]$, we have that $\Delta^{[n] \smallsetminus S'} \subsetneq \Delta^{[n] \smallsetminus S}$.
\end{note}

It will be useful to consider two subsets $\mathcal{A}$ and $\mathcal{A}' \subseteq P(T)$ to be equivalent if they produce the same simplicial subset of $\Delta^{T}$.

\begin{definition}
  We will write $\mathcal{A} \sim \mathcal{A}'$ if $\S^{\mathcal{A}} = \S^{\mathcal{A}'}$.
\end{definition}

The relation $\sim$ is an equivalence relation, but we will not use this.

The set $P([n])$ forms a poset, ordered by inclusion, and any $\mathcal{A} \subseteq P([n])$ a subposet. An element $q$ of a poset $Q$ is said to be \emph{minimal} if there are no elements of $Q$ which are strictly less than $q$. By \hyperref[note:correspondence_simplicial_subsets_power_set]{Note~\ref*{note:correspondence_simplicial_subsets_power_set}} only the minimal elements of $\mathcal{A}$ contribute to the union defining $\S^{\mathcal{A}}$. We have just shown the following.

\begin{lemma}
  \label{lemma:replace_poset_by_minimal_elements}
  For any $\mathcal{A} \subseteq P([n])$, we have $\mathcal{A} \sim \mathrm{min}(\mathcal{A})$, where $\mathrm{min}(\mathcal{A}) \subseteq \mathcal{A}$ is the set of minimal elements of $\mathcal{A}$.
\end{lemma}

Simplicial subsets of the form $\S^{\mathcal{A}}_{T}$ enjoy the following easily-proved calculation rules.

\begin{lemma}
  \label{lemma:subset_of_simplex_contains_k_simplices}
  Suppose $\mathcal{A} \subseteq P([n])$ is a subset such that for all $S$, $T \in \mathcal{A}$, it holds that $S \cap T = \emptyset$. Then $\S^{\mathcal{A}}$ contains each $k$-simplex of $\Delta^{n}$ for all $k < \abs{\mathcal{A}} - 1$.
\end{lemma}
\begin{proof}
  Let $X \subseteq [n]$. The simplex $\Delta^{X} \to \Delta^{n}$ factors through $\S^{\mathcal{A}}$ if and only if it factors through $\Delta^{[n] \smallsetminus T}$ for some (possibly not unique) $T \in \mathcal{A}$. This in turn is true if and only if $X$ and $T$ do not have any elements in common. For $\abs{X} < \abs{\mathcal{A}}$, there is always a set $T \in \mathcal{A}$ which does not have any elements in common with $X$.
\end{proof}

\begin{lemma}
  \label{lemma:add_a_simplex}
  Let $\mathcal{A} \subseteq P([n])$ and $T \subseteq [n]$. Then
  \begin{equation*}
    \S^{\mathcal{A}}_{[n]} \cup \Delta^{T} = \S^{\mathcal{A} \cup \{[n] \smallsetminus T\}}_{[n]}.
  \end{equation*}
\end{lemma}

The next lemma will be particularly useful in building inclusions $\S^{\mathcal{A}}_{[n]} \hookrightarrow \Delta^{n}$ simplex-by-simplex.

\begin{lemma}
  \label{lemma:bicartesian_square}
  Let $\mathcal{A} \subset P([n])$, and let $T \subset [n]$. Then the square
  \begin{equation*}
    \begin{tikzcd}
      \S^{\mathcal{A}|T}_{T}
      \arrow[r, hook]
      \arrow[d, hook]
      & \Delta^{T}
      \arrow[d, hook]
      \\
      \S^{\mathcal{A}}_{[n]}
      \arrow[r, hook]
      & \S^{\mathcal{A}}_{[n]} \cup \Delta^{T}
    \end{tikzcd}
  \end{equation*}
  is bicartesian, where $\mathcal{A}|T$ is the subset of $P(T)$ given by
  \begin{equation*}
    \mathcal{A}|T = \{S \cap T \mid S \in \mathcal{A}\}.
  \end{equation*}
\end{lemma}
\begin{proof}
  We have an equality
  \begin{equation*}
    \mathcal{S}_{T}^{\mathcal{A}|T} = \S^{\mathcal{A}}_{[n]} \cap \Delta^{T}.
  \end{equation*}
  as subsets of $\Delta^{n}$.
\end{proof}

\begin{notation}
  For any marked-scaled $n$-simplex $(\Delta^{n}, E, T)$ we will by minor abuse of notation reuse the same letters $E$ and $T$ to denote the restriction of the markings and scalings to any simplicial subset $S \subseteq \Delta^{n}$.
\end{notation}

In the remainder of this section we reproduce two lemmas from \cite{garcia2cartesianfibrationsii} which provide criteria for the inclusion $(\S^{\mathcal{A}}, E, T) \subseteq (\Delta^{n}, E, T)$ to be marked-scaled anodyne.

\begin{definition}
  Let $\mathcal{A} \subseteq P([n])$. We call $X \in P([n])$ an \defn{$\mathcal{A}$-basal set} if it contains precisely one element from each $S \in \mathcal{A}$. We denote the set of all $\mathcal{A}$-basal sets by $\Bas(\mathcal{A})$.
\end{definition}

\begin{definition}[\cite{garcia2020enhanced}, Definition 1.3]
  \label{def:inner_dull}
  We will call a subset $\mathcal{A} \subseteq P([n])$ \defn{inner dull} if it satisfies the following conditions.
  \begin{itemize}
    \item It does not include the empty set; $\emptyset \notin \mathcal{A}$.

    \item There exists $0 < i < n$ such that for all $S \in \mathcal{A}$, we have $i \notin S$.

    \item For every $S$, $T \in \mathcal{A}$, it follows that $S \cap T = \emptyset$.

    \item For each $\mathcal{A}$-basal set $X \in P([n])$, there exist $u$, $v \in X$ such that $u < i < v$.
  \end{itemize}

  The element $i \in [n]$ is known as the \emph{pivot point.}
\end{definition}

\begin{note}
  The last condition of \hyperref[def:inner_dull]{Definition~\ref*{def:inner_dull}} is always satisfied if $\mathcal{A}$ contains two singletons $\{u\}$ and $\{v\}$ such that $u < i < v$.
\end{note}

\begin{definition}
  Let $\mathcal{A} \subseteq P([n])$ be an inner dull subset with pivot point $i$, and let $X \in \mathcal{A}$. We will denote the adjacent elements of $X$ surrounding $i$ by $\ell^{X} < i < u^{X}$.
\end{definition}


\begin{lemma}[The Pivot Trick: \cite{garcia2cartesianfibrationsii}, Lemma 2.3.5]
  \label{lemma:pivot_trick}
  Let $\mathcal{A} \subseteq P([n])$ be an inner dull subset with pivot point $i$, and let $(\Delta^{n}, E, T)$ be a marked-scaled simplex. Further suppose that the following conditions hold:
  \begin{enumerate}
    \item Every marked edge $e \in E$ which does not contain $i$ factors through $\S^{\mathcal{A}}$.

    \item For all scaled simplices $\sigma = \{a < b < c\}$ of $\Delta^{n}$ which do not factor through $\S^{\mathcal{A}}$, and which do not contain the pivot point $i$, we have $a < i < c$, and $\sigma \cup \{i\}$ is fully scaled.

    \item For all $X \in \Bas(\mathcal{A})$ and all $r$, $s \in [n]$ such that $\ell^{X} \leq r < i < s \leq u^{X}$, the triangle $\{r, i, s\}$ is scaled
  \end{enumerate}

  Then the inclusion $(\mathcal{S}^{\mathcal{A}}, E, T) \hookrightarrow (\Delta^{n}, E, T)$ is marked-scaled anodyne.
\end{lemma}

\begin{definition}
  We will call a subset $\mathcal{A} \subseteq P([n])$ \defn{right dull} if it satisfies the following conditions.
  \begin{itemize}
    \item It is nonempty; $\mathcal{A} \neq \emptyset$.

    \item It does not include the empty set; $\emptyset \notin \mathcal{A}$.

    \item For all $S \in \mathcal{A}$, we have $n \notin S$.

    \item For every $S$, $T \in \mathcal{A}$, it follows that $S \cap T = \emptyset$.
  \end{itemize}

  In this case, we will refer to $n$ as the pivot point.
\end{definition}

\begin{lemma}[Right-anodyne pivot trick]
  \label{lemma:right-anodyne_pivot_trick}
  Let $\mathcal{A} \subset P([n])$ be a right dull subset (whose pivot point is by definition $n$), and let $(\Delta^{n}, E, T)$ be a marked-scaled simplex. Further suppose that the following conditions hold:
  \begin{itemize}
    \item Every marked edge $e \in E$ which does not contain $n$ factors through $\S^{\mathcal{A}}$.

    \item Let $\sigma = \{a < b < c\}$ be a scaled simplex not containing $n$. Then either $\sigma$ factors through $\S^{\mathcal{A}}$, or $\sigma \cup \{n\}$ is fully scaled, and $c \to n$ is marked.

    \item For all Z in  $\Bas(\category{A})$, For all $r$, $s \in [n]$ with $r \leq \min(Z) \leq \max(Z) \leq s < n$, the triangle $\Delta^{\{r, s, n\}}$ is scaled, and the simplex $\Delta^{\{s, n\}}$ is marked.
  \end{itemize}

  Then the inclusion $(\mathcal{S}^{\mathcal{A}}, E, T) \hookrightarrow (\Delta^{n}, E, T)$ is marked-scaled anodyne.
\end{lemma}

\subsection{The infinity-category of local systems}
\label{ssc:the_infinity_category_of_local_systems}

In \hyperref[ssc:the_twisted_arrow_category]{Subsection~\ref*{ssc:the_twisted_arrow_category}}, we defined the twisted arrow category $\Tw(\CC)$ of an $\infty$-bicategory $\CC$, and noted that the natural cartesian fibration $\Tw(\CC) \to \category{C} \times \category{C}\op$ classifies the mapping functor $\Map_{\CC}(-, -)$. Thus, the objects $\Tw(\CC)$ are simply the morphisms in $\CC$.

We can use this to define, for any cocomplete $\infty$-category $\category{C}$, the $\infty$-category of $\category{C}$-local systems; we simply consider $\Tw(\ICCat)$, whose objects are all functors between $\infty$-categories, and restrict to those functors whose domain is an $\infty$-groupoid, and whose codomain is $\category{C}$.

\begin{definition}
  For any cocomplete $\infty$-category $\category{C}$, we define the $\infty$-category of $\category{C}$-local systems $\LS(\category{C})$ together with a map of simplicial sets $p\colon \LS(\category{C}) \to \S$ by the following pullback diagram.
  \begin{equation}
    \label{eq:pullback_square_defining_local_systems}
    \begin{tikzcd}
      \LS(\category{C})
      \arrow[r, hook]
      \arrow[d, swap, "p"]
      & \Tw(\ICCat)
      \arrow[d, "p'"]
      \\
      \S \times \{\category{C}\}
      \arrow[r, hook]
      & \ICat \times \ICat\op
    \end{tikzcd}
  \end{equation}
\end{definition}

\begin{note}
  It is natural to wonder if we are making life unnecessarily difficult by not simply considering the category $\S \times_{\ICCat}(\ICCat)_{/\category{C}}$, which is after all also an $\infty$-category of functors from spaces into $\category{C}$. The reason for the more complicated construction given here will become apparent in \hyperref[ssc:the_monoidal_twisted_arrow_category]{Subsection~\ref*{ssc:the_monoidal_twisted_arrow_category}}, when we define the monoidal structure on $\LS(\category{C})$.
\end{note}

It is shown in \cite{garcia2020enhanced} that $p'$, hence also $p$, is a cartesian fibration. The cartesian fibration $p$ classifies the functor $\S\op \to \ICat$ sending
\begin{equation*}
  f\colon X \to Y \qquad \longmapsto \qquad f^{*}\colon \Fun(Y, \category{C}) \to \Fun(X, \category{C}).
\end{equation*}
Under the assumption that $\category{C}$ is cocomplete, each pullback map $f^{*}$ has a left adjoint $f_{!}$, given by left Kan extension. It follows on abstract grounds that $p$ is also a cocartesian fibration, whose cocartesian edges represent left Kan extension. In \hyperref[ssc:the_fibration]{Subsection~\ref*{ssc:the_fibration}}, we show this explicitly. Our goal in this section is to do some legwork to facilitate the proof of this result.

In investigating the map $p\colon \LS(\category{C}) \to \S$, it will be useful to factor the pullback square in \hyperref[eq:pullback_square_defining_local_systems]{Equation~\ref*{eq:pullback_square_defining_local_systems}} into the two pullback squares
\begin{equation*}
  \begin{tikzcd}
    \LS(\category{C})
    \arrow[r, hook]
    \arrow[d, swap, "p"]
    & \mathcal{R}
    \arrow[r, hook]
    \arrow[d, "p''"]
    & \Tw(\ICCat)
    \arrow[d, "p'"]
    \\
    \S \times \{\category{C}\}
    \arrow[r, hook]
    & \ICat \times [\category{C}]
    \arrow[r, hook]
    & \ICat \times \ICat\op
  \end{tikzcd},
\end{equation*}
where $[\category{C}]$ denotes the path component of $\category{C}$ in $( \ICat\op )^{\simeq}$. In order to show that $p$ is a cocartesian fibration, it will help us to understand the map $p''$. The $n$-simplices of $\category{R}$ are given by maps
\begin{equation*}
  Q(\Delta^{n}) = (\Delta^{n} \star (\Delta^{n})\op)_{\dagger} \to \ICCat
\end{equation*}
such that
\begin{itemize}
  \item each object in $(\Delta^{n})\op \subseteq Q(\Delta^{n})$ is sent to a quasicategory which is equivalent to $\category{C}$ (and in particular cocomplete), and

  \item each morphism in $(\Delta^{n})\op \subseteq Q(\Delta^{n})$ is mapped to an equivalence in $\ICCat$.
\end{itemize}
We can more usefully encode the second condition by endowing $\ICCat$ and $Q$ with a marking.

\begin{definition}
  We define the following markings.
  \begin{itemize}
    \item We denote by $\ICCat^{\natural}$ the marked-scaled simplicial set whose underlying scaled simplicial set is $\ICCat$, and where all equivalences have been marked.

    \item We denote by $\heartsuit$ the marking on $\Delta^{n} \star (\Delta^{n})\op$ consisting of all morphisms belonging to $(\Delta^{n})\op$, and by $(\Delta^{n} \star (\Delta^{n})\op)^{\heartsuit}_{\dagger}$ the corresponding marked-scaled simplicial set.
  \end{itemize}
\end{definition}

\begin{definition}
  \label{def:cosimplicial_obj_R}
  We define a cosimplicial object
  \begin{equation*}
    \tilde{R}\colon \Delta \to \SSetms;\qquad [n] \mapsto (\Delta^{n} \star (\Delta^{n})\op)^{\heartsuit}_{\dagger}.
  \end{equation*}

  We denote the extension of $\tilde{R}$ by colimits by
  \begin{equation*}
    R\colon \SSet \to \SSetms;\qquad X \mapsto \colim_{\Delta^{n} \to X} R(n).
  \end{equation*}
\end{definition}

The marked-scaled simplicial sets $R(\Delta^{n})$ are rather complicated, and showing that $p''$ is a cartesian fibration by solving the necessary lifting problems explicitly would be impractical. We will instead replace $R(\Delta^{n})$ by something simpler, considering the simplicial subsets coming from the inclusions
\begin{equation}
  \label{eq:inclusion_j_into_r}
  \Delta^{n} \star \Delta^{\{\overline{0}\}} \subseteq \Delta^{n} \star (\Delta^{n})\op,
\end{equation}
which we understand to inherit the marking and scaling.

\begin{definition}
  \label{def:cosimplicial_obj_J}
  We will denote by $\tilde{J}\colon \Delta \to \SSetms$ the cosimplicial object
  \begin{equation*}
    \tilde{J}\colon \Delta \to \SSetms;\qquad [n] \mapsto (\Delta^{n} \star \Delta^{\{\overline{0}\}})^{\heartsuit}_{\dagger},
  \end{equation*}
  and by $J$ the extension by colimits
  \begin{equation*}
    J\colon \SSet \to \SSetms;\qquad X \mapsto \colim_{\Delta^{n} \to X} \tilde{J}(n).
  \end{equation*}
\end{definition}

The inclusions of \hyperref[eq:inclusion_j_into_r]{Equation~\ref*{eq:inclusion_j_into_r}} induce for each $n \geq 0$ an inclusion of marked-scaled simplicial sets
\begin{equation*}
  v_{n}\colon \tilde{J}(n) \to \tilde{R}(n).
\end{equation*}

\begin{lemma}
  \label{lemma:lower_morphism_equivalence}
  For all $n \geq 0$, the map $v_{n}$ is marked-scaled anodyne, hence a weak equivalence.
\end{lemma}

We postpone the proof of \hyperref[lemma:lower_morphism_equivalence]{Lemma~\ref*{lemma:lower_morphism_equivalence}} until the end of this section.

\begin{note}
  \label{note:not_natural}
  Some care is warranted: the $v_{n}$ are not the components of a natural transformation $\tilde{J} \Rightarrow \tilde{R}$! The necessary squares simply do not commute for morphisms $\phi\colon [m] \to [n]$ in $\Delta$ such that $\phi(0) \neq 0$. This means that we do not, for a general simplicial set $X$, get a canonical weak equivalence of marked-scaled simplicial sets $J(X) \to R(X)$.
\end{note}

Despite the warning given in \hyperref[note:not_natural]{Note~\ref*{note:not_natural}}, it is still possible to get maps $J(X) \to R(X)$ in some cases. In the remainder of this section, we check that we can produce a marked-scaled weak equivalence $J(\Lambda^{n}_{0}) \to R(\Lambda^{n}_{0})$.

\begin{notation}
  Denote by $\mathring{\Delta}$ the subcategory of $\Delta$ on morphisms $[m] \to [n]$ which send $0 \mapsto 0$. Denote by $I$ the inclusion $\mathring{\Delta} \hookrightarrow \Delta$.
\end{notation}

It is easy to check the following.
\begin{lemma}
  \label{lemma:restricted_natural_transformation}
  The morphisms $v_{n}$ form a natural transformation $v\colon \tilde{J} \circ I \Rightarrow \tilde{R} \circ I$, each of whose components is an equivalence in the model structure on marked-scaled simplicial sets.
\end{lemma}

\begin{lemma}
  \label{lemma:marked_scaled_equivalence_left_horn}
  For all $n \geq 1$, there is a marked-scaled equivalence $J(\Lambda^{n}_{0}) \to R(\Lambda^{n}_{0})$.
\end{lemma}
\begin{proof}
  We can write $J(\Lambda^{n}_{0})$ as a colimit of the composition $\tilde{J} \circ b$ coming from the bottom of the diagram
  \begin{equation*}
    \begin{tikzcd}
      P_{n}
      \arrow[d, hook]
      \arrow[r, "a"]
      & \mathring{\Delta}
      \arrow[d, swap, "I"]
      \arrow[dr, "\tilde{J} \circ I"]
      \\
      (\Delta \downarrow \Lambda^{n}_{0})^{\mathrm{nd}}
      \arrow[r, "b"]
      & \Delta
      \arrow[r, "\tilde{J}"]
      &\SSetms
    \end{tikzcd},
  \end{equation*}
  where $(\Delta \downarrow \Lambda^{n}_{0})^{\mathrm{nd}}$ is the category of nondegenerate simplices of $\Lambda^{n}_{0}$. Denote by $P_{n}$ the poset of proper subsets of $[n]$ such that $0 \in S$. There is an obvious inclusion $P_{n} \hookrightarrow (\Delta \downarrow \Lambda^{n}_{0})^{\mathrm{nd}}$, and one readily checks using Quillen's Theorem A that this inclusion is cofinal. Further note that the functor $P_{n} \to (\Delta \downarrow \Lambda^{n}_{0}) \to \Delta$ factors through $\mathring{\Delta}$ via a map $a\colon P_{n} \to \mathring{\Delta}$. Thus, we can equally express $J(\Lambda^{n}_{0})$ as the colimit of the functor $\tilde{J} \circ I \circ a$. Precisely the same reasoning tells us that we can express $R(\Lambda^{n}_{0})$ as the colimit of $\tilde{R} \circ I \circ a$. The result now follows from \hyperref[lemma:restricted_natural_transformation]{Lemma~\ref*{lemma:restricted_natural_transformation}}, and the fact that each strict colimit is the model for the homotopy colimit.
\end{proof}

\begin{proof}[Proof of Lemma \ref{lemma:lower_morphism_equivalence}]
  In this proof, all simplicial subsets of $\Delta^{2n+1}$ will be assumed to carry the marking $\heartsuit$ and the scaling $\dagger$.

  We can write each $v_{n}$ as a composition
  \begin{equation*}
    \Delta^{\{0, \ldots, n, \overline{0}\}} \overset{v''_{n}}{\hookrightarrow} \Delta^{\{0, \ldots, n, \overline{0}\}} \cup \Delta^{\{\overline{n}, \ldots, \overline{0}\}} \overset{v'_{n}}{\hookrightarrow} \Delta^{2n+1}.
  \end{equation*}
  Here, the map $v_{n}''$ is a pushout along the inclusion $(\Delta^{\{n\}})^{\sharp}_{\sharp} \hookrightarrow (\Delta^{n})^{\sharp}_{\sharp}$, which is marked-scaled anodyne by \hyperref[prop:sharp_marked_right_anodyne]{Proposition~\ref*{prop:sharp_marked_right_anodyne}}. It remains to show that each of the maps $v_{n}'$ is marked-scaled anodyne.

  To this end, we introduce some notation. Let $M^{n}_{0} = \Delta^{\{0, \ldots, n, \overline{0}\}} \cup \Delta^{\{\overline{n}, \ldots \overline{0}\}} \subset \Delta^{2n+1}$, and for $1 \leq k \leq n$, define
  \begin{equation*}
    M^{n}_{k} := M^{n}_{0} \cup \left(\bigcup_{\ell = 1}^{k} \Delta^{[2n+1] \smallsetminus \{\ell, \overline{\ell}\}}\right).
  \end{equation*}
  There is an obvious filtration
  \begin{equation}
    \label{eq:filtration_by_adding_sides}
    M^{n}_{0} \overset{i^{n}_{0}}{\hookrightarrow} M^{n}_{1} \overset{i^{n}_{1}}{\hookrightarrow} \cdots \overset{i^{n}_{n-1}}{\hookrightarrow} M^{n}_{n} \overset{i^{n}_{n}}{\hookrightarrow} \Delta^{2n+1}.
  \end{equation}

  Define
  \begin{equation*}
    j^{n}_{k} := i^{n}_{n} \circ \cdots \circ i^{n}_{k}\colon M^{n}_{k} \hookrightarrow \Delta^{2n+1},\qquad 0 \leq k \leq n.
  \end{equation*}
  In particular, note that $j^{n}_{0} = v_{n}'$.

  This allows us to replace our goal by something superficially more difficult: we would like to show that, for each $n \geq 0$ and each $0 \leq k \leq n$, the map $i^{n}_{k}$ is marked-scaled anodyne. We proceed by induction. We take as our base case $n=0$, where we have the trivial filtration
  \begin{equation*}
    M^{0}_{0} \overset{i^{0}_{0}}{=} \Delta^{1}.
  \end{equation*}
  This is an equality of subsets of $\Delta^{1}$, hence certainly an equivalence.

  We now suppose that the result holds true for $n-1$; that is, that the maps $i^{n-1}_{k}$ are marked-scaled anodyne for all $0 \leq k \leq n-1$. We aim to show that each $i^{n}_{k}$ is marked-scaled anodyne for each $0 \leq k \leq n$.

  For $0 \leq k < n$, we can write $i^{n}_{k}$ as the inclusion
  \begin{equation*}
    \S^{\mathcal{A}}_{[2n+1]} \hookrightarrow \S^{\mathcal{A}}_{[2n+1]} \cup \Delta^{[2n+1] \smallsetminus \{k, \overline{k}\}},\qquad \mathcal{A}
    = \left\{ \substack{ \{\overline{n}, \ldots, \overline{1}\} \\ \{0, \ldots n\} \\ \{1, \overline{1}\} \\ \vdots \\ \{k-1, \overline{k-1}\} } \right\}.
  \end{equation*}
  By \hyperref[lemma:bicartesian_square]{Lemma~\ref*{lemma:bicartesian_square}}, we have a pushout square
  \begin{equation*}
    \begin{tikzcd}
      \S^{\mathcal{A}|([2n+1] \smallsetminus \{k, \overline{k}\})}_{[2n+1] \smallsetminus \{k, \overline{k}\}}
      \arrow[r, hook]
      \arrow[d, hook]
      & \Delta^{[2n+1] \smallsetminus \{k, \overline{k}\}}
      \arrow[d, hook]
      \\
      \S^{\mathcal{A}}_{[2n+1]}
      \arrow[r, hook]
      & \S^{\mathcal{A}}_{[2n+1]} \cup \Delta^{[2n+1] \smallsetminus \{k, \overline{k}\}}.
    \end{tikzcd}.
  \end{equation*}
  Therefore, it suffices to show that the top morphism is marked-scaled anodyne. One checks that this map is of the form $j^{n-1}_{k}$, and is thus marked-scaled anodyne by the inductive hypothesis. Therefore, it remains only to show that $i^{n}_{n}$ is marked-scaled anodyne.

  The case $n = 0$ is an isomorphism, so there is nothing to show. We treat the case $n = 1$ separately. In this case, $i^{1}_{1}$ takes the form
  \begin{equation*}
    \Delta^{\{0,1,\overline{0}\}} \cup \Delta^{\{\overline{1},\overline{0}\}} \overset{v_{1}'}{\hookrightarrow} \Delta^{3},
  \end{equation*}
  which we construct in the following way.
  \begin{enumerate}
    \item First we fill the simplex $\Delta^{\{1,\overline{1},\overline{0}\}}$ (together with its marking and scaling) as a pushout along a morphism of type \ref{item:outerms}.

    \item Then we fill the simplex $\Delta^{\{0,1,\overline{1}\}}$ (together with its marking and scaling) as a pushout along a morphism of type \ref{item:innerms}.

    \item Finally, we fill the full simplex $\Delta^{3}$ (together with its marking and scaling) as a pushout along a morphism of type \ref{item:innerms}.
  \end{enumerate}

  Now we may assume that $n \geq 2$. In this case, we can write $i^{n}_{n}\colon M^{n}_{n} \hookrightarrow \Delta^{2n+1}$ as an inclusion
  \begin{equation*}
    \S^{\mathcal{A}}_{[2n+1]} \subseteq \Delta^{2n+1},\qquad \mathcal{A} = \left\{ \substack{ \{\overline{n}, \ldots, \overline{1}\} \\ \{0, \ldots n\} \\ \{1, \overline{1}\} \\ \vdots \\ \{n, \overline{n}\} } \right\}.
  \end{equation*}


  We will express this inclusion as the following composition of fillings:
  \begin{enumerate}
    \item We first add the simplices $\Delta^{\{n, \overline{n}, \ldots, \overline{0}\}}$, $\Delta^{\{n-1, n, \overline{n}, \ldots, \overline{0}\}}$, \dots, $\Delta^{\{2, \ldots, n, \overline{n}, \ldots, \overline{0}\}}$.

    \item We next add $\Delta^{\{0, \ldots, n, \overline{n}, \overline{0}\}}$, $\Delta^{\{0, \ldots, n, \overline{n}, \overline{n-1}, \overline{0}\}}$, \dots, $\Delta^{\{0, \ldots, n, \overline{n}, \ldots, \overline{2}, \overline{0}\}}$

    \item We next add $\Delta^{\{1, \ldots n, \overline{n}, \ldots, \overline{0}\}}$.

    \item We finally add $\Delta^{\{0, \ldots n, \overline{n}, \ldots, \overline{0}\}}$.
  \end{enumerate}
  We proceed.
  \begin{enumerate}
    \item
      \begin{itemize}
        \item Using \hyperref[lemma:bicartesian_square]{Lemma~\ref*{lemma:bicartesian_square}}, we see that the square
          \begin{equation*}
            \begin{tikzcd}
              \S^{\mathcal{A}|\{n, \overline{n}, \ldots, \overline{0}\}}_{\{n, \overline{n}, \ldots, \overline{0}\}}
              \arrow[r, hook]
              \arrow[d, hook]
              & \Delta^{\{n, \overline{n}, \ldots, \overline{0}\}}
              \arrow[d, hook]
              \\
              \S^{\mathcal{A}}_{[2n+1]}
              \arrow[r, hook]
              & \S^{\mathcal{A}}_{[2n+1]} \cup \Delta^{\{n, \overline{n}, \ldots, \overline{0}\}}
            \end{tikzcd}
          \end{equation*}
          is pushout. In order to show that the bottom inclusion is marked-scaled anodyne, it thus suffices to show that the top inclusion is marked-scaled anodyne. We have
          \begin{equation*}
            \begin{tikzcd}
              \mathcal{A}|\{n, \overline{n}, \ldots, \overline{0}\}
              = \left\{ \substack{ \{\overline{n}, \ldots, \overline{1}\} \\ \{n\} \\ \{\overline{1}\} \\ \vdots \\ \{\overline{n-1}\} \\ \{n, \overline{n}\} } \right\}
              \sim \left\{ \substack{ \{n\} \\ \{\overline{n-1}\} \\ \vdots \\ \{\overline{1}\} } \right\} =: \mathcal{A}'
            \end{tikzcd}.
          \end{equation*}
          This is a dull subset of $\{n, \overline{n}, \ldots, \overline{0}\}$ with pivot $\overline{n}$. The only $\mathcal{A}'$-basal set is $\{n, \overline{n-1}, \ldots, \overline{1}\}$. One checks that $\mathcal{A}'$, together with the marking $\heartsuit$ and scaling $\dagger$, satisfies the conditions of the \hyperref[lemma:pivot_trick]{Pivot~Trick}:
          \begin{itemize}
            \item For $n = 2$, the only the marked edge not containing the pivot $\overline{2}$ is $\overline{1} \to \overline{0}$, which belongs to $\S^{\mathcal{A}'}_{\{n, \overline{n}, \ldots, \overline{0}\}}$. The only scaled simplex which does not contain $\overline{2}$, and which does not factor through $\S^{\mathcal{A}'}$, is $\sigma = \{2 < \overline{1} < \overline{0}\}$. The simplex $\sigma \cup \{\overline{2}\}$ is fully scaled.

            \item For $n = 3$, each marked edge is contained in $\S^{\mathcal{A}'}$. The only scaled triangle which does not contain the pivot $\overline{3}$, and which does not factor through $\S^{\mathcal{A}'}$, is $\sigma = \{3 < \overline{2} < \overline{1}\}$. The simplex $\sigma \cup \{\overline{3}\}$ is fully scaled.

            \item For $n \geq 4$, all scaled and marked simplices belong to $\S^{\mathcal{A}'}$ by \hyperref[lemma:subset_of_simplex_contains_k_simplices]{Lemma~\ref*{lemma:subset_of_simplex_contains_k_simplices}}.
          \end{itemize}
          In each case, the simplex $\{n, \overline{n}, \overline{n-1}\}$ is scaled, so the top inclusion is marked-scaled anodyne by the \hyperref[lemma:pivot_trick]{Pivot Trick}. We can write
          \begin{equation*}
            \S^{\mathcal{A}}_{[2n+1]} \cup \Delta^{\{n, \overline{n}, \ldots, \overline{0}\}} = \S^{\mathcal{A} \cup ([2n+1] \smallsetminus \{n, \overline{n}, \ldots, \overline{0}\})}_{[2n+1]},
          \end{equation*}
          We see that
          \begin{equation*}
            \mathcal{A} \cup ([2n+1] \smallsetminus \{n, \overline{n}, \ldots, \overline{0}\}) \sim \left\{ \substack{ \{\overline{n}, \ldots, \overline{1}\} \\ \{0, \ldots n-1\} \\ \{1, \overline{1}\} \\ \vdots \\ \{n, \overline{n}\} } \right\},
          \end{equation*}
          which we denote by $\mathcal{A}_{n-1}$.

        \item We proceed inductively. Suppose we have added the simplices $\Delta^{\{n, \overline{n}, \ldots, \overline{0}\}}$, $\Delta^{\{n-1, n, \overline{n}, \ldots, \overline{0}\}}$, \dots, $\Delta^{\{k+1, \ldots, n, \overline{n}, \ldots, \overline{0}\}}$, for $2 \leq k \leq n-1$. Using \hyperref[lemma:add_a_simplex]{Lemma~\ref*{lemma:add_a_simplex}} and \hyperref[lemma:replace_poset_by_minimal_elements]{Lemma~\ref*{lemma:replace_poset_by_minimal_elements}} can write the result of these additions as
          \begin{equation*}
            \S^{\mathcal{A}}_{[2n+1]} \cup \left( \bigcup_{i = k+1}^{n} \Delta^{\{i, \ldots, n, \overline{n}, \ldots, \overline{0}\}}\right) = \S^{\mathcal{A}_{k+1}}_{[2n+1]},
          \end{equation*}
          where
          \begin{equation*}
            \mathcal{A}_{k+1} = \left\{ \substack{ \{\overline{n}, \ldots, \overline{1}\} \\ \{0, \ldots, k\} \\ \{1, \overline{1}\} \\ \vdots \\ \{n, \overline{n}\} } \right\}.
          \end{equation*}

          Using \hyperref[lemma:bicartesian_square]{Lemma~\ref*{lemma:bicartesian_square}}, we see that the square
          \begin{equation*}
            \begin{tikzcd}
              \S^{\mathcal{A}_{k+1}|\{k, \ldots, n, \overline{n}, \ldots, \overline{0}\}}_{\{k, \ldots, n, \overline{n}, \ldots, \overline{0}\}}
              \arrow[r, hook]
              \arrow[d, hook]
              & \Delta^{\{k, \ldots, n, \overline{n}, \ldots, \overline{0}\}}
              \arrow[d, hook]
              \\
              \S^{\mathcal{A}_{k+1}}_{[2n+1]}
              \arrow[r, hook]
              & \S^{\mathcal{A}_{k+1}}_{[2n+1]} \cup \Delta^{\{k, \ldots, n, \overline{n}, \ldots, \overline{0}\}}
            \end{tikzcd}
          \end{equation*}
          is pushout, so in order to show that the bottom morphism is marked-scaled anodyne, it suffices to show that the top morphism is. We see that
          \begin{equation*}
            \mathcal{A}_{k+1}|\{k, \ldots, n, \overline{n}, \ldots, \overline{0}\} =
            \left\{ \substack{ \{\overline{n}, \ldots, \overline{1}\} \\ \{k\} \\ \{\overline{1}\} \\ \vdots \\ \{\overline{k-1}\} \\ \{k, \overline{k}\} \\ \vdots \\ \{n, \overline{n}\} } \right\}
            \sim \left\{ \substack{ \{k\} \\ \{\overline{1}\} \\ \vdots \\ \{\overline{k-1}\} \\ \{k+1, \overline{k+1}\} \\ \vdots \\ \{n, \overline{n}\} } \right\} =: \mathcal{A}_{k+1}'.
          \end{equation*}

          This is a dull subset of $P(\{k, \ldots, n, \overline{n}, \ldots, \overline{0}\})$ with pivot $\overline{k}$, and one checks that the conditions of the \hyperref[lemma:pivot_trick]{Pivot Trick} are satisfied:
          \begin{itemize}
            \item For $n = 3$, where the only value of $k$ is $k = 2$, each marked edge is contained in $\S^{\mathcal{A}_{k+1}'}$ by \hyperref[lemma:subset_of_simplex_contains_k_simplices]{Lemma~\ref*{lemma:subset_of_simplex_contains_k_simplices}}, and one can check that each scaled triangle factors through $\S^{\mathcal{A}_{k+1}'}$.

            \item For $n \geq 4$, all scaled and marked simplices belong to $\S^{\mathcal{A}'}$ by \hyperref[lemma:subset_of_simplex_contains_k_simplices]{Lemma~\ref*{lemma:subset_of_simplex_contains_k_simplices}}.
          \end{itemize}
          The basal sets are of the form
          \begin{equation*}
            \{k, a_{1}, \ldots, a_{n-k}, \overline{k-1}, \ldots, \overline{1}\},
          \end{equation*}
          where each $a_{1}$, \dots, $a_{n-k}$ is of the form $\ell$ or $\overline{\ell}$ for $k+1 \leq \ell \leq n$. In each case, the simplex $\{a_{n-k}, \overline{k}, \overline{k-1}\}$ is scaled. Thus, the conditions of the \hyperref[lemma:pivot_trick]{Pivot Trick}, the top morphism is marked-scaled anodyne.
      \end{itemize}

      We have now added the simplices promised in part 1., and are left with the simplicial subset $\S^{\mathcal{A}_{2}}_{[2n+1]}$, where
      \begin{equation*}
        \mathcal{A}_{2}
        = \left\{ \substack{ \{\overline{n}, \ldots, \overline{1}\} \\ \{0, 1\} \\ \{1, \overline{1}\} \\ \vdots \\ \{n, \overline{n}\} } \right\}.
      \end{equation*}

    \item Each step in this sequence is solved exactly like those above. The calculations are omitted. The end result is the simplicial subset
      \begin{equation*}
        \mathcal{B}_{2} = 
        \left\{ \substack{ \{\overline{1}\} \\ \{0, 1\} \\ \{2, \overline{2}\} \\ \vdots \\ \{n, \overline{n}\} } \right\}.
      \end{equation*}


    \item Using \hyperref[lemma:bicartesian_square]{Lemma~\ref*{lemma:bicartesian_square}}, we see that the square
      \begin{equation*}
        \begin{tikzcd}
          \S^{\mathcal{B}_{2}|\{1, \ldots, n, \overline{n}, \ldots, \overline{0}\}}_{\{1, \ldots, n, \overline{n}, \ldots, \overline{0}\}}
          \arrow[r, hook]
          \arrow[d, hook]
          & \Delta^{\{1, \ldots, n, \overline{n}, \ldots, \overline{0}\}}
          \arrow[d, hook]
          \\
          \S^{\mathcal{B}_{2}}_{[2n+1]}
          \arrow[r, hook]
          & \S^{\mathcal{B}_{2}}_{[2n+1]} \cup \Delta^{\{1, \ldots, n, \overline{n}, \ldots, \overline{0}\}}
        \end{tikzcd}
      \end{equation*}
      is pushout. We thus have to show that the top morphism is marked-scaled anodyne. We have
      \begin{equation*}
        \mathcal{B}_{2}|\{1, \ldots, n, \overline{n}, \ldots, \overline{0}\} \sim
        \left\{ \substack{ \{\overline{1}\} \\ \{2, \overline{2}\} \\ \vdots \\ \{n, \overline{n}\} \\ \{\overline{1}\} } \right\}.
      \end{equation*}
      One readily sees that this is a right-dull subset, and that the conditions of the \hyperref[lemma:right-anodyne_pivot_trick]{Right-Anodyne Pivot Trick} are satisfied.

    \item One solves this as before, checking that the conditions of the \hyperref[lemma:pivot_trick]{Pivot Trick} are satisfied, with pivot point $2$.
  \end{enumerate}
\end{proof}

\subsection{The map governing local systems is a cocartesian fibration}
\label{ssc:the_fibration}

Our aim is to show that the map of quasicategories $p\colon \LS(\category{C}) \to \S$ defined by the diagram
\begin{equation*}
  \begin{tikzcd}
    \LS(\category{C})
    \arrow[r, hook]
    \arrow[d, swap, "p"]
    & \mathcal{R}
    \arrow[r, hook]
    \arrow[d, "p''"]
    & \Tw(\ICCat)
    \arrow[d, "p'"]
    \\
    \S \times \{\category{C}\}
    \arrow[r, hook]
    & \ICat \times [\category{C}]
    \arrow[r, hook]
    & \ICat \times \ICat\op
  \end{tikzcd},
\end{equation*}
where both squares are pullback, is a cocartesian fibration. We now define the class of morphisms which we claim are $p$-cocartesian.

\begin{definition}
  \label{def:left_kan_simplex}
  A morphism $\tilde{\sigma}\colon \Delta^{1} \to \LS(\category{C})$ is said to be \defn{left Kan} if the simplex $\sigma\colon \Delta^{3}_{\dagger} \to \ICCat$ to which it is adjoint has the property that the restriction $\sigma|\Delta^{\{0,1,\overline{0}\}}$ is left Kan in the sense of \hyperref[def:left_kan]{Definition~\ref*{def:left_kan}}.
\end{definition}

We draw the `front' and `back' of a general $3$-simplex $\sigma\colon \Delta^{3}_{\dagger} \to \ICCat$ corresponding to some morphism $\Delta^{1} \to \LS(\category{C})$.
\begin{equation*}
  \begin{tikzcd}[column sep=large, row sep=large]
    X
    \arrow[r, "f"]
    \arrow[d, "\mathcal{F}"{swap}, ""{name=L}]
    \arrow[dr, "\mathcal{H}"]
    & Y
    \arrow[d, "\mathcal{G}"]
    \\
    \category{C}
    & \category{C}
    \arrow[l, "\id_{\category{C}}"]
    \arrow[from=L, Rightarrow, shorten=3ex, "\zeta"{swap}]
  \end{tikzcd}
  \qquad\text{and}\qquad
  \begin{tikzcd}[column sep=large, row sep=large]
    X
    \arrow[r, "f"]
    \arrow[d, "\mathcal{F}"{swap}, ""{name=L}]
    & Y
    \arrow[d, "\mathcal{G}"]
    \arrow[dl, "\mathcal{G}'"]
    \arrow[from=L, Rightarrow, shorten=3ex, "\eta"]
    \\
    \category{C}
    & \category{C}
    \arrow[l, "\id_{\category{C}}"]
  \end{tikzcd}
\end{equation*}
Here, the $2$-simplices which are notated without natural transformations are constrained by the scaling $\dagger$ to be thin. The morphism $\tilde{\sigma}$ is left Kan if and only if $\mathcal{G}'$ is a left Kan extension of $\mathcal{F}$ along $f$, and $\eta$ is a unit map.

\begin{notation}
  We endow the quasicategory $\LS(\category{C})$ with the marking $\clubsuit \subseteq \LS(\category{C})_{1}$ consisting of all left Kan edges.
\end{notation}

\begin{theorem}
  \label{thm:cocartesian_edges_of_p}
  The map $p\colon \LS(\category{C}) \to \S$ is a cocartesian fibration, and an edge $\Delta^{1} \to \LS(\category{C})$ is $p$-cocartesian if and only if it is left Kan.
\end{theorem}
\begin{proof}
  By \cite[Prop.~3.1.1.6]{highertopostheory}, it suffices to show that the map $\LS(\category{C})^{\clubsuit} \to \S^{\sharp}$ has the right lifting property with respect to each of the classes of generating marked anodyne morphisms. We check these one-by-one.
  \begin{enumerate}
    \item We need to check that all lifting problems
      \begin{equation*}
        \begin{tikzcd}
          (\Lambda^{n}_{i})^{\flat}
          \arrow[r]
          \arrow[d]
          & \LS(\category{C})^{\clubsuit}
          \arrow[d]
          \\
          (\Delta^{n})^{\flat}
          \arrow[r]
          \arrow[ur, dashed]
          & \S^{\sharp}
        \end{tikzcd},\qquad n \geq 2, \quad 0 < i < n
      \end{equation*}
      admit solutions. This follows from \hyperref[thm:mainthm_walker_fernando]{Theorem~\ref*{thm:mainthm_walker_fernando}}.

    \item We need to show that each of the lifting problems
      \begin{equation*}
        \begin{tikzcd}
          (\Lambda^{n}_{0})^{\mathcal{L}}
          \arrow[r]
          \arrow[d]
          & \LS(\category{C})^{\clubsuit}
          \arrow[d]
          \\
          (\Delta^{n})^{\mathcal{L}}
          \arrow[r]
          \arrow[ur, dashed]
          & \S^{\sharp}
        \end{tikzcd},\qquad n \geq 1.
      \end{equation*}
      has a solution. Given any such lifting problem, it suffices to find a solution to the outer lifting problem
      \begin{equation*}
        \begin{tikzcd}
          (\Lambda^{n}_{0})^{\mathcal{L}}
          \arrow[r]
          \arrow[d]
          & \LS(\category{C})
          \arrow[d]
          \arrow[r]
          & \mathcal{R}
          \arrow[d]
          \\
          (\Delta^{n})^{\mathcal{L}}
          \arrow[r]
          \arrow[urr, dashed]
          & \S
          \arrow[r, hook]
          & \ICat \times [\category{C}]
        \end{tikzcd}.
      \end{equation*}
      Passing to the adjoint lifting problem, we need to find a solution to the lifting problem
      \begin{equation}
        \label{eq:unsimplified_lifting_problem}
        \begin{tikzcd}
          R(\Lambda^{n}_{0}) \displaystyle\coprod_{(\Lambda^{n}_{0})^{\flat}_{\sharp} \amalg ((\Lambda^{n}_{0})\op)^{\sharp}_{\sharp}}(\Delta^{n})^{\flat}_{\sharp} \amalg ((\Delta^{n})\op)^{\sharp}_{\sharp}
          \arrow[r, "\sigma"]
          \arrow[d, hook]
          & \ICCat
          \\
          R(\Delta^{n})
          \arrow[ur, dashed]
        \end{tikzcd}
      \end{equation}
      such that $\sigma|\{0,1,\overline{0}\}$ is left Kan. Here $R$ is the functor defined in \hyperref[def:cosimplicial_obj_R]{Definition~\ref*{def:cosimplicial_obj_R}}.

      We note the existence of a commutative square
      \begin{equation*}
        \begin{tikzcd}
          J(\Lambda^{n}_{0}) \displaystyle\coprod_{(\Lambda^{n}_{0})^{\flat}_{\sharp} \amalg (\Delta^{\{\overline{0}\}})^{\sharp}_{\sharp}}(\Delta^{n})^{\flat}_{\sharp} \amalg (\Delta^{\{\overline{0}\}})^{\sharp}_{\sharp}
          \arrow[r, hook, "b"]
          \arrow[d, hook]
          & R(\Lambda^{n}_{0}) \displaystyle\coprod_{(\Lambda^{n}_{0})^{\flat}_{\sharp} \amalg ((\Lambda^{n}_{0})\op)^{\sharp}_{\sharp}}(\Delta^{n})^{\flat}_{\sharp} \amalg ((\Delta^{n})\op)^{\sharp}_{\sharp}
          \arrow[d, hook]
          \\
          J(\Delta^{n})
          \arrow[r, hook, "v_{n}"]
          & R(\Delta^{n})
        \end{tikzcd},
      \end{equation*}
      where $J$ is the functor defined in \hyperref[def:cosimplicial_obj_J]{Definition~\ref*{def:cosimplicial_obj_J}}. Here, $v_{n}$ is the morphism of \hyperref[lemma:lower_morphism_equivalence]{Lemma~\ref*{lemma:lower_morphism_equivalence}}. The morphism $b$ is defined component-wise via the following morphisms.
      \begin{itemize}
        \item The morphism $J(\Lambda^{n}_{0}) \to R(\Lambda^{n}_{0})$ comes from \hyperref[lemma:marked_scaled_equivalence_left_horn]{Lemma~\ref*{lemma:marked_scaled_equivalence_left_horn}}, where it is proven to be a weak equivalence in the marked-scaled model structure.

        \item The map $(\Delta^{\{\overline{0}\}})^{\sharp}_{\sharp} \to ((\Lambda^{n}_{0})\op)^{\sharp}_{\sharp}$ is marked-scaled anodyne by \hyperref[prop:sharp_marked_right_anodyne]{Proposition~\ref*{prop:sharp_marked_right_anodyne}}.

        \item The maps $(\Delta^{\{\overline{0}\}})^{\sharp}_{\sharp} \to ((\Delta^{n})\op)^{\sharp}_{\sharp}$ is marked-scaled anodyne by \hyperref[prop:sharp_marked_right_anodyne]{Proposition~\ref*{prop:sharp_marked_right_anodyne}}.

        \item The rest of the morphisms connecting the components are isomorphisms.
      \end{itemize}

      We showed in \hyperref[lemma:lower_morphism_equivalence]{Lemma~\ref*{lemma:lower_morphism_equivalence}} that the lower morphism in this square is a marked-scaled equivalence. We would now like to show that the upper morphism is a marked-scaled equivalence. Recall that, since the cofibrations in the model structure on marked-scaled simplicial sets are simply those morphisms whose underlying morphism of simplicial sets is a monomorphism, the colimits defining $b$ are models for the homotopy colimits, so the result follows from the observation that each component is a weak equivalence in the marked-scaled model structure.

      Using \cite[Prop.~A.2.3.1]{highertopostheory}, we see that in order to solve the lifting problem of \hyperref[eq:unsimplified_lifting_problem]{Equation~\ref*{eq:unsimplified_lifting_problem}}, it suffices to show that the lifting problem
      \begin{equation*}
        \begin{tikzcd}
          J(\Lambda^{n}_{0}) \displaystyle\coprod_{(\Lambda^{n}_{0})^{\flat}_{\sharp} \amalg (\Delta^{\{\overline{0}\}})^{\sharp}_{\sharp}}(\Delta^{n})^{\flat}_{\sharp} \amalg (\Delta^{\{\overline{0}\}})^{\sharp}_{\sharp}
          \arrow[r, "b \circ \sigma"]
          \arrow[d, hook]
          & \ICCat
          \\
          J(\Delta^{n})
          \arrow[ur, dashed, swap, "\ell"]
        \end{tikzcd}
      \end{equation*}
      has a solution whose restriction $\ell|\{0,1,\overline{0}\}$ is left Kan. However, we note that there exists a pushout square
      \begin{equation*}
        \begin{tikzcd}
          (\Lambda^{\{0, \ldots, n, \overline{0}\}}_{0})^{\flat}_{\flat}
          \arrow[r]
          \arrow[d]
          & J(\Lambda^{n}_{0}) \displaystyle\coprod_{(\Lambda^{n}_{0})^{\flat}_{\sharp} \amalg (\Delta^{\{\overline{0}\}})^{\sharp}_{\sharp}}(\Delta^{n})^{\flat}_{\sharp} \amalg (\Delta^{\{\overline{0}\}})^{\sharp}_{\sharp}
          \arrow[d]
          \\
          (\Delta^{\{0, \ldots, n, \overline{0}\}})^{\flat}_{\flat}
          \arrow[r]
          & J(\Delta^{n})
        \end{tikzcd}
      \end{equation*}
      in which the rightward-facing morphisms are isomorphisms on underlying simplicial sets, and where the only new decorations being added are $(\Delta^{\{0, \ldots, n\}})^{\flat}_{\flat} \hookrightarrow (\Delta^{\{0, \ldots, n\}})^{\flat}_{\sharp}$. Therefore, it suffices to solve the lifting problems
      \begin{equation*}
        \begin{tikzcd}
          (\Lambda^{\{0, \ldots, n, \overline{0}\}}_{0})^{\flat}_{\flat}
          \arrow[r]
          \arrow[d]
          & \ICCat
          \\
          (\Delta^{\{0, \ldots, n, \overline{0}\}})^{\flat}_{\flat}
          \arrow[ur, dashed, swap, "\ell'"]
        \end{tikzcd},\qquad n \geq 1
      \end{equation*}
      such that $\ell'|\{0,1,\overline{0}\}$ is left Kan. That these lifting problems admit solutions for $n \geq 2$ is the content of \hyperref[thm:left_kan_implies_globally_left_kan]{Theorem~\ref*{thm:left_kan_implies_globally_left_kan}}. The case $n=1$ is the statement that left Kan extensions of functors into cocomplete categories exist along functors between small categories.

    \item We need to show that every lifting problem of the form
      \begin{equation*}
        \begin{tikzcd}
          (\Lambda^{2}_{1})^{\sharp} \amalg_{( \Lambda^{2}_{1} )^{\flat}}(\Delta^{2})^{\sharp}
          \arrow[r]
          \arrow[d]
          & \LS(\category{C})^{\clubsuit}
          \arrow[d]
          \\
          (\Delta^{2})^{\sharp}
          \arrow[r]
          \arrow[ur, dashed]
          & \S^{\sharp}
        \end{tikzcd}
      \end{equation*}
      has a solution. Considering the adjoint lifting problem, we find that it suffices to show that for any $\sigma\colon  \Delta^{5}_{\dagger} \to \ICCat$ such that the restrictions $\sigma|\{0,1,\overline{0}\}$ and $\sigma|\{1,2,\overline{1}\}$ are left Kan and the morphisms belonging to $\sigma|\{\overline{2}, \overline{1}, \overline{0}\}$ are equivalences, the restriction $\sigma|\{0,2,\overline{0}\}$ is left Kan. Applying \hyperref[lemma:transport_left_kan_simplices]{Lemma~\ref*{lemma:transport_left_kan_simplices}} to $\sigma|\{1,2,\overline{1},\overline{0}\}$, we see that $\sigma|\{1,2,\overline{1}\}$ is left Kan if and only if $\sigma|\{1,2,\overline{0}\}$ is left Kan. Applying \hyperref[lemma:compose_left_kan_simplices]{Lemma~\ref*{lemma:compose_left_kan_simplices}} to $\sigma|\{0,1,2,\overline{0}\}$ guarantees that $\sigma|\{0,2,\overline{0}\}$ is left Kan as required.


    \item We need to show that for all Kan complexes $K$, the lifting problem
      \begin{equation*}
        \begin{tikzcd}
          K^{\flat}
          \arrow[r]
          \arrow[d]
          & \LS(\category{C})^{\clubsuit}
          \arrow[d]
          \\
          K^{\sharp}
          \arrow[r]
          \arrow[ur, dashed]
          & \S^{\sharp}
        \end{tikzcd}
      \end{equation*}
      has a solution. To see this, note that each morphism in $K$ must be mapped to an equivalence in $\LS(\category{C})$, and a morphism $a\colon \mathcal{F} \to \mathcal{G}$ in $\LS(\category{C})$ is an equivalence and only if it is $p$-cartesian, and lies over an equivalence in $\S$.
      \begin{itemize}
        \item By \hyperref[thm:mainthm_walker_fernando]{Theorem~\ref*{thm:mainthm_walker_fernando}}, the morphism $a$ is $p$-cartesian if and only if the map $\sigma\colon \Delta^{3}_{\dagger} \to \ICCat$ to which it is adjoint factors through the map $\Delta^{3}_{\dagger} \to \Delta^{3}_{\sharp}$. In particular, the restriction $\sigma|\Delta^{\{0,1,\overline{0}\}}$ is thin.

        \item The image of $a$ in $\S$ is the restriction $\sigma|\Delta^{\{0,1\}}$, which is therefore an equivalence.
      \end{itemize}
      It follows from \hyperref[eg:strictly_commuting_left_kan]{Example~\ref*{eg:strictly_commuting_left_kan}} that the morphism $a$ is automatically left Kan. Thus, each morphism in $K$ is mapped to a left Kan morphism in $\LS(\category{C})$, and we are justified in marking them; our lifting problems admit solutions.
  \end{enumerate}
\end{proof}

%{\color{red}Everything past this point is old, and probably won't survive without major revision.}
%
%
%In order to show that all left Kan edges are cocartesian, we have to show that lifting problems
%\begin{equation*}
%  \begin{tikzcd}
%    (\Lambda^{n}_{0})^{\mathcal{L}}
%    \arrow[r]
%    \arrow[d, hook]
%    & \LS(\category{C})^{\clubsuit}
%    \arrow[d]
%    \\
%    (\Delta^{n})^{\mathcal{L}}
%    \arrow[r]
%    \arrow[ur, dashed]
%    & \S^{\sharp}
%  \end{tikzcd},\qquad n \geq 2,
%\end{equation*}
%have solutions.\footnote{Note that the case $n=1$ corresponds to the condition that cocartesian lifts exist, proved above.} Unravelling the definitions, we find that these are adjunct to the lifting problems
%\begin{equation}
%  \label{eq:challenging_filling_problem}
%  \begin{tikzcd}
%    K^{n}_{\dagger}
%    \arrow[d, hook]
%    \arrow[r, "\tau'"]
%    & \ICCat
%    \\
%    \Delta^{2n+1}_{\dagger}
%    \arrow[ur, dashed, swap, "\tau"]
%  \end{tikzcd},
%\end{equation}
%where:
%\begin{itemize}
%  \item $K^{n}$ is the simplicial subset of $\Delta^{n}$ on those simplices $\sigma\colon \Delta^{k} \to K^{n}$ which satisfy at least one of the following conditions.
%    \begin{enumerate}
%      \item The $k$-simplex $\sigma$ factors through the `top face' $\Delta^{n} \subset \Delta^{2n+1} \cong \Delta^{n} \star (\Delta^{n})\op$.
%
%      \item The $k$-simplex $\sigma$ factors through the `bottom face' $(\Delta^{n})\op \subset \Delta^{2n+1} \cong \Delta^{n} \star (\Delta^{n})\op$.
%
%      \item There exists $\ell \neq 0$ such that the $k$-simplex $\sigma$ contains neither $\Delta^{\{\ell\}}$ nor $\Delta^{\{\overline{\ell}\}}$.
%    \end{enumerate}
%
%
%  \item The filler $\tau$ is left Kan. (This implies in particular that the map $\tau'$ fulfills all conditions to be left Kan which apply to the scaled simplicial subset $K^{n}_{\dagger}$.)
%\end{itemize}
%
%It is actually easier to solve these lifting problems by generalizing them somewhat. Recall that any left Kan simplex $\tau\colon \Delta^{2n+1}_{\dagger} \to \ICCat$ is in particular admissible, and thus has the property that the restriction $\tau|\Delta^{\{n+1, \ldots, 2n+1\}}$ is the constant functor with value $\category{C}$. We will relax this requirement, demanding only that each morphism in $\Delta^{\{n+1, \ldots, 2n+1\}}$ be mapped to an equivalence. We can guarantee this by introducing a marking, in addition to the scaling. We can do this using the theory of marked-scaled simplicial sets, as developed in \cite{garcia20212}.
%
%\begin{notation}
%  \leavevmode
%  \begin{enumerate}
%    \item Denote by $\natural$ the marking on $\ICCat$ containing all equivalences. Denote $\ICCat$ together with this marking by $\ICCat^{\natural}$. Thus, $\ICCat^{\natural}$ is a marked-scaled simplicial set, where a triangle is scaled if and only if it commutes up to a specified homotopy.
%
%    \item Denote by $\heartsuit$ the marking on $\Delta^{2n+1}$ containing all $1$-simplices in the `bottom face' $(\Delta^{n})\op = \Delta^{\{n+1, \ldots, 2n+1\}} \subset \Delta^{2n+1}$. Denote the marked-scaled simplicial set with this marking and the $\dagger$ scaling by $(\Delta^{2n+1})^{\heartsuit}_{\dagger}$.
%
%    \item For any simplicial subset $S \subseteq \Delta^{2n+1}$, we will denote the restrictions of $\dagger$ and $\heartsuit$ to $S$ also by $\dagger$ and $\heartsuit$ respectively.
%  \end{enumerate}
%\end{notation}
%
%\begin{definition}
%  \label{def:weakly_left_kan}
%  We will call a map $\sigma\colon (\Delta^{2n+1})^{\heartsuit}_{\dagger} \to \ICCat^{\natural}$ \defn{weakly left Kan} if it satisfies the following conditions.
%  \begin{itemize}
%    \item The objects $\sigma(\Delta^{0})$, \dots, $\sigma(\Delta^{n}) \in \ICCat$ are spaces.
%
%    \item The image of the simplex $(\Delta^{\{0, 1, \overline{0}\}})^{\flat}_{\flat} \subset (\Delta^{2n+1})^{\heartsuit}_{\dagger}$ is left Kan.
%  \end{itemize}
%\end{definition}
%
%
%We note that there is a pullback diagram of marked-scaled simplicial subsets of $\Delta^{2n+1}$
%\begin{equation}
%  \label{eq:simplification_of_lifting_problem}
%  \begin{tikzcd}
%    (\Lambda^{\{0, \ldots, n, \overline{0}\}}_{0})^{\heartsuit}_{\dagger}
%    \arrow[r, hook, "u_{n}"]
%    \arrow[d, hook]
%    & (K^{n})^{\heartsuit}_{\dagger}
%    \arrow[d, hook]
%    \\
%    (\Delta^{\{0, \ldots, n, \overline{0}\}})^{\heartsuit}_{\dagger}
%    \arrow[r, hook, "v_{n}"]
%    & (\Delta^{2n+1})^{\heartsuit}_{\dagger}
%  \end{tikzcd}.
%\end{equation}
%
%
%\begin{lemma}
%  \label{lemma:upper_morphism_equivalence}
%  For all $n \geq 2$, the map $u_{n}$ is marked-scaled anodyne, hence a weak equivalence.
%\end{lemma}
%\begin{proof}
%  We begin with a few definitions.
%  \begin{itemize}
%    \item For each $n \geq 2$, denote by $P_{n}$ the poset of strict subsets of $[n]$ which contain $0$. Note that
%      \begin{equation*}
%        \Lambda^{n}_{0} \cong \colim_{S \in P_{n}} \Delta^{S}.
%      \end{equation*}
%  \end{itemize}
%  In the remainder of this proof, all simplicial subsets of $\Delta^{2n+1}$ will carry the $\heartsuit$ marking and the $\dagger$ scaling.
%
%
%  Denote by $S^{n}$ the colimit of the upper composition above, and by $T^{n}$ the colimit of the lower composition.
%  \begin{equation*}
%    \begin{tikzcd}
%      P^{n}
%      \arrow[r]
%      & \mathring{\Delta}
%      \arrow[r, bend left, "J \circ I"]
%      \arrow[r, bend right, swap, "Q \circ I"]
%      & \SSetms
%    \end{tikzcd}
%  \end{equation*}
%  Since the cofibrations of the marked-scaled model structure are precisely those morphisms of marked-scaled simplicial sets whose underlying morphisms of simplicial sets are monomorphisms, the strict colimit of each of these diagrams is a model for the homotopy colimit. We showed in \hyperref[lemma:lower_morphism_equivalence]{Lemma~\ref*{lemma:lower_morphism_equivalence}} that each of the components of the natural transformation $v\colon J \circ I \to Q \circ I$ is a weak equivalence in the model structure on $\SSetms$. Therefore, this natural transformation furnishes a map between the colimits
%
%  We can write the inclusion $u_{n}\colon \Lambda^{\{0, \ldots, n, \overline{0}\}}_{0} \hookrightarrow K^{n}$ as the map between the colimits induced by the diagrams
%  \begin{equation*}
%    \Delta^{n} \amalg_{\Lambda^{n}_{0}} L^{n} \amalg_{(\Lambda^{n}_{0})\op} (\Delta^{n})\op
%  \end{equation*}
%\end{proof}
%
%
%
%We have just shown that left Kan morphisms $\Delta^{1} \to \LS(\category{C})$ are $p$-cocartesian, and that enough left Kan lifts exist. It remains only to show that any $p$-cocartesian lift is left Kan. We use the fact that any two $p$-cocartesian lifts of a morphism with a specified source are equivalent.


\end{document}
