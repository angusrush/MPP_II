\documentclass[main.tex]{subfiles}

\begin{document}

\section{Local systems}
\label{sec:local_systems}

In \cite{garcia2020enhanced}, the twisted arrow category of $\infty$-bicategory is defined. 

\subsection{The twisted arrow category}
\label{ssc:the_twisted_arrow_category}

In this section we give a short exposition of the material of \cite{garcia2020enhanced} that we will need.

We define a cosimplicial object $Q\colon \Delta \to \SSetsc$ by sending
\begin{equation*}
  Q([n]) = (\Delta^{n} \star (\Delta^{n})\op, \dagger),
\end{equation*}
where $\dagger$ is the scaling consisting of all degenerate 2-simplices, together with all 2-simplices of the following kinds:
\begin{enumerate}
  \item All simplices $\Delta^{2} \to \Delta^{n} \star (\Delta^{n})\op$ factoring through $\Delta^{n}$.

  \item All simplices $\Delta^{2} \to \Delta^{n} \star (\Delta^{n})\op$ factoring through $(\Delta^{n})\op$.

  \item All simplices $\Delta^{\{i, j, \bar{k}\}} \subseteq \Delta^{n} \star (\Delta^{n})\op$, $i < j \leq k$.

  \item All simplices $\Delta^{\{k, \bar{j}, \bar{i}\}} \subseteq \Delta^{n} \star (\Delta^{n})\op$, $i < j \leq k$.
\end{enumerate}

This extends to nerve-realization adjunction
\begin{equation*}
  Q : \SSet \longleftrightarrow \SSetsc : \Tw,
\end{equation*}
where by mild abuse of notation, we denote the extension of the functor $Q$ defined above by colimits also by $Q$. For any scaled simplicial set $X$, the simplicial set $\Tw(X)$ has $n$-simplices
\begin{equation*}
  \Hom_{\SSetsc}(Q(n), X).
\end{equation*}

The inclusion $\Delta^{n} \amalg (\Delta^{n})\op \hookrightarrow \Delta^{n} \star (\Delta^{n})\op$ provides, for any scaled simplicial set $K$, a functor $\Tw(K) \to K \times K\op$.

\subsection{The fibration exhibiting local systems}
\label{ssc:the_fibration}

\begin{definition}
  For any cocomplete $\infty$-category $\category{C}$, we define the $\infty$-category of $\category{C}$-local systems $\LS(\category{C})$ together with a cartesian fibration $\LS(\category{C}) \to \S$ by the following homotopy pullback diagram.
  \begin{equation*}
    \begin{tikzcd}
      \LS(\category{C})
      \arrow[r, hook]
      \arrow[d]
      & \Tw(\ICCat)
      \arrow[d]
      \\
      \S
      \arrow[r]
      & \ICat \times \ICat\op
    \end{tikzcd}
  \end{equation*}
\end{definition}

It follows from \cite{garcia2020enhanced} that $\LS(\category{C}) \to \S$ is a cartesian fibration. In this section, we show that it is also a cocartesian fibration,

\subsection{Cocartesian edges}
\label{ssc:cocartesian_edges}

In this section, we will show that 

\end{document}
