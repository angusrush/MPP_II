\documentclass[main.tex]{subfiles}
\begin{document}

\section{Miscellaneous stuff to be deleted or absorbed}
\label{sec:miscellaneous_stuff_to_be_deleted_or_absorbed}

\subsection{The triple structures}
\label{ssc:the_triple_structures}

\subsubsection{The triple structure on finite pointed sets}
\label{sss:the_triple_structure_on_finite_pointed_sets}

We define a triple structure $\triple{F}$ on $\Finp\op$, where
\begin{itemize}
  \item $\category{F} = \Finp\op$

  \item $\category{F}\downdag = (\Finp\op)^{\simeq}$

  \item $\category{F}\updag = \Finp\op$
\end{itemize}

\begin{proposition}
  This is an adequate triple structure.
\end{proposition}
\begin{proof}
  Trivial.
\end{proof}

\begin{proposition}
  There is a Joyal equivalence $\Finp \to \Span\triple{F}$.
\end{proposition}

\subsubsection{The triple structure on infinity-categories}
\label{sss:the_triple_structure_on_categories}

We define a triple $\triple{P}$ as follows.
\begin{itemize}
  \item $\category{P} = \TSCart$

  \item $\category{P}\downdag = \TSCart \times_{\Finp\op}(\Finp\op)^{\simeq}$

  \item $\category{P}\updag = \TSCart$
\end{itemize}

\begin{proposition}
  The cartesian fibration $\TSCart \to \Finp\op$ admits relative pullbacks.
\end{proposition}
\begin{proof}
  The fibers clearly admit pullbacks, and the pullback maps commute with pullbacks because products commute with pullbacks.
\end{proof}

\begin{proposition}
  This is an adequate triple structure.
\end{proposition}
\begin{proof}
  It suffices to show that $\TSCart$ admits pullbacks, and that for any pullback square in $\TSCart$ lying over a square
  \begin{equation*}
    \begin{tikzcd}
      \langle n \rangle
      & \langle m \rangle
      \arrow[l]
      \\
      \langle n' \rangle
      \arrow[u, "\phi"]
      & \langle m' \rangle
      \arrow[l]
      \arrow[u, swap, "\simeq"]
    \end{tikzcd}
  \end{equation*}
  in $\TSCart$, the morphism $\phi$ is an isomorphism in $\Finp$. But $\TSCart$ admits relative pullbacks and $\Finp\op$ admits pullbacks, so $p$ preserves pullbacks. (Just take the proof from my thesis!)
\end{proof}

\subsubsection{The triple structure on the twisted arrow category}
\label{sss:the_triple_structure_on_the_twisted_arrow_category}

We define a triple $\triple{Q}$ as follows:
\begin{itemize}
  \item $\category{Q} = \LS(\category{C})_{\otimes}$

  \item $\category{Q}\downdag = \LS(\category{C})_{\otimes} \times_{\Finp\op}(\Finp\op)^{\simeq}$

  \item $\category{Q}\updag$ consists of the subcategory of $\LS(\category{C})_{\otimes}$ of $r$-cartesian morphisms, i.e.\ thin morphisms.
\end{itemize}

These triple constructions correspond to spans of the following form.
\begin{equation*}
  \begin{tikzcd}
    \vec{X}
    \arrow[d, ""{name=L, right}]
    & \vec{Z}
    \arrow[d, ""{name=ML, left}, ""{name=MR, right}]
    \arrow[r]
    \arrow[l]
    & \vec{Y}
    \arrow[d, ""{name=R, right}]
    \\
    \category{C}^{n}
    \arrow[d, no head, dotted, bend left]
    \arrow[r]
    & \category{C}^{m}
    \arrow[d, no head, dotted, bend left]
    & \category{C}^{m}
    \arrow[l, swap, "\simeq"]
    \arrow[d, no head, dotted, bend left]
    \\
    \vec{X}
    \arrow[d, no head, dotted, bend left]
    & \vec{Z}
    \arrow[d, no head, dotted, bend left]
    \arrow[r]
    \arrow[l]
    & \vec{Y}
    \arrow[d, no head, dotted, bend left]
    \\
    \langle n \rangle
    \arrow[r]
    & \langle m \rangle
    & \langle m \rangle
    \arrow[l, swap, "\cong"]
    \arrow[from=ML, to=L, phantom, shorten=1.5ex, "\circlearrowleft"{description}]
    \arrow[from=MR, to=R, Rightarrow, shorten=2ex]
  \end{tikzcd}
\end{equation*}

According to the modified Barwick's theorem, such a span will correspond to a cocartesian morphism in $\Span\triple{P}$ if it is of the form
\begin{equation*}
  \begin{tikzcd}
    \vec{X}
    \arrow[d, "\mathcal{F}"{left}, ""{name=L, right}]
    & \vec{Z}
    \arrow[d, "g^{*}t \circ \mathcal{F}"{description}, ""{name=ML, left}, ""{name=MR, right}]
    \arrow[r, "f"]
    \arrow[l, swap, "g"]
    & \vec{Y}
    \arrow[d, ""{name=R, right}, "f_{!}(g^{*}t\mathcal{F})"]
    \\
    \category{C}^{n}
    \arrow[r]
    & \category{C}^{m}
    & \category{C}^{m}
    \arrow[l, swap, "\cong"]
    \arrow[from=ML, to=L, phantom, shorten=1.5ex, "\circlearrowleft"{description}]
    \arrow[from=MR, to=R, Rightarrow, shorten=2ex]
  \end{tikzcd}.
\end{equation*}
In particular, in the fiber over $\langle 1 \rangle \in \Finp\op$, this is simply pull-push.

%\begin{lemma}
%  Given a square
%  \begin{equation*}
%    \begin{tikzcd}
%      \langle n \rangle
%      & \langle n \rangle
%      \arrow[l, swap, "\simeq"]
%      \\
%      \langle m \rangle
%      \arrow[u]
%      & \langle m \rangle
%      \arrow[l, swap, "\simeq"]
%      \arrow[u]
%    \end{tikzcd}
%  \end{equation*}
%  in $\Finp\op$ and a solid cospan
%  \begin{equation*}
%    \begin{tikzcd}
%      y
%      \arrow[r, dashed]
%      \arrow[d, dashed]
%      & y'
%      \arrow[d, "g"]
%      \\
%      x
%      \arrow[r]
%      & y
%    \end{tikzcd}
%  \end{equation*}
%  lying over it, where $g$ is cartesian, a relative pullback always exists.
%\end{lemma}
%\begin{proof}
%  The fiber of $q$ over $\langle n \rangle \in \Finp\op$ is equivalent to $\LS(\category{C})^{n}$, which admits pullbacks by
%\end{proof}

\begin{proposition}
  The triple $\triple{Q}$ is adequate.
\end{proposition}
\begin{proof}
  We need to show that for any solid diagram in $\LS(\category{C})_{\otimes}$ of the form
  \begin{equation*}
    \begin{tikzcd}
      y
      \arrow[r, dashed, "f'"]
      \arrow[d, swap, dashed, "g'"]
      & y'
      \arrow[d, "g"]
      \\
      x
      \arrow[r, "f"]
      & y
    \end{tikzcd},
  \end{equation*}
  where $g$ is cartesian and $f$ lies over an isomorphism in $\Finp\op$, a dashed pullback exists, and for each such dashed pullback, $g'$ is cartesian and $f'$ lies over an isomorphism in $\Finp\op$. The existence of such a pullback is easy; one takes a pullback in $\Finp\op$, takes a $q$-cartesian lift $g'$, then fills an inner and an outer horn. That this is pullback is immediate because $g$ and $g'$ are $q$-cartesian. That any other pullback square has the necessary properties follows from the existence of a comparison equivalence.
\end{proof}

\subsection{Building categories of spans}
\label{ssc:building_categories_of_spans}

\begin{proposition}
  The map $p$ satisfies the conditions of the original Barwick's Theorem.
\end{proposition}
\begin{proof}
  See thesis.
\end{proof}

\begin{proposition}
  The map $r$ satisifes the conditions of the modified Barwick's Theorem.
\end{proposition}
\begin{proof}
  The first condition is obvious since $r$ is a cartesian fibration. In order to check the second condition, we fix a square $\sigma$ in $\Tw(\category{C})_{\otimes}$, lying over a square $r(\sigma)$ in $\TSCart$ and a square $q(\sigma)$ in $\Finp\op$. Fixing notation, we will let $q(\sigma)$ be the square
  \begin{equation*}
    \begin{tikzcd}
      \langle n \rangle
      & \langle n \rangle
      \arrow[l, swap, "\gamma'"]
      \\
      \langle m \rangle
      \arrow[u, "\tau'"]
      & \langle m \rangle
      \arrow[u, swap, "\tau"]
      \arrow[l, swap, "\gamma"]
    \end{tikzcd},
  \end{equation*}
  where $\gamma$ and $\gamma'$ are isomorphisms.

  The square $r(\sigma)$ thus has the form
  \begin{equation*}
    \begin{tikzcd}
      \vec{X}'
      \arrow[r, "f'"]
      \arrow[d, swap, "g'"]
      & \vec{Y}'
      \arrow[d, "g"]
      \\
      \vec{X}
      \arrow[r, "f"]
      & \vec{Y}
    \end{tikzcd},
  \end{equation*}
  where $f$ has components
  \begin{equation*}
    f_{i}\colon X_{i} \to Y_{\gamma^{-1}(i)},\qquad i \in \langle m \rangle^{\circ},
  \end{equation*}
  $f'$ has components
  \begin{equation*}
    f'_{j}\colon X'_{j} \to Y'_{\gamma'^{-1}(j)},\qquad j \in \langle n \rangle^{\circ},
  \end{equation*}
  $g$ has components
  \begin{equation*}
    g_{j}\colon Y'_{j} \to \prod_{\tau(i) = j} Y_{i},\qquad j \in \langle n \rangle^{\circ},
  \end{equation*}
  and $g'$ has components
  \begin{equation*}
    g'_{j}\colon X'_{j} \to \prod_{\tau'(i) = j} X_{i},\qquad j \in \langle n \rangle^{\circ}.
  \end{equation*}

  Note that the condition that $r(\sigma)$ be pullback is the condition that
  \begin{equation*}
    X'_{j} \simeq Y'_{j} \times_{\left( \prod_{\tau'(i) = j} Y_{i} \right)} \left( \prod_{\tau'(i) = j} X_{i} \right),
  \end{equation*}
  and that the maps $f'_{j}$ and $g'_{j}$ be the canonical projections.

  The data of the square $\sigma$ is thus given by a diagram of the form
  \begin{equation*}
    \begin{tikzcd}
      \vec{X}'
      \arrow[rrr, "f'"]
      \arrow[dr, "g'"]
      \arrow[ddd, swap, "\mathcal{F}'"]
      &&& \vec{Y}'
      \arrow[ddd, near end, "\mathcal{G}'"]
      \arrow[dr, "g"]
      %\arrow[from=R1, to=L2, Rightarrow, shorten=6ex]
      \\
      & \vec{X}
      \arrow[rrr, crossing over, "f"]
      &&& \vec{Y}
      \arrow[ddd, "\mathcal{G}"]
      \\
      \\
      \category{C}^{(n)}
      &&& \category{C}^{(n)}
      \arrow[lll, near start, "\gamma'_{\otimes}"]
      \\
      & \category{C}^{(m)}
      \arrow[ul, "\tau'_{\otimes}"]
      \arrow[from=uuu, crossing over, near start, "\mathcal{F}"]
      &&& \category{C}^{(m)}
      \arrow[lll, "\gamma_{\otimes}"]
      \arrow[ul, "\tau_{\otimes}"]
      %\arrow[from=R1, to=L3, phantom, "\circlearrowleft"{description}]
      %\arrow[from=R3, to=L4, Rightarrow, crossing over, shorten=6ex]
      %\arrow[from=R2, to=L4, phantom, "\circlearrowleft"{description}]
    \end{tikzcd},
  \end{equation*}
  where the left and right faces are thin, the top face is a pullback, the bottom face is a pushout, and the front face is a left Kan extension (i.e.\ each component is a left Kan extension). We need to show that the back face is a left Kan extension, i.e.\ that for each $j \in \langle n \rangle^{\circ}$, the square
  \begin{equation*}
    \begin{tikzcd}
      X'_{j}
      \arrow[r, "f'_{j}"]
      \arrow[d, swap, "\mathcal{F}'_{i}"]
      & Y'_{\gamma^{-1}(j)}
      \arrow[d, "\mathcal{G}'_{\gamma^{-1}(i)}"]
      \\
      \category{C}
      & \category{C}
      \arrow[l, equals]
    \end{tikzcd}
  \end{equation*}
  is left Kan.

  Using HTT 4.3.1.9, we can choose cartesian edges, transporting the pullback square in $\LS(\category{C})_{\otimes}$ above to a pullback square entirely in the fiber over $\langle n \rangle \in \Finp\op$. Since $(\LS(\category{C})_{\otimes})_{\langle n \rangle} \cong (\LS(\category{C}))^{n}$, this diagram consists of $n$ diagrams in $\LS(\category{C})$, one for each $j \in \langle n \rangle^{\circ}$, of the form
  \begin{equation*}
    \begin{tikzcd}
      X'_{j}
      \arrow[rrr, "f'_{j}"]
      \arrow[dr]
      \arrow[ddd, swap, "\mathcal{F}_{i}'"]
      &&& Y'_{\gamma'^{-1}(j)}
      \arrow[ddd, near end, "\mathcal{G}_{\gamma^{-1}(i)}'"]
      \arrow[dr]
      %\arrow[from=R1, to=L2, Rightarrow, shorten=6ex]
      \\
      & \prod_{\tau'(i) = j} X_{i}
      \arrow[rrr, crossing over]
      &&& \prod_{\tau(i) = j} Y_{i}
      \arrow[ddd, "\mathcal{B}"]
      \\
      \\
      \category{C}
      &&& \category{C}
      \arrow[lll, equals]
      \\
      & \category{C}
      \arrow[ul, equals]
      \arrow[from=uuu, crossing over, near start, "\mathcal{A}"]
      &&& \category{C}
      \arrow[lll, equals]
      \arrow[ul, equals]
      %\arrow[from=R1, to=L3, phantom, "\circlearrowleft"{description}]
      %\arrow[from=R3, to=L4, Rightarrow, crossing over, shorten=6ex]
      %\arrow[from=R2, to=L4, phantom, "\circlearrowleft"{description}]
    \end{tikzcd},
  \end{equation*}
  where the top face is pullback, the bottom face is trivially a pushout, and the left and right faces are thin. We want to show that the back face of each of these diagrams is a left Kan extension. If we can show that the front face is a left Kan extension, we will be done. The map $\mathcal{A}$ is the composition of the product of each of the maps $\mathcal{F}_{i}\colon X_{i} \to \category{C}$ with the tensor product $\category{C}^{\abs{\{\tau'(i) = j\}}} \to \category{C}$, and similarly for $\mathcal{B}$. Unravelling the conditions, one finds that the condition that $\category{B}$ be a pointwise left Kan extension of $\category{A}$ along the front-top-horizontal map above is precisely the condition that the tensor product commute with colimits.

  %OLD STUFF HERE

  %Here $\mathcal{F}$ has components
  %\begin{equation*}
  %  \mathcal{F}_{i}\colon X_{i} \to \category{C}_{i}\colon i \in \langle m \rangle^{\otimes}.
  %\end{equation*}

  %Kan extending $\mathcal{F}$ along $f$ we find a map $f_{!}\mathcal{F}\colon \vec{Y} \to \category{C}^{(m)}$, with components
  %\begin{equation*}
  %  (f_{!}\mathcal{F})_{i} \simeq (f_{\gamma(i)})_{!}\mathcal{F}_{\gamma(i)}\colon Y_{i} \to \category{C}_{i}.
  %\end{equation*}
  %Thus, we find that
  %\begin{equation*}
  %  \mathcal{G} \simeq \tau_{\otimes} \circ g^{*}f_{!}\mathcal{F},
  %\end{equation*}
  %with components given by the composition
  %\begin{equation*}
  %  \mathcal{G}_{j}\colon Y'_{j} \to \prod_{\tau(i) = j} Y_{j} \to \prod_{\tau(i) = j} \category{C}_{i} \to \category{C}_{j}.
  %\end{equation*}

  %Alternatively, starting from $\mathcal{F}$ and pulling back along the left face gives a map
  %\begin{equation*}
  %  \tau'_{\otimes} \circ g'^{*}\mathcal{F}
  %\end{equation*}
  %with components
  %\begin{equation*}
  %  (\tau'_{\otimes} \circ g'^{*}\mathcal{F})_{j}\colon X'_{j} \to \prod_{\tau'(i) = j} X_{j} \to \prod_{\tau'(i) = j} \category{C}_{i} \to \category{C}_{j}.
  %\end{equation*}

  %We need to show that
  %\begin{equation*}
  %  \mathcal{G} \simeq f'_{!}(\tau'_{\otimes} \circ g'^{*}\mathcal{F})
  %\end{equation*}
  %or equivalently, componentwise, that
  %\begin{equation*}
  %  \mathcal{G}_{j} \simeq (f'_{\gamma'(j)})_{!}(\tau'_{\otimes} \circ g'^{*}\mathcal{F})_{\gamma'(j)}
  %\end{equation*}
\end{proof}

\section{Constructing the symmetric monoidal functor}
\label{sec:constructing_the_symmetric_monoidal_functor}

The rest of the work follows my thesis pretty much exactly. We are now justified in building
\begin{equation*}
  \begin{tikzcd}
    \Span\triple{Q}
    \arrow[rr]
    \arrow[dr]
    && \Span\triple{P}
    \arrow[dl]
    \\
    & \Span\triple{F}
  \end{tikzcd}.
\end{equation*}

We pull back along $\Finp \overset{\simeq}{\to} \Span\triple{F}$, giving us isofibrations
\begin{equation*}
  \begin{tikzcd}
    P
    \arrow[rr]
    \arrow[dr]
    && Q
    \arrow[dl]
    \\
    & \Finp
  \end{tikzcd},
\end{equation*}
and it is easy to check that each of these maps has enough cocartesian lifts. The functor we are interested is the component of $P \to Q$ lying over $\langle 1 \rangle$; this classifies the functor
\begin{equation*}
  \Span(\S) \to \ICat
\end{equation*}
sending a span of spaces to push-pull, up to an $\op$. Composing with $\op\colon \ICat \to \ICat$ gives precisely the map we want. General abstract nonsense implies that this functor is lax monoidal.


\section{What does Barwick mean?}
\label{sec:what_does_barwick_mean_}

A few bits of terminology we'll refer to again and again:
\begin{itemize}
  \item An $\infty$-category is \defn{disjunctive} if it admits all limits and finite coproducts, and if finite coproducts are:
    \begin{itemize}
      \item \textbf{Disjoint:} Pushout diagrams
        \begin{equation*}
          \begin{tikzcd}
            \emptyset
            \arrow[r]
            \arrow[d]
            & A
            \arrow[d]
            \\
            B
            \arrow[r]
            & A \amalg B
          \end{tikzcd}
        \end{equation*}
        where $\emptyset$ is an initial object are also pullback.

      \item \textbf{Universal:} We have natural equivalences
        \begin{equation*}
          (X \amalg Y) \times_{B} C \cong \left( X \times_{B} C \right) \amalg \left( Y \times_{B} C \right)
        \end{equation*}
    \end{itemize}
    This looks like topos stuff. It is certainly satisfied for $\S$.
\end{itemize}

In section 1, Barwick defines some monadic generalization of CSMCs. In the first part of section 2, Barwick shows in some generality that for any $\infty$-category $\category{T}$ with some cartesian-like product, a monoidal structure is induced on $\Span(\category{T})$.

\subsection{The triple structures}
\label{ssc:the_triple_structures_bare}

\subsubsection{The triple structure on local systems}
\label{sss:the_triple_structure_on_local_systems}


We define a triple structure $\triple{Q}$ on $\LS(\category{C})$ as follows.
\begin{itemize}
  \item $\category{Q} = \LS(\category{C})$.

  \item $\category{Q}\downdag = \LS(\category{C})$.

  \item $\category{Q}\updag$ consists only of cartesian morphisms.
\end{itemize}

This corresponds to spans of the form
\begin{equation*}
  \begin{tikzcd}
    \category{D}'
    \arrow[d, ""{name=L, right}]
    & \category{D}
    \arrow[d, ""{name=ML, left}, ""{name=MR, right}]
    \arrow[l, swap]
    \arrow[r]
    & \category{D}''
    \arrow[d, ""{name=R, left}]
    \\
    \category{C}
    \arrow[r, equals]
    &\category{C}
    & \category{C}
    \arrow[l, equals]
    \arrow[Rightarrow, from=MR, to=R, shorten=2ex]
    \arrow[Rightarrow, from=ML, to=L, phantom, "\circlearrowleft"{description}]
  \end{tikzcd}
\end{equation*}

Such a span will correspond to a cocartesian morphism if it is of the form
\begin{equation*}
  \begin{tikzcd}
    \category{D}'
    \arrow[d, "\mathcal{F}"{left}, ""{name=L, right}]
    & \category{D}
    \arrow[d, "g^{*}\mathcal{F}"{description}, ""{name=ML, left}, ""{name=MR, right}]
    \arrow[l, swap, "g"]
    \arrow[r, "f"]
    & \category{D}''
    \arrow[d, ""{name=R, left}, "f_{!}g^{*}\mathcal{F}"{right}]
    \\
    \category{C}
    \arrow[r, equals]
    &\category{C}
    & \category{C}
    \arrow[l, equals]
    \arrow[Rightarrow, from=MR, to=R, shorten=2ex, "!"{description}]
    \arrow[from=ML, to=L, phantom, "\circlearrowleft"{description}]
  \end{tikzcd}
\end{equation*}

\begin{proposition}
  \label{prop:q_triple_is_adequate}
  The triple $\triple{Q}$ is adequate.
\end{proposition}

To do this, we need to prove:
\begin{lemma}
  \label{lemma:pullbacks_in_ttw}
  Let $\CC$ be an $\infty$-bicategory such that the underlying quasicategory $\category{C}$ has pullbacks and pushouts. Consider a square
  \begin{equation*}
    \sigma =\quad
    \begin{tikzcd}
      A
      \arrow[r]
      \arrow[d, swap, "a"]
      & B
      \arrow[d, "b"]
      \\
      B'
      \arrow[r]
      & C
    \end{tikzcd}
  \end{equation*}
  in $\Tw(\CC)$ corresponding to a twisted cube
  \begin{equation*}
    \begin{tikzcd}
      a
      \arrow[rrr]
      \arrow[dr, ""{name=TL, right}]
      \arrow[ddd, ""{name=R1, right}]
      &&& b
      \arrow[ddd, ""{name=L2, left}, ""{name=R2, right}]
      \arrow[dr, ""{name=TR, left}]
      \arrow[from=R1, to=L2, Rightarrow, shorten=7ex]
      \\
      & b'
      \arrow[rrr, crossing over]
      &&& c
      \arrow[ddd, ""{name=L4, left}"]
      \\
      \\
      \bar{a}
      &&& \bar{b}
      \arrow[lll]
      \\
      & \bar{b}'
      \arrow[from=uuu, crossing over, ""{name=L3, left}, ""{name=R3, right}]
      \arrow[ul]
      &&& \bar{c}
      \arrow[lll]
      \arrow[ul]
      \arrow[from=R1, to=L3, Rightarrow, shorten=3ex, "\alpha"]
      \arrow[from=R3, to=L4, Rightarrow, crossing over, shorten=7ex]
      \arrow[from=R2, to=L4, Rightarrow, shorten=3ex, "\beta"]
    \end{tikzcd}
  \end{equation*}
  in $\CC$.\footnote{Note that the top and bottom faces of any such diagram in $\Tw(\CC)$ belong to $\category{C}$, and hence must weakly commute by definition.} Suppose the right face is thin, the top face is a pullback, and the bottom face is a pushout. Then the square $\sigma$ is pullback if and only if the left face is thin.
\end{lemma}
\begin{proof}
  This corresponds to a square $\sigma$ in $\Tw(\CC)$ lying over a pullback square in $\category{C} \times \category{C}\op$, such that $b$ is $p$-cartesian. It is then classically known that $\sigma$ is pullback if and only if $a$ is $p$-cartesian.
\end{proof}

\begin{proof}[Proof of Proposition~\ref*{prop:q_triple_is_adequate}]
  We need to show that we can form pullbacks in $\Tw(\CC)$ of cospans one of whose legs is a cartesian morphism. Mapping the cospan to $\category{C} \times \category{C}\op$ using $r$, We can form the pullback, then fill to a relative pullback by finding a cartesian lift $a$, then filling an inner and an outer horn. This diagram is pullback by \hyperref[lemma:pullbacks_in_ttw]{Lemma~\ref*{lemma:pullbacks_in_ttw}}

  We also need to show that any such pullback is of this form. This also follows from \hyperref[lemma:pullbacks_in_ttw]{Lemma~\ref*{lemma:pullbacks_in_ttw}}.
\end{proof}

\subsection{Building categories of spans}
\label{ssc:building_categories_of_spans_bare}

\begin{lemma}
  \label{lemma:homotopy_pullbacks_and_overcategories}
  Let
  \begin{equation*}
    \begin{tikzcd}
      X \times_{Y}^{h} Y'
      \arrow[r]
      \arrow[d]
      & Y'
      \arrow[d]
      \\
      X
      \arrow[r]
      & Y
    \end{tikzcd}
  \end{equation*}
  be a homotopy pullback diagram of Kan complexes, and let $y' \in Y'$. Then
  \begin{equation}
    \label{eq:overcategory_commutes_with_homotopy_pullback}
    (X \times_{Y}^{h}Y')_{/y'} \simeq X \times_{Y}^{h} (Y'_{/y'})
  \end{equation}
\end{lemma}
\begin{proof}
  We can model the homotopy pullback $X \times^{h}_{Y}Y'$ as the strict pullback
  \begin{equation*}
    Y^{\Delta^{1}} \times_{Y \times Y} (X \times Y').
  \end{equation*}
  The left-hand side of \hyperref[eq:overcategory_commutes_with_homotopy_pullback]{Equation~\ref*{eq:overcategory_commutes_with_homotopy_pullback}} is then given by the pullback
  \begin{equation*}
    \left(Y^{\Delta^{1}} \times_{Y \times Y}(X \times Y')\right) \times_{Y'} Y'_{/y'}.
  \end{equation*}
  But this is isomorphic to
  \begin{equation*}
    Y^{\Delta^{1}} \times_{Y \times Y} (X \times Y^{'}_{/y'}),
  \end{equation*}
  which is a model for the right-hand side.
\end{proof}

\begin{proposition}
  The map $p\colon \triple{P} \to \triple{Q}$ of triples whose underlying map is $p\colon \LS(\category{C}) \to \S$ is adequate, i.e.\ satisfies the conditions of our modified version of Barwick's theorem.
\end{proposition}
\begin{proof}
  We need to show that for any square
  \begin{equation*}
    \sigma =\quad
    \begin{tikzcd}
      x'
      \arrow[r, "f'"]
      \arrow[d, swap, "g'"]
      & y'
      \arrow[d, "g"]
      \\
      x
      \arrow[r, "f"]
      & y
    \end{tikzcd}
  \end{equation*}
  in $\LS(\category{C})$ lying over a pullback square
  \begin{equation*}
    \begin{tikzcd}
      X'
      \arrow[r, "f'"]
      \arrow[d, swap, "g'"]
      & Y'
      \arrow[d, "g"]
      \\
      X
      \arrow[r, "f"]
      & Y
    \end{tikzcd}
  \end{equation*}
  in $\S$ (corresponding to a \emph{homotopy} pullback of Kan complexes), such that $g$ and $g'$ are $p$-cartesian and $f$ is $p$-cocartesian, $f'$ is $p$-cocartesian. The square $\sigma$ corresponds to a twisted cube
  \begin{equation*}
    \begin{tikzcd}
      X'
      \arrow[rrr, "f'"]
      \arrow[dr, ""{name=TL, below}, "g'"{swap}]
      \arrow[ddd, ""{name=R1, below}]
      &&& Y'
      \arrow[ddd, ""{name=L2, below}, ""{name=R2, above}]
      \arrow[dr, ""{name=TR, below}, "g"]
      \arrow[from=R1, to=L2, Rightarrow, shorten=6ex]
      \\
      & X
      \arrow[rrr, crossing over, "f"]
      &&& Y
      \arrow[ddd, ""{name=L4, below}"]
      \\
      \\
      \category{C}
      &&& \category{C}
      \arrow[lll, equals]
      \\
      & \category{C}
      \arrow[from=uuu, crossing over, ""{name=L3, below}, ""{name=R3, above}]
      \arrow[ul, equals]
      &&& \category{C}
      \arrow[lll, equals]
      \arrow[ul, equals]
      \arrow[from=R1, to=L3, phantom, "\circlearrowleft"{description}]
      \arrow[from=R3, to=L4, Rightarrow, crossing over, shorten=6ex]
      \arrow[from=R2, to=L4, phantom, "\circlearrowleft"{description}]
    \end{tikzcd}
  \end{equation*}
  in which the left and right faces are thin, the top face is a pullback, the bottom face is a pushout, and the front face corresponds to a left Kan extension. We need to show that the back face corresponds to a left Kan extension. For ease of notation, we flatten this out and add some more labels.
  \begin{equation*}
    \begin{tikzcd}
      X'
      \arrow[rr, "f'"]
      \arrow[dd, swap, "g'"]
      \arrow[dr, "g'^{*}\mathcal{F}"]
      && Y'
      \arrow[dd, "g"]
      \arrow[dl, "\mathcal{G}"]
      \\
      & \category{C}
      \\
      X
      \arrow[ur, "\mathcal{F}"]
      \arrow[rr, swap, "f"]
      && Y
      \arrow[ul, "f_{!}\mathcal{F}"]
    \end{tikzcd}
  \end{equation*}
  We need to show that $\mathcal{G}$ is a left Kan extension of $g'^{*}\mathcal{F}$ along $f'$, i.e.\ that for all $y' \in Y'$,
  \begin{align*}
    \mathcal{G}(y') &\simeq \colim\left[ X'_{/y'} \to X \to \category{C} \right] \\
    &\simeq \colim\left[ X \times_{Y} (Y'_{/y'}) \to X \to \category{C} \right],
  \end{align*}
  where here we mean by $X' \simeq X \times_{Y} Y'$ the homotopy pullback, and we have used \hyperref[lemma:homotopy_pullbacks_and_overcategories]{Lemma~\ref*{lemma:homotopy_pullbacks_and_overcategories}}. We know that $\mathcal{G}$ is the pullback along $g$ of $f_{!}\mathcal{F}$, i.e.\ that
  \begin{align*}
    \mathcal{G}(y') &\simeq \colim\left[ X_{/g(y')} \to X \to \category{C} \right] \\
    &\simeq \colim \left[ X \times_{Y} (Y_{/g(y')}) \to X \to \category{C} \right].
  \end{align*}
  The map $X \times_{Y} (Y'_{/y'}) \to X$ factors through $X \times_{Y} (Y_{/g(y')})$ thanks to the map $s\colon Y'_{/y'} \to Y_{/g(y')}$. The map $s$ is a weak equivalence between contractible Kan complexes (because both $Y'_{/y'}$ and $Y_{/g(y')}$ are Kan complexes with terminal objects). Thus, the map $X \times_{Y}(Y'_{/y'}) \to X \times_{Y}(Y_{/g(y')})$ is also a weak homotopy equivalence between Kan complexes, thus cofinal.
\end{proof}

\section{Extensions in infinity-bicategories}
\label{sec:extensions_in_infinity_bicategories}

\begin{definition}
  Let $p\colon \CC \to \DD$ be a functor between $\infty$-bicategories. We will call an edge $e\colon \Delta^{1} \to \CC$ \defn{locally outer cocartesian} if it is a cocartesian edge in the total space of the map $\category{M} \to \Delta^{1}$ defined by the pullback
  \begin{equation*}
    \begin{tikzcd}
      \category{M}
      \arrow[r]
      \arrow[d]
      & \CC
      \arrow[d]
      \\
      \Delta^{1}
      \arrow[r]
      & \DD
    \end{tikzcd}.
  \end{equation*}
\end{definition}

Note that if an edge in $\CC$ is outer $p$-cocartesian, then it is locally outer $p$-cocartesian.

\begin{lemma}
  Suppose $p\colon \CC \to \DD$ is an outer cartesian fibration.
\end{lemma}

\end{document}
