\documentclass[main.tex]{subfiles}
\begin{document}

\section{Miscellaneous stuff to be deleted or absorbed}
\label{sec:miscellaneous_stuff_to_be_deleted_or_absorbed}

\subsection{Lax monoidal Beck-Chevalley fibrations}
\label{ssc:lax_monoidal_beck_chevalley_fibrations}

Consider a Beck-Chevalley fibration $r\colon \category{X} \to \category{T}$ such that $\category{T}$ has a terminal object, and hence has finite products. Suppose that $\category{X}$ carries a symmetric monoidal structure $(\otimes, I)$ with the following property: suppose $r(x) = t$ and $r(y) = t'$. Then $r(x \otimes y) = t \times t'$. In other words, suppose the tensor product restricts, for each pair $(t, t') \in \category{T} \times \category{T}$, to a functor $\mu_{tt'}\colon \category{X}_{t} \times \category{X}_{t'} \to \category{X}_{t \times t'}$.

The 

We can express symmetric monoidal categories as cartesian rather than cocartesian fibrations.

\begin{note}
  For this to work out, I think we probably need to modify the second condition to be an adequate triple so that we only need to consider pullbacks of the type given in the first condition.
\end{note}

\begin{lemma}
  The category $\Finp$ admits finite colimits.
\end{lemma}
\begin{proof}
  A diagram $K^{\triangleright} \to \Finp$ is a colimit diagram if and only if the corresponding diagram $\{\ast\} \star (K^{\triangleright}) \to \Fin$ is a colimit diagram.
\end{proof}

\begin{definition}
  A \defn{CSMC} (contravariantly-presented symmetric monoidal $\infty$-category) is a cocartesian fibration $p\colon \category{O}_{\otimes} \to \Finp\op$ such that the pullback maps $(\rho_{i})^{*}\colon (\category{O}_{ \otimes})_{\langle n \rangle} \to (\category{O}_{ \otimes})_{\langle 1 \rangle}$ are the canonical projections exhibiting $(\category{O}_{ \otimes})_{\langle n \rangle}$ as an $n$-fold product.
\end{definition}

\begin{definition}
  Let 
  \begin{equation*}
    \begin{tikzcd}
      \category{C}_{\otimes} 
      \arrow[rr, "r"]
      \arrow[dr, swap, "q"]
      && \category{O}_{\otimes} 
      \arrow[dl, "p"]
      \\
      & \Finp\op
    \end{tikzcd}
  \end{equation*}
  be a map between CSMC. The map $r$ is a \defn{monoidal functor} if it sends $q$-cartesian morphisms to $p$-cartesian morphisms.
\end{definition}

\begin{lemma}
  If
  \begin{equation*}
    \begin{tikzcd}
      \category{C}_{\otimes} 
      \arrow[rr, "r"]
      \arrow[dr, swap, "q"]
      && \category{O}_{\otimes} 
      \arrow[dl, "p"]
      \\
      & \Finp\op
    \end{tikzcd}
  \end{equation*}
  is a map between CSMC such that $r$ is a cartesian fibration, then $r$ is monoidal.
\end{lemma}

\begin{definition}
  A \defn{CSMC} is \defn{cartesian} if for all objects $X$ and $Y$, the canonical maps $X \otimes I \leftarrow X \otimes Y \to I \otimes Y$ exhibit $X \otimes Y$ as the product of $X$ and $Y$.
\end{definition}

\begin{proposition}
  Suppose $\category{T}$ is an $\infty$-category with pullbacks and a terminal object. Then $\category{T}$ is adequate, and there exists a symmetric monoidal category $\Span(\category{T})^{\times} \to \Finp$.
\end{proposition}

\begin{definition}
  A \defn{lax monoidal Beck-Chevalley fibration} is a commutative diagram
  \begin{equation*}
    \begin{tikzcd}
      \category{X}_{\otimes}
      \arrow[rr, "r"]
      \arrow[dr, swap, "q"]
      && \category{T}_{\times}
      \arrow[dl, "p"]
      \\
      & \Finp\op
    \end{tikzcd}
  \end{equation*}
  with the following characteristics.
  \begin{itemize}
    \item The map $p$ is a cartesian CSMC.

    \item The category $\category{T}$ underlying $\category{T}_{\times}$ admits pullbacks.

    \item The map $q$ is a CSMC.

    \item The map $r$ is a cartesian fibration (hence monoidal).

    \item The functor $r|\langle 1 \rangle$ is a cocartesian fibration, and a morphism in $r|\langle n \rangle$ is $r$-cocartesian if and only if each component is $r$-cartesian.

    \item The functor $r$ obeys the following interchange law: for any diagram
      \begin{equation*}
        \begin{tikzcd}
          \vec{z}
          \arrow[r, "f'"]
          \arrow[d, swap, "g'"]
          & \vec{y}
          \arrow[d, "g"]
          \\
          \vec{y}'
          \arrow[r, "f"]
          & \vec{x}
        \end{tikzcd}
      \end{equation*}
      in $\category{X}_{\otimes}$ whose image in $\category{T}_{\otimes}$ is pullback, and which lies over a square
      \begin{equation*}
        \begin{tikzcd}
          \langle n \rangle
          & \langle n \rangle
          \arrow[l, equals]
          \\
          \langle m \rangle
          \arrow[u, "\phi"]
          & \langle m \rangle
          \arrow[l, equals]
          \arrow[u, swap, "\psi"]
        \end{tikzcd},
      \end{equation*}
      in $\Finp\op$, if $g$ and $g'$ are $r$-cartesian and $f$ is $r$-cocartesian, then $f'$ is $r$-cocartesian.
  \end{itemize}
\end{definition}

\begin{proposition}
  For any lax monoidal Beck-Chevalley fibration as in DEF, there is a diagram
  \begin{equation*}
    \begin{tikzcd}
      \Span'(\category{X})^{\otimes}
      \arrow[rr, "\rho"]
      \arrow[dr, swap, "\varpi"]
      && \Span(\category{T})^{\otimes}
      \arrow[dl, "\pi"]
      \\
      & \Finp
    \end{tikzcd}
  \end{equation*}
  where $\pi$ and $\varpi$ are symmetric monoidal categories and $\rho$ exhibits $\Span'(\category{X})^{\otimes}$ as a $\Span(\category{T})^{\otimes}$-monoidal category. Straightening, one finds a lax monoidal functor
  \begin{equation*}
    \hat{r}\colon (\Span(\category{T}), \widetilde{\times}) \to (\ICat, \times)
  \end{equation*}
  with the following description up to equivalence.
  \begin{itemize}
    \item On objects, the functor $\hat{r}$ sends $t \in \Span(\category{T})$ to the fiber $\category{X}_{t} \in \ICat$
    \item On morphisms, $\hat{r}$ sends a span $t \overset{g}{\leftarrow} s \overset{f}{\to} t'$ to the composition $g^{*} \circ f_{!}\colon \category{X}_{t} \to \category{X}_{t'}$.
  \end{itemize}
\end{proposition}
\begin{proof}
  We will consider the following triples.

  \begin{itemize}
    \item We define a triple structure $\triple{F}$ on $\Finp\op$, where
      \begin{itemize}
        \item $\category{F} = \Finp\op$

        \item $\category{F}\downdag = (\Finp\op)^{\simeq}$

        \item $\category{F}\updag = \Finp\op$
      \end{itemize}

      This is obviously adequate. We note that $\Finp\op$ has pullbacks, and finite limits in general, which is more than we need in this case, but will be useful later.

    \item We define a triple $\triple{T}$ as follows.
      \begin{itemize}
        \item $\category{T} = \category{T}_{\times}$

        \item $\category{T}\downdag = \category{T}_{\times} \times_{\Finp\op}(\Finp\op)^{\simeq}$

        \item $\category{T}\updag = \category{T}_{\times}$
      \end{itemize}

      To see that this is adequate, we first note that $p$ admits relative pullbacks; each fiber admits pullbacks by virtue our assumption that $\category{T}$ admits pullbacks, and the identifications $( \category{T}_{\times} )_{\langle n \rangle} \simeq \category{T}^{n}$. The functoriality coming from $p$ implements products, and therefore commute with pullbacks. Thus, $\category{T}_{\times}$ admits pullbacks, and $p$ preserves pullbacks. This immediately implies that $\triple{T}$ is adequate:
      \begin{enumerate}
        \item Pushouts of this form are simply squares with horizontal morphisms given by equivalences.

        \item Any pullback square lies over a pullback square in $\Finp\op$, so this reduces to the lemma about cartesian morphisms.
      \end{enumerate}

    \item We define a triple structure $\triple{X}$ as follows.
      \begin{itemize}
        \item $\category{X} = \category{X}_{\otimes}$.

        \item $\category{X}\downdag = (\category{X}_{\otimes})_{\Finp\op}(\Finp\op)^{\simeq}$.

        \item $\category{X}\updag$ consists only of $r$-cartesian morphisms.
      \end{itemize}

      One sees that this is adequate, since pullbacks of the necessary form exist by the usual procedure:
      \begin{itemize}
        \item Map the diagram down to $\Finp\op$, take the pullback there.

        \item Take a relative pullback in $\category{T}_{\times}$.

        \item Take an $r$-cartesian lift to produce an $r$-relative pullback in $\category{X}_{\times}$. This lies over a pullback in $\category{T}_{\times}$, hence is a pullback.
      \end{itemize}
  \end{itemize}

  Thus, we have maps of adequate triples
  \begin{equation*}
    \begin{tikzcd}
      \triple{X}
      \arrow[rr]
      \arrow[dr]
      && \triple{T}
      \arrow[dl]
      \\
      & \triple{F}
    \end{tikzcd}
  \end{equation*}
  giving us maps
  \begin{equation*}
    \begin{tikzcd}
      \Span\triple{X}
      \arrow[rr]
      \arrow[dr]
      && \Span\triple{T}
      \arrow[dl]
      \\
      & \Span\triple{F}
    \end{tikzcd}.
  \end{equation*}
  Pulling back along the map $\Finp \to \Span\triple{F}$,
\end{proof}

\section{Partial adjunctions}
\label{sec:partial_adjunctions}

\section{What does Barwick mean?}
\label{sec:what_does_barwick_mean_}

A few bits of terminology we'll refer to again and again:
\begin{itemize}
  \item An $\infty$-category is \defn{disjunctive} if it admits all limits and finite coproducts, and if finite coproducts are:
    \begin{itemize}
      \item \textbf{Disjoint:} Pushout diagrams
        \begin{equation*}
          \begin{tikzcd}
            \emptyset
            \arrow[r]
            \arrow[d]
            & A
            \arrow[d]
            \\
            B
            \arrow[r]
            & A \amalg B
          \end{tikzcd}
        \end{equation*}
        where $\emptyset$ is an initial object are also pullback.

      \item \textbf{Universal:} We have natural equivalences
        \begin{equation*}
          (X \amalg Y) \times_{B} C \cong \left( X \times_{B} C \right) \amalg \left( Y \times_{B} C \right)
        \end{equation*}
    \end{itemize}
    This looks like topos stuff. It is certainly satisfied for $\S$.
\end{itemize}

In section 1, Barwick defines some monadic generalization of CSMCs. In the first part of section 2, Barwick shows in some generality that for any $\infty$-category $\category{T}$ with some cartesian-like product, a monoidal structure is induced on $\Span(\category{T})$.

\subsection{The triple structures}
\label{ssc:the_triple_structures_bare}

\subsubsection{The triple structure on local systems}
\label{sss:the_triple_structure_on_local_systems}


We define a triple structure $\triple{Q}$ on $\LS(\category{C})$ as follows.
\begin{itemize}
  \item $\category{Q} = \LS(\category{C})$.

  \item $\category{Q}\downdag = \LS(\category{C})$.

  \item $\category{Q}\updag$ consists only of cartesian morphisms.
\end{itemize}

This corresponds to spans of the form
\begin{equation*}
  \begin{tikzcd}
    \category{D}'
    \arrow[d, ""{name=L, right}]
    & \category{D}
    \arrow[d, ""{name=ML, left}, ""{name=MR, right}]
    \arrow[l, swap]
    \arrow[r]
    & \category{D}''
    \arrow[d, ""{name=R, left}]
    \\
    \category{C}
    \arrow[r, equals]
    &\category{C}
    & \category{C}
    \arrow[l, equals]
    \arrow[Rightarrow, from=MR, to=R, shorten=2ex]
    \arrow[Rightarrow, from=ML, to=L, phantom, "\circlearrowleft"{description}]
  \end{tikzcd}
\end{equation*}

Such a span will correspond to a cocartesian morphism if it is of the form
\begin{equation*}
  \begin{tikzcd}
    \category{D}'
    \arrow[d, "\mathcal{F}"{left}, ""{name=L, right}]
    & \category{D}
    \arrow[d, "g^{*}\mathcal{F}"{description}, ""{name=ML, left}, ""{name=MR, right}]
    \arrow[l, swap, "g"]
    \arrow[r, "f"]
    & \category{D}''
    \arrow[d, ""{name=R, left}, "f_{!}g^{*}\mathcal{F}"{right}]
    \\
    \category{C}
    \arrow[r, equals]
    &\category{C}
    & \category{C}
    \arrow[l, equals]
    \arrow[Rightarrow, from=MR, to=R, shorten=2ex, "!"{description}]
    \arrow[from=ML, to=L, phantom, "\circlearrowleft"{description}]
  \end{tikzcd}
\end{equation*}

\begin{proposition}
  \label{prop:q_triple_is_adequate}
  The triple $\triple{Q}$ is adequate.
\end{proposition}

To do this, we need to prove:
\begin{lemma}
  \label{lemma:pullbacks_in_ttw}
  Let $\CC$ be an $\infty$-bicategory such that the underlying quasicategory $\category{C}$ has pullbacks and pushouts. Consider a square
  \begin{equation*}
    \sigma =\quad
    \begin{tikzcd}
      A
      \arrow[r]
      \arrow[d, swap, "a"]
      & B
      \arrow[d, "b"]
      \\
      B'
      \arrow[r]
      & C
    \end{tikzcd}
  \end{equation*}
  in $\Tw(\CC)$ corresponding to a twisted cube
  \begin{equation*}
    \begin{tikzcd}
      a
      \arrow[rrr]
      \arrow[dr, ""{name=TL, right}]
      \arrow[ddd, ""{name=R1, right}]
      &&& b
      \arrow[ddd, ""{name=L2, left}, ""{name=R2, right}]
      \arrow[dr, ""{name=TR, left}]
      \arrow[from=R1, to=L2, Rightarrow, shorten=7ex]
      \\
      & b'
      \arrow[rrr, crossing over]
      &&& c
      \arrow[ddd, ""{name=L4, left}"]
      \\
      \\
      \bar{a}
      &&& \bar{b}
      \arrow[lll]
      \\
      & \bar{b}'
      \arrow[from=uuu, crossing over, ""{name=L3, left}, ""{name=R3, right}]
      \arrow[ul]
      &&& \bar{c}
      \arrow[lll]
      \arrow[ul]
      \arrow[from=R1, to=L3, Rightarrow, shorten=3ex, "\alpha"]
      \arrow[from=R3, to=L4, Rightarrow, crossing over, shorten=7ex]
      \arrow[from=R2, to=L4, Rightarrow, shorten=3ex, "\beta"]
    \end{tikzcd}
  \end{equation*}
  in $\CC$.\footnote{Note that the top and bottom faces of any such diagram in $\Tw(\CC)$ belong to $\category{C}$, and hence must weakly commute by definition.} Suppose the right face is thin, the top face is a pullback, and the bottom face is a pushout. Then the square $\sigma$ is pullback if and only if the left face is thin.
\end{lemma}
\begin{proof}
  This corresponds to a square $\sigma$ in $\Tw(\CC)$ lying over a pullback square in $\category{C} \times \category{C}\op$, such that $b$ is $p$-cartesian. It is then classically known that $\sigma$ is pullback if and only if $a$ is $p$-cartesian.
\end{proof}

\begin{proof}[Proof of Proposition~\ref*{prop:q_triple_is_adequate}]
  We need to show that we can form pullbacks in $\Tw(\CC)$ of cospans one of whose legs is a cartesian morphism. Mapping the cospan to $\category{C} \times \category{C}\op$ using $r$, We can form the pullback, then fill to a relative pullback by finding a cartesian lift $a$, then filling an inner and an outer horn. This diagram is pullback by \hyperref[lemma:pullbacks_in_ttw]{Lemma~\ref*{lemma:pullbacks_in_ttw}}

  We also need to show that any such pullback is of this form. This also follows from \hyperref[lemma:pullbacks_in_ttw]{Lemma~\ref*{lemma:pullbacks_in_ttw}}.
\end{proof}

\subsection{Building categories of spans}
\label{ssc:building_categories_of_spans_bare}

\begin{lemma}
  \label{lemma:homotopy_pullbacks_and_overcategories}
  Let
  \begin{equation*}
    \begin{tikzcd}
      X \times_{Y}^{h} Y'
      \arrow[r]
      \arrow[d]
      & Y'
      \arrow[d]
      \\
      X
      \arrow[r]
      & Y
    \end{tikzcd}
  \end{equation*}
  be a homotopy pullback diagram of Kan complexes, and let $y' \in Y'$. Then
  \begin{equation}
    \label{eq:overcategory_commutes_with_homotopy_pullback}
    (X \times_{Y}^{h}Y')_{/y'} \simeq X \times_{Y}^{h} (Y'_{/y'})
  \end{equation}
\end{lemma}
\begin{proof}
  We can model the homotopy pullback $X \times^{h}_{Y}Y'$ as the strict pullback
  \begin{equation*}
    Y^{\Delta^{1}} \times_{Y \times Y} (X \times Y').
  \end{equation*}
  The left-hand side of \hyperref[eq:overcategory_commutes_with_homotopy_pullback]{Equation~\ref*{eq:overcategory_commutes_with_homotopy_pullback}} is then given by the pullback
  \begin{equation*}
    \left(Y^{\Delta^{1}} \times_{Y \times Y}(X \times Y')\right) \times_{Y'} Y'_{/y'}.
  \end{equation*}
  But this is isomorphic to
  \begin{equation*}
    Y^{\Delta^{1}} \times_{Y \times Y} (X \times Y^{'}_{/y'}),
  \end{equation*}
  which is a model for the right-hand side.
\end{proof}

\begin{proposition}
  The map $p\colon \triple{P} \to \triple{Q}$ of triples whose underlying map is $p\colon \LS(\category{C}) \to \S$ is adequate, i.e.\ satisfies the conditions of our modified version of Barwick's theorem.
\end{proposition}
\begin{proof}
  We need to show that for any square
  \begin{equation*}
    \sigma =\quad
    \begin{tikzcd}
      x'
      \arrow[r, "f'"]
      \arrow[d, swap, "g'"]
      & y'
      \arrow[d, "g"]
      \\
      x
      \arrow[r, "f"]
      & y
    \end{tikzcd}
  \end{equation*}
  in $\LS(\category{C})$ lying over a pullback square
  \begin{equation*}
    \begin{tikzcd}
      X'
      \arrow[r, "f'"]
      \arrow[d, swap, "g'"]
      & Y'
      \arrow[d, "g"]
      \\
      X
      \arrow[r, "f"]
      & Y
    \end{tikzcd}
  \end{equation*}
  in $\S$ (corresponding to a \emph{homotopy} pullback of Kan complexes), such that $g$ and $g'$ are $p$-cartesian and $f$ is $p$-cocartesian, $f'$ is $p$-cocartesian. The square $\sigma$ corresponds to a twisted cube
  \begin{equation*}
    \begin{tikzcd}
      X'
      \arrow[rrr, "f'"]
      \arrow[dr, ""{name=TL, below}, "g'"{swap}]
      \arrow[ddd, ""{name=R1, below}]
      &&& Y'
      \arrow[ddd, ""{name=L2, below}, ""{name=R2, above}]
      \arrow[dr, ""{name=TR, below}, "g"]
      \arrow[from=R1, to=L2, Rightarrow, shorten=6ex]
      \\
      & X
      \arrow[rrr, crossing over, "f"]
      &&& Y
      \arrow[ddd, ""{name=L4, below}"]
      \\
      \\
      \category{C}
      &&& \category{C}
      \arrow[lll, equals]
      \\
      & \category{C}
      \arrow[from=uuu, crossing over, ""{name=L3, below}, ""{name=R3, above}]
      \arrow[ul, equals]
      &&& \category{C}
      \arrow[lll, equals]
      \arrow[ul, equals]
      \arrow[from=R1, to=L3, phantom, "\circlearrowleft"{description}]
      \arrow[from=R3, to=L4, Rightarrow, crossing over, shorten=6ex]
      \arrow[from=R2, to=L4, phantom, "\circlearrowleft"{description}]
    \end{tikzcd}
  \end{equation*}
  in which the left and right faces are thin, the top face is a pullback, the bottom face is a pushout, and the front face corresponds to a left Kan extension. We need to show that the back face corresponds to a left Kan extension. For ease of notation, we flatten this out and add some more labels.
  \begin{equation*}
    \begin{tikzcd}
      X'
      \arrow[rr, "f'"]
      \arrow[dd, swap, "g'"]
      \arrow[dr, "g'^{*}\mathcal{F}"]
      && Y'
      \arrow[dd, "g"]
      \arrow[dl, "\mathcal{G}"]
      \\
      & \category{C}
      \\
      X
      \arrow[ur, "\mathcal{F}"]
      \arrow[rr, swap, "f"]
      && Y
      \arrow[ul, "f_{!}\mathcal{F}"]
    \end{tikzcd}
  \end{equation*}
  We need to show that $\mathcal{G}$ is a left Kan extension of $g'^{*}\mathcal{F}$ along $f'$, i.e.\ that for all $y' \in Y'$,
  \begin{align*}
    \mathcal{G}(y') &\simeq \colim\left[ X'_{/y'} \to X \to \category{C} \right] \\
    &\simeq \colim\left[ X \times_{Y} (Y'_{/y'}) \to X \to \category{C} \right],
  \end{align*}
  where here we mean by $X' \simeq X \times_{Y} Y'$ the homotopy pullback, and we have used \hyperref[lemma:homotopy_pullbacks_and_overcategories]{Lemma~\ref*{lemma:homotopy_pullbacks_and_overcategories}}. We know that $\mathcal{G}$ is the pullback along $g$ of $f_{!}\mathcal{F}$, i.e.\ that
  \begin{align*}
    \mathcal{G}(y') &\simeq \colim\left[ X_{/g(y')} \to X \to \category{C} \right] \\
    &\simeq \colim \left[ X \times_{Y} (Y_{/g(y')}) \to X \to \category{C} \right].
  \end{align*}
  The map $X \times_{Y} (Y'_{/y'}) \to X$ factors through $X \times_{Y} (Y_{/g(y')})$ thanks to the map $s\colon Y'_{/y'} \to Y_{/g(y')}$. The map $s$ is a weak equivalence between contractible Kan complexes (because both $Y'_{/y'}$ and $Y_{/g(y')}$ are Kan complexes with terminal objects). Thus, the map $X \times_{Y}(Y'_{/y'}) \to X \times_{Y}(Y_{/g(y')})$ is also a weak homotopy equivalence between Kan complexes, thus cofinal.
\end{proof}

\section{Monoidal categories as cartesian fibrations}
\label{sec:monoidal_categories_as_cartesian_fibrations}

\begin{definition}[contravariantly presented monoidal category]
  \label{def:contravariantly_presented_monoidal_category}
  A \defn{contravariantly presented monoidal category} is a cartesian fibration
  \begin{equation*}
    \category{C}_{\otimes} \to \Finp\op
  \end{equation*}
  such that the pullback maps lying over the inert maps $\langle 1 \rangle \leftarrow \langle n \rangle$ exhibit $(\category{C}_{\otimes})_{\langle n \rangle}$ as the $n$-fold product of $(\category{C}_{\otimes})_{\langle 1 \rangle}$.
\end{definition}

\begin{example}
  The $\infty$-category $\ICat$ has a cartesian product, giving us a monoid
\end{example}

\end{document}
