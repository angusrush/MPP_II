\documentclass[main.tex]{subfiles}

\begin{document}

\section{The non-monoidal construction}
\label{sec:the_non_monoidal_construction}

This is a classical definition, here more or less lifted from \cite{luriehopkins2013ambidexterity}.

\begin{definition}
  \label{def:beck_chevalley_fibration}
  A bicartesian fibration of quasicategories $p\colon \category{X} \to \category{T}$ such that $\category{X}$ admits pullbacks is a \defn{Beck-Chevalley fibration} if it has the following property.
  \begin{itemize}
    \item[(BC)]\label{item:beck_chevalley_condition} For any commuting square in
      \begin{equation*}
        \begin{tikzcd}
          x'
          \arrow[r, "f'"]
          \arrow[d, swap, "g'"]
          & y'
          \arrow[d, "g"]
          \\
          x
          \arrow[r, "f"]
          & y
        \end{tikzcd}
      \end{equation*}
      lying over a pullback square in $\category{T}$, if the morphism $f$ is $p$-cocartesian, and the morphisms $g$ and $g'$ are $p$-cartesian, then the morphism $f'$ is $p$-cocartesian.
  \end{itemize}
\end{definition}

The proof of the following proposition will come at the end of this section.

\begin{proposition}
  \label{prop:local_systems_are_beck_chevalley}
  The functor $p\colon \LS(\category{C}) \to \S$ is a Beck-Chevalley fibration.
\end{proposition}

The importance of Beck-Chevalley fibrations is the following.

\begin{proposition}
  Let $p\colon \category{X} \to \category{T}$ be a Beck-Chevalley fibration. Then there is a cocartesian fibration
  \begin{equation*}
    \tilde{p}\colon \Span^{\mathrm{cart}}(\category{X}) \to \Span(\category{T}),
  \end{equation*}
  where $\Span^{\mathrm{cart}}$ denotes the category of spans whose backwards-facing legs are constrained to be $p$-cartesian; that is, $\category{X}^{\dagger} = \category{X}^{\mathrm{cart}}$.
\end{proposition}
\begin{proof}
  Follows immediately from \hyperref[thm:new_barwick]{Theorem~\ref*{thm:new_barwick}}.
\end{proof}

The following is simply a reformulation of the definition of a left Kan extension (\hyperref[def:nat_xfo_exhibiting_left_kan_ext]{Definition~\ref*{def:nat_xfo_exhibiting_left_kan_ext}}) in the special case that we consider left Kan extensions along maps of Kan complexes.
\begin{lemma}
  Let $f\colon X \to Y$ be a map between Kan complexes, let $F\colon X \to \category{C}$ and $G\colon Y \to \category{C}$ be functors, and let $\eta\colon F \to G \circ y$ be a natural transformation. Then $\eta$ exhibits $G$ as a left Kan extension of $F$ along $f$ if and only if, for all $y \in Y$, the natural transformatin $F \circ \pi \Rightarrow \underline{G(y)}$ defined by the pasting diagram
  \begin{equation*}
    \begin{tikzcd}[row sep=large, column sep=large]
      X_{/y}
      \arrow[r, "\pi", ""{name=LA, swap}]
      \arrow[d, ""{name=LD}]
      & X
      \arrow[d, swap, "f"]
      \arrow[r, "F", ""{below, name=M}]
      & \category{C}
      \\
      \{y\}
      \arrow[r, hook, ""{name=LB}]
      & Y
      \arrow[ur, swap, "G"]
      \arrow[from=M, Rightarrow, shorten=2ex, swap, "\eta"]
    \end{tikzcd},
  \end{equation*}
  exhibits $G(y)$ as the colimit of $F \circ \pi$, where the left-hand square is (homotopy) pullback.
\end{lemma}
\begin{proof}
  We note that we can factor the above square into three squares
  \begin{equation*}
    \begin{tikzcd}
      X_{/y}
      \arrow[r, "\simeq"]
      \arrow[d]
      & X^{/y}
      \arrow[r]
      \arrow[d]
      & Y^{\Delta^{1}} \times_{Y} X
      \arrow[r, "\simeq"]
      \arrow[d]
      & X
      \arrow[d, "f"]
      \\
      \{y\}
      \arrow[r, equals]
      & \{y\}
      \arrow[r]
      & Y
      \arrow[r, equals]
      & Y
    \end{tikzcd},
  \end{equation*}
  where the middle square is a strict pullback. Since the map $Y^{\Delta^{1}} \times_{Y}X \to Y$ is a Kan fibration, the middle square is also a homotopy pullback. We note that the left and right squares are also homotopy pullbacks, since the horizontal morphisms are weak equivalences. Thus, the outer square is a homotopy pullback; one easily checks that the natural equivalence $f \circ \pi \Rightarrow \underline{y}$ defined by pasting the three squares agrees with the natural transformation $\alpha$ defined in \hyperref[notation:rund_um_undercategories]{Notation~\ref*{notation:rund_um_undercategories}}.
\end{proof}


\begin{proof}[Proof of Theorem \ref{prop:local_systems_are_beck_chevalley}]
  We have already shown that $p$ is a bicartesian fibration, and we know that $\S$ admits pullbacks. Therefore, it suffices to show that $p$ has \hyperref[item:beck_chevalley_condition]{Property~(BC)}.

  We consider a square $\sigma\colon \Delta^{1} \times \Delta^{1} \cong \Delta^{\{0,1,2\}} \amalg_{\Delta^{\{0,2\}}}\Delta^{\{0,1',2\}} \to \LS(\category{C})$ with the following properties.
  \begin{enumerate}
    \item The restriction $\sigma|\Delta^{\{0,1\}}$ is $p$-cartesian.

    \item The restriction $\sigma|\Delta^{\{1',2\}}$ is $p$-cartesian.

    \item The restriction $\sigma|\Delta^{\{1,2\}}$ is $p$-cocartesian.

    \item The map $\sigma$ lies over a pullback square
      \begin{equation*}
        p(\sigma) =
        \begin{tikzcd}
          X_{0}
          \arrow[r, "f'"]
          \arrow[d, swap, "g'"]
          & X_{1'}
          \arrow[d, "g"]
          \\
          X_{1}
          \arrow[r, "f"]
          & X_{2}
        \end{tikzcd}
      \end{equation*}
      in $\S$.
  \end{enumerate}
  We need to show that $\sigma|\Delta^{\{0,1'\}}$ is $p$-cocartesian.

  The map $\sigma$ adjunct to a map
  \begin{equation*}
    \tau\colon \Delta^{\{0,1,2,\overline{2}, \overline{1}, \overline{0}\}}_{\dagger}\amalg_{\Delta^{\{0,2,\overline{2}, \overline{0}\}}_{\dagger}} \Delta^{\{0,1',2,\overline{2}, \overline{1}', \overline{0}\}}_{\dagger} \to \ICCat
  \end{equation*}
  such that $\tau|\Delta^{\{\overline{2}, \overline{1}, \overline{0}\}} \amalg_{\Delta^{\{\overline{2},\overline{0}\}}} \Delta^{\{\overline{2}, \overline{1}', \overline{0}\}}$ is the constant functor with value $\category{C}$. We can read off the following further properties of $\tau$, corresponding to the properties of $\sigma$ above.
  \begin{enumerate}
    \item The simplices $\tau|\Delta^{\{0,1,\overline{0}\}}$ and $\tau|\Delta^{\{0,\overline{1},\overline{0}\}}$ are thin.

    \item The simplices $\tau|\Delta^{\{1',2,\overline{1}'\}}$ and $\tau|\Delta^{\{1',\overline{2},\overline{1}'\}}$ are thin.

    \item The simplex $\tau|\Delta^{\{1,2,\overline{1}\}}$ is left Kan.

    \item The restriction $\tau|\Delta^{\{0,1,2\}} \amalg_{\Delta^{\{0,2\}}}\Delta^{\{0,1',2\}}$ is equal to $p(\sigma)$.
  \end{enumerate}
  The condition that $\sigma|\Delta^{\{0, 1'\}}$ is $p$-cocartesian corresponds to the condition that the simplex $\tau|\Delta^{\{0,1,\overline{0}\}}$ is left Kan.

  Applying \hyperref[lemma:transport_left_kan_simplices]{Lemma~\ref*{lemma:transport_left_kan_simplices}} to the simplex $\tau|\Delta^{\{1,2,\overline{1}, \overline{0}\}}$, we see that $\tau|\Delta^{\{1,2,\overline{0}\}}$ is left Kan. Applying \hyperref[lemma:transport_thin_simplices]{Lemma~\ref*{lemma:transport_thin_simplices}} to the simplex $\tau|\Delta^{\{1',2, \overline{1}', \overline{0}\}}$, we see that $\tau|\Delta^{\{1',\overline{2}, \overline{0}\}}$ is thin.

  We study the restriction
  \begin{equation*}
    \tau|\Delta^{\{0,1,2,\overline{0}\}} \amalg_{\Delta\{0,2,\overline{0}\}} \Delta^{\{0,1',2,\overline{0'}\}}.
  \end{equation*}
  This is determined, up to specific choices of compositions, by the diagram
  \begin{equation*}
    \begin{tikzcd}[row sep=large, column sep=large]
      X_{0}
      \arrow[r, "f'"]
      \arrow[d, swap, "g'"]
      & X_{1'}
      \arrow[d, swap, "g"]
      \arrow[r, "F", ""{swap, name=M}]
      & \category{C}
      \\
      X_{1}
      \arrow[r, "f"]
      & X_{2}
      \arrow[ur, swap, "G"]
      \arrow[from=M, Rightarrow, shorten=2ex, swap, "\eta"]
    \end{tikzcd}.
  \end{equation*}
  In particular, $\tau|\Delta^{\{0,1,\overline{0}\}}$ is homotopic to the pasting of the square and the triangle above. Thus, it suffices to show that for all $x \in X_{1}$, the pasting diagram
  \begin{equation*}
    \begin{tikzcd}[row sep=large, column sep=large]
      (X_{0})_{/x}
      \arrow[r, "\pi"]
      \arrow[d]
      & X_{0}
      \arrow[r, "f'"]
      \arrow[d, swap, "g'"]
      & X_{1'}
      \arrow[d, swap, "g"]
      \arrow[r, "F", ""{swap, name=M}]
      & \category{C}
      \\
      \{x\}
      \arrow[r, hook]
      & X_{1}
      \arrow[r, "f"]
      & X_{2}
      \arrow[ur, swap, "G"]
      \arrow[from=M, Rightarrow, shorten=2ex, swap, "\eta"]
    \end{tikzcd}
  \end{equation*}
  exhibits $G(f(x))$ as the colimit of $F \circ f' \circ \pi$. But applying the pasting law for homotopy pullbacks, this follows directly from the assumption that $\eta$ exhibits $G$ as a left Kan extension of $F$ along $f$.
\end{proof}

\subsection{The triple structures}
\label{ssc:the_triple_structures_bare}

\subsubsection{The triple structure on local systems}
\label{sss:the_triple_structure_on_local_systems}

We define a triple structure $\triple{Q}$ on $\LS(\category{C})$ as follows.
\begin{itemize}
  \item $\category{Q} = \LS(\category{C})$.

  \item $\category{Q}\downdag = \LS(\category{C})$.

  \item $\category{Q}\updag$ consists only of cartesian morphisms.
\end{itemize}

This corresponds to spans of the form
\begin{equation*}
  \begin{tikzcd}
    \category{D}'
    \arrow[d, ""{name=L, right}]
    & \category{D}
    \arrow[d, ""{name=ML, left}, ""{name=MR, right}]
    \arrow[l, swap]
    \arrow[r]
    & \category{D}''
    \arrow[d, ""{name=R, left}]
    \\
    \category{C}
    \arrow[r, equals]
    &\category{C}
    & \category{C}
    \arrow[l, equals]
    \arrow[Rightarrow, from=MR, to=R, shorten=2ex]
    \arrow[Rightarrow, from=ML, to=L, phantom, "\circlearrowleft"{description}]
  \end{tikzcd}
\end{equation*}

Such a span will correspond to a cocartesian morphism if it is of the form
\begin{equation*}
  \begin{tikzcd}
    \category{D}'
    \arrow[d, "\mathcal{F}"{left}, ""{name=L, right}]
    & \category{D}
    \arrow[d, "g^{*}\mathcal{F}"{description}, ""{name=ML, left}, ""{name=MR, right}]
    \arrow[l, swap, "g"]
    \arrow[r, "f"]
    & \category{D}''
    \arrow[d, ""{name=R, left}, "f_{!}g^{*}\mathcal{F}"{right}]
    \\
    \category{C}
    \arrow[r, equals]
    &\category{C}
    & \category{C}
    \arrow[l, equals]
    \arrow[Rightarrow, from=MR, to=R, shorten=2ex, "!"{description}]
    \arrow[from=ML, to=L, phantom, "\circlearrowleft"{description}]
  \end{tikzcd}
\end{equation*}

\begin{proposition}
  \label{prop:q_triple_is_adequate}
  The triple $\triple{Q}$ is adequate.
\end{proposition}

To do this, we need to prove:
\begin{lemma}
  \label{lemma:pullbacks_in_ttw}
  Let $\CC$ be an $\infty$-bicategory such that the underlying quasicategory $\category{C}$ has pullbacks and pushouts. Consider a square
  \begin{equation*}
    \sigma =\quad
    \begin{tikzcd}
      A
      \arrow[r]
      \arrow[d, swap, "a"]
      & B
      \arrow[d, "b"]
      \\
      B'
      \arrow[r]
      & C
    \end{tikzcd}
  \end{equation*}
  in $\Tw(\CC)$ corresponding to a twisted cube
  \begin{equation*}
    \begin{tikzcd}
      a
      \arrow[rrr]
      \arrow[dr, ""{name=TL, right}]
      \arrow[ddd, ""{name=R1, right}]
      &&& b
      \arrow[ddd, ""{name=L2, left}, ""{name=R2, right}]
      \arrow[dr, ""{name=TR, left}]
      \arrow[from=R1, to=L2, Rightarrow, shorten=7ex]
      \\
      & b'
      \arrow[rrr, crossing over]
      &&& c
      \arrow[ddd, ""{name=L4, left}"]
      \\
      \\
      \bar{a}
      &&& \bar{b}
      \arrow[lll]
      \\
      & \bar{b}'
      \arrow[from=uuu, crossing over, ""{name=L3, left}, ""{name=R3, right}]
      \arrow[ul]
      &&& \bar{c}
      \arrow[lll]
      \arrow[ul]
      \arrow[from=R1, to=L3, Rightarrow, shorten=3ex, "\alpha"]
      \arrow[from=R3, to=L4, Rightarrow, crossing over, shorten=7ex]
      \arrow[from=R2, to=L4, Rightarrow, shorten=3ex, "\beta"]
    \end{tikzcd}
  \end{equation*}
  in $\CC$.\footnote{Note that the top and bottom faces of any such diagram in $\Tw(\CC)$ belong to $\category{C}$, and hence must weakly commute by definition.} Suppose the right face is thin, the top face is a pullback, and the bottom face is a pushout. Then the square $\sigma$ is pullback if and only if the left face is thin.
\end{lemma}
\begin{proof}
  This corresponds to a square $\sigma$ in $\Tw(\CC)$ lying over a pullback square in $\category{C} \times \category{C}\op$, such that $b$ is $p$-cartesian. It is then classically known that $\sigma$ is pullback if and only if $a$ is $p$-cartesian.
\end{proof}

\begin{proof}[Proof of Proposition~\ref*{prop:q_triple_is_adequate}]
  We need to show that we can form pullbacks in $\Tw(\CC)$ of cospans one of whose legs is a cartesian morphism. Mapping the cospan to $\category{C} \times \category{C}\op$ using $r$, We can form the pullback, then fill to a relative pullback by finding a cartesian lift $a$, then filling an inner and an outer horn. This diagram is pullback by \hyperref[lemma:pullbacks_in_ttw]{Lemma~\ref*{lemma:pullbacks_in_ttw}}

  We also need to show that any such pullback is of this form. This also follows from \hyperref[lemma:pullbacks_in_ttw]{Lemma~\ref*{lemma:pullbacks_in_ttw}}.
\end{proof}

\subsection{Building categories of spans}
\label{ssc:building_categories_of_spans_bare}

\begin{lemma}
  \label{lemma:homotopy_pullbacks_and_overcategories}
  Let
  \begin{equation*}
    \begin{tikzcd}
      X \times_{Y}^{h} Y'
      \arrow[r]
      \arrow[d]
      & Y'
      \arrow[d]
      \\
      X
      \arrow[r]
      & Y
    \end{tikzcd}
  \end{equation*}
  be a homotopy pullback diagram of Kan complexes, and let $y' \in Y'$. Then
  \begin{equation}
    \label{eq:overcategory_commutes_with_homotopy_pullback}
    (X \times_{Y}^{h}Y')_{/y'} \simeq X \times_{Y}^{h} (Y'_{/y'})
  \end{equation}
\end{lemma}
\begin{proof}
  We can model the homotopy pullback $X \times^{h}_{Y}Y'$ as the strict pullback
  \begin{equation*}
    Y^{\Delta^{1}} \times_{Y \times Y} (X \times Y').
  \end{equation*}
  The left-hand side of \hyperref[eq:overcategory_commutes_with_homotopy_pullback]{Equation~\ref*{eq:overcategory_commutes_with_homotopy_pullback}} is then given by the pullback
  \begin{equation*}
    \left(Y^{\Delta^{1}} \times_{Y \times Y}(X \times Y')\right) \times_{Y'} Y'_{/y'}.
  \end{equation*}
  But this is isomorphic to
  \begin{equation*}
    Y^{\Delta^{1}} \times_{Y \times Y} (X \times Y^{'}_{/y'}),
  \end{equation*}
  which is a model for the right-hand side.
\end{proof}

\begin{proposition}
  The map $p\colon \triple{P} \to \triple{Q}$ of triples whose underlying map is $p\colon \LS(\category{C}) \to \S$ is adequate, i.e.\ satisfies the conditions of our modified version of Barwick's theorem.
\end{proposition}
\begin{proof}
  We need to show that for any square
  \begin{equation*}
    \sigma =\quad
    \begin{tikzcd}
      x'
      \arrow[r, "f'"]
      \arrow[d, swap, "g'"]
      & y'
      \arrow[d, "g"]
      \\
      x
      \arrow[r, "f"]
      & y
    \end{tikzcd}
  \end{equation*}
  in $\LS(\category{C})$ lying over a pullback square
  \begin{equation*}
    \begin{tikzcd}
      X'
      \arrow[r, "f'"]
      \arrow[d, swap, "g'"]
      & Y'
      \arrow[d, "g"]
      \\
      X
      \arrow[r, "f"]
      & Y
    \end{tikzcd}
  \end{equation*}
  in $\S$ (corresponding to a \emph{homotopy} pullback of Kan complexes), such that $g$ and $g'$ are $p$-cartesian and $f$ is $p$-cocartesian, $f'$ is $p$-cocartesian. The square $\sigma$ corresponds to a twisted cube
  \begin{equation*}
    \begin{tikzcd}
      X'
      \arrow[rrr, "f'"]
      \arrow[dr, ""{name=TL, below}, "g'"{swap}]
      \arrow[ddd, ""{name=R1, below}]
      &&& Y'
      \arrow[ddd, ""{name=L2, below}, ""{name=R2, above}]
      \arrow[dr, ""{name=TR, below}, "g"]
      \arrow[from=R1, to=L2, Rightarrow, shorten=6ex]
      \\
      & X
      \arrow[rrr, crossing over, "f"]
      &&& Y
      \arrow[ddd, ""{name=L4, below}"]
      \\
      \\
      \category{C}
      &&& \category{C}
      \arrow[lll, equals]
      \\
      & \category{C}
      \arrow[from=uuu, crossing over, ""{name=L3, below}, ""{name=R3, above}]
      \arrow[ul, equals]
      &&& \category{C}
      \arrow[lll, equals]
      \arrow[ul, equals]
      \arrow[from=R1, to=L3, phantom, "\circlearrowleft"{description}]
      \arrow[from=R3, to=L4, Rightarrow, crossing over, shorten=6ex]
      \arrow[from=R2, to=L4, phantom, "\circlearrowleft"{description}]
    \end{tikzcd}
  \end{equation*}
  in which the left and right faces are thin, the top face is a pullback, the bottom face is a pushout, and the front face corresponds to a left Kan extension. We need to show that the back face corresponds to a left Kan extension. For ease of notation, we flatten this out and add some more labels.
  \begin{equation*}
    \begin{tikzcd}
      X'
      \arrow[rr, "f'"]
      \arrow[dd, swap, "g'"]
      \arrow[dr, "g'^{*}\mathcal{F}"]
      && Y'
      \arrow[dd, "g"]
      \arrow[dl, "\mathcal{G}"]
      \\
      & \category{C}
      \\
      X
      \arrow[ur, "\mathcal{F}"]
      \arrow[rr, swap, "f"]
      && Y
      \arrow[ul, "f_{!}\mathcal{F}"]
    \end{tikzcd}
  \end{equation*}
  We need to show that $\mathcal{G}$ is a left Kan extension of $g'^{*}\mathcal{F}$ along $f'$, i.e.\ that for all $y' \in Y'$,
  \begin{align*}
    \mathcal{G}(y') &\simeq \colim\left[ X'_{/y'} \to X \to \category{C} \right] \\
    &\simeq \colim\left[ X \times_{Y} (Y'_{/y'}) \to X \to \category{C} \right],
  \end{align*}
  where here we mean by $X' \simeq X \times_{Y} Y'$ the homotopy pullback, and we have used \hyperref[lemma:homotopy_pullbacks_and_overcategories]{Lemma~\ref*{lemma:homotopy_pullbacks_and_overcategories}}. We know that $\mathcal{G}$ is the pullback along $g$ of $f_{!}\mathcal{F}$, i.e.\ that
  \begin{align*}
    \mathcal{G}(y') &\simeq \colim\left[ X_{/g(y')} \to X \to \category{C} \right] \\
    &\simeq \colim \left[ X \times_{Y} (Y_{/g(y')}) \to X \to \category{C} \right].
  \end{align*}
  The map $X \times_{Y} (Y'_{/y'}) \to X$ factors through $X \times_{Y} (Y_{/g(y')})$ thanks to the map $s\colon Y'_{/y'} \to Y_{/g(y')}$. The map $s$ is a weak equivalence between contractible Kan complexes (because both $Y'_{/y'}$ and $Y_{/g(y')}$ are Kan complexes with terminal objects). Thus, the map $X \times_{Y}(Y'_{/y'}) \to X \times_{Y}(Y_{/g(y')})$ is also a weak homotopy equivalence between Kan complexes, thus cofinal.
\end{proof}


\end{document}
