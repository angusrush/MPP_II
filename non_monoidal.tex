\documentclass[main.tex]{subfiles}

\begin{document}

\section{The non-monoidal construction}
\label{sec:the_non_monoidal_construction}

\subsection{Con extensions}
\label{ssc:con_extensions}

We provide a conjectural answer to the following question. Let $f\colon \category{D} \to \category{D}'$ be a functor of quasicategories, and let $\mathcal{F}\colon \category{D} \to \category{C}$ be similar. What universal property does the $\infty$-Kan extension $f_{!}\mathcal{F}\colon \category{D}' \to \category{C}$ satisfy?

A Kan extension, conjecturally, should be modelled by a 2-simplex in $\ICCat$
\begin{equation*}
  \label{eq:left_kan_ext}
  \begin{tikzcd}
    \category{D}
    \arrow[rr, "\mathcal{F}", ""{below, name=M}]
    \arrow[dr, swap, "f"]
    && \category{C}
    \\
    & \category{D}'
    \arrow[ur, swap, "f_{!}\mathcal{F}"]
    \arrow[to=M, Rightarrow, "\eta"]
  \end{tikzcd},
\end{equation*}
where $\eta$ is the unit map.

\begin{definition}
  \label{def:con_extension}
  A 2-simplex $\sigma\colon \Delta^{2} \to \ICCat$ is a \defn{left con extension} if for all
  \begin{equation*}
    \tau\colon \Lambda^{3}_{0} \to \ICCat
  \end{equation*}
  such that $d_{2}\tau = \sigma$, and such that $d_{3}\tau$ is thin, the space
  \begin{equation*}
    \Map(\Delta^{3}, \ICCat) \times_{\Map(\Lambda^{3}_{0}, \ICCat)} \{\tau\}
  \end{equation*}
  of fillings to a full 3-simplex is contractible.
\end{definition}

We conjecture that left con extensions are not really a con at all, but rather bona fide left Kan extensions; that is, a 2-simplex in $\ICat$ is a left con extension if and only if it is of the form pictured in \hyperref[eq:left_kan_ext]{Equation~\ref*{eq:left_kan_ext}}.


\subsection{The bare functor}
\label{ssc:the_bare_functor}

Let $\category{C}$ be a category admitting small colimits. We consider the pullback
\begin{equation*}
  \begin{tikzcd}
    \LS(\category{C})
    \arrow[r]
    \arrow[d, swap, "r"]
    & \Tw(\ICCat)
    \arrow[d]
    \\
    \S \times \{\category{C}\}
    \arrow[r, hook]
    & \ICat \times \ICat\op
  \end{tikzcd}.
\end{equation*}

A general morphism in $\LS(\category{C})$ is a 3-simplex of the form
\begin{equation*}
  \begin{tikzcd}[column sep=large, row sep=large]
    X
    \arrow[r]
    \arrow[d, ""{name=L, right}]
    \arrow[dr]
    & Y
    \arrow[d, ""{name=R, left}]
    \arrow[from=l, to=R, phantom, near end, "\circlearrowleft"{description}]
    \\
    \category{C}
    & \category{C}
    \arrow[l, equals]
    \arrow[from=L, Rightarrow, shorten=3ex]
  \end{tikzcd}
  \qquad
  \begin{tikzcd}[column sep=large, row sep=large]
    X
    \arrow[r]
    \arrow[d, ""{name=L, right}]
    & Y
    \arrow[d, ""{name=R, left}]
    \arrow[dl]
    \\
    \category{C}
    \arrow[from=R, phantom, near start, "\circlearrowleft"{description}]
    & \category{C}
    \arrow[l, equals]
    \arrow[from=L, to=u, shorten=3ex, Rightarrow]
  \end{tikzcd},
\end{equation*}
We will generally denote such a morphism as follows.
\begin{equation*}
  \begin{tikzcd}
    X
    \arrow[r]
    \arrow[d, ""{name=L, right}]
    & Y
    \arrow[d, ""{name=R, left}]
    \\
    \category{C}
    & \category{C}
    \arrow[l, equals]
    \arrow[Rightarrow, from=L, to=R, shorten=1.5ex]
  \end{tikzcd}.
\end{equation*}

There is a potential to be confused here. The natural transformation notated above does not correspond to a single natural transformation, but rather to two natural transformations. However, both of these natural transformations agree up to a homotopy provided by the interior of the 3-simplex.

Such a morphism is $r$-cartesian if and only if all 2-simplices are thin; that is, if the 3-simplex making up the twisted square weakly commutes. We will use the symbol $\circlearrowleft$ to denote a weakly commuting twisted morphism, and call such morphisms \emph{thin:}
\begin{equation*}
  \begin{tikzcd}
    X
    \arrow[r]
    \arrow[d, ""{name=L, right}]
    & Y
    \arrow[d, ""{name=R, left}]
    \\
    \category{C}
    & \category{C}
    \arrow[l, equals]
    \arrow[Rightarrow, from=L, to=R, shorten=1.5ex, phantom, "\circlearrowleft"]
  \end{tikzcd}.
\end{equation*}

The functor $r\colon \LS(\category{C}) \to \ICat$ thus classifies the functor
\begin{equation*}
  \S\op \to \ICat;\qquad X \mapsto \Fun(X, \category{C}),
\end{equation*}
which sends a map $f\colon X \to Y$ to the pullback
\begin{equation*}
  f^{*}\colon \Fun(Y, \category{C}) \to \Fun(X, \category{C}).
\end{equation*}
Under the assumption that $\category{C}$ admits colimits, each of these functors admit a left adjoint $f_{!}$ given by left Kan extension. More explicitly, we have the following.
\begin{proposition}
  A morphism $\Delta^{1} \to \LS(\category{C})$ is $r$-cocartesian if and only the 3-simplex it corresponds to in $\ICCat$ has the form
  \begin{equation*}
    \begin{tikzcd}[column sep=large, row sep=large]
      X
      \arrow[r]
      \arrow[d, ""{name=L, right}]
      \arrow[dr]
      & Y
      \arrow[d, ""{name=R, left}]
      \arrow[from=l, to=R, phantom, near end, "\circlearrowleft"{description}]
      \\
      \category{C}
      & \category{C}
      \arrow[l, equals]
      \arrow[from=L, Rightarrow, shorten=3ex]
    \end{tikzcd}
    \qquad
    \begin{tikzcd}[column sep=large, row sep=large]
      X
      \arrow[r, "f"]
      \arrow[d, "\mathcal{F}"{swap}, ""{name=L, right}]
      & Y
      \arrow[d, ""{name=R, left}, "f_{!}\mathcal{F}'"]
      \arrow[dl, near end, "f_{!}\mathcal{F}"]
      \\
      \category{C}
      \arrow[from=R, phantom, near start, "\circlearrowleft"{description}]
      & \category{C}
      \arrow[l, equals]
      \arrow[from=L, to=u, shorten=3ex, Rightarrow, "\eta"]
    \end{tikzcd},
  \end{equation*}
  where $f_{!}\mathcal{F}$ and $f_{!}\mathcal{F}'$ are both left Kan extensions of $\mathcal{F}$ along $f$, and $\eta$ is the counit map. Note that the data of the second diagram determines the data of the first, and the filling 3-simplex, up to contractible choice.
\end{proposition}

For short, we will denote cocartesian morphisms
\begin{equation*}
  \begin{tikzcd}
    X
    \arrow[r, "f"]
    \arrow[d, ""{name=L, right}, "\mathcal{F}"{left}]
    & X'
    \arrow[d, ""{name=R, left}, "\Lan_{f}\mathcal{F}"{right}]
    \\
    \category{C}
    & \category{C}
    \arrow[l, equals]
    \arrow[Rightarrow, from=L, to=R, shorten=1.5ex, "!"{description}]
  \end{tikzcd}.
\end{equation*}

We define a triple structure $\triple{Q}$ on $\LS(\category{C})$ as follows.
\begin{itemize}
  \item $\category{Q} = \LS(\category{C})$.

  \item $\category{Q}\downdag = \LS(\category{C})$.

  \item $\category{Q}\updag$ consists only of cartesian morphisms.
\end{itemize}

This corresponds to spans of the form
\begin{equation*}
  \begin{tikzcd}
    \category{D}'
    \arrow[d, ""{name=L, right}]
    & \category{D}
    \arrow[d, ""{name=ML, left}, ""{name=MR, right}]
    \arrow[l, swap]
    \arrow[r]
    & \category{D}''
    \arrow[d, ""{name=R, left}]
    \\
    \category{C}
    \arrow[r, equals]
    &\category{C}
    & \category{C}
    \arrow[l, equals]
    \arrow[Rightarrow, from=MR, to=R, shorten=2ex]
    \arrow[Rightarrow, from=ML, to=L, phantom, "\circlearrowleft"{description}]
  \end{tikzcd}
\end{equation*}

Such a span will correspond to a cocartesian morphism if it is of the form
\begin{equation*}
  \begin{tikzcd}
    \category{D}'
    \arrow[d, "\mathcal{F}"{left}, ""{name=L, right}]
    & \category{D}
    \arrow[d, "g^{*}\mathcal{F}"{description}, ""{name=ML, left}, ""{name=MR, right}]
    \arrow[l, swap, "g"]
    \arrow[r, "f"]
    & \category{D}''
    \arrow[d, ""{name=R, left}, "f_{!}g^{*}\mathcal{F}"{right}]
    \\
    \category{C}
    \arrow[r, equals]
    &\category{C}
    & \category{C}
    \arrow[l, equals]
    \arrow[Rightarrow, from=MR, to=R, shorten=2ex, "!"{description}]
    \arrow[from=ML, to=L, phantom, "\circlearrowleft"{description}]
  \end{tikzcd}
\end{equation*}

\begin{proposition}
  \label{prop:q_triple_is_adequate}
  The triple $\triple{Q}$ is adequate.
\end{proposition}

To do this, we need to prove:
\begin{lemma}
  Let $\CC$ be an $\infty$-bicategory such that the underlying quasicategory $\category{C}$ has pullbacks and pushouts. Consider a square
  \begin{equation*}
    \sigma =\quad
    \begin{tikzcd}
      A
      \arrow[r]
      \arrow[d, swap, "a"]
      & B
      \arrow[d, "b"]
      \\
      B'
      \arrow[r]
      & C
    \end{tikzcd}
  \end{equation*}
  in $\Tw(\CC)$ corresponding to a twisted cube
  \begin{equation*}
    \begin{tikzcd}
      a
      \arrow[rrr]
      \arrow[dr, ""{name=TL, right}]
      \arrow[ddd, ""{name=R1, right}]
      &&& b
      \arrow[ddd, ""{name=L2, left}, ""{name=R2, right}]
      \arrow[dr, ""{name=TR, left}]
      \arrow[from=R1, to=L2, Rightarrow, shorten=7ex]
      \\
      & b'
      \arrow[rrr, crossing over]
      &&& c
      \arrow[ddd, ""{name=L4, left}"]
      \\
      \\
      \bar{a}
      &&& \bar{b}
      \arrow[lll]
      \\
      & \bar{b}'
      \arrow[from=uuu, crossing over, ""{name=L3, left}, ""{name=R3, right}]
      \arrow[ul]
      &&& \bar{c}
      \arrow[lll]
      \arrow[ul]
      \arrow[from=R1, to=L3, Rightarrow, shorten=3ex, "\alpha"]
      \arrow[from=R3, to=L4, Rightarrow, crossing over, shorten=7ex]
      \arrow[from=R2, to=L4, Rightarrow, shorten=3ex, "\beta"]
    \end{tikzcd}
  \end{equation*}
  in $\CC$.\footnote{Note that the top and bottom faces of any such diagram in $\Tw(\CC)$ belong to $\category{C}$, and hence must weakly commute by definition.} Suppose $b$ is thin, the top face is a pullback, and the bottom face is a pushout. Then the square $\sigma$ is pullback if and only if $a$ is thin.
\end{lemma}
\begin{proof}
  Sorry for the writing, I'm really tired. This corresponds to a square $\sigma$ in $\Tw(\CC)$ lying over a pullback square in $\category{C} \times \category{C}\op$, such that $b$ is $p$-cartesian. It is then classically known that $\sigma$ is pullback if and only if $a$ is $p$-cartesian.
\end{proof}

\begin{proof}[Proof of Proposition~\ref*{prop:q_triple_is_adequate}]
  This is literally the lemma right above us.
\end{proof}

Well, it's nice to see that this stuff works as intended. We know that the triple $\triple{P}$ of unconstrained spans of spaces works as intended, so let's just move on.

\begin{lemma}
  \label{lemma:homotopy_pullbacks_and_overcategories}
  Let
  \begin{equation*}
    \begin{tikzcd}
      X \times_{Y}^{h} Y'
      \arrow[r]
      \arrow[d]
      & Y'
      \arrow[d]
      \\
      X
      \arrow[r]
      & Y
    \end{tikzcd}
  \end{equation*}
  be a homotopy pullback diagram of Kan complexes, and let $y' \in Y'$. Then
  \begin{equation}
    \label{eq:overcategory_commutes_with_homotopy_pullback}
    (X \times_{Y}^{h}Y')_{/y'} \simeq X \times_{Y}^{h} (Y'_{/y'})
  \end{equation}
\end{lemma}
\begin{proof}
  We can model the homotopy pullback $X \times^{h}_{Y}Y'$ as the strict pullback
  \begin{equation*}
    Y^{\Delta^{1}} \times_{Y \times Y} (X \times Y').
  \end{equation*}
  The left-hand side of \hyperref[eq:overcategory_commutes_with_homotopy_pullback]{Equation~\ref*{eq:overcategory_commutes_with_homotopy_pullback}} is then given by the pullback
  \begin{equation*}
    \left(Y^{\Delta^{1}} \times_{Y \times Y}(X \times Y')\right) \times_{Y'} Y'_{/y'}.
  \end{equation*}
  But this is isomorphic to
  \begin{equation*}
    Y^{\Delta^{1}} \times_{Y \times Y} (X \times Y^{'}_{/y'}),
  \end{equation*}
  which is a model for the right-hand side.
\end{proof}

\begin{proposition}
  The map $p\colon \triple{P} \to \triple{Q}$ of triples whose underlying map is $p\colon \LS(\category{C}) \to \S$ is adequate, i.e.\ satisfies the conditions of our modified version of Barwick's theorem.
\end{proposition}
\begin{proof}
  We need to show that for any square
  \begin{equation*}
    \sigma =\quad
    \begin{tikzcd}
      x'
      \arrow[r, "f'"]
      \arrow[d, swap, "g'"]
      & y'
      \arrow[d, "g"]
      \\
      x
      \arrow[r, "f"]
      & y
    \end{tikzcd}
  \end{equation*}
  in $\LS(\category{C})$ lying over a pullback square
  \begin{equation*}
    \begin{tikzcd}
      X'
      \arrow[r, "f'"]
      \arrow[d, swap, "g'"]
      & Y'
      \arrow[d, "g"]
      \\
      X
      \arrow[r, "f"]
      & Y
    \end{tikzcd}
  \end{equation*}
  in $\S$ (corresponding to a \emph{homotopy} pullback of Kan complexes), such that $g$ and $g'$ are $p$-cartesian and $f$ is $p$-cocartesian, $f'$ is $p$-cocartesian. The square $\sigma$ corresponds to a twisted cube
  \begin{equation*}
    \begin{tikzcd}
      X'
      \arrow[rrr, "f'"]
      \arrow[dr, ""{name=TL, below}, "g'"{swap}]
      \arrow[ddd, ""{name=R1, below}]
      &&& Y'
      \arrow[ddd, ""{name=L2, below}, ""{name=R2, above}]
      \arrow[dr, ""{name=TR, below}, "g"]
      \arrow[from=R1, to=L2, Rightarrow, shorten=6ex]
      \\
      & X
      \arrow[rrr, crossing over, "f"]
      &&& Y
      \arrow[ddd, ""{name=L4, below}"]
      \\
      \\
      \category{C}
      &&& \category{C}
      \arrow[lll, equals]
      \\
      & \category{C}
      \arrow[from=uuu, crossing over, ""{name=L3, below}, ""{name=R3, above}]
      \arrow[ul, equals]
      &&& \category{C}
      \arrow[lll, equals]
      \arrow[ul, equals]
      \arrow[from=R1, to=L3, phantom, "\circlearrowleft"{description}]
      \arrow[from=R3, to=L4, Rightarrow, crossing over, shorten=6ex]
      \arrow[from=R2, to=L4, phantom, "\circlearrowleft"{description}]
    \end{tikzcd}
  \end{equation*}
  in which the left and right faces are thin, the top face is a pullback, the bottom face is a pushout, and the front face corresponds to a left Kan extension. For ease of notation, we flatten this out and add some more labels.
  \begin{equation*}
    \begin{tikzcd}
      X'
      \arrow[rr, "f'"]
      \arrow[dd, swap, "g'"]
      \arrow[dr, "g'^{*}\mathcal{F}"]
      && Y'
      \arrow[dd, "g"]
      \arrow[dl, "\mathcal{G}"]
      \\
      & \category{C}
      \\
      X
      \arrow[ur, "\mathcal{F}"]
      \arrow[rr, swap, "f"]
      && Y
      \arrow[ul, "f_{!}\mathcal{F}"]
    \end{tikzcd}
  \end{equation*}
  We need to show that $\mathcal{G}$ is a left Kan extension of $g'^{*}\mathcal{F}$ along $f'$, i.e.\ that for all $y' \in Y'$,
  \begin{align*}
    \mathcal{G}(y') &\simeq \colim\left[ X'_{/y'} \to X \to \category{C} \right] \\
    &\simeq \colim\left[ X \times_{Y} (Y'_{/y'}) \to X \to \category{C} \right],
  \end{align*}
  where here we mean by $X' \simeq X \times_{Y} Y'$ the homotopy pullback, and we have used \hyperref[lemma:homotopy_pullbacks_and_overcategories]{Lemma~\ref*{lemma:homotopy_pullbacks_and_overcategories}}. We know that $\mathcal{G}$ is the pullback along $g$ of $f_{!}\mathcal{F}$, i.e.\ that
  \begin{align*}
    \mathcal{G}(y') &\simeq \colim\left[ X_{/g(y')} \to X \to \category{C} \right] \\
    &\simeq \colim \left[ X \times_{Y} (Y_{/g(y')}) \to X \to \category{C} \right].
  \end{align*}
  The map $X \times_{Y} (Y'_{/y'}) \to X$ factors through $X \times_{Y} (Y_{/g(y')})$ thanks to the map $s\colon Y'_{/y'} \to Y_{/g(y')}$. This is a weak equivalence between contractible Kan complexes (because both $Y'_{/y'}$ and $Y_{/g(y')}$ are Kan complexes with terminal objects, and $s$ sends the terminal object of one to the other.) Thus, the map $X \times_{Y}(Y'_{/y'}) \to X \times_{Y}(Y_{/g(y')})$ is also a weak homotopy equivalence between Kan complexes, thus cofinal.
\end{proof}


\end{document}
