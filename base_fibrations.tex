\documentclass[main.tex]{subfiles}

\begin{document}

\section{The base fibration}
\label{sec:base_fibrations}

The starting point of our construction is the functor
\begin{equation*}
  \begin{tikzcd}
    \Finp
    \arrow[r, "\ICat^{\times}"]
    & \Cat_{(\infty, 2)}
    \arrow[r, "\Tw"]
    & \ICat
  \end{tikzcd}
\end{equation*}
Note that this gives a monoidal category since $\Tw(\ICat^{n})\cong \Tw(\ICat)^{n}$.

We're not gonna worry too much about where the functor $\Tw$ comes from, just keep notes about what properties we're using. The relative nerve of this functor is a Cartesian fibration $\Tw(\ICat)_{\otimes} \to \Finp\op$ with the following description: an $n$-simplex $\sigma$ corresponding to a diagram
\begin{equation*}
  \begin{tikzcd}
    \Delta^{n}
    \arrow[rr, "\sigma"]
    \arrow[dr, swap, "\phi"]
    && \Tw(\ICat)_{\otimes}
    \arrow[dl, "p'"]
    \\
    & \Finp\op
  \end{tikzcd}
\end{equation*}
corresponds to the data of, for each subset $I \subseteq [n]$ having minimal element $i$, a map
\begin{equation*}
  \tau(I)\colon \Delta^{I} \to \Tw(\ICat)^{i}
\end{equation*}
such that For nonempty subsets $I' \subseteq I \subseteq [n]$, the diagram
\begin{equation*}
  \begin{tikzcd}
    \Delta^{I'}
    \arrow[r]
    \arrow[d, hook]
    & \Tw(\ICat)^{i'}
    \arrow[d]
    \\
    \Delta^{I}
    \arrow[r]
    & \Tw(\ICat)^{i}
  \end{tikzcd}
\end{equation*}

\begin{example}
  An object of $\Tw(\ICat)_{\otimes}$ lying over $\langle n \rangle$ corresponds to a collection of functors $\category{C}_{i} \to \category{D}_{i}$, $i \in \langle n \rangle^{\circ}$. We will denote this $\vec{\category{C}} \to \vec{\category{D}}$.

  A morphism in $\Tw(\ICat)_{\otimes}$ lying over the active map $\langle 1 \rangle \leftarrow \langle 2 \rangle$ in $\Finp\op$ consists\footnote{Here we mean the morphism in $\Finp\op$ corresponding to the active map $\langle 2 \rangle \to \langle 1 \rangle$ in $\Finp$.} of
  \begin{itemize}
    \item A `source' object $F\colon \category{C} \to \category{C}'$

    \item A pair of `target' objects $G_{i}\colon \category{D}_{i} \to \category{D}'_{i}$, $i = 1$, $2$.

    \item A morphism
      \begin{equation*}
        \begin{tikzcd}
          \category{C}
          \arrow[r, "\alpha"]
          \arrow[d, ""{name=L, right}, "F"{left}]
          & \category{D}_{1} \times D_{2}
          \arrow[d, ""{name=R, left}, "G_{1} \times G_{2}"{right}]
          \\
          \category{C}'
          & \category{D}'_{1} \times \category{D}'_{2}
          \arrow[l, "\beta"]
          \arrow[Rightarrow, from=L, to=R, shorten=2ex, "\eta"{description}]
        \end{tikzcd}
      \end{equation*}
      where here we mean $\eta\colon F \Rightarrow \beta \circ (G_{1} \times G_{2}) \circ \alpha$.
  \end{itemize}

  A morphism of the above form is cartesian if and only if $\eta$ is an equivalence, in which case we call the morphism \emph{thin.} A general morphism is cartesian if and only if each component of $\eta$ is an equivalence.
\end{example}

Note that $\Tw(\ICat)_{\otimes}$ admits a forgetful functor to $\TICatCart \times (\TICatCocart)\op$, where
\begin{itemize}
  \item $\TICatCart$ is the relative nerve (as a cartesian fibration) of the functor $\Finp \to \ICat$ giving the cartesian monoidal structure on $\ICat$

  \item $\TICatCocart$ is the \emph{cocartesian} relative nerve of the same.
\end{itemize}

Taking all this together gives a commuting triangle
\begin{equation*}
  \begin{tikzcd}
    \Tw(\ICat)_{\otimes}
    \arrow[rr]
    \arrow[dr]
    && \TICatCart \times (\TICatCocart)\op
    \arrow[dl]
    \\
    & \Finp\op
  \end{tikzcd}.
\end{equation*}

\begin{lemma}
  Let $\category{C}$ be a small 1-category (and, by abuse of notation, its nerve), let $F$, $G\colon \category{C} \to \SSet$ be functors, and let $\alpha\colon F \Rightarrow G$. Suppose that $\alpha$ satisfes the following conditions.
  \begin{enumerate}
    \item For each object $c \in \category{C}$, $\alpha_{c}\colon F(c) \to G(c)$ is a cartesian fibration.

    \item For each morphism $f\colon c \to d$ in $\category{C}$, the map $Ff\colon F(c) \to F(d)$ takes $\alpha_{c}$-cartesian morphisms to $\alpha_{d}$-cartesian morphisms.
  \end{enumerate}
  Then taking the relative nerve gives a diagram
  \begin{equation*}
    \begin{tikzcd}
      N_{F}(\category{C})
      \arrow[rr, "\rho"]
      \arrow[dr, swap, "\Phi"]
      &&
      N_{G}(\category{C})
      \arrow[dl, "\Gamma"]
      \\
      & \category{C}\op
    \end{tikzcd}
  \end{equation*}
  with the following properties.
  \begin{enumerate}
    \item The maps $\Phi$, $\Gamma$, and $\rho$ are all cartesian fibrations.

    \item The $\rho$-cartesian morphisms in $N_{F}(\category{C})$ admit the following description: a morphism in $N(F)(\category{C})$ lying over a morphim $f\colon d \leftarrow c$ in $\category{C}\op$ consists of a triple $(x, y, \phi)$, where $x \in F(d)$, $y \in F(c)$, and $\phi\colon x \to Ff(y)$. Such a morphism is $\rho$-cartesian if the morphism $\phi$ is $\alpha_{d}$-cartesian.

    \item The map $\rho$ sends $\Phi$-cartesian morphisms in $N_{F}(\category{C})$ to $\Gamma$-cartesian morphisms in $N_{G}(\category{C})$.
  \end{enumerate}
\end{lemma}
\begin{proof}
  We first prove 2. We already know (HTT 3.2.5.11) that $\rho$ is an inner fibration, so in order to prove 2., we need to show that we can solve lifting problems
  \begin{equation*}
    \begin{tikzcd}
      \Lambda^{n}_{n}
      \arrow[r]
      \arrow[d]
      & N_{F}(\category{C})
      \arrow[d]
      \\
      \Delta^{n}
      \arrow[r]
      \arrow[ur, dashed]
      \arrow[dr, swap, "\gamma"]
      & N_{G}(\category{C})
      \arrow[d]
      \\
      & \category{C}\op
    \end{tikzcd},
  \end{equation*}
  where $\Delta^{\{n-1, n\}} \subset \Lambda^{n}_{n}$ is mapped to a cartesian morphism as described above, i.e.\ a triple $(x, y, \phi)$, where $x \in F(\gamma(n-1))$, $y \in F(\gamma(n))$, and $\phi\colon x \to F(\gamma_{n})(y)$ is $\alpha_{\gamma(n-1)}$-cartesian. This is equivalent to solving the lifting problem
  \begin{equation*}
    \begin{tikzcd}
      \Lambda^{n}_{n}
      \arrow[r]
      \arrow[d]
      & F(\gamma(0))
      \arrow[d, "\alpha_{\gamma(0)}"]
      \\
      \Delta^{n}
      \arrow[r]
      \arrow[ur, dashed]
      & G(\gamma(0))
    \end{tikzcd},
  \end{equation*}
  where $\Lambda^{n}_{n}$ is $F(\gamma_{n-1, 0})(\phi)$. But by assumption $F(\gamma_{n-1, 0})(\phi)$ is $\alpha_{\gamma(0)}$-cartesian, so this lifting problem has a solution.

  The assumption that each $\alpha_{c}$ is a cartesian fibration guarantees that we have enough cartesian lifts, so $\rho$ is indeed a cartesian fibration.

  The $\rho$-cartesian morphisms in $N_{F}(\category{C})$ are pairs $(x, y, \phi)$ as above, where $\phi$ is an equivalence. Claim 3. then follows because the functors $\alpha_{c}$ functors map equivalences to equivalences.
\end{proof}

\begin{corollary}
  Everything in sight is a cartesian fibration.
\end{corollary}

Because both $\TICatCart \to \Finp\op$ and $\ICatCart \to \Finp\op$ classify the same functor $\Finp \to \ICat$, they are related by a symmetric monoidal equivalence. Composing a monoidal $\infty$-category $\Finp \to \ICat$ gives:

\begin{proposition}
  We can express a monoidal $\infty$-category as a functor $\Finp \to \TICatCocart$.
\end{proposition}

Fix some monoidal $\infty$-category $\category{C}$ which admits colimits, and such that the monoidal structure preserves colimits in each slot. We can thus define a functor
\begin{equation*}
  \TSCart \to \TICatCart \times (\TICatCocart)\op
\end{equation*}
which is induced by the inclusion on the first component, and given by the composition
\begin{equation*}
  \TSCart \to \Finp\op \overset{\category{C}}{\to} (\TICatCocart)\op.
\end{equation*}

Forming the pullback
\begin{equation*}
  \begin{tikzcd}
    \LS(\category{C})_{\otimes}
    \arrow[r]
    \arrow[d, swap, "r"]
    & \Tw(\ICat)_{\otimes}
    \arrow[d]
    \\
    \TSCart
    \arrow[r]
    & \TICatCart \times (\TICatCocart)\op
  \end{tikzcd}
\end{equation*}
gives us a commutative triangle
\begin{equation*}
  \begin{tikzcd}
    \LS(\category{C})_{\otimes}
    \arrow[rr, "r"]
    \arrow[dr, swap, "q"]
    && \TSCart
    \arrow[dl, "p"]
    \\
    & \Finp\op
  \end{tikzcd}.
\end{equation*}

By definition, one sees that the map $p$ is a cartesian fibration, and the $p$-cartesian morphisms in $\TSCart$ are precisely those representing products.

\begin{proposition}
  The maps $q$ and $r$ are cartesian fibrations, and a generic morphism
  \begin{equation*}
    \begin{tikzcd}
      \vec{\category{D}}
      \arrow[r, "\alpha"]
      \arrow[d, ""{name=L, right}, "F"{left}]
      & \vec{\category{D}}'
      \arrow[d, ""{name=R, left}, "G"{right}]
      \\
      \vec{\category{C}}
      & \vec{\category{C}}
      \arrow[l, "\otimes"]
      \arrow[Rightarrow, from=L, to=R, shorten=1.5ex, "\eta"{description}]
    \end{tikzcd}
  \end{equation*}
  in $\LS(\category{C})$ is
  \begin{itemize}
    \item $r$-cartesian if each component of $\eta$ is an equivalence

    \item $q$-cartesian if $\alpha$ exhibits a product and $\eta$ is an equivalence.
  \end{itemize}
\end{proposition}
One sees immediately that $r$ sends $q$-cartesian morphisms in $\LS(\category{C})_{\otimes}$ to $p$-cartesian morphisms in $\TSCart$, and has the sanity check that for any morphism $f$ in $\LS(\category{C})_{\otimes}$ whose image in $\ICatCart$ is $p$-cartesian, $f$ is $q$-cartesian if and only if it is $r$-cartesian.

\section{The triple structures}
\label{sec:the_triple_structures}

\subsection{The triple structure on finite pointed sets}
\label{ssc:the_triple_structure_on_finite_pointed_sets}

We're not gonna notationally distinguish between $\Finp$ and $N\Finp$. Who has time for that these days?

We define a triple structure $\triple{F}$ on $\Finp\op$, where
\begin{itemize}
  \item $\category{F} = \Finp\op$

  \item $\category{F}\downdag = (\Finp\op)^{\simeq}$

  \item $\category{F}\updag = \Finp\op$
\end{itemize}

\begin{proposition}
  This is an adequate triple structure.
\end{proposition}
\begin{proof}
  Trivial.
\end{proof}

\begin{proposition}
  There is a Joyal equivalence $\Finp \to \Span\triple{F}$.
\end{proposition}

\subsection{The triple structure on infinity-categories}
\label{ssc:the_triple_structure_on_categories}

We define a triple $\triple{P}$ as follows.
\begin{itemize}
  \item $\category{P} = \TSCart$

  \item $\category{P}\downdag = \TSCart \times_{\Finp\op}(\Finp\op)^{\simeq}$

  \item $\category{P}\updag = \TSCart$
\end{itemize}

\begin{proposition}
  The cartesian fibration $\TSCart \to \Finp\op$ admits relative pullbacks.
\end{proposition}
\begin{proof}
  The fibers clearly admit pullbacks, and the pullback maps commute with pullbacks because products commute with pullbacks.
\end{proof}

\begin{proposition}
  This is an adequate triple structure.
\end{proposition}
\begin{proof}
  It suffices to show that $\TSCart$ admits pullbacks, and that for any pullback square in $\TSCart$ lying over a square
  \begin{equation*}
    \begin{tikzcd}
      \langle n \rangle
      & \langle m \rangle
      \arrow[l]
      \\
      \langle n' \rangle
      \arrow[u, "\phi"]
      & \langle m' \rangle
      \arrow[l]
      \arrow[u, swap, "\simeq"]
    \end{tikzcd}
  \end{equation*}
  in $\TSCart$, the morphism $\phi$ is an isomorphism in $\Finp$. But $\TSCart$ admits relative pullbacks and $\Finp\op$ admits pullbacks, so $p$ preserves pullbacks. (Just take the proof from my thesis!)
\end{proof}

\subsection{The triple structure on the twisted arrow category}
\label{ssc:the_triple_structure_on_the_twisted_arrow_category}

We define a triple $\triple{Q}$ as follows:
\begin{itemize}
  \item $\category{Q} = \LS(\category{C})_{\otimes}$

  \item $\category{Q}\downdag = \LS(\category{C})_{\otimes} \times_{\Finp\op}(\Finp\op)^{\simeq}$

  \item $\category{Q}\updag$ consists of the subcategory of $\LS(\category{C})_{\otimes}$ of $r$-cartesian morphisms, i.e.\ thin morphisms.
\end{itemize}

These triple constructions correspond to spans of the following form.
\begin{equation*}
  \begin{tikzcd}
    \vec{X}
    \arrow[d, ""{name=L, right}]
    & \vec{Z}
    \arrow[d, ""{name=ML, left}, ""{name=MR, right}]
    \arrow[r]
    \arrow[l]
    & \vec{Y}
    \arrow[d, ""{name=R, right}]
    \\
    \category{C}^{n}
    \arrow[d, no head, dotted, bend left]
    \arrow[r]
    & \category{C}^{m}
    \arrow[d, no head, dotted, bend left]
    & \category{C}^{m}
    \arrow[l, swap, "\simeq"]
    \arrow[d, no head, dotted, bend left]
    \\
    \vec{X}
    \arrow[d, no head, dotted, bend left]
    & \vec{Z}
    \arrow[d, no head, dotted, bend left]
    \arrow[r]
    \arrow[l]
    & \vec{Y}
    \arrow[d, no head, dotted, bend left]
    \\
    \langle n \rangle
    \arrow[r]
    & \langle m \rangle
    & \langle m \rangle
    \arrow[l, swap, "\cong"]
    \arrow[from=ML, to=L, phantom, shorten=1.5ex, "\circlearrowleft"{description}]
    \arrow[from=MR, to=R, Rightarrow, shorten=2ex]
  \end{tikzcd}
\end{equation*}

According to the modified Barwick's theorem, such a span will correspond to a cocartesian morphism in $\Span\triple{P}$ if it is of the form

\begin{equation*}
  \begin{tikzcd}
    \vec{X}
    \arrow[d, "\mathcal{F}"{left}, ""{name=L, right}]
    & \vec{Z}
    \arrow[d, "g^{*}t \circ \mathcal{F}"{description}, ""{name=ML, left}, ""{name=MR, right}]
    \arrow[r, "f"]
    \arrow[l, swap, "g"]
    & \vec{Y}
    \arrow[d, ""{name=R, right}, "f_{!}(g^{*}t\mathcal{F})"]
    \\
    \category{C}^{n}
    \arrow[r]
    & \category{C}^{m}
    & \category{C}^{m}
    \arrow[l, swap, "\cong"]
    \arrow[from=ML, to=L, phantom, shorten=1.5ex, "\circlearrowleft"{description}]
    \arrow[from=MR, to=R, Rightarrow, shorten=2ex]
  \end{tikzcd}.
\end{equation*}
In particular, in the fiber over $\langle 1 \rangle \in \Finp\op$, this is simply pull-push.

%\begin{lemma}
%  Given a square
%  \begin{equation*}
%    \begin{tikzcd}
%      \langle n \rangle
%      & \langle n \rangle
%      \arrow[l, swap, "\simeq"]
%      \\
%      \langle m \rangle
%      \arrow[u]
%      & \langle m \rangle
%      \arrow[l, swap, "\simeq"]
%      \arrow[u]
%    \end{tikzcd}
%  \end{equation*}
%  in $\Finp\op$ and a solid cospan
%  \begin{equation*}
%    \begin{tikzcd}
%      y
%      \arrow[r, dashed]
%      \arrow[d, dashed]
%      & y'
%      \arrow[d, "g"]
%      \\
%      x
%      \arrow[r]
%      & y
%    \end{tikzcd}
%  \end{equation*}
%  lying over it, where $g$ is cartesian, a relative pullback always exists.
%\end{lemma}
%\begin{proof}
%  The fiber of $q$ over $\langle n \rangle \in \Finp\op$ is equivalent to $\LS(\category{C})^{n}$, which admits pullbacks by
%\end{proof}

\begin{proposition}
  The triple $\triple{Q}$ is adequate.
\end{proposition}
\begin{proof}
  We need to show that for any solid diagram in $\LS(\category{C})_{\otimes}$ of the form
  \begin{equation*}
    \begin{tikzcd}
      y
      \arrow[r, dashed, "f'"]
      \arrow[d, swap, dashed, "g'"]
      & y'
      \arrow[d, "g"]
      \\
      x
      \arrow[r, "f"]
      & y
    \end{tikzcd},
  \end{equation*}
  where $g$ is cartesian and $f$ lies over an isomorphism in $\Finp\op$, a dashed pullback exists, and for each such dashed pullback, $g'$ is cartesian and $f'$ lies over an isomorphism in $\Finp\op$. The existence of such a pullback is easy; one takes a pullback in $\Finp\op$, takes a $q$-cartesian lift $g'$, then fills an inner and an outer horn. That this is pullback is immediate because $g$ and $g'$ are $q$-cartesian. That any other pullback square has the necessary properties follows from the existence of a comparison equivalence.
\end{proof}

\section{Building categories of spans}
\label{sec:building_categories_of_spans}

\begin{proposition}
  The map $p$ satisfies the conditions of the original Barwick's Theorem.
\end{proposition}
\begin{proof}
  See thesis.
\end{proof}

\begin{proposition}
  The map $r$ satisifes the conditions of the modified Barwick's Theorem.
\end{proposition}
\begin{proof}
  The first condition is obvious since $r$ is a cartesian fibration. In order to check the second condition, we fix a square $\sigma$ in $\Tw(\category{C})_{\otimes}$, lying over a square $r(\sigma)$ in $\TSCart$ and a square $q(\sigma)$ in $\Finp\op$. Fixing notation, we will let $q(\sigma)$ be the square
  \begin{equation*}
    \begin{tikzcd}
      \langle n \rangle
      & \langle n \rangle
      \arrow[l, swap, "\gamma'"]
      \\
      \langle m \rangle
      \arrow[u, "\tau'"]
      & \langle m \rangle
      \arrow[u, swap, "\tau"]
      \arrow[l, swap, "\gamma"]
    \end{tikzcd},
  \end{equation*}
  where $\tau$ and $\tau'$ are isomorphisms.

  The square $r(\sigma)$ thus has the form
  \begin{equation*}
    \begin{tikzcd}
      \vec{X}'
      \arrow[r, "f'"]
      \arrow[d, swap, "g'"]
      & \vec{Y}'
      \arrow[d, "g"]
      \\
      \vec{X}
      \arrow[r, "f"]
      & \vec{Y}
    \end{tikzcd},
  \end{equation*}
  where $f$ has components
  \begin{equation*}
    f_{i}\colon X_{i} \to Y_{\gamma^{-1}(i)},\qquad i \in \langle m \rangle^{\circ},
  \end{equation*}
  $f'$ has components
  \begin{equation*}
    f'_{j}\colon X'_{j} \to Y'_{\gamma'^{-1}(j)},\qquad j \in \langle n \rangle^{\circ},
  \end{equation*}
  $g$ has components
  \begin{equation*}
    g_{j}\colon Y'_{j} \to \prod_{\tau(i) = j} Y_{i},\qquad j \in \langle n \rangle^{\circ},
  \end{equation*}
  and $g'$ has components
  \begin{equation*}
    g'_{j}\colon X'_{j} \to \prod_{\tau'(i) = j} X_{i},\qquad j \in \langle n \rangle^{\circ}.
  \end{equation*}

  Note that the condition that $r(\sigma)$ be pullback is the condition that
  \begin{equation*}
    X'_{j} \simeq Y'_{j} \times_{\left( \prod_{\tau'(i) = j} Y_{i} \right)} \left( \prod_{\tau'(i) = j} X_{i} \right),
  \end{equation*}
  and that the maps $f'_{j}$ and $g'_{j}$ be the canonical projections.

  The data of the square $\sigma$ is thus given by a diagram of the form
  \begin{equation*}
    \begin{tikzcd}
      \vec{X}'
      \arrow[rrr, "f'"]
      \arrow[dr, "g'"]
      \arrow[ddd, swap, "\tau'_{\otimes} \circ g'^{*}\mathcal{F}"]
      &&& \vec{Y}'
      \arrow[ddd, near end, "\mathcal{G}"]
      \arrow[dr, "g"]
      %\arrow[from=R1, to=L2, Rightarrow, shorten=6ex]
      \\
      & \vec{X}
      \arrow[rrr, crossing over, "f"]
      &&& \vec{Y}
      \arrow[ddd, "f_{!}\mathcal{F}"]
      \\
      \\
      \category{C}^{(n)}
      &&& \category{C}^{(n)}
      \arrow[lll, near start, "\gamma'_{\otimes}"]
      \\
      & \category{C}^{(m)}
      \arrow[ul, "\tau'_{\otimes}"]
      \arrow[from=uuu, crossing over, near start, "\mathcal{F}"]
      &&& \category{C}^{(m)}
      \arrow[lll, "\gamma_{\otimes}"]
      \arrow[ul, "\tau_{\otimes}"]
      %\arrow[from=R1, to=L3, phantom, "\circlearrowleft"{description}]
      %\arrow[from=R3, to=L4, Rightarrow, crossing over, shorten=6ex]
      %\arrow[from=R2, to=L4, phantom, "\circlearrowleft"{description}]
    \end{tikzcd},
  \end{equation*}
  where the left and right faces (as well as the top and bottom faces) are thin, and the front face is a left Kan extension, i.e.\

  . Here $\mathcal{F}$ has components
  \begin{equation*}
    \mathcal{F}_{i}\colon X_{i} \to \category{C}_{i}\colon i \in \langle m \rangle^{\otimes}.
  \end{equation*}

  Kan extending $\mathcal{F}$ along $f$ we find a map $f_{!}\mathcal{F}\colon \vec{Y} \to \category{C}^{(m)}$, with components
  \begin{equation*}
    (f_{!}\mathcal{F})_{i} \simeq (f_{\gamma(i)})_{!}\mathcal{F}_{\gamma(i)}\colon Y_{i} \to \category{C}_{i}.
  \end{equation*}
  Thus, we find that
  \begin{equation*}
    \mathcal{G} \simeq \tau_{\otimes} \circ g^{*}f_{!}\mathcal{F},
  \end{equation*}
  with components given by the composition
  \begin{equation*}
    \mathcal{G}_{j}\colon Y'_{j} \to \prod_{\tau(i) = j} Y_{j} \to \prod_{\tau(i) = j} \category{C}_{i} \to \category{C}_{j}.
  \end{equation*}

  Alternatively, starting from $\mathcal{F}$ and pulling back along the left face gives a map
  \begin{equation*}
    \tau'_{\otimes} \circ g'^{*}\mathcal{F}
  \end{equation*}
  with components
  \begin{equation*}
    (\tau'_{\otimes} \circ g'^{*}\mathcal{F})_{j}\colon X'_{j} \to \prod_{\tau'(i) = j} X_{j} \to \prod_{\tau'(i) = j} \category{C}_{i} \to \category{C}_{j}.
  \end{equation*}
   
  We need to show that
  \begin{equation*}
    \mathcal{G} \simeq f'_{!}(\tau'_{\otimes} \circ g'^{*}\mathcal{F})
  \end{equation*}
  or equivalently, componentwise, that
  \begin{equation*}
    \mathcal{G}_{j} \simeq (f'_{\gamma'(j)})_{!}(\tau'_{\otimes} \circ g'^{*}\mathcal{F})_{\gamma'(j)}
  \end{equation*}
\end{proof}

\end{document}
