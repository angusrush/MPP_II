\documentclass[main.tex]{subfiles}

\begin{document}

\section{The base fibration}
\label{sec:base_fibrations}

The starting point of our construction is the functor
\begin{equation*}
  \begin{tikzcd}
    \Finp
    \arrow[r, "\ICat^{\times}"]
    & \Cat_{(\infty, 2)}
    \arrow[r, "\Tw"]
    & \ICat
  \end{tikzcd}
\end{equation*}
Note that this gives a monoidal category since $\Tw(\ICat^{n})\cong \Tw(\ICat)^{n}$.

We're not gonna worry too much about where the functor $\Tw$ comes from, just keep notes about what properties we're using. The relative nerve of this functor is a Cartesian fibration $\Tw(\ICat)_{\otimes} \to \Finp\op$ with the following description: an $n$-simplex $\sigma$ corresponding to a diagram
\begin{equation*}
  \begin{tikzcd}
    \Delta^{n}
    \arrow[rr, "\sigma"]
    \arrow[dr, swap, "\phi"]
    && \Tw(\ICat)_{\otimes}
    \arrow[dl, "p'"]
    \\
    & \Finp\op
  \end{tikzcd}
\end{equation*}
corresponds to the data of, for each subset $I \subseteq [n]$ having minimal element $i$, a map
\begin{equation*}
  \tau(I)\colon \Delta^{I} \to \Tw(\ICat)^{i}
\end{equation*}
such that For nonempty subsets $I' \subseteq I \subseteq [n]$, the diagram
\begin{equation*}
  \begin{tikzcd}
    \Delta^{I'}
    \arrow[r]
    \arrow[d, hook]
    & \Tw(\ICat)^{i'}
    \arrow[d]
    \\
    \Delta^{I}
    \arrow[r]
    & \Tw(\ICat)^{i}
  \end{tikzcd}
\end{equation*}

\begin{example}
  An object of $\Tw(\ICat)_{\otimes}$ lying over $\langle n \rangle$ corresponds to a collection of functors $\category{C}_{i} \to \category{D}_{i}$, $i \in \langle n \rangle^{\circ}$. We will denote this $\vec{\category{C}} \to \vec{\category{D}}$.

  A morphism in $\Tw(\ICat)_{\otimes}$ lying over the active map $\langle 1 \rangle \leftarrow \langle 2 \rangle$ in $\Finp\op$ consists\footnote{Here we mean the morphism in $\Finp\op$ corresponding to the active map $\langle 2 \rangle \to \langle 1 \rangle$ in $\Finp$.} of
  \begin{itemize}
    \item A `source' object $F\colon \category{C} \to \category{C}'$

    \item A pair of `target' objects $G_{i}\colon \category{D}_{i} \to \category{D}'_{i}$, $i = 1$, $2$.

    \item A morphism
      \begin{equation*}
        \begin{tikzcd}
          \category{C}
          \arrow[r, "\alpha"]
          \arrow[d, ""{name=L, right}, "F"{left}]
          & \category{D}_{1} \times D_{2}
          \arrow[d, ""{name=R, left}, "G_{1} \times G_{2}"{right}]
          \\
          \category{C}'
          & \category{D}'_{1} \times \category{D}'_{2}
          \arrow[l, "\beta"]
          \arrow[Rightarrow, from=L, to=R, shorten=2ex, "\eta"{description}]
        \end{tikzcd}
      \end{equation*}
      where here we mean $\eta\colon F \Rightarrow \beta \circ (G_{1} \times G_{2}) \circ \alpha$.
  \end{itemize}
\end{example}

Note that $\Tw(\ICat)_{\otimes}$ admits a map to $\TICatCart \times (\TICatCocart)\op$, where
\begin{itemize}
  \item $\TICatCart$ is the relative nerve (as a cartesian fibration) of the functor $\Finp \to \ICat$ giving the cartesian monoidal structure on $\ICat$

  \item $\TICatCocart$ is the \emph{cocartesian} relative nerve of the same.
\end{itemize}

Taking all this together gives a commuting triangle
\begin{equation*}
  \begin{tikzcd}
    \Tw(\ICat)_{\otimes}
    \arrow[rr]
    \arrow[dr]
    && \TICatCart \times (\TICatCocart)\op
    \arrow[dl]
    \\
    & \Finp\op
  \end{tikzcd}.
\end{equation*}
where
\begin{proposition}
  Everything in sight is a cartesian fibration.
\end{proposition}
\begin{proof}
  The left-hand map is a cartesian fibration by construction, as is the right-hand map. The 
\end{proof}

\begin{proposition}
  There is an equivalence $\ICatCart \simeq \TICatCart$ which is homotopic to the identity on $\ICat$.
\end{proposition}

\begin{proposition}
  We can express a monoidal $\infty$-category as a functor $\Finp \to \TICatCocart$.
\end{proposition}

Fix some monoidal $\infty$-category $\category{C}$ which admits colimits, and such that the monoidal structure preserves colimits in each slot. We can thus define a functor
\begin{equation*}
  \ICatCart \to \TICatCart \times (\TICatCocart)\op
\end{equation*}
which is the equivalence on the first component, and given by the composition
\begin{equation*}
  \ICatCart \to \Finp\op \overset{\category{C}}{\to} (\TICatCocart)\op.
\end{equation*}

Forming the pullback
\begin{equation*}
  \begin{tikzcd}
    \LS(\category{C})_{\otimes}
    \arrow[r]
    \arrow[d, swap, "r"]
    & \Tw(\ICat)_{\otimes}
    \arrow[d]
    \\
    \ICatCart
    \arrow[r]
    & \TICatCart \times (\TICatCocart)\op
  \end{tikzcd}
\end{equation*}
gives us a commutative triangle
\begin{equation*}
  \begin{tikzcd}
    \LS(\category{C})_{\otimes}
    \arrow[rr, "r"]
    \arrow[dr, swap, "q"]
    && \ICatCart
    \arrow[dl, "p"]
    \\
    & \Finp\op
  \end{tikzcd}.
\end{equation*}

\begin{proposition}
  The map $p$ is a cartesian fibration, and the $p$-cartesian morphisms in $\ICatCart$ are precisely those representing products.
\end{proposition}

\begin{proposition}
  The maps $q$ and $r$ are cartesian fibrations, and a generic morphism
  \begin{equation*}
    \begin{tikzcd}
      \vec{\category{D}}
      \arrow[r, "\alpha"]
      \arrow[d, ""{name=L, right}, "F"{left}]
      & \vec{\category{D}}'
      \arrow[d, ""{name=R, left}, "G"{right}]
      \\
      \vec{\category{C}}
      & \vec{\category{C}}
      \arrow[l, "\otimes"]
      \arrow[Rightarrow, from=L, to=R, shorten=1.5ex, "\eta"{description}]
    \end{tikzcd}
  \end{equation*}
  in $\LS(\category{C})$ is
  \begin{itemize}
    \item $r$-cartesian if $\eta$ is an equivalence

    \item $q$-cartesian if $\alpha$ exhibits a product and $\eta$ is an equivalence.
  \end{itemize}
\end{proposition}
One sees immediately that $r$ sends $q$-cartesian morphisms to $p$-cartesian morphisms, and has the sanity check that for any morphism $f$ in $\LS(\category{C})_{\otimes}$ whose image in $\ICatCart$ is $p$-cartesian, $f$ is $q$-cartesian if and only if it is $r$-cartesian.

\section{The triple structures}
\label{sec:the_triple_structures}

\subsection{The triple structure on finite pointed sets}
\label{ssc:the_triple_structure_on_finite_pointed_sets}

We're not gonna notationally distinguish between $\Finp$ and $N\Finp$. Who has time for that these days?

We define a triple structure $\triple{F}$ on $\Finp\op$, where
\begin{itemize}
  \item $\category{F} = \Finp\op$

  \item $\category{F}\downdag = (\Finp\op)^{\simeq}$

  \item $\category{F}\updag = \Finp\op$
\end{itemize}

\begin{proposition}
  This is an adequate triple structure.
\end{proposition}
\begin{proof}
  Trivial.
\end{proof}

\begin{proposition}
  There is a Joyal equivalence $\Finp \to \Span\triple{F}$.
\end{proposition}

\subsection{The triple structure on infinity-categories}
\label{ssc:the_triple_structure_on_categories}

We define a triple $\triple{P}$ as follows.
\begin{itemize}
  \item $\category{P} = \ICatCart$

  \item $\category{P}\downdag = \ICatCart \times_{\Finp\op}(\Finp\op)^{\simeq}$

  \item $\category{P}\updag = \ICatCart$
\end{itemize}

\begin{proposition}
  The cartesian fibration $\ICatCart \to \Finp\op$ admits relative pullbacks.
\end{proposition}

\begin{proposition}
  This is an adequate triple structure.
\end{proposition}

\subsection{The triple structure on the twisted arrow category}
\label{ssc:the_triple_structure_on_the_twisted_arrow_category}

We define a triple $\triple{Q}$ as follows:
\begin{itemize}
  \item $\category{Q} = \LS(\category{C})_{\otimes}$

  \item $\category{Q}\downdag = \LS(\category{C})_{\otimes} \times_{\Finp\op}(\Finp\op)^{\simeq}$

  \item $\category{Q}\updag$ consists of the subcategory of $\LS(\category{C})_{\otimes}$ of $r$-cartesian morphisms.
\end{itemize}

These triple constructions correspond to spans of the following form.
\begin{equation*}
  \begin{tikzcd}
    \vec{\category{D}}
    \arrow[d, ""{name=L, right}]
    & \vec{\category{D}}'
    \arrow[d, ""{name=ML, left}, ""{name=MR, right}]
    \arrow[r]
    \arrow[l]
    & \vec{\category{D}}''
    \arrow[d, ""{name=R, right}]
    \\
    \category{C}^{n}
    \arrow[d, no head, dotted, bend left]
    \arrow[r]
    & \category{C}^{m}
    \arrow[d, no head, dotted, bend left]
    & \category{C}^{m}
    \arrow[l, swap, "\simeq"]
    \arrow[d, no head, dotted, bend left]
    \\
    \vec{\category{D}}
    \arrow[d, no head, dotted, bend left]
    & \vec{\category{D}}'
    \arrow[d, no head, dotted, bend left]
    \arrow[r]
    \arrow[l]
    & \vec{\category{D}}''
    \arrow[d, no head, dotted, bend left]
    \\
    \langle n \rangle
    \arrow[r]
    & \langle m \rangle
    & \langle m \rangle
    \arrow[l, swap, "\cong"]
    \arrow[from=ML, to=L, phantom, shorten=1.5ex, "\circlearrowleft"{description}]
    \arrow[from=MR, to=R, Rightarrow, shorten=2ex]
  \end{tikzcd}
\end{equation*}

According to the modified Barwick's theorem, such a span will correspond to a cocartesian morphism in $\Span\triple{P}$ if it is of the form

\begin{equation*}
  \begin{tikzcd}
    \vec{\category{D}}
    \arrow[d, "\mathcal{F}"{left}, ""{name=L, right}]
    & \vec{\category{D}}'
    \arrow[d, "g^{*}\mathcal{F}t"{description}, ""{name=ML, left}, ""{name=MR, right}]
    \arrow[r, "f"]
    \arrow[l, swap, "g"]
    & \vec{\category{D}}''
    \arrow[d, ""{name=R, right}, "f_{!}(t \circ g^{*}\mathcal{F}t)"]
    \\
    \category{C}^{n}
    \arrow[r]
    & \category{C}^{m}
    & \category{C}^{m}
    \arrow[l, swap, "\cong"]
    \arrow[from=ML, to=L, phantom, shorten=1.5ex, "\circlearrowleft"{description}]
    \arrow[from=MR, to=R, Rightarrow, shorten=2ex]
  \end{tikzcd}.
\end{equation*}
In particular, in the fiber over $\langle 1 \rangle \in \Finp\op$, this is simply pull-push.
\end{document}
